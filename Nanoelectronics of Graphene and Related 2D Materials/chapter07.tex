\fig{72}{Nanoelectronics of graphene and related 2D materials 2024}
Okay, so it's time now to calculate quantum capacitance and to do so we're just going to use definition. So basically we have to find the relationship between the charge and the shift of the band structure, that is by how much the band structure should move down so that we get the charge Qg, and from that we can then simply take the derivative, calculate the quantum capacitance, which is then done at the following slide. So what we're going to do is now I'm going to redraw this picture here, just more precisely. So this picture is now shown here. So what we have here in the upper left is the situation when VGS is not applied, so this is the case VGS equals zero, in which case we have only impurity charges in the channel Q in, okay. And then what you see on the right hand side is the other part of the plot from the previous slide where the band capture moved down by EV channel S. So that means if the direct point here was at zero, it means here is moved down by EV channel S, so that means the direct point is here, Z minus EV channel S. And then this Fermi level which was here is now moved down by EV channel S, so it's here, EF minus evic channel S, but because the actual permeability in the system is this one, the one we had previously, obviously here we have extra filled states, which means we have fewer empty states in the valence band by pumping electrons inside the system to the capacitor, right? And that's why the charge we have in the channel is not any more impurity charge, but impurity charge minus the charge induced by the applying voltage QG, where the QG is this extra charge between these two levels. Okay, so how are we going to now find the relationship between this charge QG and by how much the band structure moves down. So that's what I said, we have to find out by how much we should move band structure down so that here we get charge QG, so we need some relationship. And this is very easy to do. First, we should look at this picture and simply say, if you now go from this Fermi level up by EV channel S, so by the same amount the band structure on the right went down, that means that here we have now if you look at this red level here which is EF plus EV channel S the charge we have here is identical to charge we have here right so because then the pictures become identical and that means you can use this Fermi level and this charge simply to find relationship how you go back to the slide with equilibrium and if you go back no VGS. You will find out I showed you there is a relationship between the impurity charge and the Fermi level in the system. Okay that's what we found in slide 66. And now basically this is the same. We can use exactly the same formula for impurity charge, we just have to replace Fermi level with this, which is EF plus EV channel S and the value we are going to get is this Qin minus Qg. So basically I'm basically using the same formula from slide 66 in which I found the relationship between the impurity charge and the Fermi level. So if you go back to slide 66 you will find this relationship here which I calculated through the Fermi-Dirac integral, if you remember. And this picture here is the same. This is basically the same case, it's just we have a different Fermi level and different amount of charge. they are related to the formula I gave you on slide 66, which is given by, if you remember, given by Fermi-Dirac internals. And now the only difference is that what I did here was to say that you're going to get this Q-in-Qg, so the charge we need, if we go back to this expression from slide 66 and replace EF by EF plus EV channel source. And that means if you go back to slide 66 you will find this expression for the impurity charge. The only difference here you have X and now where X was if you remember Fermi level divided by kVT but now instead of EF we have EF plus this. So that means that instead of X we will have Y where Y is X plus EV channel source divided by KBT that's the only difference. That's it basically that's the expression we need because now we have relationship we need because this is Q if minus QG so we have a QG here and here we have the EV channel S which is hidden in the expression for y and that gives you relationship between QG and B channel s that's what we need to find the quantum capacitance. Although you may argue now okay if you solve this now by QG this is what you're going to get so you put QG on the right hand side you take this put on the left hand side with Qimp you get this and you may argue okay but I have a Qimp here how to get rid of it well you don't have to get rid of it simply because you're going to take derivative to find the quantum capacitor because this is the expression I put on the previous slide definition of the quantum capacitors and when you take derivative with respect to the V channel s this Q imp is of course constant because Q imp is impurity charge which is defined by the doping and it's not related to the applied gate source voltage so this is what you have Q imp in the absence of the gate source and for that reason this Q is constant so when you take the derivative with respect to V channel S derivative of Q is zero and then basically you just have to take derivative of this expression here given by Fermi-Dirac integral and then okay then let's take the derivation of this expression so you say derivative with respect to V channel S is because here we have a pi which is a function of V channel s, then you simply say derivative of qg with respect to v channel s is derivative of qg with respect to y, times the derivative of y with respect to v channel s. Okay, and then the first derivative is derivative of this expression with respect to y. So this is basically this constant here, you put it here, and you remember I showed you that the derivative of the Fermi-Dirac integral is the Fermi-Dirac integral of the lower order, y1. So this becomes F0 and this becomes also F0. So you get this. And finally, we have derivative of Y with respect to V channel S, which is just E divided by KBT. And then again, if you go back to slide 66, you will find out the expression for F0. I told you F0 is the only fermi-zero integral which can be solved in the flow of 4. If you get the expression for F0, you will find that F0 of X, for instance, is logarithm of 1, natural logarithm of 1 plus e to x. So then you write down expression for f0 of y like this, logarithm of 1 plus e to y and this is logarithm of 1 plus e to minus y, you get this, and then logarithm plus logarithm is logarithm of the product, so here is logarithm and the product is 1 times 1 is 1, e to y times minus e to y is also 1, which is 2, and you have e to y plus e to minus y, which is 2 times hyperbolic cosine of y, which is given by this expression here. Y is, remember, is ef plus ev channel s divided by k vt. That's what you get. And basically, you see that at room temperature, at any finite temperature, you can get the expression for quantum capacitance in the globe of the delta and the trojan. Okay, and that's the finite temperature at zero kelvin. You can also of course get the expression at zero kelvin you basically replace, you basically repeat what I've done here exactly the same, and you just here use the expression for impurity charge at zero kelvin. You take impurity charge at zero kelvin and then you put, when you replace EF by this, this is what you're going to get. You get that this charge is a square function of VF, which is, if you remember from slide 66, and then of course to find again the quantum capacitance, you write down QG as Qimp minus this, which is this expression here, and then when you take the derivative similar to there, this is constant and derivative will give you this at the end. Notice at the end you get the true temperature that you get that the quantum capacitance is a linear function of the Fermi level, which is understandable. Why? Because if you remember how the charge is calculated at zero kelvin, it will be clear. Why? Because you remember the dose is a linear function of energy and then when you integrate it to get the when you integrate the dose and the Fermi-Dirac distribution function which is only one or zero zero kelvin, you get of course integration of the linear function, you get the square function of energy. And then when you take the derivative to calculate the quantum capacitance, you come back to the linear function. Okay? So that's finally the expression for the quantum capacitance at zero kelvin, and then when you draw it, it looks like this. So let's comment on the plot we got. First, let's comment on the plot at zero kelvin, because this one is easier to understand before switch to the other one. So if you look at the blue plot, that's the quantum capacitance of zero kelvin as a function of this eV channel s, which is the band structure shift. And obviously, quantum capacitance is zero when this expression is zero, that is when eV channel s is minus eF. So when the eV channel s is minus EF we have a zero and left and right from that you have like absolute x function which looks like this So mathematically, I hope this is clear, but now let's try to understand what's going on here. So first of all What does it mean what happens when the eV general S is equal to minus EF? Notice that if we have a p-doped sample then the EF is in valence band as we discussed which means EF is negative because if you remember the Dirac point is at zero so EF is negative in the valence band which means that minus EF is positive okay and by the way these numbers here mean nothing this is just some arbitrary scale so basically this is the value of minus EF which is a positive number okay so then the EV channel S is equal to minus EF, what do we have? When the band structure is so large so that it's equal to minus EF which is a positive number. Look at this picture here. Here you have Dirac point at minus EF channel S. When the EF channel S is minus EF, that means that this value is how much? It's exactly EF. Which means that when the EF channel S is minus EF, this positive number EF, minus EF, the band structure move down so much that the Dirac point hits the Fermi level. So basically that means this is the Dirac point here. Okay, so add a Dirac point, add a Dirac point, so when the minus eV channel S is EF, so when we are here, so when this point hits the Fermi level, add the Dirac point, the quantum capacitance is zero. Why? Well, first of all, you remember I showed you that the device, I'm not going to go back to the previous slide, I will just draw it here, it's faster. You remember the device between gate and source can be represented as the series connection of two capacitors, right? We have the quantum capacitance and we have here the geometrical gate capacitance, you remember this, okay. And that means that the equivalent capacitance is, as we already discussed, is this. And now if the quantum capacitance is zero, this is infinite, in other words, you get that the total capacitance is just equal to the quantum capacitance which is zero. That means quantum capacitance totally dominates the device capacitance because it's equal to zero. What that means, that it's zero? Well, if you remember the explanation I gave you from the previous slide, what do we model by quantum capacitance? With quantum capacitance we model the fact that to get some charge QG, we have to shift down the band structure, right? But at zero kelvin, at the direct point, how much is DOS? Zero. And now consider this, if the DOS, if the density of state is smaller, you have to shift the band structure more to get the same amount of charge. But if the DOS is not small, but zero, you basically have to shift band structures by infinite value to get some finite number of charge. Which means that if you look at the definition of the quantum capacitance, in that case this is infinitely large and hence the quantum capacitance is equal to zero. That's a very simple explanation because those are the Dirac points of zero. So we basically have to shift by infinite amount, then structure down to get any finite charge. And when you put the dv channel as infinite you get that this is zero. So that means that the Dirac point c is equal to cq equals zero. That's it. As you go away from the direct point either you go to the left or to the right what's to the left of this point what to the left of this point to the left of this point is that EV channel s is smaller than this value so it's not large enough so that this point hits the Fermi level, which means it's somewhere here, right? Which means Fermi level is inside the valence band. So to the left you have the valence band and of course to the right there is a conduction band. Because when eV channel S becomes larger than this positive number minus Cf, then this point goes below Fermi level and the Fermi level gets inside the conduction band, right so as you go to the left or to the right from the Dirac point as you go to the valence or conduction band the DOS finally increases it's not Dirac so as you go more to the left or more to the right the density of state increases which means you have to shift band structure for less to get the same charge and that means as you go more to the left and to the right this decreases because you have to shift band structure for less and that means CQ increases because it's inversely proportional to this, it increases and that's why, so that's a simple physical explanation why when you go to the left and to the right, quantum capacitance simply increases, right? And then if you look at this expression, if quantum capacitance increases that means it dominates less less and less the total capacitance. And if the quantum capacitance becomes very large, then you can neglect this term and basically quantum capacitance doesn't play any role anymore. But for that to get the total capacitance equal to geometrical gate capacitance, you have to drive the system far away from the limited dose. So to drive the system far away from the Dirac point, so we have so many, so the density of state is so large that you don't have to basically shift the band structure down to get some charge. Basically you are approaching your conventional metal situation in which there is no band structure shift and in which this quantum capacitance is so large that it becomes negligible in this series. Okay? Then you neglect it and you get that this is equal to this. So that's the situation of Dirac-Helvig. happens at finite temperature which is now plotted here in orange. So you can also plot this at home because you have a hyperbolic cosine so this is easy to plot. But you will notice that this function is always above that one, above the one at 0 Kelvin Pi, because even if the system is biased to the Dirac point due to thermal fluctuations, there are occupied states both in, you have both electrons in the conduction band and holes in the valence band, which means that you have states even at available states even at Dirac point, which means you don't have to shift the band structure by infinite amount to get the finite charge. Which means obviously in that case the quantum capacitance is larger, so that's why it's larger, it's above zero, and also similar to zero Kelvin, the more you go away from the dear point the density of state is higher that means you need to shift the band structure by smaller amount to get a finite charge and that's why CQ also temperature I mean a finite temperature also goes up okay so that's the behavior of the quantum capacitor and now the question so is this a positive or negative thing this quantum capacitance so it's obviously a negative thing because basically when you apply a voltage some voltage will drop on quantum capacitance rather than this capacitance here and that's why you should would like to get rid of it but there is no way to get rid of it in the system with limited loss. That's what it is. I mean one option to get rid of this quantum capacitance would be to have CG extremely small because if CG is extremely small, if you set it, okay, you cannot make it of course smaller than zero, that's impossible, but anyway you usually don't operate device at zero Kelvin. So if you operate device at some finite temperature, you may argue because CG is a constant, if I put this constant CG way way below this minimum, then 1 over CG is going to dominate this sum, so it becomes much lower capacitance and it dominates the series connection, but that makes no sense. Why? Because you want this capacitance to be as large as possible, not as small as possible. Because you want to modulate the charge in the channel more and for that reason you cannot play with this. In other words there is no way around it. It's always present in the system with limited dose. And finally how is this measured experimentally? Usually CV measurement is done to find the total capacitance of the system because when you, as Eldrei told you last time, this point here is the channel and the capacitance between the gate and the channel, which is the capacitance which determines how much, how well you modulate the current of the transistor, is given by these capacitance between these two points. Unfortunately, as I already told you, you cannot access this point here, you can only access gate and source externally. So when we do the so-called CV measurement to find out capacitance of the device, we can only find this series connection, we cannot find individually these two, because we have no access to the channel, because the only point we have is gate or source. But actually we can find out how much is CG, and you can immediately ask me but why do we need why do we need Tg? Wasn't Tg constant defined by geometry of the device that is width and length and the electric constant and thickness of the insulator gate insulator? Yes that's true but the problem is that in small devices it's very hard to understand how much is exactly this constant. Why? Because imagine if you make something on a nanoscale then you make a gate, it is going to have a huge impact whether the gate length is 15 nanometers or 16 or 17 nanometers. That will significantly change the gate capacity. So you have to be able, if you would like really to know it, you have to be able to fabricate gate with exactly the same parameters you need. And that's not always easy to do. So that's problem number one when make such small devices. And the problem number two is apart from the geometry as I said it depends on the thickness and the electric constant of the insulator and these two are all are also not always known in advance. Why? Because the thickness of the oxide depends how thick you deposit oxide and because the oxide is extremely thin we are talking about cubes that I'm indicating insulator is extremely thin, we are talking about a few nanometers. So even one nanometer off will change significantly the capacitance. And the biggest problem is actually the electric constant, because the values you care for the electric constant for different materials are given for value. But in a gate insulator, you care about very thin gate insulators, few nanometers. And because this material is amorphous, you cannot really know in advance be the electric constant. That's why in order to fully characterize semiconductor device, you have to actually measure gate capacitance. You cannot really rely on numbers and the formula for capacitance. And the one way to do it is to do so-called CV measurements in which a DC bias is applied. So basically you sweep the DC bias, so it's a gate source voltage, you sweep the DC bias and at each point, at each DC value of VGS, capacitance is calculated how? Basically AC signal is sent to the device, AC current is measured, and the ratio will give you impedance from impedance you can get capacitance. So that's how CV measurement is done. Of course, sounds easy when I say it, but in reality you need a semiconductor parameter analyzer to do it, because if you try to do it just doing that comb like you're connecting transistor by wires and applying some AC voltage it's probably not going to work you really need a very expensive instrument to do it we have it in comb i can show you to come with a hundred thousand euro okay not only for measuring cv curves but also other things it's done for full semiconductor analysis, parameter analysis, but the point is that you really have to measure the capacitance of the device to find out the GFET. And how it's done, so I put here one result, the result obtained from one measurement like this. Basically when you have a GFET, when you do CV, and by the way, CV measurements are done for all semiconductors. You also do it for the diodes and transistors and MOSFETs to be able to fully characterize them. Anyway, if you do this measurement for GFET, this is going to be the total capacitance C, the one which is given by this formula for the series connection. Or you can also use this formula if you want, for instance. Something like this. And now you can very easily understand this plot if you look at these plots. So you basically here in series connection, you have one capacitance which changes like this, like I plotted here, changes like this, and another capacitance which is constant. So basically when you calculate, you will see that this C of course behaves as CQ. So it's going to follow this shape, which means that in this device, in this device, here is the Dirac voltage here. It's around one volt. is the Dirac voltage here because of this simple formula this capacitance C must follow the shape of CQ it goes like this so this is the Dirac voltage is here but when the quantum capacitance becomes large enough becomes negligible and C gets stuck to this constant understand so basically if you go far away this will go something like this and you will see the constant which is CG. This will be CG then. You understand? Because if you go far away, CQ becomes very large. This is zero and you get that C is equal CG. And that's why this curve looks like this. Which means that the geometrical gate capacitance you can get from the constant value which is around here. So if you continue this plot, you will get that it's about 1.4 micro farad per square centimeter that's going to be C ox C over a okay and that's how you can get to the CG and then when you find CG if you really need plot for a quantum capacitance then you can go back and say I know this one I measured it with this orange line I know this constant I just calculated from here it's about 1.4 microfarad per square centimeter and then CPU I can get from this formula. That's how you can get the quantum capacitance. But what you really need is EG to be able to understand the properties. That's it, that's the quantum capacitance, I hope this is clear.
\fig{73}{Nanoelectronics of graphene and related 2D materials 2024}
Now the other thing I like to consider is the band structure shift in each channel X. So I'm going to do it now only at zero Kelvin because if you would like to do it at finite temperature becomes very complicated due to the Fermi-Gerak integrals and that's why I've already told you quite often people just look at the expressions for, at the expressions of zero Kelvin to try to understand the and then at room temperature you know the the constant of the capacitor is just larger and so on. So let's look at the band structure shift at 0 kV because that's what we can calculate in closed form. So I would like to now find out this band structure shift. Basically, the question is this. I apply some gate source voltage and the band structure moves down. But by how much? By how much it goes down? That is, what's the relationship between band structure shift and the applied voltage Vgs? Or to reformulate the question, if I for instance apply gate source voltage 1 volt, does it mean that the band structure goes down by 1 electron volt? That would be great, because then it would be very easy to understand what's going on. but the answer is no. You can easily see why no if you look here. Because Vgs is the voltage you apply here between gate and source, but the band structure goes down by this much. When you multiply this by E, the rest goes here. The rest goes on the gate capacitor, which is voltage between the gate and the channel so if i put one volt here i'm not i'm not going to get one electron volt here because i'm not going to get one volt here because this one volt is going to be split between these two okay and that's now the question if i apply one volt by how much band structure goes down obviously it doesn't go down by one electron volt because of this this is a capacity voltage divider, right? It doesn't go down by one electron volt, but by how much? In other words, what's the relationship which connects this voltage here and Vgs? So, you start from this relationship which I put here, and And what you have here is this voltage divider here. You basically say VGS is V channel S plus VG channel. So VGS is V channel S plus VG channel. And if you remember, VG channel was VG divided by CG, because VG channel is the voltage on the gate capacitor, and for that you use the single relationship between the two. And if you remember QG, you can calculate it here, like this, why? Because if you remember the total charge Q, the total charge Q in the channel is the impurity charge minus what you induce by the gate. That's what I wrote a long time ago. So the total charge in the channel is initial impurity charge minus what you induce by the gate. this is M and then you calculate QG QG is Q in minus QA you get this okay so this is the expression I wrote long time ago and now in order to find in order to find the V channel s as a function of VGS basically what you have to do is to get rid of these two. So Qimp, you go back to slide 66 in which I investigated state without VGS. Then I showed you how to work the connection between the charge Qimp and the Fermi level. That's what we derived on slide 66. You can find this expression there and that's the expression of zero kelvin right and then we have to get rid of this q q is as you as you can understand from here q info minus q g and that's q info minus q g you can find on the previous slide that's the expression i used on the previous slide if you remember i told you this is the same expression as this one in which ef is replaced by ef plus ev channel x that's what you get and now if that's the case you should just insert these two expressions for q if and q here to get this expression to get v channel s is vgs vgs okay minus one over cg here it is and now you have this minus bit basically if you insert q input q in the first expression you're going to get this expression here and now before i continue let me just explain one thing i'm surprised no one else especially guys from electronics. Because if I have this voltage divider why am I complicating my life like this? Why cannot I simply calculate V channel S as CG divided by CQ plus CG times VGS? Why don't I use the simple formula for the for the simple capacitive divider to find the relationship between V channel S and VGS, why am I complicating my life like this? Answer is because CQ derived in the previous slide was derived in AC regime. You remember it was DQG divided by D V channel S, it was the CQ valid in AC regime, in regime of small changes. And And here I'm interested in the total change, here I'm interested in the value of ebit generalized. Not, I'm not interested in its differential. You understand? Because CQ was derived from the differential. Look at the definition. If you remember from the previous lecture I told you, when I derived, when I used this first formula, I said I can define quantum capacitance from this formula, but that would be a static quantum capacitance, which I don't need. What I need is, what's the capacitance of the device in AC regime, because that's where capacitance is important. That's why I calculated two differentials. You remember, I took differential of this formula, and then defined the quantum capacitance as you have it in the previous book, like the dQG divided by dV channel S. And that's why such derived quantum capacitance cannot be used here, because here I'm looking for v channel s not dv channel s understand so that's why I have to go hard way unfortunately had I derived previously expression for the static quantum capacitance I would be able to use it now in the formula for the capacity divider but I didn't so let's go hard way ok you basically get because it's not really so hard you basically get this expression I hope you understood how I got it from this and then you just rearrange it so if you look this line here this is basically identical to this line the only difference is that this expression here is placed on the same side as V channel S you see V channel S and this big expression is this one here and on the left hand side we have VGS and this one which we have here and in addition in addition I added EF over E on both sides so that doesn't change the equation. I just added EF over E on both sides so this one was not present here but that doesn't change equation because it's the same number on both sides. Okay and now if you look very closely at this expression I wrote here, you will notice that on the left hand side we have a VGS minus something and this is minus this value which is this one which is actually the Dirac voltage. And now immediately comes the question, hang on, how do you know this? How do you know that this value here, green, is minus Dirac voltage? So this green value here shaded in green that's the minus Dirac voltage which means that Dirac voltage is minus this which is written here but how do I know this how do I know that this is minus Dirac voltage so that the left-hand side of the equation is this to understand how do I know this you have to rewrite the right-hand side of the equation like this which is basically not much work you just have to put these two together so you get EF plus EV channel S divided by E and then you just have to add this term so not much work and now look when EV channel S is equal to minus EF that's when the system is biased such that the Fermi level is where? I told you previously, at the Dirac point, right? So when the eV channel S is minus EF, the time structure moves down so the Dirac point hits the Fermi level. But when the eV channel S is minus EF, this is zero and this is zero, which means that the left-hand side must also be zero. And because the left-hand side can be written as Vgs minus some voltage then this voltage must be the Dirac voltage because of the Dirac voltage the Fermi level is Dirac 1. You understand? So if the right-hand side gets 0 on the Dirac voltage the left-hand side must also be 0 on the Dirac voltage. So that's why this is indeed left hand side is indeed DGS minus VPH to the standard. Well I can uh you should just go back to the previous slide to what I've done and I think some of you will conclude why. Because you remember when we have the plot of the quantum capacitance which is zero kelvin look like this. The plot look like this. And I explained you that when the eV channel S is minus EF, the premium level is exactly the zero quantum, right? And that's what you have exactly here. When the eV channel S is minus EF, right hand side is zero, which means that the left hand side must be zero, but because the left-hand side is Vgs minus some voltage, we know that the Dirac point Vgs is equal to the Dirac voltage, so therefore this voltage must be the Dirac voltage, very simple. So you see, very simple, we got the expression for the Dirac voltage here, which you can then rewrite in this form, how you can just write Ef to be equal to the sine of Ef multiplied by the absolute value of EF, right? You can just use this expression. EF is sine of EF times the absolute value of EF. And then you take this sine from here and here, put in front, and what you get inside is a positive number. The reason I put it in this way was simply to indicate that if EF is negative, that is when the Fermi level is in the valence band, sine of EF is negative, minus is plus, this is always positive, you get positive threshold. So when the Fermi level is inside the valence band, then the Virak voltage is positive. Yes, of course. This is zero. This is EF negative. so if the ef is negative fermi level is in the valence band and then the direct voltage must be positive you can see in this expression why because when you apply a positive voltage band structure moves down hence you need a positive voltage so the band structure moves down so the direct point hits the fermi level right so it makes sense so this is correct and now to however what we are looking for is not the threshold or the Dirac voltage we don't need this we are looking for a dependence of each NLS on the VGS to simplify this expression notice another thing from here notice that this value here is positive always positive and here we have a sign of EF plus eV channel source and EF plus eV channel source again can be written as the absolute value of this times this sine which means that the sine of the right-hand side of the equation is equal to sine of this okay on the other hand sine of this is sine of Vgs minus Vth so you can write down that sine of the left-hand side which is this one can be equal to the sine of the right-hand side which is this one. Why did I do it in this way? Because if you replace this by sine of vgs minus vth, you put it here, what do you get with respect to ef plus ev channel s? You get a simple quadratic equation. Where is vgs in this quadratic equation? It's here. So basically, if you solve this quadratic equation by this EF plus EV channel S, you will get how the band structure depends on the applied gate source voltage. You will get how the band structure shift depends on the gate source voltage. So do it. Do it at home. When you do it at home, you're going to get this as a solution of the quadratic equation. Quadratic equation has two solutions. You will see that one is physically meaningless, you will discard it and then the other one which is meaningful is going to be this one so please do this at home I'm not going to waste time now solving quadratic equations and look what we got here we got here some NQ and NC in the old version of the slides this NC is denoted by NG I decided to call it NC I think it's better. Why? Because first what is nq? nq is some carrier density defined through this expression here. This simply comes from solving the quadratic equation. There is no mystery here. When you solve a quadratic equation and if you write it in this form, nq will be given by this. Notice you have an h bar here, which means that this nq is a carrier density associated with quantum capacitance. The other carrier density you see in this expression is the NC. I decided to change name to C as classical. So this is the classical carrier density, so the one which will end up here after solving quadratic equation. And look at the expression for this NC. You will understand why I decided to call it classical the classical carrier density the one you get without quantum capacitance and how do we know this because if you look at this expression you see that this directly comes from this expression here in which in which this QG is replaced by CGVGS. And this is CGVTH, so you basically get this expression here. You get the total charge, Q, Q, divided by AE. that's the classical carrier density that's the expression you have without quantum capacitance with quantum capacitance what changes here in this expression so with quantum capacitance this first line is still correct because the charge you get is always Q-imp minus QG but this is a problem because the voltage on the capacitor is not Vgs it's voltage between the gate and the channel so in case of the in realistic case with the quantum capacitance you should replace VGS by VG channel that's the difference but if you assume classical case without quantum capacitance and normal density of states then instead of VG channel you can use VGS and then that would be the carrier density classical carrier density but that's not a real carrier density that's just classical carrier density in case of the absence of the quantum capacitance. Anyway, this is the expression you get for the EF plus EV channel S as a function of VGS. Notice that VGS you have here in this sign, which is obvious because I already told you sine of this is sine of this and all this is positive. How do I know all this is positive? Because this is positive, this square root is larger than 1 so minus 1 is positive okay so that's clear and VGS is also hidden where in NC in the expression for the classical carrier density so that basically gives you how each NLS how the band structure shift depends on the gate source voltage which is here and inside the classical and the expression for the density and that can be plotted. So what I plotted here is the band structure shift as a function of the applied gate source voltage. You may argue that maybe this plot is not clear because I don't have here ebgenl as as a function of egs, but I have this as a function of this. But bear in mind that this is just the translation of the coordinate system. Why? Because for a given sample for a given ambient impurities Fermi level is set by ambient impurity so this is constant it doesn't depend on the applied voltage and the same for the Dirac voltage because once you set the Fermi level you also set the Dirac voltage that's the voltage you need to shift band structure so that the Dirac point hits the Fermi level so that means that in this plot actually this is constant and this is constant so if you would like to plot just eV channel s as a function of VGS it's going to look the same the only difference is that instead of origin 0 0 you get point with these coordinates V TH and minus VF ETH and minus EF you understand but the plot is still going to look the same and that's why I just plot it like this because this is constant as I said and this is constant so it's not important okay so here we see how the band structure shifts when we apply gate source voltage so where is the Dirac Dirac point here this plot now it's obviously the origin the point at which Vgs is equal Vth because when Vgs is equal to the Dirac voltage Fermi level, actually the Dirac point is the Fermi level. Okay, so the Dirac point is here, left from here for smaller voltages, as I already plotted here for smaller voltages, the Dirac point still didn't hit the Fermi level and we have a situation like here when the Fermi level is inside the valence band and the VGS is larger than the Dirac voltage then this level is below the Fermi level and the Fermi level is inside the valence band. Similar to previous plot, left values below Vth that's when the Fermi level is inside the valence band and this is when it's in the conduction band. Okay, so let's see now what happens around the Dirac point here. Around the Dirac point here this function which shows you the band structure shift as a function of the applied gate source voltage can be approximated with its Taylor series at this point and if you take only up to the first term you will get exactly this so basically we approximate just with a tangent at this point we take first two terms from the Taylor series and it's very interesting that the tangent had this equation because that means that V channel s is equal to Vgs basically if you look that means that around that means that around Dirac voltage if we apply one millivolt the band structure will shift by one millielectron volt but only for small values around the Dirac point and how do I know this is really the tangent well in theory you can take the Taylor series but there is no reason for this you can very easily simplify this expression so basically what I'm telling you when VGS minus VTH approaches zero if you look here at impression for the classical carrier density when VGS approaches VTH classical carrier density approaches zero okay and then you look at this square root you know that square root of you know that square root of 1 plus x approaches 1 plus 1 over 2x, then x approaches 0, right? You know this. So when you now apply this formula, so when nc approaches 0, this is x which approaches 0, okay? so we get 1 plus half of this minus this so 1 and 1 cancels out 2 and 2 cancels out nq nq cancels out you get nc in front and because nc is given by this you will end up with this simple expression here you will get exactly what i said which means that if we change the gate source voltage so if we go around this point one millivolt gate source voltage either to the left or to the right e v channel s will change by one millielectron volt so if you if you go by one millivolt left to the right, band structure will shift up and down by one milli-electron volt. So you really get this one-to-one correspondence. Voltage that you apply entirely shifts the band structure. Why? You can go back to this expression here. Just know that in this case we should look at the static quantum capacitance, because we look at these small changes around the Dirac voltage, we can also look at the dynamic. It doesn't matter. The point is this, at the Dirac point, because remember this is zero Kelvin, at the Dirac point, entire series connection is dominated by the quantum capacitance, which means if I change this by one millivolt, this one millivolt is going to entirely drop here, okay because this is CG divided by CG plus CQ VGS so when this is zero and I change this by one millivolt I change this by one video like I change this also by one millivolt so each ILS changes by one video like so basically at the direct point where the capacitance is totally dominated by the 1-2 capacitance, any voltage you apply, assuming it's very small, will entirely shift the band structure. You change by 5 millivolts, band structure goes down by 5 millialectron volts. But notice that this happens only for small changes around the direct voltage, around the direct point, sorry, around the direct voltage. Because as soon as you apply a larger voltage. Let's say we change Vgs by one volt. We change Vgs by one volt. As soon as you apply, as soon as you change by one volt, then the point is here, not here. Okay, so that means this is the voltage drop on the quantum capacitance. This is by how much the band structure moves down. Actually, this device is going to move down by approximately 0.3 electron volts actually go up because i should look here but doesn't matter if you shift by 0.3 electron volts and the remaining 0.7 electron volt will drop where will drop on the geometrical gate capacity okay and the further away further away the further away you go from the direct point the larger the difference between the voltage which drops on geometrical gate capacitance and the quantum capacitance which means if you go far away enough from the direct voltage basically the entire voltage drops on the geometrical gate capacitor and then you can use this simple relationship when when the quantum capacitance is not important okay but if you are close to to the Dirac point, you have a non-negligible drop on the Dirac, on the quantum capacitance, and that voltage drop is the one by how much the band structure goes down. As I said, when you go far away, then everything drops on the gate capacitance. Why? Because when you go far away, the density of state is very large, and that means that to change charge you need to negligibly small shift the band structure. That means that quantum capacitance becomes irrelevant. That's exactly what you see here. If you go far away you will see that basically this term totally dominates the one here. But for the physics close to the Dirac point you have to take into consideration the quantum capacitance and therefore the band structure shift. Okay, I hope this is clear. And finally I just want to write down one expression I for this line, but they're kind of dry, I mean, but we have to do it, okay? And then very soon it will become again more interesting. I mean, if you're a physicist, you'll probably like this. I don't know. But, okay. Anyway, another interesting consequence of this derivation was this expression for the threshold voltage. It basically shows you how you can calculate the threshold voltage in device. The only problem is that in this expression is expressed as a function of the Fermi level, which is kind of hard to use in applications and that's why it would be much better to have this as a function of the carrier density rather than the Fermi level. So basically it means if you would like to express it as a function of the carrier density, you have to get rid of the Fermi level. You have to eliminate Fermi level and express basically Fermi level as a function of the carrier density. And that can very easily be done how? Well, how the Fermi level is related to the impurity charge. If you go to slide 66, the one with the Fermi Dirac integral, you will find out that I wrote that the impurity charge which comes from the ambient doping is given by this expression here that's what I derived this is expression again of the zero Kelvin okay I do everything at zero Kelvin and then we have an impurity charge as a function of the Fermi level I have to get somehow the carrier density but that's simple you then use this expression here because what's the total impurity charge? It's the charge of one carrier times carrier density times surface area, right? And the carrier charge I wrote like this, so it's elementary charge E times the sine of the charge, which is written in this rather strange form, but the reason I wrote it in this form is just to get rid of this sine of EF on the right-hand side. How do I know this is really okay? well if the Fermi level is in the valence band EF is negative sine of EF is minus 1 times minus 1 is 1 so the charge is E okay is if EF is positive it's in the conduction band sine of EF is positive charge is minus E so it's correct. So the reason I wrote it in this rather strange form is that you can cancel out this sign and this sign and basically from here calculate Fermi level as a function of the carrier density which is written here. Okay and now you take this expression for the Fermi level and you put it here and you will get threshold voltage as a function of carrier density. And carrier density I denoted by, in general, by NEH imp. Imp is impurity, right? EHE means electrons, or just shortly, an imp, and H means holes, PE. So whether EF is positive or negative you will have a concentration carrier density of electrons per volt. And this is interesting expression because you can use this expression to calculate the threshold voltage or the Dirac voltage of a transistor if you know the carrier density density, okay, or other way around. Because you usually don't, if you make a transistor, you basically don't know the density of carrier. But what you can measure, I'm going to show you very soon, is the Dirac voltage. That's very easy to measure. And then you can basically use this relationship to find out carrier density as a function of the measured Dirac voltage. Okay? That's why this expression is very useful. And very soon we are going to do exercise and I'm going to show you this. Remember this expression for the exercise. Okay?
\fig{74}{Nanoelectronics of graphene and related 2D materials 2024}
And the last thing which is left and then we are done with this boring derivations is the, how does this carrier density, how does the carrier density depend on the applied VGS? The previous expression showed you the, how the carrier density in the absence of VGS depends on the Fermi level or the measured voltage, but this is the charge of the impurity this is the charge which comes from impurities in general when you apply VGS you change the charge in the channel and the question is okay what is then the carrier density in the channel how do we calculate carrier density in the channel when we apply VGS because when we apply VGS we change the carrier density in the channel now calculation is again not that hard if you use the previous expression. The first start from the expression for the total charge which I derived on slide 72, that was one of the first expressions I used today when we started lectures. You remember you start from the expression for the impurity charge and then you just replace Fermi level with EF plus EV channel S and that will give you the charge in the system then you're applying voltage between the gate and source. And that's the expression which was valid at 0 Kelvin. So for simplicity I'm going to use the 0 Kelvin. And from here, in theory, from here we can find what we need. Because Q will give us the carrier density in the channel and the VGS is hidden in the expression for EF plus EV channel SY because just previously I derived you how this depends on VGS right so that's the way to do it that's the simplest way to do it okay so let's because we're looking for carrier density let's express carrier density so let's basically take expression for Q and write it via carrier density. Then again, the Q, which is the charge in the channel, is the charge times carrier density times the surface area. In this case, carrier density for electrons is just N and for coke is not P, because that's the actual carrier density. N and P, these are the carrier density in the absence of the applied gate source voltage. That's what you have in equilibrium without applying any voltage which runs from end to end. But when you apply VGS then you have N and P. These are the actual carrier densities. And again notice how I denote its charge. You can already guess why I did it because this expression I have here. So if I express the charge like this, these two signs will cancel out, right? And how can I prove that this is really correct? If you remember this plot, you will understand this. If you remember this plot, when EV channel S is smaller than EF, that's the valence band. When EV channel S is larger than minus EF, that's the conduction band. So here the charge is positive, here is negative. So look, for instance here, e to the channel s is smaller than minus cf, which means that e to the channel s plus cf is smaller than 0. Sine is negative times negative is positive, correct, valence band. And here is the other way around, this is the conduction band, okay. And now you just take this expression you say this is equal to this these signs cancels out and you can solve by EF plus EV channel S you will get this expression when you take these two equations to be the same and now how do we get then get rid of this EV channel S you remember in the previous slide I calculated how much is this. We solve the quadratic equation to find this. That was the previous slide and that's what's written here. This was the solution. This is the solution from the slide 73. This is the solution from the previous slide. The only difference is here we have a sign of Vgs minus Vth which I'm going to put back to sine of Ef plus Ev channel s which you can see that these two are the same in the previous slide and that was the solution of the quadratic equation okay and now you say that this absolute value of this is if you look here you take the absolute value of this is basically equal to this and there you go from one side this absolute value is given by this expression you have the gate source voltage hidden here in the expression for the classical carrier density on the other side this absolute value is given by this so when you take that this is equal to this you get this expression here and that's what you need that's the expression which connects the actual carrier density when you apply the gate source voltage and the gate voltage which hidden here in the expression for the classical carrier density. And then just to solve it, you can take these terms in front of Nq and put on the left-hand side to get this. Why? Because then this term here can be eliminated to the expression for the this quantum carrier density which I wrote in the previous slide. If you go to the previous slide, you will see that there is expression for this nQ and from that you get that this expression here which we have here is just square root of 2 over pi nQ. So when you take this 2 over pi nQ and put here, pi and pi will cancel out and you will get square root of 2 nQ n is equal to this. So when you solve by this carrier density this is what you're going to get. You basically take the square of this equation and then you solve it this is what you're going to get where the where the where the gate voltage is hidden here in the expression for NC and this is this NQ as I said which I used here that's it that's the expression you need I change this line by the way if you have the old version this slide looks a little bit different I try now to simplify it to simplify the solution. So that shows you, so now if we put this map aside, that shows you the carrier density as a function of the gate voltage, so when you apply the gate source voltage what happens with the carrier density in the channel? And the gate source voltage is hidden here in the expression for the classical carrier density. Notice that the carrier density in the channel is the classical carrier density minus something which is always positive. It's always positive, why? Because this square root is always larger than one. So it means that the carrier density is always smaller than the classical carrier density. Why? Because of the quantum capacitance because when you apply a voltage in the absence of the quantum capacitance the entire voltage drops on Cg so basically it means the carrier density is the classical carrier density and C however because of the presence of the quantum capacitance you have some charge also on CQ, lost charge on CQ, charge which is used to shift the band structure down. And that's why the carrier density in the channel is reduced by this quantum term. Okay. Far away from the Dirac point, you can neglect second term, you just get the classical carrier density. And you can plot this here. So the solid line, that's what we just derived, the carrier density of the holes and the electrons, holes in the valence band, holes, electrons in the conduction band, and here is the classical carrier density. The one without assuming the one you obtained by neglecting the quantum capacitance and you see at the Dirac point which is here these two are quite comparable and you cannot neglect quantum term but as you go far away from the Dirac point this difference becomes larger and you can basically just assume sorry sorry, I didn't express myself correctly. As you go far away from the direct point, the difference between these two, the actual and the classical, becomes negligible with respect to the total value of either one of the two, you understand? So basically, far away, you can take any of these two expressions, they're basically the same, because they're much, much larger than the difference between them. Okay? So that's how you can calculate the carrier density. Of course, this is at zero Kelvin. Notice how it's going to look like at finite temperature. How much is the carrier density here? Zero because we are at zero Kelvin. At finite temperature, it goes up. That's it. On top of this, we have electron hole puddles, which also shift this up. Okay. Okay, so we are done with these derivations, and now we can try to apply some of them to some exercise.
\fig{75}{Nanoelectronics of graphene and related 2D materials 2024}
So let's have a look at this exercise. So I didn't use animations here, so you have to look slowly going from the top to the bottom. Anyway, so the exercise here is to find the carrying density, ambient carrying density of this system. If the eigen curve looks like this so this is the transfer curve of a transistor measured so what you get is the drain current as a function of the gate source voltage typically in electronics when you see transistor curves they are normalized with respect to the channel width because the larger the channel the sorry the wider the channel the larger the current so to get idea whether transistor is good or no you should actually plot the drain current normalized to the channel width so typically drain current is just divided by the channel width so hence the unit is amp per meter good transistor at the on operating point should have this current equal to i should remove these microns because it should have this current around thousand amps per meter in the good on state in the good transistor in the on state the current should be around 1000 amps per meter here is very small because the applied bias is very small it's only 0.1 volt and the reason for small bias is because we still haven't learned what happened to transistor when the vds is not small because so far i just showed you transfer curves for negligibly small VDS and that's why VDS here is very small okay so what you have here are actually three measurements of three different transistors by the way you can see transfer curves of three different transistors and you have to find out how much is the ambient doping in this transistor how much is the ambient doping in these transistors well you do what I told you to do just previously you basically have the IV curves from which you can understand the Dirac voltage and then if you have the Dirac voltage Vth you can then calculate the carrier density so in this case you can see that the minimum of the curves is at about half a volt so we can write down that the Dirac voltage is half a volt and now if you know the Dirac voltage we can now calculate the carrier density The easiest way to calculate the carrier density is to assume that the quantum capacitance is negligible. That of course is going to be a source of error. But let's try to do first this way. We try to do it by neglecting the quantum capacitance. If you neglect the quantum capacitance, then here, as I already told you, you don't have here voltage between the gate and the channel, and here you have VGS. So that's the equation you can use to calculate it. Right? So you can write down that this is therefore equal just CG times VTH minus VGS. That's it. and what we need is just this term. I think I wrote this expression a long time ago in one of the previous slides, that QMPCG VTH, and if you know VTH is half of volt and if you know how much is CG you can calculate how much is CG because the data for the capacitor are given then it's very easy to calculate Qimp that's it so let's do it so basically use this expression here you see Qimp is CG Vth this expression here we have CG which we can calculate from here Vth is half a volt and Qimp is what? What's the charge in this case? Well, first you have to understand whether it's a p- or m-type transistor. Which transistor is this? If the Vth is half a volt. What kind of doping do we have? Is it p-type or m-type? Please don't tell me now after all this is a slide, you're going to think whether this is p- or m-type. It's obviously In the absence of the applied gate voltage, we are in the valence band. So this is a p-type transistor. You have to apply positive voltage to hit the Dirac point. You have to shift band structure down so that the Dirac point hits the Fermi level. This is exactly the situation we have here. you have to shift the band structure down so that the direct voltage hits the Fermi level. So that's the p-type, which means we have p-type impurities, and therefore the impurity charge is the elementary charge, which is E times surface area times carrier density, Cg Vth we already have, right? So when you solve by PIMP, by the carrier density of holes, you get CGVTH divided by AE. As you know, CG divided by A is COX. COXVTH divided by E. COX is epsilon over T, which is given. You get exactly 1.4 microfarad per square centimeter. This is the transistor for which one of the CV curves is given previously, where I wrote you that C ox is 1.4. Anyway, you have C ox, this value, half a volt, you have elementary charge, you calculate, you get carrier density of holes in the absence of applied gate source voltages, 4.4, 10 to the 12. However, that's just the approximate value, why? Because it's totally neglect, okay, apart from the electron hole, pardon me, if you cannot model easily, then we always neglect. That's obviously wrong because we also have quantum capacitance which is not included. And if you would like to include the quantum capacitance, we are going to use only expression of zero Kelvin because it's the only one we derived. So you take the expression at zero Kelvin, which I just derived in the previous slide, which gives you the Dirac voltage as a function of the carrier density. Okay? So this is the expression I just derived previously. Go to the previous slide, you will find this expression. Dirac voltage expressed as the carrier density. That's it. This is the quantum term which comes from quantum capacitance and this term is the classical term, right? If you neglect this term, if you neglect the first term, what you are going to end up is what? The previous expression, okay? If you neglect the first term with H, you also know from physics to quantum physics, if you set h to zero, you get classical physics, right? That's what they say. So, if you neglect the first term, you set h to zero, you get classical physics. Basically, you will get this expression here, which I already used, okay? But with this, we have to do the calculation in the presence of the quantum capacitance. How do we solve carrier density from here? Basically, what we have is the quadratic equation with respect to the square root of carrier density. So if you rewrite this expression like this, which is written on the here, so this is pin and this is square root of pin, you basically get the quadratic equation with respect to square root of pin, right? And then when you solve this quadratic equation, because you know everything here, here. a over cg, that's 1 over cox. This is e, h bar vf you know, h bar is the Dirac constant, vf is approximately speed of light divided by 100, I showed you a long time ago. Elementary charge pi, this is half a volt, we know everything. You solve quadratic equation and you get, you see, much smaller, no not much, but you get smaller carrier density. You get 2.6 instead of 4.4 10 to the 12th per square centimeter why do we get smaller value where is the difference on the quantum capacity that's because of the presence of the quantum capacitance the actual carrier density is smaller okay I hope this exercise is clear then we have another exercise and that will be the end of today.
\fig{76}{Nanoelectronics of graphene and related 2D materials 2024}
So the other exercise is to repeat calculations for the back gate. Because if you look at the structure which I put here, this is the structure we have. Right? That's the G-FET, that's the G-FET placed on top of an insulator. Now the insulator is typically silicon dioxide, right? I showed you how the color of silicon vapor changes to the thickness of silicon dioxide, and when people used to make G-pads, they usually place on a 90 nanometer or 290 nanometer thick silicon dioxide on a silicon wafer because in that case you can very easily see monolayer graphing. But, below this silicon dioxide insulator there is a wafer and if the silicon wafer is heavily doped, you can treat it as a metal. So basically you get another gate in the device which is called backgate. Why? Because again you have the MOS capacitor, you have the metal, you have the oxide or insulator, and you have the channel. So basically this is device with two gates. You have one gate on the top, but you have another one at the bottom if you use a heavily doped silicon as a vapor. which means in theory you can use the one at the bottom to operate transistor two and you may argue that this is maybe even easier right because if you make the same device without top gate you're going to get a larger carrier why because carriers in graphing graphing scatter on for instance impurities and imperfections in the material which is in contact with graphing when you place graphene on insulator you always have scattering from here but if you do not make the top gate then there is no scattering from the top gate into it for one source of scattering removed because the the one at the back this insulator you must have anyway so this carrier scattering you can't get rid of unless you use xbn so from one the from one point of view this looks better because uh if you go to the font source of the scattering and the carrier mobility goes up the other issue is you have no issue with alignment because i explained you last time it's very hard to place gate to almost entirely fill the space between source and drain without making short circuits with source and drain. If you don't have the top gate you just put flake two contacts and you fully gain the channel without any problem right looks great but there are several problems because of which this is never used as the real gate. One is the one i'm going to show you now because as you can already guess you're going to need much much larger voltage is proper and this device from the back gate that's one problem that's the solving the exercise which i'm going to do but there are other few problems which one electronics guys do you see the problems those of you who study electronics what's the problem with this back gate there are two huge problems. Come on, quickly. Now we all feel it. So I'm going to adopt the same approach. You ask me something, is it? I don't know, I do electronics. You ask me something electronic, no idea. Hopefully knowing the same way. This guy doesn't know anything. problem number one if you use this kind of gate it's common to all devices on the chip so it's stupid because the transistors must have individual gates which are used to operate transistor and if you use the back gate then all gates of all transistors are the same basically all connected to the same point which makes circuit with such circuit makes no sense you understand problem number one. Problem number two. These contacts that you have, this is not visible from the image, but these contacts you have here, they must go somewhere. So this metal will go somewhere else. so for instance this is looking at the structure from the top for instance this is source this is drain contact looking from the top graphene is here right but these contacts must go somewhere and make for instance large pads to be able to externally access this device pads which are used to land with electric probes I'm talking about the pads which are 200 by 200 microns in size what does this give you you have a huge metal here huge piece of metal you have an insulator and another piece of of metal, you get a huge parasitic capacitance here. You understand? That's the second problem. So that would be extremely slow. But I mean the bigger, you may argue, okay, but if I make a circuit, I'm not going to connect this transistor with another one via probes, because that would be the otic solution, then you will really kill everything. You will kill the bandwidth by parasitic capacitance. That's true. If you make a circuit, you're going to integrate everything here and only external contacts will go away. Okay? But nevertheless, this makes a problem. But the biggest one is of course the problem that everything is in the same gate, or all gates are short circuit. The other even bigger problem is that this is highly inefficient gate and that's the point of this exercise. So basically this exercise tells you recalculate everything, assuming device is operated either by the top gate or the bottom gate. So take the previous exercise, if the device is operated only by the top gate, you get these transfer curves that I showed you here. Okay, sorry this WebEx is so slow. So that's what you get when you measure device when you operate by the top gate. Now the question is, how is this going to look like if I operate by using the back gate rather than the top gate. So it's still going to make a curve like this, but where the threshold or where the Dirac voltage is going to be? Well, let's recalculate it. There are two ways to recalculate this. The first is again conventional way to basically totally neglect quantum capacitance and that's what we are going to do on Wednesday.