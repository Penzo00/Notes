\fig{43}{Nanoelectronics of graphene and related 2D materials 2024}
Okay, so let's construct now armchair tube so in armchair carbon nanotube the chiral vector is in the armchair direction so if you go back now to the crystal lattice of graphene that we had previously this is the zigzag direction and then you remember okay let me show you something before this because some people asked me last time about some directions in the lattice I updated this slide and put here these directions you see if you're really interested in them you can find them there although the only two which you really need are only these two this one and this one because everything else you get rotating by 60 degrees so there is no point to investigate the other. But okay, so let me go back to this one. Okay, so as we discussed, armchair is 30 degrees from the zigzag, so in this orientation this is the zigzag, so the first armchair is going 60 degrees clockwise, and for instance, as an example, Let's make the armchair tube 3-3. The tube is 3-3, it means its parallel vector is 3-3. So this is the parallel vector 3-3, right? 1, 2, 3. 1, 2, 3. So it's 3a1 plus 3a2. And how do we construct nanotube? I already told you. You draw the lines perpendicular to the armchair vector which pass from the beginning and the end of the armchair vector. So this line here and this line here. And then you take the scissors, you cut the lattice along these lines, and then you simply roll the tube so that the beginning of the chiral vector coincides with the end. So what are you going to get in that case? because the because the chiral vector is in the armchair direction what you're going to get is the armchair direction goes around the tube around the circumference of the tube why because I already told you for each armchair direction there is a zigzag direction perpendicular to it because these lines are perpendicular to the armchair you will get that the tube is in zigzag direction. So this is zigzag and this is zigzag. Okay?
\fig{44}{Nanoelectronics of graphene and related 2D materials 2024}
And for instance this is one tube like this. So here is the armchair, you see, which goes around the tube, so it's this here, while if you, I mean it goes around the tube, around the circumference, while this is the zigzag which goes in the direction of the tube. So as I said last time, this is the main confusion me the tubes because when someone tells you the tube is armchair it means it's in the zigzag direction, while the armchair is the one which goes around the circumference of the tube.
\fig{45}{Nanoelectronics of graphene and related 2D materials 2024}
And finally let's construct some arbitrary tube which is not necessarily in either armchair nor zigzag direction, so that kind of tube is called chiral tube and in that case the chiral vector does not point along these two directions which we already investigated. So we are coming back to the vector 4,2 I showed you before. This vector is so 1,2,3,4,1,2, okay. So this vector is between two neighboring zigzag and armchair directions so it's in some arbitrary direction between. And how do we construct this chiral tube? Again the same. you draw the lines perpendicular to the chiral vector like this, which pass through the beginning and the end of the vector. Then you take the scissor, of course, you cut the lattice along these lines, and then you simply roll the tube so that this point here, at the end of the chiral vector, coincides with this one here. So I think you understand, right? I already said it three times for three different tubes. So if you have a tube, you get chiral vector going like this, right? So the chiral vector goes around the circumference so that beginning and the end of the vector coincide. Okay. And now when you roll it, what you're going to get, where you're going to get tube, chiral tube, 4-2, and it's interesting to see what happens in that case with, for instance, zig-zag direction. So the zig-zag direction is this one. In zig-zag tube, this direction will go around circumference. In armchair tube, this direction will go exactly along the axis of the tube. But what about chiral, for instance, tube like this? Because these two points coincide, to understand where does this continue when you roll the tube, you should translate this point here at the edge parallel to the parallel vector. Which means that after you roll the tube, this point here will coincide with this. And then the zigzag direction will continue like this. And then this point will continue after rolling here, and then zigzag will go like this. So you see in case of the chiral tube, zigzag direction helicoidally goes around the axis of the tube to infinity. So it never closes on itself like in a zigzag tube. Or it's not straight like in armchair tube. It's just helicoidally going around the axis of the tube until the infinity. And of course the same is for armchair, but I'm not going to draw it here.
\fig{46}{Nanoelectronics of graphene and related 2D materials 2024}
And, okay, so that's about the classification of the tubes. It's very interesting that the properties of the tubes actually depend heavily on this chiral vector, that is, whether the tube is metallic or semi-conducting can be easily understood from geometry and knowing the chiral vector of the tube. to do so we have to understand which are the possible carrier states in case of carbon nanotubes. So because the carbon nanotube is obtained by rolling graphene sheet, the first Brillouin zone look the same because it's exactly the same type of the lattice. The only thing which is different is that carbon nanotube is not two-dimensional but one-dimensional object. And what that means? So first First of all you should understand that in case of the nanotubes you should look not at kx, ky but you should look at the other pair of the k numbers which describe motion of the carriers in the tube. So you already understood from what I told you that the chiral vector goes around the circumference of the tube, okay, so it goes like this, while the direction which is perpendicular to the chiral vector is the one which goes along the axis of the tube. From that you can already understand that in contrast to graphene sheet where kx and ky were important to understand the motion of carriers, here it's much better to look at the k vectors in these two directions. Because you see, if you look at the direction along the chiral vector, because the chiral vector closes on itself, we have a quantum confinement along the circumference of the tube. Because we have a limited motion of the carrier, we just have a circumference of the tube. On the other side, the direction which goes along the axis is basically free. I mean the carrier can go freely from one side to the other of the tube without any restriction and for that reason it's more important to look at the k vectors, at the components of the k vector along the chiral vector and the vector perpendicular to the chiral vector rather than kx and ky and that's actually drawn here. So what you see here is the first Brillouin zone of graphene which is hexagon. This hexagon we know very well with 6 direct points out of which only k and k' are non-equivalent. And then I put here two directions. So one direction is the direction of the Carroll vector, which is this one. Now, just don't get confused, this is reciprocal space. And this Carroll vector is vector from the direct space. So you cannot plot this vector here directly because the units on the axis are different. Okay, however, so you cannot tell how much is the... You cannot tell where is the end of this vector on this scale. However, I don't need the exact coordinates of this vector. What I need is just the direction of the vector and direction of the vector is just defined by this angle and this angle is the same as in the direct space. So you basically draw the vector the Karel vector in direct space, you find the angle between this vector and the x-axis and you draw a vector in this direction. So here is the Karel vector, which goes in this direction, and now we have a direction perpendicular to the Karel vector, which is the direction of the axis of the tube. So then you draw another axis, which goes like this, which is perpendicular to the Karel vector. and now we are going to denote these two k-axis with k-parallel parallel indicates motion parallel to the axis of the tube so this k-number, k-parallel is the one which runs from minus infinity to plus infinity in the tube while this one, sorry, it's not like that is the chiral vector which corresponds to the motion of the carrier which can go from minus to plus infinity in the tube, so parallel to the axis. So, k parallel is this direction. And then we have the other direction, the direction of the chiral vector which is k perpendicular, so k perpendicular in the sense that it's perpendicular to the parallel k So the vector, so the wave vector which is more important in case of nanotubes is the wave vector defined by k perpendicular and k parallel rather than kx and ky. Okay, and that means for instance if you look at the properties of the carbon nanotubes close to the direct point, you should then translate this k perpendicular, k parallel coordinate system to the k point, so you get here cube perpendicular and cube parallel, okay, not kx and ky, as I said, because these two are important, not kx and ky. Anyway, the point is, now, if you understood that, then this is the first Brillouin zone, so that's the only place where we are looking for the possible carrier states, and now the question is what are the allowed states in the first Brillouin zone? Because in case of graphene basically all states inside the hexagon were allowed. That's not strictly speaking true, you remember there is quantization in graphene along kx and ky, but if you look at microscopically large graphene flake, I mean the flake which is large enough, this is so small that you can assume that basically you have almost continuous k vectors, which means basically all states inside the hexagon, inside the first Brillouin zone are allowed, and that's why when I talked about graphene, I filled all this with green color. The entire hexagon was green, meaning all states are allowed. And now have a look at possible allowed states for carbon nanotubes. Now you see what's the difference. The difference is that not all states along k perpendicular are allowed. Why? Because you see we have a confined motion along the chiral vector, which means we have a quantization along the chiral vector. And what is the quant along this direction k perpendicular? As always, the quant is 2 pi divided by the length of the space in this direction, which is in this case just the magnitude of the chiral vector. And the magnitude of the chiral vector is obviously the circumference of the carbon nanotube, right? But because carbon nanotubes are very, very small, I already showed you for instance that for a single-volt nanotube you get a diameter of like 1.2 nanometers. You remember this TN image by Gima. Well, if diameter is 1.2 nanometers, then this circumference is 1.2 nanometer times pi, which is extremely small. so that means this quant is rather large, which means we cannot assume that we have almost continuous states along k perpendicular but we actually have a discrete state, okay, with this quant 2 pi divided by magnitude of the current vector and that's depicted here, you see, when you start from zero, one possible state is zero another possible state is delta k perpendicular, which is this quant here, and then you have a 2 times delta k perpendicular, then 3 times delta k perpendicular, and so on. And also going to the negative side, minus delta k perpendicular, minus 2 times delta k perpendicular, minus 3, and so on. So only those are available states along k perpendicular. Regarding K-parallel, because K-parallel is the axis along the length of the tube, which could be microscopic, this length of the tube is rather large, so the equivalent quant is rather small. So, you can pretty much assume you have almost like quasi-continuous motion along the K-parallel. And for that reason, all available states inside the hexagon, inside the first Brillouin zone are those for a fixed value of this quant, okay, like for instance zero, and then basically continuous values of k parallel, so this line, you see, so these are the available states. Inside the first Brillouin zone is just this part of this green line which is inside the hexagon, and then for k perpendicular equal to delta, k perpendicular, first quant, these allowed states, okay? So this is the k perpendicular equal to first quant and then parallel we have a green line which corresponds to all possible states along k parallel, you understand? And that's how you draw the other lines. Is this clear? So that means that in case of the nanotube instead of having all states inside the hexagon pretty much available, here we have only those states which are along the green lines inside the hexagon. That's it. Because of the quantization along the chiral vector. And now from that you can already understand when the tube is metallic and when it's semiconducting. For instance, the tube I draw here, is it metallic or semi-conducting? What do you think? When the tube is metallic, the tube is metallic or semi-metallic if you wish, when there are states at the Dirac point, right? Because at the Dirac point, the valence and conduction band touch and there is no band gap. So, in other words, you draw these lines for your tube and if one of these lines hits the direct point like here, it means there are states at the direct point, which means that there are states at which conduction and valence bands touch, because the band structure of the tube is the same as that of graphene, it's just a matter of variable states, and And that means that this tube is metallic. It's actually semi-metallic because the graphene is semi-metal, but somehow when people talk about tubes they say metallic, not semi-metallic, but that's the point. Alternatively, if you draw these lines and they miss the K-point, it means there are no states at the K-points which are allowed. So that means in the band structure there are no states at the point at which valence and conduction band touch and that means that that tube has a band gap, which means it's semi-conducting. So obviously this tube is metallic. And that means from very simple geometry we can understand if we know the current vector of the tube, we just have to draw the lines and understand whether they hit the Dirac point or not. And in order to avoid drawing all this for every tube, for instance I give you at the exam for instance, what you should do, actually what we should do now is to derive a simple geometrical condition which comes from this and then you don't have to draw all these lines you can simply calculate, you use simple calculation which I'm going to derive now. Okay? So how does this work? So I think we understood that carbon nanotube is metallic if one of the lines hits the direct point like what happens here. But what is mathematically conditioned for that to happen? What is mathematically conditioned for that to happen? this is going to happen? When between this point here and this point here we have integer number of quants, that's it. If between the gamma point and projection of the k-point to the chiral vector we have integer number of quants, I know that one of the lines is going to hit the direct point, that's it. it's very simple. So basically mathematically it means this, if the projection of the k point on k perpendicular is this upper case k perpendicular, so this is the projection of the k point to the chiral vector, so if this is the, if upper case k perpendicular is the projection of the k point on the chiral vector, I know the tube is metallic, if this value, this uppercase K perpendicular is simply some integer multiplying the quant along the chiral vector direction. To simply say this coordinate uppercase K perpendicular, this here must be equal to integer number of quants. In my particular drawing this is 3, in my particular drawing this L, so this is not 1, this is L, lower case L, in my case L is equal to 3, 3 times delta k perpendicular, okay? So that's the condition and now how much is mathematically this projection, how do you calculate projection of one vector on another? You take the scalar product and divide by the magnitude of the other vector, So that's it. So this is the projection of the k vector. k vector is this vector here, vector of the Dirac point. You take the scalar product with the chiral vector and then you divide by the magnitude of the chiral vector and that's the projection. So that, when you calculate this, must be equal to integer number of delta k perpendicular, right? And now if you remember I already brought you the delta k perpendicular this quant is 2 pi divided by magnitude of the chiral vector and then when you cancel out here magnitude and magnitude of the chiral vector you get that this scalar product between the k vector and the sorry not the k vector between the vector of the direct point uppercase k vector and the chiral vector must be equal to 2 pi l. must be equal to integer number of two piles, that's the condition for tube to be metallic, so that's the condition for tube to be metallic and then the rest is really trivial math, you simply say okay I need this scalar product and now to calculate the scalar product I need the expression for the vector of the k-point which is given by this, I derived this long time ago you will appreciate again the reason I took k-point along the horizontal axis, so we don't have too much to calculate because we have a zero here, right? Okay, so this is the vector of the k-point, and this is the, remember, this is the chiral vector. So it is n times A1, n times this is A1, plus n times A2, this is A2. These are the unit cell vectors in direct space, which I also wrote on one of the first slides. Okay. And then when you calculate this scalar product, please do it at home, I'm going to skip this, this is really trivial, you're going to this expression, this is the scalar product. Or you can write it like that, the scalar product is you put 2 inside, you get 2 pi times 2n plus n divided by 3, which is this one. Okay. And that, as I said, must be equal to 2 pi integer, which is this one. So, for instance, from that you can, if you cancel out 2 pi and 2 pi, you get that this integer must be 2n plus n divided by 3. In other words, one condition for a tube to be metallic, actually the condition for the tube, the only condition for the tube to be metallic is this one. If you look at the chiral vector, which is defined by the coordinates n, n, I mean coordinates in the direct space with respect to the vectors a1 and a2. so if you take n and m of the carat vector if you calculate this 2n plus m divided by 3, if this turned out to be an integer well, that's the metallic 2 that's it, however typically people don't write it like this because this sounds like very complicated, who is going to calculate 2n plus m divided by 3, so what people do they rewrite it in the following form so if you multiply everything by 3 you will get 3L on one side on other side you get 2N plus M and 2N plus M can be written like this because 2N is 3N minus N, that's 2N and M is minus minus N so the reason why it's written in this form is because then N minus M is 3 times L minus N which is again integer. In other words, the simplest way to say whether the tube is metallic or not is to find n-m. If n-m is divisible by 3, the tube is metallic, which means if n-m can be written as 3 times integer, the tube is metallic. So if you take a tube and calculate n-m, if this number is divisible by 3 the tube must be metallic because in that case you will have integer number of quants up to the projection of the direct point onto the para vector. Ok, that's it, that's a very simple condition for having a metallic tube. So remember this condition n-m divisible by 3 and also remember the value of L. The reason why I calculated value of L is because L is the number of quants. It gives you the number of quants from the gamma point to the projection of the k point in the Kerr vector, which means the number of lines which go from a gamma point to the k point is L plus 1. We have three quants, four lines, one, two, three, four, okay. And okay so that's what I already told you we are going to use this later and does anyone see what's the problem with my plot? I hope you understood this but I made one mistake in this plot. I did it deliberately because I didn't want to start discussing some things before time but now if you look at the plot you will actually notice one problem one error does anyone see the error in the plot why did they write the wrong direction what about other four equivalent points I mean you see the lines pass through K and K prime. What about the other four points? If they are equivalent, it means the situation must be the same, which means I did not draw green lines correctly, direction of my Carroll vector is wrong. Why? Because if this is really metallic tube and if one of the green lines pass through k and another one through k prime some of the other lines must pass through the other four points you understand because these other four points are equivalent to k and k prime so obviously the way I draw the picture is not good I'm going to show you in the next slide the actual orientation in case of metallic tube you will understand that the only reason I didn't do it in the beginning because if I have done it immediately you will ask me, oh but why this really coincides with this? You know, and then I will explain this, but now I understand. And that's actually what I'm going to show you now.
\fig{47}{Nanoelectronics of graphene and related 2D materials 2024}
For instance, we are going to look at the band structure of armchair carbon nanotube. So, what about armchair tubes? If you go to the plot at the very beginning, when I explained you directions in the tubes, this one here, I told you we should look only at one armchair direction, everything else you get rotation by 60 degrees. So let's have a look at this armchair, okay? We know that this is 30 degrees from horizontal axis, okay, and NM are actually NN. So, what's the conclusion for armchair tube? Are they metallic or semiconducting? The condition for having metallic tube is N minus M divisible by 3. How much is here N minus M? It's 0. 0 is divisible by 3, or by any number, but also 3, which means all armchair tubes are metallic. Okay? ok so let's draw it this now ok so how do we draw these lines these possible states so remember first you draw the first you draw the direction along the chiral vector I already told you the Carroll vector in case of armchair tube is 30 degree clockwise from the horizontal axis. This angle here is 60 okay this is the zigzag direction horizontal and then the armchair is this one this is 30 degree from this one this one here is again zigzag okay zigzag zigzag and in between we have armchair and 30 degrees from horizontal axis. This is the direction of the Carroll vector and this is K perpendicular and then K parallel is perpendicular to K perpendicular which is 90 degree from here, this angle is 30 and this angle is 60, so basically it goes like this K parallel passes through these two direct points fine and now let's have a look at armchair tube 5-5 for instance I have to pick up some numbers so let's pick 5-5 so if the tube is 5-5 how many quants do we have from gamma point to the projection of the k point to the k perpendicular well you remember it's this number here so you have 2 times 5 plus 5 which is 15 divided by 3 is 5. So you have 5 quants, here it is 1, 2, 3, 4, 5 or 6 lines from this to this. So first quant, second, third, fourth, fifth. Ok. And now, you get what I already told you, you see. Because this tube is metallic we have an integer number up to this point here, this line passes through the k point. But not only through here, you see, this line passes through this too. And if you keep drawing, okay, this line passes also through this one, and if you keep drawing on the negative side, because the image is symmetric, you will get, if this is the first line, sixth line going going to the left will pass through this tube. That's what I told you. If the line passes through one k-point, it must pass through its equivalence. But I didn't draw the negative side of the spectrum because it's symmetric. It's just enough to understand this. Okay. So these are quantization lines in case of the tube 5-5. Okay, fine. But how does the bent structure of this carbon nanotube look like? Well, the bent structure is the bent structure of graphene. So basically if this is the first Brillouin zone, if you remember when I plotted the band structure of graphene, you have like conduction band at the top, you have valence band at the bottom, they touch at six, I have only five fingers sorry, they touch at the six Dirac points from the top and from the bottom and that's the band structure of the tubes. The only difference is in case of the tubes, not everything inside hexagon is allowed. You have only states allowed along the lines, which means that if, for instance, if you fix k perpendicular to be zero, the only states inside the first Brillouin zone are dosed, which means out of the entire band structure of graphene you should take only states, you should take only points on the plot of the band structure of graphene which correspond to this state here. So what's the easiest way to plot this? The easiest way to plot this is to take this band structure of graphene, take a plane which is perpendicular to the hexagon, which intersects hexagon along this line and when this plane goes up and intersects the band structure of graphene, the line you get is the band structure of carbon nanotube for k perpendicular equal zero. Which is this, you see this line here, sorry this plane here, I don't know can you see it, so this is kx, this is ky going there, this plane here is the plane I already told you, and when this plane goes up it intersects here the band structure of graphene. I can actually show it better because I already made this plot so why don't I show you the original which is this one. Can you see it? So this X is here. Can you see mouse? So this X is here is KX. KY is the one which goes over there. you can see, maybe you can see it from the top, you see? this is looking from the top so let me restore the image, ok and now look, we just draw the line which passes through the gamma point which is perpendicular to the hexagon that is to kxky plane and it's under angle like in my plot and you see this is where it cuts the band structure of graphene. So basically if you look at this the band structure of tube this one Is this line this intersection? you understand is this intersection, so if you go now back to the Slide This is the line at the very top the black one you see so this line here is this one, of course I plotted it because the k perpendicular is fixed it's equal to zero so parameter along this plot, black plot is k perpendicular equals zero, I plotted it as a function of k parallel because k parallel freely changes from this value to this value, okay. Now notice that this plot does not fully correspond to this hexagon Y, because if you go along this line inside the first bilan zone, this plane should cut this one also in Dirac points, which means that this goes down to zero, that will be this point here, and this goes down to zero here, this will be this point, but this is not included here, maybe I should have recreated this plot by myself, I just took it from one book, but anyway you understand the point okay because if you go back to the plot you will see that actually also as in my image this plane goes through the Dirac points do you see so here is the Dirac point here so if we now rotate you will see that actually cuts you see goes exactly to the Dirac point it means that this line has a maximum at the gamma point because that's where this band structure, so I'm talking about the conduction band, we have a maximum of the conduction band of this like Mexican hat and then the intersection goes down down down and then it hits zero at the Dirac point which you cannot see in the band structure image which is shown here because simply the axis was not, the horizontal axis, the k-parallel was not extended enough. Okay? And now look, so that's the band structure for K perpendicular equals zero. Then you keep plotting. Okay, the next one is for K perpendicular equal Delta K perpendicular. So now if we fix Delta K perpendicular, we are along this line, we now put plane here, go up and cut Mexican hat. And I understand what's going to happen because this is more to this side, this plane is going to miss the maximum and it's going to be moved closer to us so the intersection will go down you understand because we are not cutting because this black one cuts exactly the gamma point where you have maximum I can also show you this with this 3d plot for instance if I change this like okay one one would be exactly this one so now if you look at the 3d plot you see the plane now is closer to this because we are looking from this side is closer to us and you see it doesn't cut it at maximum it's moved more to the right and for that reason the line goes down and also pay attention it misses the Dirac point as you can see also from from the first Brillouin zone which means this one will not hit zero this line for delta for k perpendicular equal one times delta k perpendicular will not hit zero and how it's going to look like you can see it here maybe this is the best way to see so it starts from the lower value goes down but then it goes up it doesn't it doesn't it doesn't pass through the direct point. So that's why the black one, the top one will for instance hit 0 here and hit 0 here but the red one which is moved closer to us will go down and then go up, go down and then go up because it doesn't hit the Dirac points, you understand? And now you see, if you plot a band structure, so the red line here corresponds to this red line here. Then this yellowish like line color, if you look at this line color, that's for k perpendicular equal two times delta k perpendicular, so the line is even more further away from the maximum you see the the band structure goes down and even misses by more the direct points okay and then you keep going like this so I can show you maybe one more example but I think it's clear what I'm doing right guys okay so let me just show you once more this is the third one you see so we totally miss the maximum, now it goes very very, so it's actually this line here, if you plot the band structure of the tube, it will be actually this line here, okay, which you can see on my plot, not mine, I actually picked it up from some book, you see they go down, go down and miss by more the Dirac points, but then when we come finally to the purple one at the end, then we are going to hit this Dirac point. Okay, so how then the band structure is going to look like? We are far away from the maximum, so this is the lowest one, but then for this value of k parallel or minus this value, we are going to hit the Dirac point, so then we have Dirac point here and Dirac point here, and after that it goes up so I can also show you this in the 3d plot so this is for this is the last one for five you see it goes exactly through these two direct points so how the band structure looks like it has this noximum and then it goes to Dirac point Dirac point and it goes up because it cuts the Dirac cones exactly through the Dirac point so here you have basically linear curve here and here because you really cut the Dirac cone through the Dirac point okay So you see these lines here, like a cross, that's what you get by cutting Dirac cones through the Dirac point. So it's linear here, linear here, this is the Dirac point, so it's this one and this is this point, which is as you remember equivalent to K prime, that's it. So that's the band structure and then of course totally symmetric for the valence band, so so far I just discussed the conduction band, but as you remember the band structure of graphene is totally symmetric, so because the expression for the valence band structure is just a minus of the conduction band, so for the valence band which is here at the bottom you just take the conduction band and multiply everything by minus one, you basically mirror it with respect to the zero axis and bear in mind this one here goes like this, here and then it goes up, doesn't go down. Down is the valence band okay don't get confused. So it goes like this and then up okay and then at which number here we can find the zero now this is very easy to calculate how so let's finish this and then we are done with the first hour so you simply say how much is this here how much is this here? this distance here, how much? this is this distance here, right? this is the because this is plotted as a function of k parallel, this is the value of k parallel at this point, which is either this or it's simpler to say this one so this distance here must be identical to this distance here because this is the coordinate of k parallel at which gamma point, sorry at which the k point is hit, okay, and how much is this? Now notice that here it's not plotted as a function of k parallel but as a function of a1 times k parallel where a1 is the magnitude of the direct space vector a1. Why? Because you want to have the numbers here because if you remember k vector is vector from reciprocal space so So the physical unit of the k vector is 1 over meter. So to avoid using 1 over meter you plot in something which has a dimension of length multiplied by the k vector. So we finish after the break. So finally just let's calculate this number, this is very easy. So the reason I am calculating just to show you this is really the zeros of the top line are outside of this plot. So A1 is the magnitude of the unit cell vector, if you go back to one of the first slides you can see that magnitude of this vector is the carbon-carbon distance I denoted by A times square root of 3, okay. Then we have, that's it and then how much is this? this is a unilateral triangle, triangle with all sides the same, right? That means that this here is half of this, right? So it's one half of this and this is actually this one, which is Kx. the kx you get from the, I wrote you long time ago how much are the coordinates of the Dirac point so it's a1, a square root of 3 times this and this is half of this which is 1 over 2, this is the x coordinate of the k point and then when you put the numbers you will get 2 over 9 which is exactly here and now you understand why the zeros that you get from the top line, the black one are outside of the plot so at which k-parallel coordinate you will get this zero here? at this value and this value is twice this one, so at 4.18, okay? so that means the black curve will hit zero at 4.18 here and minus 4.18 here and this plot is only between minus pi and pi so that's why those zeros are outside of the plot But I hope this is clear. So that's how you get the band structure of this metallic tube. And as I already said, there is no band gap because of the decay point the valence and conduction bands touch. And here I'd like to point out that here we have a correct direction in contrast to the previous plot. Because this line hit the direct point, also other direct points are hit by the lines. ok so let's have a look now at the zigzag tube.
\fig{48}{Nanoelectronics of graphene and related 2D materials 2024}
So the zigzag tube for instance is the tube which is defined by the chiral vector n0 if you remember. So zigzag direction is the horizontal one which means that in this particular case kx is actually k perpendicular because that's the zigzag direction and the direction along which carriers can move freely is perpendicular to the chiral vector is this one, so ky is the k-parallel. And now how do we construct the band structure? Again, we have to know the type of the tube, we need a chiral vector to be able to calculate how much is the quant, how much is the quantization along the k-perpendicular, and then for each possible value of this quant, 0, 1, 2, 3 and so on, we draw the line perpendicular to k perpendicular to get the state along k parallel. And because this is k perpendicular, then you simply have this line for the gamma point and this is for quantum 1, 2, 3, 4 and so on. So basically they just keep going like this. And now you already understand from here that zigzag tubes do not necessarily, the zigzag tubes do not have to be in the most general case metallic, why? Because now it's a matter of hitting these k points or not, because if you go back to the previous slide, if you go back to the armchair tube, I already told you armchair tube is always metallic because n minus m is 0 divisible by 3, but how else can you see it from this plot. Because the armchair direction is 30 degree from here, parallel direction is this one, so the parallel direction already passes through two non-equivalent k points. You see, you basically don't, in case of armchair trip, you don't have to keep drawing the lines, because you already see from the first line that it goes through the k-point, two non-equivalent k-points on the opposite sides of the hexagon. That's it. So that's why arm-chamber tubes are always metallic. However, in case of zig-zag, that's not necessarily the case. Why? Because when you start to draw in here, one, two, three, you have no guarantee whether those lines are going to hit the Dirac point or not, which means not all zig-zag tubes are metallic. Some are semi-conducting, some are metallic. of course we are mostly in electronics interested in semiconducting tubes, which means from the electronics standpoint, archer tubes are useless, because they are semi metals. So let's now try to draw some band structure, so for instance tube 9-0, so it is N0, is zigzag tube 9,0 metallic or semiconducting, use criterion I gave you, n minus n must be divisible by 3, 9 minus 0 is 9, 9 is divisible by 3, which means that this tube is metallic, okay, which means that when you draw lines from, so how many quants do we have? The number of quants or L from gamma point to one side was L, you remember, 2N plus N divided by 3. So 2 times 9, 18 plus 0, 18 divided by 3, 6. You have 6 quants. If you draw 6 quants, the third line will hit this one, the seventh line will hit this one, okay. What I want to say after 3 quants, actually the fourth line, because the first one is at the gamma point, the fourth one, so after 3 quants, a line will hit this one, and after another 3 quants, which is 6 quants, that is the seventh line, will hit this 0 point, okay? So that tube is metallic. And what you can see at the top is the band structure of this tube, okay? So how do we draw it? Again, remember, we plot it as a function of k parallel, which is the one which changes, which is the k vector which pretty much continuously changes from minus some value to plus some value, in this case from this value to this at the gamma point. So, if you set k perpendicular to be zero, that means you are plotting band structure which corresponds to this line, so take now the plane, I'm not going to now waste time with 3D plot you understand what's going to happen. Take now the plane, place it perpendicular to the first Boulogne zone so that it intersects the first Boulogne zone here, so this plane, and how it intersects the band structure, again it goes through the middle because it goes through the gamma point, so the line which is here, it has the response to this band structure here at the very top. So this black line at the very top, okay? And as the line moves to the right you're going away from the maximum so the band structure goes down you see the band structure goes down and the seventh line is going to hit the k point, okay, and at which value of k parallel we are going to hit it? 0. That's why the k point is now here at 0, you understand? Because this k point you plot as a function of k parallel which is ky. The corresponding value of k parallel or ky which corresponds to this point is 0. So that's why we have 0. So basically here we have a plane cutting exactly through Dirac cones and again we get like X, right, we get Dirac cones here. Okay, so here I don't have to calculate at which point this happens, okay. And then for instance if you look and then of course you metric for the balance band and then if you look at the, if you look at the tube 8, 0, tube 8, 0 is semiconducting because 8 minus 0 is 8, it's not divisible by 3, which means that if you plot these lines in case of tube 8, 0, you're going to end up with a situation similar to what we have in the plot, which means none of the lines is going to hit the Dirac point and that means we have the band gap. Where is the band gap? It's again at 0, why? Because as you plot the lines let's assume this line let's assume this line here corresponds to this band structure here what do you get you have a plane cutting the Dirac cones but not through the Dirac point so let me try to visualize this so these are the Dirac cones you're very close to the Dirac point which is here okay should be symmetrical, forget it. And now you're not cutting here with the plane, you're cutting here somewhere before, which means that the cut is line which looks something like this. You understand? When you cut you get one line like this, another one line like this. So there is a band gap in between. So this is the band gap. This is the band gap. And the band gap is this distance here and it happens at zero. Why? Because here again we are closest to the k point and the corresponding coordinate is zero. So it happens again at zero. We have a small band gap. And finally let's discuss the, now let's make a summary. So what's the summary we can say from here? So we saw that all armchair tubes are metallic, that's clear. So if someone tells you the tube is armchair, you know immediately it's metallic, nothing to discuss. If the tube is zigzag, then you don't know, you have to use criterion like for any other chiral tube. You basically have to look at the N minus N and to check whether it's divisible by 3 or not. and if it is divisible by 3 the tube is metallic, if it is not divisible the tube is semiconductor, there is a band gap. Okay? And now you can already understand from this, you can already understand why the band gap in tubes is larger if the diameter of the tube is smaller. you can actually understand from this point here if the tube if the, this is very important to understand because of course the question which comes immediately is how much is this band gap and can we use it in electronics that's immediately the question so the answer is yes we can use it in electronics assuming it's very close to the band gap of silicon right? But band gap of silicon is 1.1 eV, something like this. So how do I get band gap of 1 eV? What should I do with the tube to get band gap of about 1 eV? The answer is you have to look at the very narrow tube with a diameter close to 1 nm because in the tube the smaller the diameter, larger the band gap. Why? Now you can understand it from this plot here. Because if the tube has a smaller diameter, if the diameter of the tube is smaller so if the diameter of the tube goes down it means that the magnitude also of the chiral vector goes down which means that quantization goes up the quants are more and more separated and that means if you have this plot if those green lines are more separated you have a higher chance with a certain tube to miss k point by more you understand? And because imagine these two situations. You have a k-point here and you have a quantization like this. Okay, I hit it here. Or you have a quantization like this. Obviously, if quant is larger, you can miss k-point by more. if you miss the K point by more it means this plane is moved more away from the Dirac point and these two lines are separated more which means the band gap is larger to understand this that's why for a larger band gap you need a smaller tube and it turns out experimentally if you look at the numbers we can put here.
\fig{49}{Nanoelectronics of graphene and related 2D materials 2024}
The band gap approximately behaves like this it's one electron volt divided by the diameter of dube nanometers, which means if the tube has a diameter of 1 nanometer, you are going to get band gap of 1 electron volt. If the tube has a diameter of 10 nanometers, you are going to get 0.1 electron volt, which is not sufficient for digital devices, it's too small. That means that for applications in electronics, you are looking only at tubes of extremely small diameter. We are talking about single-volt carbon nanotubes diameter about nanometer and that's what you should use if you would like to make circuit like I showed you at the beginning this risk processor they made you remember they made like spaghetti of very very narrow tubes to somehow simulate like a thin semiconducting layer with a large enough band-aid And finally I have to tell you that the theory of the tubes I gave you now is quite understandable, I hope, but it's also a bit simplified, because in reality it's not that simple. The problem is that because of the curvature and the finite side of the tube, also the first Boulogne zone is deformed. It's not actually identical to that of graphene. In reality, the first Brillouin zone of carbon nanotube is somehow shrink because of this phenomena. So basically all these Dirac points go closer to the gamma point. That's how it's realistically in a carbon nanotube. So for instance, what I want to say is that if this hexagon here is the first Brillouin zone of graphene in a real nanotube, I mean in a nanotube, the real first Brillouin zone is compressed towards gamma point. this K point is not here but here and this K point is not here but here you see so you move all K points in the direction of the gamma point you kind of compressed everything how does this change the properties of the tubes basically the properties of everything what I told you for armchair tubes is correct why because even if these K points move here and here they still stay along this line, which means that when you draw the quantization line, they are going to hit the K point and you will get the benghe. Sorry, what I am talking about, you are going to hit the K point and you are not going to get the benghe, you will get metallic tube. So armchair tubes stay metallic, while tubes which are not metallic can actually turn to be metallic in other way around, if they are not armchair. For instance, look at the tube with a zig-zag tube with quantization lines going like this, like I showed you in the previous slide. And let's look at the zig-zag metallic tubes where quantization line passes through this point. I told you this tube is metallic. In reality, it's not. Because the gamma point is moved more to the left, and basically this quantization line is going to most likely miss the gamma point. and this tube is not going to be metallic but in reality it is, why? because this compression is rather small so the bang gap you get in this way is like maybe 10 mEV which is nothing at room temperature, at room temperature thermal energy is 25 mEV which means that this is metallic at room temperature The band gap is too small to be observable basically at room temperature. So this tube pretty much stays metallic. So what I want to say just with this last comment is that the theory I gave you is rather simplified. It's good enough to understand the tubes, but you should know that in reality if you really would like to investigate tubes in more detail, the theory is a bit more complicated because the Brillouin zone is not identical to that of graphite because of the size of the tubes.
\fig{50}{Nanoelectronics of graphene and related 2D materials 2024}
But this is really, let's call it a second order effect, so we can neglect it and I think this is enough for you for this course because this course is not really about nanotubes, is enough to understand the properties of the tubes. And finally to finish with the tubes, I would like just to comment on mobility in the tubes because as you already guessed if the tubes are obtained by rolling graphene sheets we may expect to have also high mobility in the tubes and that's true and how can we explain this well we can explain it in the same way as in graphene but as i'm going to show you this quantization is going to change situation a little bit for non armchair tubes. So how did I show you that in graphene we have very high carrier mobility? I showed you by looking at the transition matrix element you remember and we found out that due to the fact that in graphene we have a spinor wave function. This transition matrix element is proportional to the cosine half angle difference between the final and initial state and that's equal to zero for backscattering in which the angle difference is pi, you have cosine pi half, so basically probability of backscattering is zero. Backscattering is heavily suppressed in graphene. It's very interesting that similar similar reasoning can be used to justify high mobility in armchair tubes but not all tubes and that's what I would like to discuss before we finish with the tubes so for instance if you look at the armchair carbon nanotubes the only difference between the tubes and graphene is allowed states. So, I remind you again, not entire hexagon, but only quantization lines inside the hexagon, right? So, what that means in case of armchair tubes? Well, again, because we are looking at the transport close to the Fermi level, I'm going to look only at the same situation close to the Dirac point, okay? And if you go back now to armchair tubes, is important to understand, if you go back to the armchair tube and now look at the quantization line, there is always quantization line which passes to the K point, okay? That's what I showed you. So if you now come here and look at the situation in case of armchair tubes, so here is one quantization line, that's the purple line in my previous plot, now we just put it in green for simplicity, so we have this quantization line, we calculate k vector with respect to the direct point, so we use the q vector, as I already told you, we look at the q vector, and again this is the armchair direction, okay, so this is the quantization we have along this axis and along this axis, this is the parallel axis, we have basically unrestricted motion of carriers, right. So, let's have a look at the initial and final q states that we have here. So, let's say we have a state here at the green line, so let's say the initial vector before scattering is this one, orange vector Qy, and now after scattering we have, after backscattering, what vector we have? Vector with the opposite Q parallel vector and vector with the opposite Q parallel vector, why opposite Q parallel, why not opposite Qx or Qy? Because the carriers in the tube travel along Q parallel. If they get backscattered Q parallel will change the sign. That means if this is the QI vector this is also the Q parallel vector and after backscattering we are going with the opposite Q parallel which is now this one and again similar to graphene we get that the angle between these two states is pi. Cosine pi half is zero hence probability of backscattering is zero in armchair tubes and that's why armchair carbon nanotubes have extremely high carrier mobility. Revaling that of graphene basically above 70-80 thousand at room temperature. So you would say so what? That's the same. Well yes, but that's not the case for zigzag tubes. Because in case of the zigzag tubes we have a problem. So let's have a look at the tube which is zigzag, so that means we have a quantization line if you remember which are along the, you remember in case of the zigzag, this is the zigzag direction, this is the direction of the chiral vector, so this is the Q perpendicular, Q parallel is this one perpendicular to that one and the quantization lines go like this, perpendicular to Q perpendicular, they go along the Q parallel. And now if you look at the zigzag tube for instance, which is semiconducting, if the zigzag tube is semiconducting that means that those quantization lines are going to miss K point, so we have a bank gap. Okay, so we have a situation like this, you see, we don't have a quantization line here because if you would have a, if there were a quantization line there, that would be a metallic tube, but let's have a look at semiconducting because that's what we need in electronics, right? So, we have, so we miss with quantization lines, we miss the k-point, so the quantization line goes like this, okay? And now, let's assume we have a carrier traveling along a few parallel directions, so that means that the, for instance, state of the carrier is here. and now notice because this line does not pass through the origin of the Q coordinate system it means that the QI vector which corresponds to this state is this one you see, is this one and this vector not only has Q parallel component which is this one but also it has a small red component which is the Q perpendicular component clear? because the state is here now when this carrier flips due to backscattering when it flips to the opposite direction it changes its q parallel changes sign right because it was going in one q parallel direction then it hits well the wall and then it backscatters so that means the q parallel component q parallel must change the sign of this vector right so that means if this was the q parallel component going in this direction, this is the Q parallel component going in the opposite direction. Only Q parallel changes sign. And that means that the vector which corresponds to the final state is this purple vector QF. And now look, what's the angle between the initial and final state it's not Pi anymore. When it would be Pi? If there were no Q perpendicular component that means if this tube were metallic because if this tube were metallic this green line would pass here. Q initial will be like this this Q final will be like this and the angle would be pi. So if this tube were to be metallic, then angle would be pi and there would be no backscattering, high mobility. But if the tube is semi-conducting, quantization line does not pass through the k-point and the vectors Qi and Qf do not make angle pi, which means cosine pi half is not anymore you have a probability of scattering, you understand? Which means the mobility in this kind of zig-zag cube is smaller. And now from this you can already understand something I told you before. You remember I told you before, I'm going to show you this plot later on at some point, but now let's discuss it. I think I told you before that for all materials there is a trade-off between the band gap and mobility. The higher the band gap, the smaller the mobility, unfortunately. Remember this. And now you can already understand it. For instance, for tubes. Armchair, no band gap, high mobility. Zigzag metallic, no band gap, high mobility. Zigzag semi-conducting, band gap, reduced mobility. And you can already see how it gets reduced by increasing the band gap. How? How you increase the band gap in this structure? By pushing the green line more away, for instance here. How do you do this? By making a tube of the smaller diameter, so the green line is split more, the quant is larger, and you miss direct point by more. And look what happens then. Then we know we have a larger band gap, as I already explained you before. before. But look what happens with mobility now. Because now if you look at the carrier traveling again in the same direction with the same k parallel vector with the same k parallel value which is this one, now the state is here. You see this is the vector. And now when it backscatter, when it flips and it goes to the opposite direction, you flip the sign of Q parallel and this is this one. So the final state is now this one. So So you see what's happening as the green line goes away from the K point and the band gap increases. These vectors, the angle between the vectors reduces, you see, which means the cosine increases. Probability of scattering just goes up and up and up. So larger the band gap, larger the probability scattering, smaller mobility. That's it. So here you can very easily understand from this plot this trade-off. larger band gap, smaller mobility, because this line goes more away, angle is getting smaller, cosine half of this angle goes up.
\fig{51}{Nanoelectronics of graphene and related 2D materials 2024}
Okay, so we are done then with carbon nanotubes, I hope this was interesting, I hope yes, because now you see if you understand graphene it's very easy to understand the tubes. But now let's go back to the main topic of the first half of our lectures which is graphene, not tubes. So coming back to graphene I would like now first discuss methods for synthesizing graphene and how actually graphene was originally discovered and what's used today to synthesize graphene. The method which was originally used to obtain graphene was based on scotch tape exfoliation. I already explained you how it's done. You take the scotch tape and press against highly oriented pyrolithic graphite or HOPG, which is just a nice piece of crystalline graphite. You rip it off, you get some flakes on the tape, you keep doing like this with the tape, adhesive part against adhesive part, doing couple of times to make flakes thinner and thinner and at some point you press against the substrate, you remove the tape and you get lots of stuff on the chip. And what you are going to get on the chip is something which looks like this. So what you see here is the part of the chip. This is the optical image, the image taken by optical microscope of part of the chip. You can see here all sorts of stuff. So basically what you see here are pieces of graphite of different thicknesses. So obviously when you do this process of exfoliation, this is not really a controllable process and you can get all sorts of flakes on the substrate. In this particular case, those flakes which have a gold color, this is actually rather thick graphite, maybe up to 100 nanometer thick, so many many layers of graphite, of graphene sorry, and then as the color goes from yellow to like whitish, then the thickness goes down, thickness goes down and then it becomes darker like bluish, it's even thinner and basically in order to find the monolayer graphene you have to look for something which is very very pale so basically it's very hard to see and that's something which is shown here I don't know can you see it on this image it's better if you look at your tablets but maybe even here you can see it that for instance this is the substrate here is maybe by three graphene, but here in the middle you see very very pale object, basically almost the same contrast as that of substrate which is here and that's monolayer graphene. So the first point I like to make is that it's possible actually to see monolayer by optical microscope. Of course you need a very good optical microscope. The microscope from Amazon for 100 euro will not do it. You need a microscope which is very good. talking about 50,000 euros, so you really need a good magnification and high quality optics, but you can actually see and that was actually key to discover graphene. Why? Because when you do this when you do this process of exfoliation on a chip you will end up with maybe just few monolayer graphene flakes on a chip which is, I don't know, two by two centimeter in size. And these flakes are rather small, in the best case you will get something in the range of maybe 50 microns. So I'm talking about few extremely laterally small objects on a rather large chip, which is few by few centimeters in size. So how can you actually find it? Well, the first thing which comes to mind, let's use some of the microscope which has a very high lateral resolution, I mean sorry, a very high resolution in the vertical direction, so you can measure the thickness of this object we see on the chip and then we can guess whether it's graphene or not. Unfortunately that's not going to work. There are two main reasons. The first reason is that the high resolution microscopes are very slow. So let me give you an example. For instance, atomic force microscope will need 10 to 15 minutes to scan area 10 by 10 micron in size. Now imagine how much time you can easily calculate. It will take to scan by AFM the entire like 2 by 2 centimeter chip. It will take forever. So you can't do it. It's absolutely impossible. On top of that if you remember what I told you if you scan by AFM because of the dead layer for a monolayer graphene you will measure 1.2 nanometers not 0.33. So you will never be sure whether it's really monolayer or not. That's additional problem. Then you may argue, okay, but maybe what about scanning tunneling microscope? No. Scanning tunneling microscope is even worse. In terms of speed, it's much slower. And also, scanning tunneling microscope requires metallic substrate. Otherwise, you will crash the tip. So that's also out of question. Do we have some other high-resolution microscopes? Yes, maybe SCM, scanning electron microscope. But the answer is no again. Because scanning electron microscope cannot resolve 0.3 nanometers thickness. That would be extremely hard. With SEM you usually scan like this under right angle and therefore you don't see the thickness. You may look under angle, tilt scan, but you will never be able to resolve 0.3 nanometer with SEM. And SEM, although faster than AFM, it's still too slow to scan because if you would like to resolve 0.3 nanometers, you really need to scan in high resolution, so very small scan field. and if you have a small scan field it's going to take again forever to scan the entire chip. Not to mention that SEM contaminates surface as it scans it. It introduces carbon contamination so if you and you're looking for a monolayer carbon so obviously it's not going to work. So basically out of options. In transmission electron microscope you can see maybe monolayer I showed you but for that you need a special substrate which is a membrane and then how do you put this in the membrane and also TM has a high resolution so again how are you going to distinguish, I mean it's going to take forever again. So basically the key to discovery of graphene was that it was possible to see graphene under optical microscope, why? Because optical microscope has a rather, well this field of view is not really large but the point is you can have a field of view which is more than 100, let's say typical field of view 100 by 100 micrometers, but if a person looks at the image like you, you immediately see whether you have some pale objects or not. So if you have a motorized, if you have optical microscope with motorized stage, we just scan sample like this, goes quickly through the chip, you can just look at it and when you see something stop it. So basically that was the key to discovery of graphene, to use optical microscope to find monolayer graphene, that was the key. In other words, you scan by optical microscope the chip and when you see the object which doesn't look like this, but looks something like this, something with very very poor contrast, what you may expect from monolayer, then you stop and scan with some high resolution microscope at that point to understand whether it's monolayer or not. But I have to tell you there was also a bit of luck involved in all this. Because it turned out that because graphene is monolayer you can't actually see it on all substrates. It's actually quite surprising you can see it by optical microscope because it sounds almost unbelievable that something which is just one atom thick can be seen by optical microscope. And the reason for that is that you can see it only if the substrate has this purple color. If the substrate has a purple color, then interference of light, which goes through graphene, reflects back from the substrate, goes through graphene again, with the impinging light, will give you this very weak contrast, which you can observe by naked eye, I mean not naked eye, by looking on the optical microscope. Either looking through the eyepiece or what we do today of course we look at the screen. Because it's much more convenient to look at the screen than to look through the eyepieces of the microscope. So only if the color is purple, that is only if the wavelength is in the range of visible light close to purple. Only that you can actually see. Why that was a bit of luck involved? Because the group which synthesized graphene of course tried to synthesize it actually to exfoliate it on silicon dioxide silicon wafer. Of course you want to have insulating wafer because if you would like to make electrical contacts you don't want the device to be short circuited by conductive substrate. So you need insulating substrate and of course the standard is to use just silicon substrate with some silicon dioxide on top. And now comes the luck. Because it turns out, well, not turns out, that's known from semiconductor industry from many decades ago, that the color of the silicon wafer depends on the thickness of the silicon dioxide on top. What you see here is the color chart of the silicon wafer depending on the thickness of the silicon dioxide on top which is given here. So you see if the thickness of the silicon dioxide is about 50 nanometers, the color is like light brownish. And then as you increase the thickness of the silicon dioxide, the color goes here. So you see around 100 nanometers the color has a, the substrate has purple color and then it goes to blue light blue almost white yellowish orange and then turns back to purple at about 300 nanometers and if you ask a company to make your silicon wafer with some silicon dioxide on top if you like to make a device you're going to ask them to specifically grow silicon dioxide for you. Why? Because although silicon wafer exposed to air will oxidize immediately, the oxide you get by this native oxidation, so-called native oxide, is just few nanometers thick. Because once few nanometers is oxidized, it prevents further oxidation. But few nanometers is not good electrical isolation. That's too thin to be electrically isolated from the underlying substrate which has certain conductivity because it has certain doping. So basically if you want to make a device, what we also did in the past, I did it too, we asked company for instance for carbon nanotubes, you ask company to actually grow you thicker, much thicker silicon dioxide wafer so when you deposit the tubes you have no short circuits with the wafer, right? and you basically have to specify the thickness of the silicon dioxide. So what thickness are you going to specify? Totally arbitrary. But people like round numbers. So they typically ask for 100 nanometer thick, 200 nanometer thick, 300 nanometer thick silicon dioxide. And guess what? At 100 and through 300 you get exactly the purple color which will give you enough contrast to see monolayer graphene. graphene. So, basically that is a bit of luck, because this group in Manchester, which exfoliated graphene from graphite, actually had the wafers of exactly this thickness, which gave purple color. Had they had any other thickness with a different color, they would never be able to see monolayer graphene, because it can be only seen here. And then in 2007, people actually calculated. They actually calculated the what you see here in this contour plot, actually density plot, you see the contrast between the graphene and the wafer and you see the highest contrast is here. This blue is high, yellow is low. For a wafer with the thickness of 100 nanometer and 300 nanometers. You basically get invisible range, so that's what you look by your eye, you get invisible range, good difference between the contrast between the wafer and graphene and that happens only for purple wafer with 100 or 300 nanometer thickness. So that was a bit of luck involved and that's why if you look at this wafer you see it's purple and this one is also purple. On any other wafer it's very hard to see, it's almost impossible because the contrast difference is very very cool.
\fig{52}{Nanoelectronics of graphene and related 2D materials 2024}
But then once you see something which looks like very thin object, how do you actually characterize it, how do you figure out whether it's really monolayer or not. Well the first guess is as I said to try to do AFM, so in this first paper they published in science, they use atomic force microscopy to try to figure out what's the thickness of these pair objects they've seen on the wafer. And then it works like this, so what you see here is the image from our lab, I deliberately put image from I think 15 years ago, something like this, no wait, no yes 15 years ago, extremely old image to show you how we tried to do it at the beginning when large scale material was not available, so basically we exfoliated like they did and then we took optical microscope and we looked for a pale object, as you see at the beginning we couldn't make really large object, so here is the scale, this is 10 micron, but you can see here among these flakes, it's very hard to see on this image, here there is something which is very pale, which may indicate very very thin object, now the graphene we exfoliate is much larger in size, could actually fill all this image but back then that was the size we were able to produce and then we will continue after the break, ok? but then there is another problem once you find something which looks promising how do you find it again to be able to scan it with another microscope and that's why you have crosses here. So you see in order to find the structure afterwards you need some sort of coordinate system patterned everywhere on the chip so that you can use this coordinate system as a reference to find the structure. And these numbers are actually x and y coordinates of these crosses which are patterned every 80 microns by eBIM lithography in both directions. so then you know the numbers which corresponds to this area so you can very easily find this place later and now I put you here on the right an AFM image of the same promising area so you see this very small square here is not scanned here by atomic force microscope and that's what you get and now I can tell you immediately that this region here is a monolayer and this one is bilayer. I know that there is a difference between this region and this region which is a one layer graphene. Why? Because if you measure here by atomic force microscope you will see that the high difference between this area and this area is around 0.3 nanometers. Why? Because this is part of graphite so there is no dead layer in between and that's why the bilayer sits on top of the monolayer at the distance which corresponds to their distance in graphite which is 0.33 nanometers. The only problem is that this monolayer cannot easily be identified because it sits on a dead layer on a substrate so when you measure thickness by AFM you get a thickness difference between this area in this area to be around 1.2 nanometers and that's why you need another way to figure out whether this is really a monolayer or not and before I show you how it's done let me just say that if you look at this image which is very interesting to see that when this flake was exfoliated from graphite, because by exfoliation you can also break a flake along crystallographic lines, you can actually see them here, because this looks rather irregular, which tells you that this structure here is probably a crystal, right? Because if this would be something else, like maybe glue residue, it wouldn't have such a nice well defined sharp edges you see. Anyway, so how then to identify whether this is really a monolayer or not well one way to do it I showed you, you can on this part of the flake make a hole bar, measure quantum hole effect and if you get half integer effect you know that this is a monolayer but after you did it this is already gone, you already wasted this So, obviously this is not the best way to identify material because it's a method which will totally consume this part of the flake and the procedure is quite complicated, you have to do by lithography here the whole bar, then you have to cool down everything at cryogenic temperatures put in magnetic field, so it's really complicated, it's not really way to do it, it's too time consuming and basically destructive for the flake.
\fig{53}{Nanoelectronics of graphene and related 2D materials 2024}
So, that's why the method which is used even today to identify whether this is monolayer or not is the method developed in 2006 by Professor Ferrari and his colleagues. So, Professor Andrea Ferrari is a Cambridge professor who by the way graduated from Politecnico di Milano. Anyway, he developed method which is based on micro Raman. I'm not going to go into details what the Raman method is, that's topic for another course is just you should just understand how it works so you basically shine a laser light on a material it's called micro Raman because this laser spot is the size of a micron so it's a rather small and then you take so called Raman spectrum which looks like this and it turns out the Raman spectrum of monolayer graphene is quite distinctive no other material has it like this and and it can be used to actually as a fingerprint of graphene. What you see here are Raman spectra exactly of this flake. So when the laser shines on this area here which we now know it's a monolayer we got this kind of so called 2D peak in Raman spectrum but when the laser shines on this area which is bilayer now we know the spectrum looks like this. So there is a quite distinguishable of difference between the spectrum of the monolayer graphene and all other number of layers and that's actually method we use today not only to identify whether we have a monolayer graphene or not but also other materials so for other materials other 2D materials this method would also be used because all these other layered monolayers have distinguishable round spectra which can be used to identify the type of material and also the number of layers and this method is very good why because first it's very fast you just shine a laser and basically within few minutes you get the spectrum and it's non-invasive because you don't damage the flame 514 nanometers that's the wavelength of the laser lens. So it's a very nice and quick way how you can actually understand whether material is monolayer or not.
\fig{54}{Nanoelectronics of graphene and related 2D materials 2024}
And now that's the way how it was done in the past by exfoliation, but since people were very interested in electronics properties of graphene because of its high mobility immediately the focus was on trying to make electronic devices. Now in the way how it was done in the past You can actually make electronic devices you can go back and you can make a transistor here But that's not the way to make a more complex circuits because for more complex circuits You need to cover the entire wafer basically with monolayer graphene and then to pattern by lithography I don't know whatever whatever circuit you would like to make So having a monolayer flake here and there on the chip It's maybe good to make individual devices to understand how they work but that's not the way how you make electronic devices you want to make electronic devices on a wafer and that's why after this initial discovery of graphene there was a rush to develop method which can be scalable which can be used on a large scale which you can use to make a graphene as large as possible now the initially what people try to do was basically just to upscale exfoliation. So here is one example. So what you do is you take a stamp, could be a polymer stamp whatever, you press it against graphite and due to adhesion when you release the stamp from graphite it will pick up some flakes which will be placed on this side and then if, I mean this image is not important this is just to optically inspect whether you are happy with what you picked up or not. If you then take this stamp and go around the wafer and stamp it, you can get areas like this. So this looks like an interesting way to exfoliate graphene on a large scale. Why because when you use a stamp it prints against the wafer, some flakes will stick to the wafer and then you go to another place and another place and another and basically stamp it like this. I mean this, if you look at like this, looks nice but in practice is not really reliable. Why? Because you have no control of number of layers left after stamping. How can you control after pressing the wafer by the stamp, how many layers will be left once you remove the stamp. One monolayer, maybe two, three, you never know. But anyway, that was one of the ideas people try to do and this is like FM, so this is the image of one of these stamped parts and the kite profile, so like it's really thin. So because this thickness is like, as I already told you, it's about one point something nanometer, this is probably a monolayer here. But as you can see already here, they cannot control. This is a monolayer, this is a bilayer, then what, monolayer again, so it's not really controllable way.
\fig{55}{Nanoelectronics of graphene and related 2D materials 2024}
And then another way people try to do it was instead of using a stamp to try to directly press with HOPG against the substrate. You should know that HOPG crystals are rather small, that's approximately this size and you know semiconductor industry uses 300 millimeter wafers, so obviously that's a lot of stamping, but in theory can be automatized, right, providing you can really control how many layers you can leave after stamping. I mean I'm just showing you how another interesting idea which was put forward in 2009. So basically you take the HOPG and then you pattern it, you make these kind of pillars and then you turn it around and start stamping on the substrate and hopefully somewhere you get a monolayer graphene like here. So it's not really controllable but you see people tried many different things.
\fig{56}{Nanoelectronics of graphene and related 2D materials 2024}
And then there was attempt basically the same year to try to somehow control the number of layers you can leave on the substrate. So the idea was not to physically stamp it, so to physically press against the substrate, but to come very close to the substrate and then to apply a voltage between the HOPG stamp and the wafer because then the electric field will simply exfoliate, simply pull some of the flakes on the wafer due to the electrostatic attraction and you know that's possible because as we already discussed interlayer force in graphite is very weak, if you were talking about Wonderwell's force it's very easy, for the same reason you can exfoliate with the take you can actually do this and then idea was probably that if I control the voltage Maybe with the voltage I can control how many flakes I get. Of course it didn't work like this, but anyway, it got published in a very prestigious journal.
\fig{57}{Nanoelectronics of graphene and related 2D materials 2024}
And then there was another idea, I like this one very much because very early basically they got a very large flake, so you're talking about the flake here, which is like 100 micron. At that time, like 15 years ago, that was unheard of, that you can really exfoliate such a large monolayer. here it's anodic bonding, so it's another method used in semiconductor industry to bond glass on silicon wafer like Pyrex glass, but if you then replace glass with graphite, maybe then because bonding will bond graphite to graphene, will actually bond few layers on the wafer because of the poor Van der Waals force which keeps graphite together and sounds like a crazy idea but it worked, they got rather large, rather large flakes. However these are just attempts and as you can already guess none of these methods is used today.
\fig{58}{Nanoelectronics of graphene and related 2D materials 2024}
I like now to point out on another method which is kind of used today, which is I would say industrially scalable but it's not used in electronics, it could be used to disperse graphene in solvents, so you can mix it for instance with polymers and such to make for instance conductive polymers, because for instance you know most of the polymers are not conductive, but if you mix them up with carbon fibers or with graphene they can turn into conductive. So that could have some industrial applications especially because the method used is scaly And to give you immediately one application, for instance winter is coming, you know, and in the winter, in most of the cases, if you put the gloves, you cannot operate the smartphone. You know why? Because the smartphone uses capacitive touch sensor, which may, there is electronics which measures capacitance at each point of the screen and because your finger is slightly conducted when you touch the screen, this changes capacitance at that point and then electronics recognized where you press the screen and then processor takes the action upon this. If you have a glove, this is probably not going to work because the glove is usually made of insulating material, so when you press the touch screen capacitance doesn't change and nothing happens. Of course now you can buy conductive gloves, so for instance you can make conductive gloves by using standard material for a glove mixed up with something which is conductive like graphene. And the way how you can produce graphene in large quantities is basically pretty much the method which was described by Bohm. I told you this paper from 1950s or 60s in which he demonstrated something which looked like graphene. You basically start, it's basically liquid phase exfoliation. You put graphite in a solvent and then you can try to exfoliate it by using sonication. that usually doesn't work very well I told you because graphite being made of carbon atoms is rather hydrophobic so then everything you exfoliate aggregates so then the best way is to use a method for oxidizing graphite I told you use some strong acids like sulfuric acid to oxidize to get oxides and then that can very easily be dispersed and then you get in that way graphene oxide and then if you really need graphene you can reduce it back to graphene and you get solutions with for instance graphene. And that could actually create monolayer graphene, here is one example, I put here a couple of papers which are based on this method but I have to tell you that although this can be used industrially because you can put huge amounts of graphite in the liquid, you can sonicate on a large scale, you can reduce if you want on a large scale so chemically so that's possible. That's of course not the way to make electronic devices. Why? First of all, all exfoliation, whatever type of exfoliation is, is going to break the flakes too. So not only to exfoliate flakes from graphite but also it will break them. So they will have a rather small size you see this is one of the largest flakes they managed to make in this paper which I told you is pretty much copy paste of the paper of bone where it's very similar not copy paste but very similar and that's not what you really need in electronics 10 micron is nothing I told you you need on a wafer scale and the other problem is if you really need the graphene this graphene is going to be of pure quality maybe I told you this already because graphite graphene Oxide is really garbage material I have to say. It's full of defects because of the way how it's made because of the aggressive acid treatment. You have lots of carboxyl oxygen groups but the material itself is quite damaged. It also looks like urine. So it's really very bad material and then if you manage somehow to reduce it you get heavily defective graphene flake. with mobility which is extremely small because you get the flake which is not perfect, it has lots of defects, mobility is really very low, there is no point to use it. So that's why this material could be used for polymers, for mixing with other stuff, it's not going to be used of course in electronics. Of course at the beginning we also tried to do this in my group. We tried everything back then because there was no large-scale fabrication method, we tried, but the problem was that this, it turns out well the reducing agent, so you see this is graphene oxide, you see carboxyl hydroxyl group and so on and here you have like a graphene which is clean of them, so this is reduction agent. The problem is this hydrazine, unfortunately I bought it without looking at its safety data sheet. That's probably the most dangerous chemical I've ever seen in my life, I mean it's everything. If you look at the data sheet all hazards are there, toxic, bad for environment, absolutely everything. The only thing which is missing is radioactivity, few luckily it's not radioactive, everything is there. So when I saw it I just gave to my technicians and put it somewhere and never open it. You're not going because I don't want to kill anyone with this. So well these guys are good chemists so they used it, but I really didn't want. I mean we tried later with other more nature friendly reducing agents like ascorbic acid for instance, it's a very good reducing agent. You know what is this ascorbic acid? It's vitamin C, it's healthy. So no danger, but again the problem was that material you get is comes from defective graphene oxide and then you get defective graphene so forget it, it's not really for electronics.
\fig{59}{Nanoelectronics of graphene and related 2D materials 2024}
And now we are coming to electronics, so how to make this on a wafer scale? So the first very early attempt to make it on a wafer scale actually quite successfully was done just a few years after, only two years after the discovery and that was done by epitaxial growth on silicon carbide, kind of epitaxial growth, because you know epitaxial growth is used to grow semiconductors, so what these people did from Georgia Tech, they took the silicon carbide wafer, exposed it at very high temperature in UHV, and they found out that depending on the orientation of the silicon carbide wafer, with certain orientation you can actually get graphene after this treatment. what happens is that at very high temperature silicon sublimates and what's left is carbon which then rearranges in monolayer graphene. So they basically managed to get in 2006 graphene on a large scale on a silicon carbide wafer. So now you may ask if you look at the previous method people tried why not everyone use this method because this is just two years later why people try with stamping and everything well the problem was that this material was not widely available at that time unfortunately and the other problem is that when you grow it in this way it turns out that this monolayer graphene is not really free it's bound to silicon carbide wafer so its properties are slightly different and and it's very hard to remove it from silicon carbide. So that's not the conventional monolayer graphene you get by exfoliation on any other substrate. That was the main reason why this was not really widely used. And this also triggered a big dispute about the Nobel Prize, you know, but I'm not going to go into this. I think Manchester group got it deservedly because they not only showed very simple way how everyone can exfoliate graphene, but also they actually invited other people from other research groups to show them how to do it, to share their knowledge which is extremely nice and they also did lots of experiments, really striking experiments in this material. I think they deservedly got the Nobel Prize, these people were not happy.
\fig{60}{Nanoelectronics of graphene and related 2D materials 2024}
But anyway, I'm coming closer to the methods which are used now. What's the problem with this method here? You can already guess from what I explained. Apart from the fact that you have graphene monolayer which is not bound to silicon carbide wafer, the other problem is that silicon carbide wafers are expensive and the method itself is very expensive because it requires UHV ultra high vacuum and requires sublimation at very high temperatures, we are talking about 1000 Celsius. And now I'm coming back to the methods which are actually used widely today to grow graphene on large scale and that's actually growing graphene on metal surfaces because it turns out it's very easy to grow graphene on large scale on a catalytic surface, typically metal surface. The first method that was developed was growing graphene on nickel and the method is chemical vapor deposition which is pretty much identical to chemical vapor deposition of silicon and germanium that is used in semiconductor industry. So basically you know how to grow or germanium by CVD. You have a wafer substrate on which you want to grow silicon and germanium, typically you grow silicon on silicon and what you do? You heat up wafer to high temperature and then flow precursor gas in the chamber and the precursor gas is something which contains atom you would like to grow. So in case of silicon typically use silane, SiH4. In case of germanium germane, GeH4. So in the, for instance in case of silicon, when the silane comes in contact with the hot silicon wafer, because the wafer is hot, it has enough, this extremely high temperature has enough energy basically to break apart silane. So in contact with the hot wafer, silent molecules break apart, hydrogen goes away, silicon gets deposited on the wafer, that's how you could actually grow silicon. And the same with graphene. What you need is precursor gas, which is then CH4, which is methane. You use methane, CH4, you flow over catalytic surface like nickel, which is hot, methane molecule gets broken apart into carbon and hydrogen, hydrogen goes away and carbon starts spontaneously growing monolayer on nickel surface. The only problem with the nickel surface is that once the monolayer is grown, reactivity of the process does not drop dramatically, which means after the monolayer is grown, the bilayer will be grown, so after the first monolayer, the another monolayer will be grown and another and another and another. In other words, if you would like to grow monolayer on nickel you have to be very careful to understand when to stop the process to get monolayer, otherwise you end up with multilayer graphene, right? And the other problem of course with this process is that after the growth you end up with monolayer graphene grown on metal surface, which you can't use to make a device because everything is short circuited on a metal. So you have to then transfer it from the metal surface to some another substrate. So basically that's what's shown on this slide here. You see, so here is the wafer with the nickel on top, so they basically did not use the the nickel, they simply use silicon wafer and then just evaporate it thin layer of nickel on top, so that's the cheapest and best way to do it and they put G here just to make it fancy, so this is like G in nickel, so everything else which is shiny is nickel and this blue G is where the nickel is missing, they just wanted to make it like, I don't know, like more attractive than it is. And then you do CVD on this, you get graphene grown on nickel and then once you get it, once you get graphene on nickel you have to somehow remove graphene from nickel and the easiest way is then from here you go, you can either do it this way or this way, the easiest one to do it is to, let me see which one it is, okay this one. You basically put everything in nickel etchant, it's the chemical which etches nickel and basically when you put it, it will etch nickel, under etch nickel and then graphene will simply fly away in the solution and you just have to fish it out and that's done. If you may ask, okay but hang on, how do I fish it out? Because if I put everything in nickel etchant, okay I will under etch nickel, graphene will be released but how do I fish it out if I don't see it right well the other way you can do it is basically before you do this you put some polymer on top like they put here PDMS you put polymer on top so then when you under etch nickel graphene will get stuck in polymer and then you fish because polymer you can see and then you fish out everything polymer and then if you after dissolve polymer you will get the graphene on the top or you can even stamp it because in PDMS you even don't have to dissolve it, you can just press PDMS against the substrate and then once you release it graphene will, if the temperature is regulated well, you can get graphene stamped on arbitrary substrate. Okay, this way here is like a harder way to do it, this one here, because here what they basically do they under act silicon dioxide which is below nickel, so this is harder way to do it. you should just remember this one this is a simple rate because here they first under edge silicon dioxide and then under edge nickel and then they have it okay and then finally by the way what I wanted to say here just before I forget and then you end up you can stamp it as I said then it's on the PDMS but here you have for instance they show the picture of graphene on PDMS, so you can actually flex it and graphene do not break if the flex angle is not too much. That's the flexibility of graphene we discussed, which can be used to make flexible device and transparent you see, but we talked about this.
\fig{61}{Nanoelectronics of graphene and related 2D materials 2024}
And then finally this is the method which is today widely used, was developed in 2009 iron by Texas Instruments. Luigi Colombo was one of the key authors. He worked for very many years in TI. Okay, anyway, so the point is this is basically the same as nickel, the only difference is it's grown on copper. So the catalytic surface is not nickel but copper and basically you can grow it on a copper foil which is rather cheap. And why is this used not nickel? The main issue is that with copper it's much easier because once the first monolayer is grown on copper, reactivity of the process suddenly drops down dramatically which basically means that by layer it would be much harder to grow than mono layer. I mean it will be possible to grow of course but it will take much longer time than growing mono layer on nickel directly and that gives you plenty of time to stop the process on time and get just the mono layer. So that's the reason why this process is better because you have what's called self limiting growth in some sense. Once it's grown mono layer it basically limits, suppresses further growth. I mean it will be grown also by layer but much much slower and that's why this is the method which is used today because these are copper foils which are used for growth are very easy to manipulate you very easily get monolayers and the values you see here for carrier mobility that's carrier mobility of graphene on silicon dioxide substrate and it's very interesting that that even this is grown on a large scale by CVD process, if you take this CVD graphene and encapsulate in HBM, you will get pretty much the same mobility as with exfoliated graphene from HFGG. So we are talking about hundreds of thousands of square centimeter per volt second. So that's why this is the absolutely the best method which is widely used today. So what do we do in my lab when we need graphene? Well depends on the application. for us it's easier to exfoliate material although nowadays we don't exfoliate from HOPG we buy high quality powder graphite powder and then exfoliate from graphite powder that's for some special applications but mostly when we make electronic devices we simply buy you can buy this on the market graphene grown on oh thank you graphene grown on copper foil and then you simply cut the piece you need you under-etch in copper etchant to get rid of the copper of course before that you put polymer on top to be able to fish it out what we typically use is E-beam resist PMMA which is placed on top and then you fish it out, get rid of the polymer you get monolayer and arbitrary substrate and you can do it on any scale you want.
\fig{62}{Nanoelectronics of graphene and related 2D materials 2024}
And to show you that you can really do it on any scale you want I put here what Samsung did in 2010 where they demonstrated you can actually do all this process at 39 inch scale this is crazy 39 inch that's one meter right that's 39 times 2.54 centimeters that will give you one meter size like this that's really large and how they did it? well they took a large copper foil put in a obviously very large CVD chamber they grew the monolayer graphene on this copper then they took the copper foil you see this This yellow here, this is copper foil, this grayish area that's graphene grown on copper and then they simply use the roll to roll process which is very similar how newspapers are made if you ever seen. So basically you pass it through these two rolls, you roll it, so it pulls material, from one side they place graphene on copper foil, on the other side it's polymer support because if you would like to under etch copper you need some support to pick up graphene and then you get with this which you then roll it again into another roller you put it into a copper etchant and as it goes through the copper etchant copper foil gets etched so you basically remove copper from this and you end up with graphene on polymer substrate and then again you can use this roll to roll process to for instance place graphene on another polymer or substrate if you press with the original polymer with graphene on and that's how you get graphene. Of course you need a certain temperature for this separation, but that can very easily be controlled and you get graphene on another substrate. And why do you think Samsung was interested in this? And why did they make it in such a large scale? Samsung is mainly known by making what? I mean Samsung is a big company, they make absolutely everything. But apart from smartphones and other things, we also make large TV sets, right? Because if you would like to make a TV screen, flat panel display, what you need is a conductive and transparent electrode. Must be transparent so you can see the picture, okay, generated by the TV set. And typically what the industrial use is ITO, Indium Tin Oxide. This is very expensive material again because it has indium. So they thought maybe we can use graphene because graphene is also transparent and it's also conductive. So that's why they invested money in investigating this. I think they even made a prototype with graphene, but at the end they decided to stay with ITO. There are a couple of reasons, one was the contact resistance I mentioned before, because this electrode should also have a very low contact resistance to be able to easily operate TV set. set that was one problem and another one was the conductance because with the ITO layer which is thicker than graphene because ITO is transparent material you don't lose anything in transmission of light but you can get very very low resistance or very very high conductance. graphene you're limited by we will see later when we do calculations it has a high mobility but you cannot reach the conductance of thin metals or thin conductors like ITO because they are much thicker of course but since ITO is transparent that's it but that's the reason they were interested in this.
\fig{63}{Nanoelectronics of graphene and related 2D materials 2024}
However as you already guessed we are not interested in this course in TV sets, you are interested in electronics, so let's now discuss why we would like to have graphene in electronics. So what are the possible reasons we would like to use graphene in electronics, okay? So I mentioned couple of reasons at the very beginning why we would like to use 2D materials in electronics. Let me try to be more specific and discuss graphene. Okay, so I listed here on this slide, what are the advantages and also disadvantages of graphene from the point of view of electronics, right? So the point number one, the most important point is this. We have a very high carrier mobility and this mobility is not only extremely high, but it's also equal between holes and electrons because of the symmetry of the band structure so this is extremely extremely good why because I already explained you that high carrier mobility means high velocity high drift velocity right because the drift velocity was equal to mu times E. I'm going to write it down because I'm going to need it very soon And if the drift velocity is very high, it means carrier needs shorter time to go from one side of the device to the other, which means device is interestingly faster if material has a higher mobility. So that's obvious. However, you see I put one asterisk here. Yes, that gives you the high speed, but read the fine print. So this is something like you can see well in the past, now I think the industry is kind of regulated, so the mobile phone industry is trying to avoid this kind of traps, but you know maybe in the past you buy a mobile phone contract everything looks great, but then you see some asterisk and you have to look down fine print to understand some strings attached to this offer that it's not great as you think you know. What I would like to just point out here is that yes, high mobility is excellent for electronics, but that's not all you need. Let me give you an example. Imagine Ferrari, I mean like a car. Like it's very fast, it can go very fast, but what if it doesn't have a steering wheel? Is it useful? Not very much. You're going to crash at the first corner, right? So what I would like to say, high speed is not all. you need also control over this speed. That's the issue. And as I'm going to show you, that's a small issue with graphene. Because graphene does not have a band gap, you basically don't have a full control over this high speed. But we'll discuss that later in more details. Okay, second point. It has a very high saturation velocity, which means very high current. And there is no asterisk here, because this is actually good and there is no discussion about it. But what is saturation velocity? As you know, this linear relationship between the drift velocity and electric field is valid only up to a certain electric field. So if If you plot this drift velocity as a function of electric field, indeed it goes like this and this slope is given by mobility, but at some point this will saturate and you end up with a constant velocity which is called saturation velocity Vsat. That's the maximum velocity of carrier you can get in material and that what determines the highest possible current in material because the current density as you know is proportional to the drift velocity. Which means, let me give you an example, if you have another material with much smaller mobility like this, which material is better in high fields? Well if in high fields. Well if you stay here if you stay here obviously material with higher mobility is better but if you have application in which you need very high current then the second material is going to be better because it can reach high velocity despite having smaller mobility. So that's what ultimately limits the current in device. You can't go above the limit set by the saturation velocity. Luckily you don't have many examples in which we have this kind of situation because usually you operate devices somewhere here because this would also require very high electric field which means very high bias, very high voltage, right? So in most of the applications you need really high velocity, you need high saturation velocity but usually materials with high mobility will provide you in most of the cases sufficiently large also saturation velocity. And it's interesting that in case of graphene that's really indeed the case because if you compare graphene and silicon you see that graphene has saturation velocity which is four times that of silicon which is 10 to the 7 centimeter per second. But it's also interesting to note that despite huge difference in mobility, difference in saturation velocity is only 4 to 1, you see. And there is no asterisk here, that's what you really need if you want to get a high drive current. And then the other advantage is, and people really got excited about this, was that graphene has a very large current carrying capability or ability. Basically means you can use it for instance to make interconnect in microprocessors for instance. Why? Why is this interesting? Well, I don't know did I show you, no I didn't show you this. I will show you later, much later I will show you the cross section of microprocessor. Anyway the point is you should understand because you have billions of transistors in microprocessors at the zero level, at the substrate level, they have to be somehow connected together. You need metal contacts between all those transistors and you can imagine if you have billions and billions of transistors that's really lots of connections. So that's why connections in transistors which are called integrated circuits, which are called interconnects, actually go in levels. Basically first you have contacts between neighboring transistors and then you go to the further one on the next level and then because the next level conductors carry more current because they connect more transistors, they must be thicker and as you go up and up you have a larger and larger conductors. Why because of the fact that you need very large currents to at the end, if you basically look at the cross sectional microprocessor, these metal layers they totally dwarf the transistor layer. Transistor layer is just the surface of the semiconductor and everything else about it's just like a jungle of metal layers and because of they take a lot of space and there is also So current which flows through them and their resistivity is not zero, you get certain power dissipation, the current flows through all these jungle of metal contacts. People were thinking how to make interconnects more efficient and there are two problems with interconnects. One is as I told you it's a finite resistivity and with resistivity you have joule heating, problem number one so would be better to replace for instance copper with material with higher carrier mobility and the other issue is electro migration because the main reason of integrated circuit failure is not failure of the semiconductor it's not failure of the transistor is the failure of one of the metal layers because the current density in them exceeds what copper can withstand and then basically they break and they break by the process called electron migration which is very interesting. It's not related to Joule heating, it's slightly of course it gets mediated by the heating but the process is because when you have a high current you have lots of electrons moving in certain direction and those electrons create so called electron wind which pushes the atom physically and basically all atoms try to migrate in certain direction and then if you have a current flowing in this direction you may end up with for instance structure which looks like this so it's better to draw current in this direction and now what's going to happen if the same current flows through this conductor here the current density is much higher than here okay because you have a much smaller section so that means that this electron wind is much stronger here than here so So basically while, let me just finish this. While these atoms are not going to move to the right, because the current density is not sufficiently large to move, then these will move and that basically will create breakdown here. This part of the wire will slightly move to the right and the wire will break down. That's a breakdown by electromigration, because the atoms migrate due to this electrical effect.\\
So, to prevent the breakdown of copper interconnects in integrated circuits, it's necessary to keep the current density in copper below the level which is written here, one mega amp per square centimeter. And that's why people thought originally graphene would be great for replacing copper in interconnects because the breakdown current density in graphene is much larger. So the highest current you can pass through graphene is about 1.2 milliamp per micron. It's given per micron because graphene is a 2D material, so it's given with respect to its width, while in case of copper it's given with respect to the cross-section of the conductor. and I have to say this value of 1.2 milliampere micron is quite conservative value because you can get actually higher currents than that in graphene even up to almost three times that but if you go much higher than this then you have a device heating so that could deteriorate the device property so to be on the really safe side I mean really conservative estimate is to keep current at this level. So that means if the graphene channel is 1 micron wide, the highest current you can pass is 1.2 milliamp. Actually, as I said, it could go even up to 3 milliamp without problem. But let's say, let's be conservative. Now to compare this to copper, you should scale this to the cross section. So instead of having it expressed in amps per meter, you should simply divide by the thickness of graphene, which is 0.33 nanometers. And then if you take this value 1.2 and divide by 0.33 nanometers, you will get the current density in graphene of about 360 megaamps per square centimeter. So you see, compared to copper, it's 360 times larger. It's much, much, much larger. However, okay, there was a great excitement about this, but at the end, as you know, in microprocessors, still copper is used. Why? Well, the problem is, again, this is extremely high current density, but again, it's the current density calculated with respect to the atomic thickness of graphene. So, what I want to say similar to what we discussed with the breaking strength of graphene, in realistic applications you don't care much about relative strength, whether mechanical or current carrying strength or capability, what you're interested in is the absolute strength and that's why you should consider what happens in realistic case. So if you have a microprocessor with lots of transistors at some point you need a current to pass through the conductors of let's say one amp. So let's assume 1.2 amps for simplicity. So if you would like to get 1.2 amps through an interconnect that means that in case of graphene you will need what, thousand times this, right? And thousand times this is, how do you get thousand times this? You can either make a graphene sheet thousand times wider, but that makes no sense because if this is one micron and you multiply it by thousand, you get one millimeter, right, so it's rather large. Or alternatively, you can stack graphene sheets on top of each other, so basically you will need then thousand graphene sheets to get this thickness, which doesn't sound so bad, well because one sheet is 0.33 nanometer, if you multiply it by thousand you will need 330 nanometer thick stack of graphene sheets to get current 1.2 Amps safely through, but there is one problem, when you stack thousand graphene sheets that's not monolayer graphene more that's graphite and in graphite the breaking density is much smaller it's like 10 times smaller than that of copper. So obviously this is not going to work which means that you have to always look at the absolute value and that's why I put here asterisk. So yes you can use for interconnects if you need some ultrathin interconnects for some special applications but in reality for interconnects in which you have to pass large current that's not going to work because you basically turn into graphite. And then there is another thing which was very exciting at the beginning, especially physicists were extremely excited about it and that is the fact that graphene is an bipolar material. What does it mean, ambipolar? Ambipolar material means that you can get conduction of both types in the same material, which means both electrons and holes. Because if you remember, we're going to talk about this soon, but if you remember the band structure of graphene, so we have a Dirac cone, Fermi level through the Dirac point, and that means basically at any finite temperature you will have both electrons and holes involved in the transport. And if you would like to, for instance, get an anti-transistor, you don't have to dope it. What I have to do is to just push Fermi level into, for instance, conduction band. Great. This classroom is really nightmare. Does anyone see a chuck? okay plan B So that means, I thought you found Chuck. So that means if you push a Fermi level inside the conduction band electrostatically, for instance using the gate, then you get n-type transistor or if you lower it down into the valence band, then you get p-type transistor. So basically, if you would like to get a certain type of a transistor made of graphene, the only thing you have to do is simply to just electrostatically control the position of a Fermi level and you will get the type of the transistor you want. And that sounds cool because it looks like you can electrostatically control the type of the transistor and you don't need any doping, right? So this is really interesting. However, the problem is that in reality this is not a good idea. The main problem is that if you need to use the gate to make transistor into the, I mean I'm going to show you later how is this done with the gate. If you have to use the gate of transistor to push the Fermi level either in conduction on valence band, then you already use the gate to set the type of the transistor. but you know that you need to use the gate to operate the transistor, to operate the logic gate, for instance. So basically that means that that's going to interfere with the circuit operation because you have to use the same contact both for doping structure and operating the circuit, which really complicates things a lot. So, this is not really something that is recommended and but people were super excited about it because especially if you look at the... If you look at the nanoscale transistors now in microprocessors, doping actually represents a problem. It is a problem because if the device is very small and the channel is very, very short, then the issue is that even one or two doping atoms can change the situation a lot. You can get different property because if your channel is just, I don't know, 15 nanometer long and it's very thin and you have to dope it, then every doping atom counts. So at this scale it's very hard to control doping, that's really an issue and for that reason, as I said, people are very excited about this, having an bipolar graphene, because you can electrostatically, without any doping, without any issues, control the properties of material but as I said that just complicates things because you need to both dope and control the device and that's not really a great idea. So for that reason I put here asterisk, so it's not really something that you would like to do. And then there was also another issue that people were exciting because already we already discussed for suppressing short channel effects, we need to have ultra-thin channel, so from that point of view graphene looks great, but the problem... It doesn't work when it's charging, come on. The problem is that you use thin material to suppress short channel effect so that you get, for instance, as I already showed you, good drain current saturation so that current is constant in saturation, it doesn't increase with the drain source voltage and for that you need very thin channel. So graphene seems a perfect candidate for doing this because it can't get thinner than this, but the problem with graphene is that because it doesn't have a band gap, it's going to have a bad saturation as I'm going to show you. So yes, you can control short channel effect because it's very thin, but on the other hand the fact that it doesn't have a band gap makes things actually worse. So that's why another asterisk here. And then another advantage, everyone was exciting, that's no asterisk here because that's really true. already discussed, flexible and transparent electronics, there is no really doubt that graphene is good for that. And especially if you look at the competition we already discussed, because conventional semiconductors are neither transparent nor flexible. So basically you compete with organic polymers or such stuff where mobility is really, really low. I mean, when they get circuit operating at like one megahertz, they're extremely happy. That's considered high frequency in organic polymer electronics. And finally, I would like also to make one point regarding the bandgap, as I already said. We already understand that because graphene doesn't have a bandgap, it's not advisable to make digital devices out of graphene, because transistors can't be turned off, so you have a huge static power dissipation, which is a huge issue, of course. And that's why you would like to use it mainly in analog applications, not digital because in analog applications you can use circuits which, there are circuits which do not require transistor to be turned off, okay.