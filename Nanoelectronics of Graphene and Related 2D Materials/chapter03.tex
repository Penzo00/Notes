\fig{21}{Nanoelectronics of graphene and related 2D materials 2024}
What are we going to do I already announced last time now we are going to repeat the same we've done with by multiplying wave function from the left by psi a now we're going to do the same with sorry not with phi b okay so now you basically repeat everything I've done previously which means you take the wave function and now you put phi b to the left of the entire wave function and you get these two long lines and now again I can leave this to you for a homework because now the rest is just repetition of what I've done previously. Sanjay is going to go through this very quickly. So first of all, if you look at this term here, you have a scalar product between the atomic orbital function of one of the specific B atoms. So as I already said before, this is the wave function of one specific B atom, one that you pick up and keep fixed. Okay. so you have a scalar product between the wave function of that specific B atom and all atoms A and because the atom B is never equal to any of the atoms A there's a scalar product of two different wave functions which is zero you know this okay so this is zero this term is gone and then if you look this term here again you have the scalar product of two different wave functions at less, index i corresponds to atom b that you picked up. So basically here, you're going to get only terms for rb i equal rb when this scalar product is 1, and you get just psi b times e to jk r. This is b, okay, jk rb, that's it. So that's the only thing which is left here. and then you go back to the first line which is again more complicated also like the last time so if you look closely what you have here in the second term you have a matrix transition element for hopping from site B to any other site B I and if you remember I already explained you so this is this plot sorry this graph at the top sketch we had before now we have this one at the bottom when you look at some specific atom B the nearest neighbors are always from atoms A. So that means that the hopping between B and B is impossible because atom B is never surrounded by the nearest neighbors of type B. This is atom B. Which means that then again, this term is zero unless index I corresponds to the atom B. Because then you have a hopping between B and B which is just the site energy of atom B. and previously I told you that the site energy of atom A is 0 and now because of the symmetry of crystal lattice the same is for atom B so this value here is the same as the site energy at site A at atom A and for that reason the value must be the same we took it previously to be 0 so this is now then also 0 so the only term which is left from the sum is 0 so you don't have this term basically the only thing which is left this one so this is the hopping from atom B to any other atom A and again like last time the only hopping which is possible is between the nearest neighbors so the only three terms which are left out of the entire sum are terms which correspond to the hopping to the nearest neighbors of atom B so that's why when you rewrite this expression here you just take the same expression and put just indices which correspond to the nearest neighbors of atom B and then again similar to last time I decided to denote this nearest neighbor A as A1 this as A2 and this as A3 we have here the position vector RA1 RA2 RA3 and these these are indices which are here from one to three. Okay. And then you can continue. Then you say, okay, so here I have the hopping energy between atom B and one of the three nearest neighbors. And again, because of the symmetry of the crystal lattice, it doesn't matter whether atom sorry whether electron hops from a to B or from B to a the energy must be the same so that's why we should put here the same value we used previously which was remember minus gamma and then you put this minus gamma in front of the Sun and that's now the eigen equation you get and again we have here some where index I runs from 1 to 3 and now we are going into again transform it in the same way as I've just done a few minutes ago. So what you do is you have here a rai vector which is one of the position vectors, this one, this one or this one and then you represent it as a function of position vector of atom B and one of these vectors which connect them. So for instance let's have a look at ra1, now you will understand why did they put a1 to be here at the bottom because ra1 is obviously rb plus this vector and this vector is minus this one so it's minus u1 and the same for the other two rb2 is sorry ra2 is rb plus this vector and this vector is exactly minus u2 this one and finally the same for our a3 our a3 is our b plus this vector in this vector is exactly minus u3 so basically we can write again the similar expression to the one we used previously so RAI is RB now minus UI okay and then if you do this you again replace here this vector RAI with RB minus UI and then this term with RB and this term with RB can be cancelled out they are on on the opposite sides of the equation and then the only thing you get is this is just minus gamma psi a this sum e to j k minus ui with this term here and here you have just e times psi B and now if you remember the definition of the function f which I previously introduced function f if you go back to the previous slide was exactly this expression just without this minus here if you remember sum of e to j k u i and here we have minus j k u i so this is basically the same function just complex conjugated value so this is f star it's a complex conjugate of the previously introduced function f so basically with this final substitution we get the second eigen equation here so this is now the second one.
\fig{22}{Nanoelectronics of graphene and related 2D materials 2024}
And now let me rewrite everything together so the one you see at the top that's the one that we got previously and now this is the second one and this is finally the system we can solve because this is system two by two so there's this system of eigen equations we can solve it's very interesting that for instance you can write down this system in a matrix form which is written here because if you introduce now the wave function as vector with components psi a and psi b then you can see that right hand side of these two equations can be simply written in matrix form as e times psi a psi b vector that's what you have here and then what you have on the left hand side is just this how do I know this because when you multiply this with this you will get the 0 times psi a so nothing and here we have minus gamma f psi b which is exactly what we have here on the left hand side okay and then when you multiply the second row of this matrix with this column you will get minus gamma f star of psi a plus 0 times psi b which is nothing we get exactly what we have here on the left hand side which is minus gamma f star of psi a a okay so that means that what we get here in the matrix form that's actually the eigen equation we should solve so this matrix here 2 by 2 that's the Hamiltonian in this case and psi this column wave function is the eigen function in our problem so we get a matrix form rather than scalar form why because in graphing lattice we have two sublattices so that's the consequence of that we have two sublattices so we got two equations which we can done simply write down in a simple form using a matrix notation however of course now what we have to do is to find the energy that's the point that the reason we've done all this to find the band structure of graphene and we have to solve the eigen equation and of course you can use the matrix form if you like but I think it's much easier and faster to do the scalar form it's already expanded matrix form in two scalar equations so how are we going to find the energy the easiest way to find the energy from these two is to simply multiply these two equations you multiply the left hand side with the left hand side right hand side with the right hand side and what do you get when you multiply the left hand side you get minus gamma times minus gamma which is gamma squared then you have F times F star which is F magnitude squared and then you get psi B psi A but if you look at your right hand side you will also have a psi B so they simply cancel out and you just get on the right hand side E squared which means when you solve this by E you get the D is equal to equal to plus minus gamma magnitude F or absolute value of f because that's a scalar complex number okay so that's it there's the band structure of graphene you see you can actually get by solving problem on the paper you don't need any computer to do it numerically because if you remember semiconductors and you try to solve the band structure you can't it's too complicated and basically the only thing you do at the end is to do Taylor expansion to get to the band structure close to the bottom of the conduction band on the top of the valence band where you get that the energy is quadratic with respect to the magnitude of the wave vector right but there's just Taylor expansion close to the minimum or maximum of the band but to get the actual shape of the band for any value of the wave vector regardless of its position with respect to the maximum of minimum of the band it's close to impossible I mean you get basically you can't solve it have to do numerically if you really want to get the full band structure. However, here because of the simplicity and symmetry of graphing lattice, we actually got the exact band structure. So basically the only thing which is left before I draw you the band structure is to find function f of k because that is the function of the k vector I already solved, to put here inside that's it. before I do this because that's now again boring map let's look at the physics of this solution so already from this simple solution we have here we can understand something about the back structure of graphene so we get we get the energy of electron in graphene lattice is plus minus something which is always positive because gamma if you remember was introduced as a positive number was 2.8 electron volts and absolute value of f is always positive number so basically gather the band structure of graphene is plus minus something which means if you if you try to okay you cannot to draw it now exactly because we still don't know the function f I mean we did not solve it but that basically means plus is above minus right so if a zero is somewhere here and I would like to remind you that the zero level is equal to what that's the site energy you remember that's the reference level we have something plus plus something and minus something that means we got two bands one with the positive energy which is above the other one with the negative energy so we got what we got exactly conduction and the valence band conduction band is with the sine plus that is the band with electrons and and and and the sine minus is the valence band which is below that the bandwidth hole as you can already expect a zero Kelvin this top band must probably be completely empty and the one below completely filled because these bands evolved from half field PZ state it was only half field and if you start from eight which is half field you should get then two bands which are half field so obviously the one which is below must be filled at zero Kelvin the one above must be empty so the Fermi level is somewhere in between and that's it so there's the first conclusion we got the conduction and valence bands I already told you before that the conduction band is called pi star band that's how it was called in physics before and the one below was called pi and it's also very interesting that this band structure of graphene has been known since 40s, 1940 something. Although at that time graphene was not isolated of course and people even didn't think it's possible, they actually found bent structure already back then, why? Because if you would like to find the bent structure of graphite, you should first try to find the bent structure of graphene and then see what happens when you stack a graphene sheets together to get graphite and that's why people already solve this problem like 80 years ago but they didn't think that this material could be alone but this solution has been known I told you since 80 years ago anyway so there's the first conclusion we got a conduction valence band grade there is another important conclusion you can already see from this equation even though it's not sold yet fully, which is the bands are plus minus something, the bands are fully symmetric. There is no difference apart from the sign between the conduction and valence band, this is very interesting. That's something you don't have with conventional semiconductors because if you remember, if you have a conventional semiconductors, the bands are always a symmetric, you never have the same conduction band which is, you never get the conduction band symmetrical to the valence band. How do we know this? Very simple mobility for instance, you know that mobility in silicon and n-type silicon is much larger than that of p-type silicon. Where does it come from? You may argue from different effective mass, but where does different effective mass come from? Well, what's effective mass. How do you introduce it? You introduce effective mass through Taylor expansion of the band structure, right? It's related to the second derivative. So that means that the band structure is not symmetric because the effective mass of electrons and holes is not the same. You understand? Obviously, here we don't have that problem. That's great. That means the graphene band structure is fully symmetric. And so let me switch immediately to electronics, why is this important for electronics? Because that, as I already said, electrons and holes have identical mobility, which means when you make a P-type transistor out of graphene, you don't care. You can use exactly the same layout as for the N-type transistor, because the properties of the materials are the same. It doesn't matter whether the carriers are electrons or holes. With silicon, it's not like that, because the P-type mobility is much smaller. As I already told you, you have to make P-type much wider than the n-type to be able to get the same current or the same bias basically to get the same channel resistance right okay fine so nice that we understood this but now let let me keep going before I show you the band structure it's very interesting what I'm going to do which is something you probably done before with semiconductors I'm going to find the wave function have Have you ever actually found a function in conventional semiconductors? You basically just find the band structure and stop there, right? You usually do not look further for the wave function. Here, in this case, I'm actually going to also find the wave function because I can do it simply because the band structure is simple. And I'm going to do it because one of the main properties of graphene, which is its enormous carrier mobility, is directly related to its wave function. it's actually consequence of the spinner type of its wave function and that's why I'm going to find now also the wave function because I get the value for the function f so the second one in other words I'm going to fully solve the eigen equation I would like now to find the wave function okay I found the energy is this and now I need the wave function the easiest way to wave function is to look at one of these two equations and get rid of the energy that we just found so for instance if you look at the second equation if you divide the second equation by Psi a you will get on the right hand side Psi b over Psi a which is written here right and then if you put energy to the left hand side you will get minus gamma f star which is this divided by energy e okay and then here in this equation you just put energy we derived you put the energy is plus minus gamma absolute value of s so minus and plus minus will give you minus plus which means sine minus corresponds to electrons to the conduction band and sine plus corresponds to the host or the valence band so the top sign is always electrons like I introduced here okay so it's minus plus gamma gamma cancels out and you just get here in the denominator absolute value of f so that's the expression for psi b over psi a and now we can find the wave function you say the eigen wave function is psi a psi b you take in front psi a you get here one and instead of psi b you get psi b over psi a then you replace psi b over psi a with what we just got here which is this minus plus f star divided by absolute F and then you just get rid of psi a by normalization you basically say that the scalar product of psi with psi must be one so you take hermitian of psi which is you know you transpose conjugate this is what you get and this is the original function so this is the hermitian of this wave function and now when you take the scalar product you will get what this and this is psi a squared magnitude squared and this time this is just 1 plus 1 because 1 times 1 is 1 and when you multiply these two you get 1 why because minus plus times minus plus is minus is 1 and you have F F star which is F magnitude squared divided by again F magnitude squared. You get 1 plus 1. This must be equal to 1 if the function is to be normalized. So you basically get that psi a squared is 1 half or psi a magnitude is 1 over square root of 2. And we are going to simply assume that this is the value of psi a. So that's how you get this expression for the wave function done. that's it we have the eigen value we have the eigen function and we basically solve the problem so the only thing which is left is finally to get expressions for the function so fine again I'm going to leave to you because here in the energy we just need the absolute value of f and in the wave function we need also absolute value of f and but we also need f star f star you can get by conjugating expression from one of the previous slides when I wrote the expression for function f and regarding the magnitude of the function f well you take the expression for function f and you say the magnitude is of course square root of real part squared plus imaginary part squared so this is what you get and then with some small transformation you come up to this expression yes it's not that trivial but at least you can do it on a paper it's not really that hard so I'm going to leave this to you for a homework okay so that's the 30s you can just follow this and get this expression no no reason to waste time on this fine so let's plot the energy so the first thing I'm going to do is to plot the energy to find the band structure so you already got that the energy is plus minus gamma absolute value of f sorry magnitude of f and magnitude of f is this so the band structure of graphene as a function of the k vector is plus minus gamma multiplied by this plus minus gamma magnitude f so it's plus minus gamma times this but before I plot it this is obviously going to be a 3D plot because I have two variables here, kx and ky, unfortunately. So that's a bit harder to draw on the paper, but it's doable, you're going to see.
\fig{23}{Nanoelectronics of graphene and related 2D materials 2024}
But before I do this, let me raise one question. I have to plot this as a function of kx and k y but which values of k x and k y should i plot for all possible values of k x and k y from minus to plus infinity for both of them so for the entire k plane do i really have to plot it for the entire k plane or i can maybe plot just for a certain part of the plane and then say that everything else gets repeated as i go in the k x k y plane which is, by the way, called reciprocal space. As you know, if you remember from solid state physics, it's just enough to plot it within the so-called first Brillouin zone because everything else gets repeated after first Brillouin zone. So the first Brillouin zone is the equivalent of a unit cell in reciprocal space in the space of the k vector. which means that everything else is then obtained by simple translation okay so the only thing we have to find out is what the unit cell but what's the primitive cell in reciprocal space and the primitive primitive cell reciprocal space is defined as the minimum cell, as the primitive cell to which one k-value corresponds, starting from the gamma point. Well, okay, it will be clear when I do it. It's Wigner-Zyde cell, right, you know, this one, solid-state physics. So basically, okay, forget about this, how it's done. So let me just give you a simple way how to do it. have to first find the reciprocal vectors in the reciprocal space which are which correspond to the unit cell vectors in the direct space so for that first we have to go back to our direct lattice under lattice in which we have unit cell vectors a1 and a2 and then we have to simply calculate reciprocal vectors or unit cell vectors in reciprocal space these are the B vectors and these B vectors can be calculated using these formulas I put here. So you see the B1 vector, so any of these vectors is always 2 pi divided by the volume of the volume of the unit cell in direct space, which is just a mixed product of vectors a1 and a2 and a3. The only tiny problem is that because this is two-dimensional space, we don't have vector a3. And for that reason, to use the same formula, you can simply assume that the vector a3 is just the unit vector in the z direction, because we have these two vectors a1, a2 in the xy direction. Did I put this? There is no axis. Okay, I forgot to put maybe the axis, xy, but you remember, so this is the xy plane, these are the unit cell vectors in the xy plane and the a3 vector is missing because it's a 2d space so for the a3 vector you just take the vector perpendicular to the xy plane so any vector will lose so 0 0 1 is enough and then you can use these formulas you have all three vectors so any of these three vectors B is simply 2 pi over the volume of the unit cell vector in the of the unit cell in the direct space multiplied by and then you start from the index if this is one you keep counting in the circular way so one two three two after two is three and then you go back to one and then three you go back to one and then two that's it we have two pi over volume times the vector product between the other two direct vectors you simply count like this okay and now if you do it you will get the following for these three vectors so let me first discuss B3 B3 is going to be this when you calculate 2 pi 0 0 1 again this one is irrelevant because there's the one out of plane which is perpendicular to the xy plane you can forget about this B3 it comes to the consequence of a3 which I introduced to get all three vectors okay so forget about B3 what you need really is B1 and B2 and when you solve B1 and B2 you will get this with the zero at the end the z component is obviously it must be 0 because this is 2d space that's what you're really going to get when you do this calculation at home and because it's a 2d space and just going to ignore this third coordinate so that's why I didn't put 0 and 0 here that's it these are the vectors b1 b2 these are the unit vectors in reciprocal space let's plot them b1 is this one you see it has both components positive so it points like this and B2 is this one it comes downwards because it doesn't have X component and that component sorry Y component is negative you will also notice that both vectors have the same magnitude okay and that's it and how do we now construct the first B1 zone so that's the primitive cell to which only one site, which primitive cell with only one site per cell and the site is actually the gamma point you start from the center of the coordinate system so you say like this if I have a site here at the center at the origin which is by the way called gamma point now translate this gamma by any integer linear combination of the vectors b1 and b2 and find the other equivalent sites because those sites can be obtained by translation so that these are equivalent sites and then the limit the space so simply split the space in half between this site and the one you get by translation so that's then at the end and closing space only has one site per cell which is the gamma point okay so let me do step by step you will understand so for instance if you translate gamma points by vector b1 you're going to get this point here okay and now you split the space exactly half in the middle you take the line because this is a 2d space in 3d space you will put a plane which is perpendicular to this vector and exactly passes through the middle of this one because this is 2d space is just a line in 3d space you will place the entire plane perpendicular to b1 vector and passing exactly through the half of it that's it and then you just keep going until you enclose the minimum space in which we have only gamma point they say like this okay if I now translate gamma point by b2 I get this point here I'm going to split the space in half by this line which is exactly in the middle okay let's keep translating further you say okay this is minus b1 if I translate gamma by minus b1 you end up in this point here leave the space here between these two points and that's given by this line here and then take a vector minus b2 which is this one here when you translate gamma by minus b2 you end up in this point here, split the space in half between these two, you simply draw the line in the middle, you get something like this and then we exhausted the linear combinations 1 0 minus 1 0 0 1 0 minus 1, now we have to go to 1 1 1 minus 1 and so on, okay, so let's go to combination 1 1, this is vector b1 plus b2, okay, b1 plus b2 is exactly this vector, when you translate gamma point by b1 plus b2 you get this point here. Split the space in half and you get this one and then we have a vector minus b1 minus b2 then you translate gamma by that one you get this point here then you split this place in between you get this one and and now you should keep going so this is combination 1 1 minus 1 minus 1 but if you now keep going further 1 minus 1 or minus 1 1 you will get a vector which are far away from this so splitting will not reduce the space around gamma point for instance if you take 1 minus 1 which is b1 minus b2 b1 because this is b2 b1 minus b2 is this vector here okay then you split the space it's not going to get inside this interior of this hexagon around gamma point and so on the same for minus 1 1 and then 1 2 2 1 1 minus 1 so on you will always get the vectors away so that's it so the minimum space enclosed by all these splitting is actually this hexagon here so this green hexagon here and because of that reason this green hexagon is the first Brouillon zone of graphene all other parts of all other parts of k-space reciprocal space can be obtained starting from any point here and then doing in integer linear translation we have a repetition translational symmetry also in reciprocal space so that's why we don't have to plot the energy for all possible K values we just have plot for the K values inside this hexagon and that's it everything else is obtained by repetition by translation it's very interesting that due to hexagonal symmetry of the direct lattice we got that the first Boulogne zone of graphene is also hexagon that's very interesting okay and now before I finally plot it let me just introduce the corners let me just introduce the corners of the first real ones on so these dots here these big dots here the reason I'm introducing them because they will have a very very special meaning they have a very they're extremely important in graphene physics and for that reason I'd like to introduce them now. So we can introduce the six corners of the hexagon. Notice that only two out of these six are non-equivalent, which means if you set two, the other four can be simply obtained by translation. Why? Because they're all at the edges, so obviously something will get repeated. So what's the easiest way to pick up two non-equivalent points? One way is just to pick two neighboring points. Whatever two neighboring points you pick up, they're going to be non-equivalent because if they are neighboring there is no way you can by translation go from here to here by using these vectors b1 and b2 because this is too small you understand but i'm not going to pick it up like this i'm going to pick it up so that one non-equivalent point is here and another one is here because why did they pick it up like this well this is very easy to understand if these two are non-equivalent that means that these two are non-equivalent which means these two are equivalent and that means that these two are non-equivalent so these two are non-equivalent because every second is non-equivalent when you go from one point. The reason I decided to pick up these two rather than for instance these two is because I will need the expression for the vectors of the this point and these two are the easiest why because if I know this one then this one just minus this one I don't have to calculate anything that's the only reason is just for convenience and if you don't like you can take the two neighbors and you will get the same results at the end at the at the very end so how they are called so this point here is typically called the K point this is then called the K prime that is the another non-equivalent point while the other four points can be obtained by translating one of these two these points in the corners of the hexagons are called Dirac points so these are the so called Dirac points why they are called your points what are why are they called your points you will find out very soon because they have a very special meaning there is a very they have very special importance in their extremely important in grapheme physics. So that's why they have a special name, which is, they are called Dirac points. So let's have a look, as I said, at these two non-equivalent Dirac points, and then if you don't trust me, you can try to double check that these are really, that the other four are really equivalent. For instance, if you look at the point k1, how can you get point k1? Like this. It is k plus this vector, and this vector is minus b1, which is the first equation here. So this shows you how you can get point k1 if you know the properties of point k. Actually, it shows you how you get k1 by translating point k. And then the same for the other points like k2. If you look at the point k2, how do you get the point k2? You say like this, to get point k2, you take point k, and then you have a k vector plus this vector here, and this vector here is minus this one, which is minus b1 minus b2, okay? And the same for the other two. k1 prime is equivalent of this. k1 prime is this one. How can you get it? k1 prime is k prime plus this vector, which is b1, which is written here. And the same with k2 prime. k2 prime vector you get as k prime vector plus this vector, and this vector is b1 plus b2. That's it. that's the proof that the other four are equivalent and we should only if you are really interested in the and as you're going to see they're very we are going to be very interested in the corners in these six points then we should just look at K and K prime the other four equipment and we can ignore them okay fine and because of their importance because of the importance of these K points we have to find the coordinates of K and then K prime is just minus K vector that's easy so the easiest way to find the coordinates of the K point is to first find the coordinates of the end point which is here at the intersection between this line and this line so it's exactly in the half along this line or half of the vector b1 as you like and if you very easily see vector m is just half of the vector b1 because it's exactly in the middle you can write down that the vector m is just b1 divided by 2 and the b1 we already derived so if you divide this by 2 you will get the m vector okay so now you know the magnitude of the of the m vector which is magnitude of this and it's equal to this value now you can easily find the coordinate of the k point it all it obviously has only the kx component ky zero that's the reason I picked up this point it's very easy so we just need to find this distance and you can do it from this triangle because here is the right angle we have a 90 degree here so simply say that kx squared is equal to magnitude m squared which is this which is this squared plus this squared and this value here is half of this and because this is the triangle with identical sides this is the same as kx this kx half squared okay and then if you finally introduce the value for the magnitude m you get that kx is given by this value and that finally gives you the coordinates of the k and k prime point so the K prime is just K X you derived 0 so it's this times 1 0 and K prime is just minus K is just on the other side that's it the reason I calculated them because I'm going to need them you're going to see very soon okay is this clear this first belongs on of graph and fine so if this is clear I'm just going to note that if you take a and multiply K X you will get this value we are going to need it later. So this is not that important, so they put in the bottom right corner, that's not important right now, we will need it later, for a later reference. Okay, so let's finally plot the band structure, so let's plot it.
\fig{25}{Nanoelectronics of graphene and related 2D materials 2024}
\fig{26}{Nanoelectronics of graphene and related 2D materials 2024}
So how do you plot it? You, okay, you put this function f, its magnitude, multiply by plus minus gamma, put in a PC and that's what you get when you plot. So it's plus minus gamma magnitude f, which is the, this value here that we derived previously and then if you plot as a function kx ky for certain values of kx ky you get something which looks like this now of course this is just put in PC and curry and plot it but now I'm going to explain you how do we get it the first of all notice that we have a plus minus something and that's exactly what we got the top one the top surface is obtained for sine plus so the orange surface is obtained for sine plus the bottom surface is obtained for sine minus so the top one orange one that is the conduction band the bottom one is valence band okay let's make a break because I was late seven minutes because of the problems with this, can we have a break only seven minutes, so we start on time but we don't finish later is this okay? okay did you think about the room for Monday? which one do you want? 25? okay, you don't want to sit on the you can always bring a pillow I told you. OK. So I will ask for room 25. But let me just stop. OK. so first the surface of the top that's the conduction band the surface of the bottom blue the blue one is the valence band so that's plus minus you can already understand I mean that's clear also from the expression we have of course symmetric band structure but what we didn't know so far was this if you look very closely you will see that these two surfaces touch at one two three four five six points they touch if they touch if the conduction and valence bands touch it is graphene is what type of material semi-metal it's not a semiconductor you know people used to say to in order to sound better they say the zero bank gets any conductor so what are the metals then negative bank gets any conductor well anyway so graphene is a semi-metal because the conduction and balance band starts and that will have a profound impact on its electronic properties in terms of electronic devices which means it doesn't have a band gap which means you can't turn it off because you turn off transistor by placing the states in the band gap where there are no states and this is obviously not possible anyway so it's a semi-metal then the other interesting point is what did I say one two three four five six ah what a coincidence right so if you now draw first belongs on here what are you going to see is that these six points are exactly six K points from the first belongs on or if you look at the same plot from the top so this is now what they put here what they put here on the left there's the top view so if you look from the bottom you're going to see this notice that this is a different color color scheme this orange one when you look from the top looks like this and then I just change the color and then when you look from the top you see that actually this actually resembles a hexagon and these are the six points these six points here because you see in the center when you look from the gamma point from the center you're at the maximum which is here and then whether you go left right whatever direction the conduction band goes down so here is the maximum it goes down and then it reaches the minimum here at these six points which are denoted here and these six points actually form a hexagon so that's what I said if you now plot first Brillouin zone of graphene that I mentioned it turns out that these two bands exactly touch in the six corners of the first Brillouin zone or if in this plot you add the green hexagon from the previous slide it will end up exactly here so notice this is K X so the K X axis actually goes like this okay K Y axis goes like this I didn't plot it maybe I should plot it and this is the K point here this is the K prime point and you have the other six points so that's why those K points are so important they have a special meaning for graphene and call our Dirac points why they are called Dirac that would be you will understand very soon but that's why they are important because of these six points the conduction and valence bands touch but why is this important why the point at which these two touch are important well and this you have to remember what we discussed previously: a zero Kelvin we know that valence band should be completely filled with electrons and the conduction band completely empty and that means the Fermi level must be somewhere in between but if these two bands touch where is then the Fermi level it passes exactly through these six points which I drawn here so here is the Fermi level so this is now on another I use another color scheme so what you have a red at the top that's the conduction band at the bottom you have the valence band and because the Fermi level at zero at zero Kelvin must pass somewhere in between but they touch it must pass exactly to the point at which these two bands touch why because otherwise if the Fermi level is somewhere up from this position then it will be inside a conduction band and conduction then band would not be empty if it's below then it will be inside the valence band and then in the valence band you will have some empty states okay and for that reason that the Fermi level exactly goes passes through the Fermi level plane exactly passes through these six points and now you understand why are they why they are important why because the transport of carriers always happens somewhere close to the Fermi surface right so that's why the physics of this material we are interested in is basically physics in the vicinity of the Dirac point I don't care for instance for gamma point here the maximum because this is far far far away from the Fermi level nothing is going to happen there I have no business looking at this no reason because everything happens close to the Fermi level so that's why these six points are so important because the entire physics actually comes from there all electronic transport properties from the transport around these six points And that's why these six points are so important. And that means that if you would like now to better a band structure of graphene, you can now finally do what you've done with other semiconductors. Remember, we're interested only what happens at the bottom of the conduction band, right? So what you do instead of having a complex expression for the conduction band, you say forget about this, I'm going to use Taylor expansion to find the approximate expression for the spectrum for the band structure close to the bottom of the conduction band. The same for the top of the valence band. So that means we should do something similar here. Basically, although we have an exact expression for the band structure, so this function given by the square root, you remember, it would be much easier to somehow simplify this expression further. because although we have the expression for the band structure in the closed form, it's a bit inconvenient to use. There are lots of terms there. There is some square root. So why don't we use again the same approach like with regular semiconductors and expand the band structure here. But there is a small tiny problem. If you closely look at these points, you will notice that they form something which is called a cusp in mathematics, which means they are extremely sharp like this. Where does this come from? Why do they have, why they are so sharp? Well, if you go back to the expression for the, for the band structure for the function f, you will find there the square root of something. It actually comes from that. Because, for instance, if you try to plot, for example, it comes from the absolute value if you remember for the magnitude if you plot the absolute value of X it's going to look like this right so this sharp point here at origin is called cost and cost is any point that means the left and the right derivative are not the same it is a small problem why if you have a cost you can't use Taylor series around that point because the derivatives are not same and if you remember in our function what we had was the square root so the square root looks like this right but you basically have some sort of square root of absolute X so you end up with something like this you get a cusp again so basically this absolute value this magnitude makes this unfortunate cost which actually prevents us from using Taylor series to get the approximate expression for the function. So what are we going to do? Well, the solution is very simple. The point is we should not try to expand the structure directly in the sense if you remember the energy, the band structure was what? It was plus minus gamma times the magnitude of f you remember so we are not going to expand the magnitude of f in Taylor series because it's the magnitude of f which makes these cuts but we are going to expand in Taylor series is function f and then we will find the magnitude of that you understand so that will allow us to get simplified expression for the function without dealing with the cusp and the Taylor series because if you look at the function f the function f you remember was a complex function it has a real part has imaginary part but if you plot separately real and imaginary parts there are no cusps this is a very nice and smooth function and you can both of them both really imaginary you can simply expand in Taylor series around any point the cusps comes as a consequence of this magnitude that's the problem the same as function function x doesn't have a cost but when you put magnitude you get a cost the same story so instead of expanding energy directly because you remember the energy was I wrote you before the band structure was this if you remember plus minus gamma magnitude of f of k instead of expanding this the entire function in Taylor series I'm going to expand only function f because function f doesn't have magnitude and can be expanded in Taylor series. And then I will take the magnitude multiplied by gamma and get the spectrum, okay? So that's it.
\fig{27}{Nanoelectronics of graphene and related 2D materials 2024}
So let me now try to find out the Taylor series, this function f, around points of interest. So at which point we are going to find the Taylor series? As I said, point of interest. What are the points of interest? The Dirac points. Why? Because that's where the Fermi level is at zero Kelvin, okay? So let's now expand function f around, for instance, k points. Later on, we will do it around k prime points, and that's enough. Why? Because the other four points are equivalent. We don't have to bother about them. We should just look at the point k. Okay, so there you go. Here is the expression for function f that I wrote at the beginning of this lecture, like something like an hour ago. This expression here, right? And this is what we are going to expand into Taylor series around point k. not its magnitude where are we going to expand it we are going to expand it here at point K so this is just a part of the first bill one zone I didn't drop everything because it's based on this slide to do the derivation so I'm going to expand this function exactly at point K okay the K vector of electron close to this point is some some decent vector key so this small vector case this one the upper case vector k is the vector of the point k okay this is okay this is error this is this will come out later so ignore this Q at the moment okay I think I should fix it later fine so let's do it you say okay so let's start first in general so how do you I think you know how do we expand the function in Taylor series around arbitrary point a for instance well you say this this expansion is what this is the f of k so it's the value function of this value at this point plus we have a first partial derivative y partial because f of k is the function actually of kx ky we have two variables we go the partial derivative with respect to kx numerically evaluated at point k of expansion this point here times kx which is the variable minus the corresponding component at the point of expansion, which is the x coordinate of point k, which is uppercase Kx. Thus, we have then the same for the other variable ky, the other partial derivative. So these are the linear terms, plus now we have the second order terms and so on. You always, in physics for simplicity, stop at the first non-zero term. And as you're going to see here, it's going to turn out that all these linear terms are actually non-zero and we can stop here. When you do expansion with conventional semiconductors, where do you stop? When you try to do this with conventional semiconductors, where do you stop? After the second term, why? Because with conventional semiconductors, there is no actually a linear term. If you try to expand, you will get zero, nothing. we have to go to the second term here as we are going to see the linear term is not 0 and the second order term is going to be 0 so that's also the additional reason not to go to the second term ok oh this is enough we just stop at the linear term and how this can be simplified in this expression that's what comes now because you are expanding around this point k it's quite convenient to introduce a new coordinate system with the origin at the point of expansion. Okay? So if this is the state of the electron here, described by this vector, lowercase k, this vector here, if you introduce the new coordinate system at the point of expansion, which I'm going to just call Q, so it's Qx, Qy, this is basically, again, the k coordinate system just shifted to the point of expansion, which is the Dirac point k but I have to use the different names because it's obtained by translation from the original system so I just translated my original system here and got this one q ok then I can express everything as a function of the new q vector because if the state of the electron is here in the original coordinate system the vector k is this but in this new coordinate system the vector is this which is denoted by lowercase q and the relationship is very simple uppercase k plus q is lowercase k which means that the q is lowercase minus uppercase k and if you introduce it in that way then this can be simplified in the following form why because the lowercase k minus k means that the lowercase kx minus uppercase kx is qx and the same here with QY. So from now on, because I'm looking only at the point of expansion, I'm going to use a new coordinate system Q. It's very important to know that you should have actually done the same with semiconductors, with conventional semiconductors. When you get a spectrum, it's the H bar K squared divided by two effective mass. You should actually write down instead of K, Q, because it's very rarely that the bottom, for instance, of the conduction band is a D-gamma point. So when you expand the conventional semiconductor band structure around, for instance, the bottom of the conduction band, you should move the coordinate system there, and this expression, which is written h-bar k-squared, this k is actually q. And here I would like really to stay precise, and that's why I'm going to use q instead of k, because that's the correct way to do it. Okay? Fine. Anyway, you know the Taylor series, so now this is very easy to do you take the first partial derivative you take this function and find the first partial with respect to kx which means when you take it you assume that ky is a constant so when you take it this is what you get I'm going to skip you can do this at home that's just partial derivative and then you do the same you need another partial which is with respect to ky you do it you get this again please do it at home I'm not going to do it here instructions for the exam. At the exam, you may get this question. I usually give this question quite often, right? What I would like to understand that you understand what you're doing. So please at the exam, do not write me these expressions out of blue. Please derive them. Don't write them like I wrote them on the slide because I personally wouldn't be able to recreate them just from my head. I always wonder how the students do it on the exam. Just write these partials like this without any derivation. So please derive it on the exam if you would like to get all points. So please take the function, okay, this is 0, derivative of this is 2 times cosine of this times 1 over 2a, and so on, so that they really see that you derived it, that you know what you're doing, okay? So please do it on the exam. Finally, okay, so this is what you get. and now finally to get the expansion you do the following what we need for this expansion is this we need what we need the value of the function are the K point you basically take the coordinates of the K point put here what you're going to get 0 Y we know that we must get 0 Y because the band structure because the conduction and valence bands touch at zero. So the magnitude of F at touching point must be zero. And if the complex value has a zero magnitude, then the entire complex number must be zero. So that's why F of K must be zero. So this is the comment you can write down at the exam. You can write down because they touch, this must be zero. Or, hard way, take the coordinates of point K and put it there if you want. so the point k is this if you want insert this at the top you should get zero but I think that's just a waste of time at the exam you can just write down but please write an explanation why why zero okay okay that was easy and then these two unfortunately here you have no choice but to do it so basically first partial you take this around k points so you basically take this and insert here and then you get this so now you understand why I picked up this point to be the kx axis because if this coordinate is 0 it's much easier to calculate this that was the reason because if ky is 0 you know that this term is 0 sine you have 0 0 there is no reason to calculate anything you understand so that was the reason and then the same for the partial with respect to y you take again these coordinates here you put inside you get this and now from here we can get the function f of k around the k point it is 0 plus minus 3 over 2 times a times qx plus j times 3 over 2 a times qy and that's how you get this expression for function f of k okay and then we have another non-equivalent Dirac point which is k prime, k prime is this, and if you do, if you would like to find out the approximate expression for function f around the k prime point, it's just this, the same, with this inserted here and here, that's it. Around k prime point 2, the function must be 0 because the band starts, and these two, you must calculate them, sorry, you take this expression, again, you see, now you understand why I picked it up like this, you take this, put here and here you get this and this and what you have here at the bottom that's the approximate expression for a function f of K around the K prime point so these are approximate expressions when the state of the electron is very close to the K point if you go far away then it's not really valid anymore but that's like any Taylor expansion you know this if you go too far away then that becomes symbolic but close to close to the K point this is okay and that's what we need because the Fermi level is there.
\fig{28}{Nanoelectronics of graphene and related 2D materials 2024}
Let me go back just to the k point again so let me go back to the k point again so around the k point this is this is what we got as the as the approximate expression for the function f of K and now I can already calculate energy but I'm not going to do it right now I'm going to do a one diversion I'm going to go and calculate Hamiltonian why because it's very interesting to see the shape of the Hamiltonian so before I take this function take the magnitude multiplied by plus minus gamma to get the energy which is by really trivial to do let me calculate Hamiltonian for us and then we'll come back to energy okay because Hamiltonian is going to be very interesting so this is the Hamiltonian I showed you the beginning right it was because we had a system two by two it was actually this matrix if you go few slides back and then you say okay let's plug inside this function to see what's actually Hamiltonian close to the k point was the Hamiltonian close to the Dirac point okay you take out the gamma fine here is the function and now put here we have minus f of K so because we have here both functions they both have 3 over 2a you can put in front of the matrix 3 over 2a and you get minus this so it's qx minus jqy and here you get minus f star so it is qx minus because it's star you get plus jqy this is the Hamiltonian and now if you remember Pauli matrices now this is quantum physics so if this is a bit too too much into the quantum physics so if you are studying electronics, you've probably never seen these matrices before. So if you ever get a question like this, what's the shape of the Hamiltonian, I'm going to give you Pauli matrices at the exam. You can then transform it by yourself without memorizing Pauli matrices, because there is no reason to memorize them. Anyway, if you now remember Pauli matrices sigma x and sigma y, which look like this, you can then rewrite this entire matrix in this form. Because here you have QX, QX, so it's QX times 1, 1, so it's matrix sigma X. And you have minus J and here you have plus J, which is exactly what you have in sigma Y. So basically what you get is 3 over 2A sigma, this is the scalar here. and then you have a kind of scalar product between the sigma vector Rho which is this and this so you take these two matrices in our like a row like this bear in mind these are two by two matrices so you make like this and multiply by this which is Q and this is Sigma you get a scalar product between Sigma and Q sorry well it's a kind of scalar product because the result is as you can see the matrix but looks like a scalar product so this is the global I mean the total Sigma matrix and this one anyway you get Sigma Q why did I write it in this form because if you now go into the momentum operator representation in which the wave vector is replaced by main minus J nabla you know this so the wave vector in momentum operator representation can be replaced by minus J nabla so again if you are from electronics you probably have no idea what I'm talking about but it's not important you don't have to really understand this fully the point is you're going to get you're going to get this Hamiltonian here so they give you simply instead of Q instead of Q you put minus J nabla here and you get here minus J in front and you get nabla J remember is the imaginary you you get this so why did I do all this because again this is again very deep into quantum physics there is something which is called Dirac-Weil Hamiltonian in quantum relativistic physics it's called Dirac-Weil Hamiltonian in the Hamiltonian which describes relativistic massless particles it has this shape here it's minus J H bar remember H bar is Dirac constant so it's H divided by 2 pi times C which is speed of light because the particles are relativistic time the product between Sigma and nabla and now if you compare these two Hamiltonians you basically see that Hamiltonian V got has exactly the same shape as Dirac whale Hamiltonian which means what which means that the particles of the carrier when they are close to the Dirac point they behave as massless Dirac fermions relativistic massless Dirac fermions it basically means that the carriers in graphene close to the Dirac point behave as relativistic particles and that's very very interesting that's not what you have in conventional semiconductors of course there is a one small difference they behave as relativistic particles but they do not travel at speed of light because there is no speed of light here in the original Hamiltonian we got. So now I made some change in the slides yesterday evening, so I uploaded the new version. So if you have the old version of the slides, maybe you see this written in slightly different form, but it doesn't matter. You should always upload the newest version. So basically what you do is you say like this. they have the same shape I have minus j minus j some scalar times sigma nabla sigma nabla okay so that means I can rewrite this Hamiltonian in this shape like minus j h bar some velocity not really speed of light times sigma nabla like this and that means that this scalar which multiplies sigma nabla which is minus j h bar vf is equal to this scalar here minus j 3 over 2 a sigma sorry gamma and from that after you cancel out minus j j and divide everything by h bar you get the equivalent relativistic speed of particles close to the dirac point which i denoted by Vf which corresponds to Fermi velocity and it's equal to this value here, 3a gamma divided by 2 h bar. When you put the numbers it turns out to be something like 300 times smaller than the speed of light. The basic conclusion is that due to the symmetry of graphene lattice when the carrier is close to the Dirac point it behaves as relativistic massless particle traveling at Fermi velocity Vf, which is 300 times smaller than the speed of light. The fact that the velocity is smaller, it's not some surprise, but what is really surprised is that we got relativistic band structure, basically. That's very interesting. That means that the graphene is a very simple system for investigating relativistic phenomena in a solid-state device which you can put on a tabletop. you don't need a large Hadron Collider to do it basically to accelerate anything at speed of light of course in a large Hadron Collider whatever accelerator you use you can never reach of course the speed of light but basically this tells you that carriers behave as relativistic massless particles that's it they just don't travel at speed of light they have this velocity here but that happens only where Hamiltonian has this place which means it happens only here around the k point where this Taylor expansion is valid if you go further away this breaks down of course but if you stay here where the Fermi level is this is the k point then this is okay and okay so this is a diversion with this Dirac-Vale Hamiltonian maybe I don't know maybe I should have done it I don't know because this is actually much easier to see through the expression of for energy and before I do it okay so you can just also rewrite now this expression if you want this is not really important if you want you can now rewrite this expression because you can now eliminate this 3a gamma over 2 you see 3a gamma over 2 is equal to h bar VF so this is just h bar R V F you can write it down like this okay that's not important so let's go back to energy so this is maybe more important so let's calculate the energy finally what's exactly the band structure here so what's the band structure here close to the Dirac point I already told you right plus minus gamma magnitude F okay plus minus gamma magnitude F is the magnitude of this function at top so it's 3 over 2 a magnitude of this complex number magnitude of this complex number is the same as the magnitude of the two vectors because magnitude of this is the square root of 2x squared plus Q Y squared but this is also the magnitude of the Q vector which is Q X Q Y that's why this can be written in this form here okay and now finally what we can do again to get rid of this 3a gamma over 2 which is Vf times h bar 3a gamma over 2 so this is Vf h bar is this and the magnitude of the Q vector I just wrote like a scalar Q okay and finally h bar Q is what h bar times wave vector is momentum right you get plus minus Vfp and And now you can understand why did I say massless Dirac fermions. You remember this expression from relativistic physics. So this is the band structure in relativistic physics. In case of the Dirac-Weil Hamiltonian, this will just turn out to this because Dirac-Weil Hamiltonian is valid for massless relativistic particles for which the rest mass is zero. You just get plus minus speed of light momentum. Compared to this, exactly the same. Plus minus F momentum. That's it. That's maybe easier to understand why this corresponds to the relativistic massless particles rather than using Dirac-Wehl-Hamiltonian. I mean, I showed you also why a Dirac-Wehl-Hamiltonian, because I know there are physicists here, but that's not something I'm going to ask at the exam because I know there are people from electronics here. So the worst thing you can get at the exam, for instance, is I'm going to give you, if I really would like to ask about Dirac-Fehl Hamiltonian, I'm going to give you these two matrices, and then I'm going to ask you, please show me these matrices are like this. What original Hamiltonian will turn into? And then you derive this expression, that's it. But I think for the general audience, this is more understandable, because probably everyone knows this expression from the relativistic physics you put mass 0 you get basically exactly the same you get that this corresponds to relativistic massless fermions and until recently until recently people thought that such one such particle is which one which other particle apart carriers in graphene obeys Dirac whale Hamiltonian neutrinos but they found recently they have a mass it's not anymore because in neutrinos they found people for very long time thought that they don't have a mass but it turned out they have very very small mass so basically is the same expression just with this mass extremely extremely small so basically pretty much plus minus CP so you can say that carriers in graphene are very very similar to neutrinos but now it turns out neutrinos have mass these are the recent experiments which confirm they have very small tiny mass but they still have it okay that's why like ten years ago people said these carriers are like neutrinos they are not it turns out neutrinos have a mass And I like just to point another thing. Sometimes you hear people say, errors in graphing do not obey Schrodinger equation, they obey relativistic equation. That's not true and true. Why? Because this equation we actually got from Schrodinger equation, right? We got it from Schrodinger equation and Taylor expansion. So it's not true that they obey some relativistic equation rather than Schrodinger equation. This comes from Schrodinger equation, so that's partially true and partially not true, right? Okay. And finally, if you now zoom in this area close to the Dirac point, what's actually the shape of this? Now, if you know relativistic physics, you know what the shape is. But if you don't know relativistic physics, you can very easily guess it. what is the what is geometrical shape of the conduction band here and the valence band here what is this geometrical shape here well it's written on the slide it's a cone and how do you know it's a cone because the equation of this for instance if you look at the conduction band what's the equation of the conduction band close to the direct point it's plus VFP so you can write down close to the conduction band so for the electrons is this but because this is Q space then it's maybe better to write momentum simply like as I already wrote h bar Q it's a constant time Q when you plot in the coordinate system Q X Q Y this function what do you get you get a cone why what's the easiest way to plot a 3d function plot a surface in space what's the easiest you simply say this is equal to some constant if you say that this is equal to constant you're going to get what you're going to get the equation of the line which is the intersection between this surface and plane whose equation is constant you understand so basically you take a plane equation is that equal constant actually it's not that but okay in the third okay energy right so the third axis is energy e so energy e which is that axis right is equal constant and this plane makes intersection with your surface and the result is the now what I said is the intersection between your surface and the plane so when the energy is zero when the energy is zero zero here that means when the constant is zero q is zero what is q equals zero is the equation of this point here when the constant goes up if you have a larger constant you will get equation new equation of the curve which is constant divided by VFH bar right what's the equation in QX KQI plane of this sorry what what the curve corresponds to this equation there's the equation of circle right because this is the magnitude of the QX QI vector this is the equation of the circle between when you start plotting are the K point you just get zero you increase constant you have a plane which is now above zero and in this plane you have a circle right and then if you increase the constant in the next plane because the constant is larger you have a larger radius you get this and then this and this and that's why okay maybe this is not the best way to draw it something we and that's why the surface this is surface of our cone okay that's how you understand that this is a cone so with the plus sign we get positive cone which is the conduction band with the negative sign we got valence band we got this blue cone these are so called Dirac cone because this is cone with the node of the Dirac point okay so now you understand the band structure of graphene close to the Fermi level we just have cones if you go far away then you know the cone will then expand and you get this very like a hat like a Mexican hat shaped structure but that's not important because it's far away from the Dirac point okay and finally you remember I told you that one of the main properties of graphene it's extremely high carrier mobility comes from its wave function from the specific shape of its wave function and that's why I'm going to also calculate the wave function close to the Dirac point so if you remember this is the expression for the wave function idea already derived remember 1 over square root of 2 times this vector here and now we can find out how much is this close to the Dirac point basically what you have to do is to take the expression for the function f of K and insert here you put here f star divided by the magnitude here it is this is the f star which is the complex conjugated value right notice 3 over 2 a cancels out because you have it both in the numerator and the numerator here you just get the conjugate of this which is minus Q X minus J Q Y divided by magnitude of this and now if you take this minus in front you will get plus minus so that's why you get plus minus here you get Q X plus J Q Y and here the magnitude of this complex number is the same as the magnitude of this complex number, that doesn't matter. The reason why I wrote it like this is because I already have this complex number in the numerator, right? So, when you have a complex number divided by its magnitude, what you get just is the phase factor. You get e to the j theta q, where theta q is the angle between the vector and the horizontal axis, okay? So, that's the final expression for the wave function that I'm going to use later to show you why graphing has such enormous mobility okay fine homework repeat everything around K prime point I'm going to skip this just now it's clear you take the expression for the function close to the K prime point and you go through all the same steps notice that here you are not going to get the identical expression as I got before but this is just a different shape of the Dirac-Weil-Hamiltonian which I already told you it's not important so for if you really don't like it you don't have to really look at this part just derive the expression for energy that's the most important thing we need and if you derive the expression for energy you will of course got get the same expression because of the symmetry plus minus VF times H bar Q or plus minus F plus minus VF times momentum okay and then similarly you can get a similar expression for the wave function but going to skip it.
\fig{29}{Nanoelectronics of graphene and related 2D materials 2024}
\fig{30}{Nanoelectronics of graphene and related 2D materials 2024}
And then another homework if you want not go out to this or the exam so it's just for your exercise you can try now to repeat everything from the very beginning in which you swap names of the lattices B and A to see what's going to happen if the blue atom is atom B and not also me you can try to do it you will actually get very similar expressions but not again identical so the Hamiltonian is going to be slightly different but at the end energy is going to be the same that's the only important thing you're going to get at the end plus minus VFP or plus minus VFH bar you close to the Dirac point okay is this okay fine so now we know the band structure and now if we know the band structure we can now finally try to understand the physical properties of graphing however...
\fig{31}{Nanoelectronics of graphene and related 2D materials 2024}
There is always however before I do it I need to find first density of states in graphene why do I need it because for some further calculations I'm going to do later like to find the carrier density or something like this I need the density of states so that's the last thing last heavy physics based thing that's going to do it because they need it for calculations and then it's going to become more interesting I mean we are just going to apply what we got so what What is the density of states? Now of course, physicists know it, but again I have to go through it very quickly trying to explain. So if you look at the energy scale, because we've just found the bench structure of graphene, okay, if you look at the energy scale, the question is now, at certain energy, e within the interval so between e and e plus d so in this small infinitesimally small interval of energy how many electron states do we have so why do i need this if you would like to concentrate the carrier density in material you need this because you have to go to the band structure and basically sum up all electrons available within the band Right? Or within the range of energy you're interested in. But before you come up to the point of summing up all electrons in the certain energy range, you first have to find how many electrons you have in infinitesimally small range. And then based on that, you can integrate and get the full value. Okay? So that's related to something that in physics is called density of states. it basically tells you what the density of states basically how many states do you have in this small energy interval here between e and e plus d e and how density of states or dose is defined is defined like this you say density of states is something when i when multiply by this infinitesimally small range d, I get the number of electrons inside this range here divided by the, because this is 2D material graphene, divided by the surface area of graphene. Why divide it by surface area of graphene? To make the value independent of the size of the sheet. Because we never talk about the number of electrons. We always talk about concentration of electrons. Because when you multiply concentration or density by surface area, you get how many electrons you have in total. And that's why always when you calculate the number of electrons, you actually calculate the density. And then if you really need to know how many are there, you will just multiply density with the surface area. So that's why the density of states is defined. You multiply it by this small range here, and that will give you exactly how many electrons you have here. Of course, normalize with respect to the size of the sample. Because the larger the sample, the more electrons you have. How do I know this? Because when the atoms join the crystal, each state gets split into a band but the number of sub states in a band is equal to the number of atoms you join the more atoms you have the larger the surface area of the sample in this case you will get more electrons so that's why it scales with the surface area right and that's why we have to divide by surface area to get density so that's why we have this here and of course in case of 3d materials you will divide by the volume in case of 1d materials like carbon nanotube you will divide by the length but graphene is 2d material so we define we divide by the surface area which means that I'm going to assume that graphene my graphene sheet looks like this XY okay XY here is that which is not important because this is to the system so here along X we have size LX here we have ly so the surface area a is simply LX ly however notice one thing sorry I put here the absolute value y I didn't put the I put the absolute value the reason I put it is because the value we have on the right hand side is always positive this is defined as the number of electrons it's a positive number or or holes it doesn't matter so the number of carriers divided by surface area which is positive which means that the left-hand side must be positive density of states is always defined to be positive so that's why to get a sign matching this D must be positive so it's just a warning be careful the range that that multiplies the density of states must be positive, it cannot be negative. You will understand very soon why this is required to put. Okay, that's it. But now the question is, fine, but how do I calculate this? How do I know how many states I have here in this energy band D? This is not the solid state physics you already learned, right? I'm going to go a bit quicker through this. So the point is, you have to count the states, but you cannot count the states directly here. The place where you can count the states is the k space, because the k vector is the one which gives you the number of states. You remember, we have two quantum numbers, kx and ky. Why is no one interrupting me? because I forgot to start the timer and you guys just keep silent and I just keep going and going again okay let's make a break 15 minutes okay we will start at seven minutes past half past four okay okay so but what we know is how to count states in the case space that's what we know because in the case space we have two quantum numbers K X Y those numbers count the state along the X and Y axis which means that the quant along the x-axis is the distance between two nearest states along the x-axis or kx-axis which is in this direction and the quant, the smallest distance along the x-axis is simply, you know from quantum physics, is 2 pi over Lx where Lx is the size of the sample in the x direction and the same for the quant along the y direction so basically if this is 2 pi over Lx and this is 2 pi over Ly and this is the interior again I didn't draw the entire hexagon if this is the interior of the first Brillouin zone the only possible states are those green dots you start from gamma point and going quantum to the right quantum to the top left down and so on and only available only allowed states are those with green dots now if you remember my original plot of the first Boulogne zone I put the entire hexagon to be green and the reason for that is that we are considering here microscopically large graphene sheets which means that Lx and Ly are rather large and because of that these quanta these quants 2 pi over Lx and 2 pi over Ly are very very small so basically when you look at this plot you should basically just see bunch of green pole points pretty much filling the entire pretty much continuously filling the entire blue one zone you know this but anyway to be more specific this is the actual quantization right and that means that what I can identify now I cannot tell right now how many states I have here in this range D but what I can tell you immediately is how many k states I have inside the first Boulogne zone how because I know that the surface area which corresponds to this small rectangle this gray rectangle here corresponds to a single state so I have in the first Boulogne zone I have one state per this small gray rectangle how do I know this because in the corners I have states but because of the neighboring rectangles I have a situation that each of these corners belong to the rectangle with quarter of the value we have four times one over four is one state you can one allowed k state within this surface area so basically in order to understand how many states they have here D I have to somehow associate this energy interval d e to certain surface area in this space and then I will know how many states I have because I know how many states I have for this small gray rectangle so let me show you this so we have to somehow do this association and how are we going to do it okay so the first step so let's draw again the Q coordinate system so I'm going to put the coordinate system of the k point this is okay because they calculated the band structure with respect to this coordinate system and then I'm going to write down the expression for the energy close to the Dirac point because that's what I'm all interested in because my electrons are probably somewhere here Fermi level is somewhere there so I know that my energy is already showed you plus minus h bar VFQ okay and now if you plot this this is what you get if you plot the energy as a function of the Q which is the magnitude of the Q vector magnitude of the vector Q and Qy this is how it looks like plus h bar vs Q and minus h bar vs Q so this is the energy spectrum right so this is like a part of the cone that we already showed you before for the electrons and the blue is for holes and now you see I have finally connection between this and this between these two plots because now here I have Q I have it also here okay so that means if I have energy E this energy if I fix this energy E I know that this energy corresponds to a certain Q vector whose magnitude is here and I also know based on this plot that this energy interval this range d corresponds to a certain range dq here ok and now you understand why I put here the absolute value because I've been showing you so far only for the electrons but you can actually calculate for the whole but if you calculate for the whole notice that because this function is decreasing function dE is actually negative you understand so that's why you have to put here the absolute value otherwise you will get that the number of states is negative which is stupid because the number of states is just the number we assume to be positive we just count how many states do we have so that's the reason why you have to put here absolute value you put absolute value to be absolutely sure that whether you look at the electrons or holes you're always going to get positive numbers so now you understand the reason Okay, and now we just have to associate somehow this energy E to this value Q here and this band here D to this infinitesimally small range of Q which is denoted by DQ. How are we going to do it? Well, very simple. We have the band structure. you say like this if I select energy E if I fix this energy E I fix the certain magnitude of the momentum vector so if this is vector Q for instance and if this vector Q has magnitude which corresponds to this value Q here so that's why I denoted it with Q of P so for this energy e we have this Q which is this magnitude Q then what are all Q vectors which correspond to the same magnitude Q where they placed in space here along this circle because for any state along the circle the magnitude doesn't change so that means that for this energy E, this is the corresponding Q and the corresponding Q vectors are vectors which start here and end up somewhere on this circle, okay. Now let's associate it with infinitesimally small energy dE. I'm looking for the energy range from E to E plus dE, which means I'm looking for the range of the magnitude Q from Q to Q plus dQ where dQ corresponds to dE okay so if this is Q and if I draw another circle to form a ring with the width dQ where this dQ is this value here then all Q vectors with the magnitude from Q to Q plus DQ are vectors which have a beginning here and end up somewhere inside this ring here because then they will have the magnitude between Q and Q plus DQ so all these vectors which end up with the end somewhere inside this orange ring that's it and now I can calculate it why because I can very easily say now how many states do I have here why because the number of states here is the same as the number of states here but the number of states here is the number of states inside this orange ring they have to calculate the surface area of the orange ring which is very simple there's the circumference of the inner circle multiplied by the width of the ring and then if I mal if I divide this surface area by the surface area which corresponds to a single state I got how many states I have within the orange ring which is going to be the same number of states within this energy range from e to plus e plus d okay and now you get that therefore the number of possible Q states is simply the surface area of the orange ring which is denoted by the Omega divided by the surface area of this gray rectangle which is 2 pi over LX times 2 pi over L Y and then when you put LX L Y to be a right this is what you get it is a DC the omega divided by 4 pi squared that's it and then finally because the omega is 2 Q pi D Q if you now replace instead of the omega 2 Q pi D Q 2 pi and 2 pi cancel out you get just 2 pi you have a times Q D Q divided by 2 pi that's the number of yes now the first B1 zone has a hexagonal symmetry but the K vector is just quantized number, kx is just quantized, zero, so this is quantum mechanics. You have the, so for the kx possible values are zero, delta kx which is 2 pi over Lx, two times delta kx, three times and then also minus delta kx minus two delta kx and so on this is the quantum you just it's just quantized the same for ky so that's why all these points are like this but hexagonal symmetry comes from the shape of the first Boulogne zone which is hexagon you should only look inside the hexagon okay and that's it that's the number of Q states inside the ring which is the same as the number of states here because this energy band D corresponds to this ring here the only catch is that this is the number of Q states not the number of electrons or holes so we have to go from the states to the holes or electrons how do we do it that's this step here you simply see the number of for instance electrons or holes is what in each Q state you can get two electrons or holes spin up spin down you have so called spin degeneracy two so to get the number of possible electrons of holes you have to multiply by two okay and then also you have a valley degeneracy which is now the specifics of graphene why because if the electron or hole is close to the k-point you should also consider the other non equivalent point which is k prime because if you know the energy of the electron to be close to the Fermi level level this electron could be either close to the k point or a prime point you have two possibilities electron close to K or electron close to K prime they have twice the number of electrons because of the specifics of the band structure of graphene they have the so called valley degeneracy which means we can multiply by two to account for two valleys one in the sense of in the Fermi level when you fill the state in the band structure of graphene with electrons they got filled not only close to the k point but also close to the other non-equivalent point which is k prime we have the twice the number in the single value so that's why you have to multiply this d and q with spin degeneracy which is 2 and valley degeneracy which is 2 and basically means you multiply this by 4 or you multiply this by 4, 4 over 2 is 2 and you get this expression here so this is finally the number of electrons or holes within band D and therefore by definition I already told you when you multiply dot by this value you should get this number divided by a so you take this number divide by a and you get this so therefore the density of state of a 2d system is given by this you simply divide this by D and you get this expression so far this has been valid for any 2d system basically why well the only the only place where I actually use specifics of the graphene band structure was here but they have valid degeneracy too but everything else is basically repetition for a derivation you have for any 2d system I just did it for a case of graphing but you can repeat it for any this system okay now comes the bent sort of graphing that you really need it you get a dot is equal to this but you have some differential you have to get rid of it basically what we have to do now is to find the D over D Q what was this so we have to take the we have to take the band structure of graphene and find the first derivative of energy with respect to Q to get rid of this derivative it's easier to find the energy with respect to Q than the other way around we're going to find reciprocal value of this one okay and at this point when you go to replace this derivative now what really comes into play is the actual band structure of material if we ignore this if you put here value degeneracy for silicon then This is also valid for 2D silicon, let's say. Okay? Fine. So let's do it. If this is the band structure of graphene, then derivative is very simple. Differential of C is what? You have a plus minus H bar VF DQ. That's it. But remember, we need absolute value. So you take the absolute value of this, and you just end up with H VF DQ. and now dq over absolute value of d is 1 over h bar vf which we are going to put here so you get as i said that density of state is finally 2q over pi times 1 over h bar vf and the density of state is here expressed as a function of e on the left hand side but on the right hand side you have Q to get rid of this Q again you take the band structure of graphene and you say the absolute E is H bar VFQ so this Q here is the absolute E divided by H bar VF so our absolute E divided by H bar VF will give you H bar VF squared and in the front you have 2 over pi that's the density of states of graphene It looks like this when you plot it as a function of energy. So this is basically absolute E function. It looks like this. How does it look in case of the two-dimensional conventional semiconductor? What's the density of state in case of two-dimensional semiconductor? Let's do it quickly. if you don't know it by heart E is proportional to Q squared and you take derivative it's Q Q over Q 1 it's constant in conventional 2D semiconductors you get a constant density of state the density of state which looks like this just a flat line that's a case of the conventional 2D semiconductor from here you can see one problem with graphene which one from here you can see one problem with graphene at Fermi level which is a zero Kelvin here density of state is zero you basically have no state that's a pretty big problem in electronics. That means that, okay, of course, if you go to higher temperatures, you will get transport around the Fermi level, also due to the doping, Fermi level can shift from zero, but basically is going to stay around here, where the density of states is very low. That means you don't have so many states and that's the reason why it's very hard to get a low contact resistance with graphene, because the low cost of what is the contact resistance and you have a graph in transistor for instance there but you have any transistor you have to make contact the source and drain right and these contacts are typically made of metal contact between the metal and semiconductor you have to make sure of course that contact is omic not Schottky because if you get the Schottky you block transistor you get the two diodes back-to-back which are going to block transistor but if you get omic contact if you manage to get omic you still have contact between two different materials and that gives rise due to the different band structure to something which is called contact resistance so whenever you make a contact between two different material there is a resistance associated with the difference in their band structure which is called contact resistance the point in electronics is always to make this contact resistance as small as possible why because this is parasitic resistance connected in serious to source on one side and drain to the other and basically deteriorates transistor properties because that means that in reality if you have a transistor the realistic transistor like nMOS will be actually drawn like this this is the actual drain and this is the actual source and this is the contact resistance of the drain so the contact resistance between the drain metal and the drain material and this is the contact resistance RS the resistance between okay this is gate between the metal contact and the actual material of the source source material and these two resistances are parasitic resistances connected in series to the transistor. In order to get rid of them, you would like to reduce them to a smallest possible value. Let me give you immediately the typical value of electronics. This is mostly critical in high frequency electronics because if you have it here, it's going to reduce the bandwidth. You know that bandwidth is always somehow inversely proportional to RC constant of the circuit, right? Because you have to charge discharge capacitances and that goes through resistances that gives rise to the time constant which actually limits the bandwidth of a device. And this contact resistance is therefore must be as small as possible to have a highest possible bandwidth. For high frequency transistors, this typical value, so let's denote it by RC, which is the contact resistance must be, it's typically expressed as multiplied by the channel width, must be around 50 ohm micron, which means if you have a resistor channel which is one micron wide, the contact resistance with this channel should be 50 ohms at most, not more than that, which means that if you want to reduce the contact resistance, you can for instance make the channel 10 times wider, like 10 micron and then the contact resistance is going to be, you have to divide by the channel width, will be 5 ohms you have a great carrier mobility which allows you to have a high bandwidth but actually low density of state reduces bandwidth because contact resistance is high you know and this value in graphene 200 ohm micron if you make pretty old contacts with graphene because gold turned out to be the best to make low contact resistance with graphene and how is this related to the density of state well it's very simple because the contact resistance come from the dissimilar band structure of two material you have a graphene and metal basically in order to reduce the contact resistance you must be able to inject as many carriers as possible from graphene channel to the metal okay but if your density of state is low you have not so many carriers to inject and that's the reason why you have a relatively high contact resistance with graphene it's basically limited by the density of state understand that's why this kind of density of state looks very cool it's like a linear it's not constant like other 2d materials but that's actually very bad for electronics they have to really complex stuff with the contact if you want to go below 200 ohm microns so So we, for instance, manage in our devices to go down to about 25 ohm microns. That was very complicated stuff, we had to actually drill holes in graphene below the contact, so we created edges in graphene which then promoted the injection into the metal because of the edge state. It's possible to do, it's possible to reduce, but this is really making a problem, this density of states. so now you already understand how a physical property of material have detrimental impact on the transistor operation this is one example so from that point of view if graphene were to have the constant dose like other materials 2d would be much better because then you would have here a higher density of state of course bear in mind that the graphene is actually better if you go far away from the Fermi level because then those goes up and at some point can exceed that of the other 2d semiconductors but for that you need really to push Fermi level far away from the Fermi from the from the origin here and that means you have to apply very large gate source voltage if you want to do it you have to go really far away from the from this point okay fine.
\fig{32}{Nanoelectronics of graphene and related 2D materials 2024}
And now if this is clear since I'm already comparing graphene to other 2d materials so let me do a full comparison I just compare it now in terms of the density of state now you understand that unfortunately this relativistic band structure of graphene has a negative consequence of the on the this let's try to globally now compare because we know the band structure of graphene so now we more let's globally compare now graphene with other 2d semiconductors to understand what we can already guess from the band structure okay so let's start with the comparison so the first thing is this one I put here two columns the column on the left is graphene and its properties the column on the right is some other two-dimensional semiconductor okay so let's start with the energy we have so the band structure we already got for graphene is plus minus VF h bar u okay in case of the in case of the other two the semiconductors we have like this it's quadratic you remember this is again close to the bottom okay I have a pointer this is close to the bottom of the this is valid only close to the bottom of the conduction band or top of the valence band okay so it is the energy this is the conduction band or valence band edge and then plus minus HQ squared divided by two times effective mass either of electrons or holes the sine plus is for the electron so we have C conduction band and we have E electrons effective mass of electrons and for sine minus we have a valence band and holes okay so that's the band structure which one is better for electronic applications now the one on the left looks cool right linear spectrum like relativistic particles but when you make transistor do you really care about the relativistic physics I mean you really want to make transistor to work well whether it's relativistic or not so from the point of view of a transistor which band structure is better the one on the left or the one on the right the one on the right why because on the right we have a band gap that's it EC minus EV is EG is the band gap conventional 2d semiconductors well semiconductors they do have a band gap EG which is EC minus EV you know this very well and that's what you need to make a good transistor this is bad it has no bang gap basically you can of course by the gate source voltage when you make a graph in transistor you can change the you can change the drain current but you can't turn it off you can't set it down to zero you can apply whatever gate source voltage you want it never goes to zero because there is no way to put the state to align with the bang gap where the carriers cannot travel through that's the problem so one zero semiconductors as expected okay band structure on the left symmetric on the right asymmetric why because as you know in conventional semiconductors you rarely get symmetry it's always that electrons have a different band structure than holes so which one is better graphene or semiconductors point for graphing because it's symmetric because that means as I already told you you can use the same layout for N and P type transistor it's going to work the same no difference these are going to be different because of the asymmetry of the band structure you will get different effective mass and therefore different carrier mobility which is inversely proportional to the effective mass and because of that as I already told you P type transistor made of silicon should be wider than that of N otherwise you don't match the currents you don't match the curves so one one okay density of states I already explained you the one on the right is better because close to the Fermi level you have a constant density of states so you get a reasonably small contact resistance with graphene you have to go on the an extra mile you have to do extra processing to somehow manage to reduce the contact resistance to compensate for the loss of the density of states basically as i told you one way is to edge graphene because the edge state introduces additional density of state of the edges and that can improve injection from the metal to the graphene so i would say two one semiconductors because this is better then in terms of electric mass we can't compare it why because as defined electric mass here which I put here in the band structure such effective mass obviously doesn't exist here it's nowhere to be seen why because I already told you that this effective mass this band structure this is nothing but a Taylor expansion to the second term and this effective mass is basically this if you put one over effective mass you will get this second derivative in the numerator this is basically coefficient of the Taylor series you know this of the second order graphene spectrum around the Dirac point is linear and it doesn't have a it doesn't have a second term so you can't define the effective mass in this in that way okay so that's why this comparison is meaningless because there is no effective mass here and sometimes you will find in the books people say well would it be meaningless effective mass in graphene is zero why well because the graphene is the massless zero you have carriers like massless Dirac fermions well I wouldn't call it like that because this zero effective mass if you would really think that this is effective mass then you would have problems to understand like cyclotron effect in graphene because you know that the radius of cyclotron orbit in graphene in any material is somehow related to the mass of the carrier right so that means if you have a zero mass then what's the radius what does it mean we cannot do we cannot find the cyclotron orbit in graphene and the answer is no of course I will show you how it's done so basically that's why sometimes people say zero because they talk about massless Dirac fermions but you can just ignore this it doesn't have this you don't see this zero physically anywhere apart from the band structure it doesn't show up realistically anywhere and sometimes people say it's infinite because if you take derivative and derivative is zero and you take reciprocal value you get infinity so it's even it's even more stupid than to say so I just say just remember forget about this it's meaningless definition because it comes from the second derivative which doesn't exist in case of graphene so here we can't compare it but then comes to the point of this experiment I said if I subject the ruff into magnetic field and by the way why is this important because if you subject the ruff into magnetic field then you will get half integer quantum Hall effect I measured mentioned before so if you subject it how do you then find the cyclotron orbit what is then the mass which is involved in this motion and And then you can, so this is now physics, you can calculate first the group velocity, which is defined like this in physics. If you apply to this band structure here, I put now derivation just for the electrons. In case of graphene, you will just get that the group velocity is Fermi velocity, very easy, because the derivative of this is just Vf h bar. When you divide by h bar, you just get Vf. In case of the 2D semiconductors, and of course this is just for electrons, you will get this value, and then the cyclotron effective mass, which is defined like this, will turn into a regular effective mass for the semiconductors, but for graphene, it will turn into this. So it is finite, it exists, you see. So that means in the presence of the magnetic field, carriers in graphene will also make cyclotron orbits. It's just that the effective cyclotron mass is different, because it's not this one, obviously, this one doesn't exist in graphene. but okay this is more physics not much not so much related to electronics because I mean you don't use magnetic field in electronics right so forget about this so still 2-1 for semiconductors and then finally the last one carrier mobility so the carrier mobility of graphene is enormous at room temperature more than hundred thousand square centimeter per volt second to these semiconductors not even close to this these values I put here these are for bulk I will correct this these are bulk values 1500 that's n-type silicon bulk high fields drop to 500 very thin silicon thinner than 2 nanometer drops below 100 so very very low this by the way 77,000 that's pointless that's indium tymonite you can get higher mobilities in three fives I already told you even higher than that but not at room temperature in HEMS you have it only at very low temperatures so basically this is a clear advantage for graphene it has enormous carrier mobility so 2-2 basically a draw it means what graphene has a potential but you have to understand where to use it you can't use it everywhere because of the absence of the band gap you're not going to use it in digital electronics but but because it has a huge mobility you can try to use it in analog electronics and also there are other possible applications of graphene which are not present here I didn't put them in this table because they actually don't come directly from the band structure but come from the from the from the graphene from from the fact the graphene is 2d material which chart first flexibility graphene is rather flexible material why because in plane bonds are very strong sigma-coulomb bonds extremely strong so you can basically flex graphene you can make flexible devices with graphene and also you can make transparent devices with graphene because graphene is only one atom peak and therefore it has a transparency above 97\% so basically dropping transistor can be made to be both flexible and transparent so that's very attractive for flexible and transparent electronics so that's something where conventional semiconductors can't go why try to flex silicon you know what's going to happen it's like a piece of glass this is going to break immediately. No way. Is it transparent? No, it's opaque. That's it. So there is no competition basically if you go to the field of transparent or flexible electronics. Conventional semiconductors can't compete. You compete basically with polymer materials, organic polymers for instance. That's the competition. But you know what's the mobility in organic polymer semiconductors? One. draw and they get 10 they're extremely happy so that basically that's a clearly arena in which graphing should compete with so basically draw means find a way how to compete with how to find the correct application for material don't use it for something which is not good you use it for something which is good and by the way now just we have a half a minute just to tell you when graphene was discovered everyone talked about transparent and flexible mobile phones. Nokia invested some money in this in developing this kind of stuff because everyone thought this is going to be the future like to make a phone to be transparent and flexible can you imagine and but the only problem is of course if you want to make this transparent and flexible you have to make everything transparent and flexible you can't make battery transparent and flexible this is the biggest problem you can maybe make transistor channel but not everything so yeah this is nice promise but you have to then make everything transparent and flexible right and you have some flip phones so we are going in this direction but transparency very hard I actually don't know whether it's required or not I usually have problem I lose my phone have a problem to find it if it's transparent I will never be able to find.