\fig{33}{Nanoelectronics of graphene and related 2D materials 2024}
Okay, so we finished with the plant structure graphing, and now it's time to discuss some physical properties before we switch to the challenge. So one of the main important properties of graphene is very high carrier mobility and now it's time to explain where does it come from. So from one point of view it's easy to understand why graphene should take high carrier mobility. If you take it outside of graphite you get the perfect crystal, right? Why? Because the covalent bonds in, because the in-plane covalent bonds are very strong, So they're very hard to break, and that's why when you get graphene outside of graphite, you get crystal lattice without any defects. And as you know, defects are one of the main reasons for reduction of carrier mobilic in crystals. So that's the first reason, because this is the standard column bond known to us, and therefore it's very hard to make defects in such lattice, and you get a high carrier mobilic. However, it turns out that there is another reason for high carrier mobility. Actually, as you understand already, because grafting is just a 2D material, if you want to make a device, like for instance electronic device to measure carrier mobility like a whole bag, you have to place grafting on something. And then you place it on substrate, and then it could be very much influenced by the substrate, because grafting does not have a valve. and therefore everything happens on this surface, so if you place it on a substrate, it will be influenced for instance from charged drops in the substrate, surface roughness of the substrate and so on. So therefore, even though graphene has a, you may expect because of its perfect crystal structure, that it has a very high carrier mobility, the problem is that its environment has a detrimental impact on its carrier mobility. However, even despite all this, the carrier mobility is very large. So for instance, I told you, if graphene is not influenced by the substrate or environment, for instance, if you encapsulate it in HBN, so you make HBN graphene, HBN sand, which you are going to get from temperature mobility in excess of 100,000 cm2 per volt-second, which is enormous. But if you place it on a silicon substrate, like silicon dioxide substrate, which is full of defects because it's a morpheus surface, full of travel straps. Of course, the carriers in Jaffin are going to scatter on those defects. But nevertheless, if you measure carrier mobility, it's going to drop to something like 4-5 thousand. But that's still much more than that in silicon. silicon is only 1500 in bulk, 500 in high field, less than 100 when silicon becomes below 2 nm. Therefore, there must be something else in addition which actually makes graphene to have such a very high mobility. And the main reason is actually it's a strange wave function, which is the reason I actually derived this question for the wave function last time. So in order to understand this, we should look at the transition from some initial state denoted by i from the initial to some final state denoted by f from final and calculate probability per unit time for a carrier to jump from the initial to the final state. Why? Because if you calculate this probability, this probability must somehow be related to be the carrier scattering right because if carrier jumps from the initial to the final state due to some scattering obviously the more intense the scattering when the carrier jumps from initial to the final state the lower the mobility the more scattering you have, the smaller the mobility. And obviously, if you would like to have a very high carrier mobility, it means that this probability of scattering from some initial to some final state must be very, very small. And if you remember when we discussed the derivation of the time structure of graphing, I told you there is one scattering matrix element, if you remember, with Hamiltonian in the middle, which I told you at that time we called it hopping energy, if you remember, but now you can see how this energy is actually related to the scattering probability. And that's given by this expression here, which is so-called Fermi's golden rule. So that's something which comes from quantum mechanics, I'm not going to go into detail on that. It's just important to note that here inside you will find out that the rate of this probability probability per unit time is proportional to the square of this transition matrix element, and this element is integral of this one. So basically what you would like to have is to minimize this one, where this Hamiltonian in the middle is this scattering Hamiltonian. So what these equations here describe is the probability basically of scattering on some scattering potential given by scattering Hamiltonian from some initial state to some final state. What we would like to have is that this expression here is as small as possible because that would mean that the mobility of scattering is very small and that's the video 5 k-ray mobility. So basically what we have to do here in case of graphing is to calculate this transition matrix element here which gives us from which we can then calculate the scattering probability per unit time. And of course we are not going to calculate it just everywhere, but we are going to calculate it next to the Dirac point. Why? Because we already discussed last time that everything happens around Dirac point. There is no reason to use this complete expression for the band structure of Gapin, we can just restrict ourselves to the expression for the band structure close to the Dirac point, and if it's close to the Dirac point, then we can easily find out the expression for the wave function, which is this bra here, sorry, this cat here, and this cat here is given by this. So what we have here on the right hand side is the eigen function we found by solving the Hamiltonian. You remember that was the vector with two components. So this kind of wave function which we just derived previously is also called spinner wave function because it has like two components, like two states, right? Of course these are not the spin states, but because you have two states, because of the two sublattices, this wave function is simply called spinner wave function. That's the vector we found previously, I'm going to write it very soon, and then of course because here we have also the position vector, you have to add the dependence on the position, but due to the translational symmetry of the lattice, you just have to add the block phase factor, which is E2jkR. Okay, and now let's calculate this transition matrix element. So this is the wave function here, okay, this psi of k, so this one here. psi of k is this vector here you remember we found that normalization gave us one over square root of two and these were the two components of this vector okay and now to get the total wave function we have to multiply it by the block phase factor which is given here so what you see here on the slide this part here that's actually this cat here okay and then in the middle we have the scattering Hamiltonian which is scattering potential if we assume that the carriers are scattered on some potential v then this will be in the entire scattering of the Hamiltonian but in this case that would be the the only component but as you know here we have two components so if you remember the Hamiltonian also was two by two matrix so that's why instead of scalar v we have actually this one, which is nothing but V multiplied by identity matrix of the dimension which corresponds to the wave function. Okay, you can just write it down as V times identity matrix. Okay, and then here we have we just have a brow this which is the permission. So you transpose, so this vector turns into the row, so this vector column turns into the row vector and then you have because this is Hermitian you have to take the complex conjugate you just put minus here and that's how you get it. Notice that here we have the initial state before scattering so that's why you have here i and also here i and here we have the final state after the scattering so here we have f and here we have f okay i hope this is clear right and now uh as i said because this is v times identity matrix you put v in front so one over square root of two one square root of two is one over two you have a v over two okay you have identity matrix when identity matrix multiplies this you just get the same uh you just get the same column so you basically end up with this exponential factor times this one which is given by here so you have e to j ki minus k fr which is this one and then you have this row multiplied by this column so it's one plus minus times plus minus is just plus and you have e to j qi minus ql so this is what you get so this is this uh scattering matrix element which gives you at the end through this formula here probability of scattering on the scattering Hamiltonian from the initial state i to the final state f. Okay and now if you look at this here you can take a half of this phase factor in front of the parentheses and what you get is e to minus this plus e to plus this. Basically you get this expression here and as you know this is two times cosine of this argument without j. So basically you get that this scattering matrix element is proportional to the cosine of the half angle difference between the initial and final state. So that's the only important thing here. I didn't write everything just enough to show that it's proportional to this one. And now why? Well the main reason I wrote it in this form is because you should understand that the main cause of reduction scattering this backscattering. So when the carrier goes into one direction and gets scattered to the final state, if this final state is in the opposite direction, if you're talking about backscattering, so carrier going like this and then backscattering, that's the main cause of the reduction of the carrier mobility-wise, because that means if the carrier goes from source to drain, it will not actually reach drain, it will just this head back and that's the main cause of the reduction of carrier mobility and why is this important well have a look now at this if the carrier goes in certain direction that direction is defined by its wave vector so let's say this is a QI this is the vector close to the of course we are talking about in the gear in the vicinity of the k point so the qi vector points in the direction of the movement of the carrier and then carrier gets back scattered so that means now the final qf vector is in the opposite direction because we have back scattering right and now what's the angle between these two vectors because if you remember these two were angles between the vector and the x-axis therefore the difference of these two, so you should go back, okay, so this is, this angle here is the, this angle here is the one of these, then the difference of these two, the difference of these two in case of the backscattering is simply equal to pi. That's the angle between the vectors, and that means the cosine, you have there cosine of pi half which is zero that's it that means that in graphene backscattering is forbidden it's forbidden because each spin or wave function does not allow backscattering which means in graphene there is no backscattering and that's the main reason why there is a very high carrier mobility in graphene of course you may argue that yeah that's true but but backscattering is not only a source of reduction of mobility. Here, if you for instance have a scattering at the angle of, let's say, 160 degrees, that's going to also reduce the carrier mobility, yes, but then you will have here cosine of 80 degrees, right, which is still pretty small. So basically you have a full suppression of backscattering and partial suppression of all angles which are close to pi and that basically suppresses reduction of carrier mobility which is the main reason for very high carrier mobility in graphing and that's it. That's the reason I derived the expression for the wave function because from here you can understand why graphing has high carrier mobility. For two reasons. First, perfect crystal lattice due to the very strong color and bond and then the second is the spinner wave function which totally suppresses any source of like scattering okay so now I hope this is clear yes no no no yes I mean any scattering reduces mobility but the scattering with the angles close to pi reduces the most. Understand? So for that reason, graphene does not have infinite carrier mobility, but finite, but it's very large. Understand? Because if you try to calculate the, let's say, contribution of each of the angles to the reduction of the carrier mobility, you will find that those with backscattering angle or close to backscattering angle has the highest weight. Okay? So that's why you have the high mobility. It's true that also under other angles you will get reduction of carry mobility, but most of the reduction is contributed to the very high angle flow to the pilot. Okay? Okay? That's the reason. Okay.
\fig{34}{Nanoelectronics of graphene and related 2D materials 2024}
And now I'm going to discuss a couple of more interesting physical properties of graphene. Some of the phenomena can be observed in graphene. And the reason why they are so interesting is simply because of this strange graphene band structure, which corresponds to relativistic malpouse particles, is that the physics of graphene is completely different with respect to conventional semiconductors. And that's why I picked up a few interesting properties which I think we should discuss to simply understand how graphing is different from other materials. So one interesting property I would like to discuss is so-called Klein paradox. So before I explain what is it, let me just discuss what we have in this picture here. So if you look at this picture here, you have a creation of the carrier, in this case, for instance, electron, doesn't have to be electron, but okay, electron impinging on a potential barrier with energy E0. I think in the original version of the slide it was written E0, so I put back E0 because E0 is the energy. You're going to use V4 electric potential to avoid any confusion. You see this is the old version, you have E0. I put E0, okay. So you have a carrier moving to right and impinging on potential barrier of the height e d. Now, in conventional physics, you know what's going to happen, backscattering. Because in the conventional physics, when the carrier faces a potential barrier higher than its energy, it must scatter back, it cannot go through the barrier. In classical physics, that's impossible, you know this. However, of course, we are talking about very small systems, very small objects like carriers, charged carriers, and therefore we should look at quantum physics, not conventional physics. Now in quantum physics you know that in theory a carrier can end up on the other side even if it doesn't have enough energy to go over the barrier, which would be the case in conventional physics, because in conventional physics the only way to end up on the other side is to jump over the barrier, to have energy higher than that of the barrier. However, as you know in quantum physics there's effect called quantum tunneling, which allows carrier to pass through the barrier if the barrier is thin enough and short enough. If these two conditions are satisfied you can get the certain transmission probability for the carrier ending up on the other side of the barrier, like tunneling through the barrier. Hence the name of the effect, tunneling effect. You know that the transmission probability in quantum physics is exponentially dependent, as I said, both on the barrier thickness and barrier height. How? The thinner the barrier, the higher the probability of tunneling, it actually exponentially increases with the thickness of the, it exponentially increases with decreasing barrier thickness. So the thinner the barrier, exponentially you get a higher transmission. And also with the height of the barrier, the shorter the barrier, again exponential increases the durability of tunneling. So that means that in quantum physics, if you increase this E0 to very high value, if you make it very very tall, there will be no tunneling through the barrier even if the barrier is thin because of this exponential dependence on the thickness. However it turns out that in relativistic physics if the barrier is too high actually transmission probability is equal to one which is totally counterintuitive. So in total contrast to quantum physics when the barrier height becomes very high transmission instead of dropping to zero actually increases to one and that phenomenon is called coin paradox i mean it's not paradox but it's called paradox because it's paradox in contrast to quantum physics because also you will expect that transmission goes to zero as the height of the barrier goes up but here in relativistic physics is the other way around. The barrier goes up and the transmission goes to one, which clearly makes no sense, but actually could be understood. Because if you remember the solution of the Dirac equation for relativistic particles, you remember that there are actually two solutions, one with positive energies and another with negative energies, which correspond to positrons. you remember this. So in relativistic physics, the minimum energy of the particle is m0 c squared, where m0 is the rest mass, and these are the possible energies. However, there is another solution down here with negative energies, which corresponds to positrons. So you have electrons here and you have positrons here, with negative energies. Okay? And now how do we explain this phenomenon? Well, in this picture, it's actually easy to explain why. Because, for instance, if this level here, 0, which I denoted by 0, if this is, for instance, M0C squared, that means that we have somewhere here from minus M0C squared, we have positron state. Okay? And now look, if you have a potential barrier which moves everything up, that means this band structure I put here will go up. and the higher the potential barrier the more band structure will move up and at some point positron states will end up here so basically what you get is that transmission is now suddenly possible because electrons can transmit through the barrier over the positron states which are below there, you understand? and that's the explanation for the Klein paradox because for a potential barrier high enough, positron state will end up here in the barrier and basically align with the energy mc squared of the relativistic particle which is strictly panel two of n over these positron states, available states. However, this effect, the other reason why this effect is called paradox because it has never been proven experimentally why. Because unfortunately in quantum, sorry, in relativistic physics, standard relativistic physics, to get this you need, so to push, because this rest energy m0 c squared is very large, you have to have extremely high potential barrier to put positron states here, which requires extremely high electric fields. So you basically We need electric fields in the excess of 10 to the 16 volt per centimeter to be able to get the position states aligned with the state of the electron. That is the first problem. So we need enormous potential to get this. And the other problem is that the barrier thickness must be on the order of Compton length, basically of the wavelength, which you can calculate as the h bar, Vera constant divided by mc. And then if you put these numbers, you get that because of the enormous speed of light, this is extremely short. So we are talking about, well, you can put in the calculator, you get something like one nanometer or something like this, very, very small. So you need, over the very, very short distance, extremely, extremely high potential barrier, which is not possible to practically obtain. And that's why this phenomenon hasn't been observed experimentally until the uping of the theory. And this can be also, why, okay, I think it's recording, right? Guys online, do you hear me? I suppose yes, otherwise you would be complaining already. I'm a bit confused by this, but it's blue, so I suppose it's recording. Okay, maybe they didn't hear me I think it's okay because this one is blue I think it's okay I don't know why it's showing here I think it's E, or the yellow gel. Well, okay. Anyway, so, alternatively you can look at the fine structure constant in quantum electrodynamics and say that this constant is extremely small, because you have, in case of electrons, you have electron charges, because of very large speed of light, this is extremely small, so that basically prevents you to find out this. However, the funny thing is that this can actually be observed in graphene. Why? Well, because the carriers in graphene, they carry out the relativistic masses of particles, and the consequence of that is that in case of graphene, you have here not speed of light but you have Fermi velocity which is 100 times smaller and that means when you calculate this value you don't get one nanometer you basically get something which is closer to 100 nanometers so it is more realistically possible to do and also other reason for this is that in case of graphene in case of graphene you don't need really enormous potential barriers. Why? Because the reason you need, in the case of a relativistic particle, this kind of enormous potential barrier is actually this very large gap here. Because basically you have to push this band structure by this much up to overcome this gap and to put positron states to align with the electron state. But in case of graphene, there is no such case, such thing. Why? Because the band structure of graphing looks like this. At zero kelvin, curvy level is here, right? Here is the empty conduction band and here is the field, the valence band, and there is no band gap in between, right? So in other words, if you look at this picture, Let me draw it here. If you, for instance, have this potential barrier, and let's say, for simplicity, this level here is the Fermi level, so you have something like this. Typically, band structure of graphene is drawn like this in 2D, because this represents two cones which touch each other. okay if you remember close to the Dirac point we have two clones, Dirac clones touching at the Fermi level and that's usually drawn like this and that means when you have a potential barrier this will end up here okay and now if the electron from the conduction band impinges on the barrier with this energy E here you have the cold state and if you'll be able to pass through okay that's it you don't have to really have enormous potential to overcome this huge gap which is 2m0 c squared or you can already understand from this picture because if you set if you set M0 to B0 for massless particles, these two will join together, right? So there is no gap. You get like a band structure of graphene. That's alternative explanation. And that means the higher the potential, the higher transmission probability. Why? Because the higher the potential, the more you move the Dirac point up and here you go deeper and deeper into the valence band right because if this band structure of graphene moves up at this level this level goes into deeper and deeper into the valence band and you remember density of states of graphene Looks like this. The deeper you go into the valence band, the higher the density of states. There are more states, and if you have more states, you have a higher contribution to the probability to get through. And for that reason, you get basically the same as in relativistic physics. Okay? And then people first calculated this, and then this phenomenon was then experimentally confirmed in case of gupping. So what you see here is the calculation of transmission probability for some quite reasonable parameters. parameters for instance if you look at the barrier height the barrier height is around between 0.2 and 0.3 volts 0.3 electron volts let's say so between 200 so so the actually not between what I'm talking about so the the red value 200 electron volts is not milli electron volt is electron volt is this one and and this 285 rho is this one so this is the potential barrier so up to 285 electron volts that's not really something which is unreasonable that means the voltage is up to 280 volts that will give you 200 electron volts and the barrier thickness as I told you is 100 nanometers which is again reasonable and the impinging energy is about 80 mEV above the Fermi level, okay? And what we have plotted here is the transmission probability as a function of the incidence angle. So, what you see here, these angles here, that's the angle of incidence under which carrier impinges on the barrier. So, in this picture here that I have here, when the carrier goes like this, angle is zero. Okay? Angle is zero. And if you look here at angle 0, transmission probability is how much? So you should look at these circles here. So here is the transmission probability and each circle is the isoline of transmission probability. So this is 1, this is 0.8 and so on. So if you look now at the angle 0, in both cases you have the point here, which means transmission probability is equal to 1. So if the carrying things with the barrier under these parameters if you simply pass through with transmission probability one okay it's very interesting and as I said it was experimental also confirmed in graphing and now because this is of course in partially physics partially electronics is this good or bad for electronics. What do you think? It's bad because you put potential barrier and carrier doesn't care, just goes through. So basically you can't cut the current through transistor. Okay, of course I have to say that this is a bit idealistic, but nevertheless you see the problem. Then you may argue, okay, hang on, but we can actually reduce the transmission probability. Why? Because you see that for different angles we have a smaller probability. So for instance if the angle of incidence is not zero but 30 degree, so if the carrier comes under angle of 30 degrees, then the probability will drop down. For instance in both cases dropped to about, you see, a 30 degree drops to about 60\%. This is better but still not good because for a transistor, if you would like to turn it off with the potential barrier, you want to get zero, or as low as possible to zero percent, right? And of course, the only way to do it is to come here, but then the angle of incidence is 90 degree, and then of course it's zero, because you totally missed the transistor. It means that if this is the potential barrier created by the gate, which is here, and this is source and drain, if the angle of incidence is 90 degree, then the carrier goes like this, so it doesn't go through the transistor, of course you get zero probability. So for any reasonable angle in which carrier goes from source to drain and keeps the gate barrier, you end up with a non-zero, non-zero transmission probability, which is a probability. That's another indication that with graphene we cannot really turn off transistors properly.
\fig{35}{Nanoelectronics of graphene and related 2D materials 2024}
Now it's time to explain how it was actually confirmed and the graphene was discovered in 2004 how it was confirmed because if you remember I told you I told you okay we will have ten minutes more because I started later but you calibrate them for you. So I told you that the problem was that when they showed first FM images of graphene, they couldn't actually confirm that this is a monolayer. They had to figure out other way how to understand whether they have a monolayer graphene or not. And it turned out that there is a way to do it and it goes via quantum Hall effect. But before I explain you why quantum Hall effect in graphene is different compared to other materials, other semiconductors. I would like to discuss the baryphase because they're intertwined. I mean, baryphase in graphene and integer quantum, sorry, quantum Hall effect, which is not integer in graphene, are actually intertwined. So the specific quantum Hall effect in graphene is directly related to the very phase. So what is the very phase? So if you have a carrier moving in a circular motion, so for instance carrier starts from this point and then moves in a circle and comes back to the same point, which is a typical situation you have in magnetic field, tense Hall effect, okay, quantum Hall effect. So if you have a carrier which moves in a circular fashion, in some materials like graphing, once the carrier goes back to the initial state, its wave function will pick up a phase. In other words, if you look at the wave function of carrier after, if you look at the wave function of carrier after carrier completes one circle, this function will have additional phase with respect to the original wave function, which means as if the new wave function you get after turning angle 2 pi is the old one multiplied by e to j pi by this factor. This phase that carrier picks up as it goes in a full circle is called Baryphase. You probably never heard of it because Baryphase in conventional semiconductors is zero. So basically once carrier comes back, the phase of the wave function is the same as it was before it started moving in a circle. However, in graphene, actually this space is not zero. And now the question is how to calculate this space. Well, this space comes from the time evolution of the wave function as the carrier goes in a circle. So basically we are talking about the transition from state kr, this initial state, back to the same state after making a time evolution. and therefore this space can be calculated as the total value proportional to this transition matrix element between the initial state, final state where the operator is the time operator because we're talking about the time evolution and what you have on the left hand side is j phi and you get that j phi this exponential term j5 is equal to this integral. So when you calculate phi, you divide by j and then you get minus j in front of the integral. The integral is calculated along the closed trajectory of the carrier, hence this sign for the line integral along the closed path. Okay? That's the only thing you have to know. I I mean, we don't have to go deeper into quantum physics to understand it. And as I said, in conventional semiconductors, this very phase is zero. That's why you probably never heard of it. However, in roughing, it's not zero. That's very interesting. And again, this comes from its strange spin-or-wave function. So let's calculate this. Basically, this calculation is similar to what we've done with scattering. when we calculated scattering, so we calculated very similar transition matrix element, the only difference is we don't have a scattering Hamiltonian here, but we have a time operator, del over del t, and notice the initial and final states are the same, because the carrier goes along the closed path and ends up on the same k-R state, okay? So we have k-R, k-R. And now to calculate that is just a simple map. You can do it. I will just give you the main tips. So this K R, this cat here is the wave function we had before, right? So this is the eigenfunction that we derived for graphing. Again, I'm calculating all this for the K point. Okay, so we have this expression for the wave function and then multiplied by e to jkR which gives out the, this is the block phase factor, so this is this ket here, it is this term here, then in the middle we have the time del over del t, and then here we have a commission of this which is again transposed and conjugated. And then, okay, you put these two square roots of 2 you put in front and okay we have this the problem is here we have this time operators we have to take the derivative partial derivative with respect to time so then the only way coming the easiest way to do it is to take this exponential term put inside the vector you get here e to j k R and you get plus minus e to j k r plus this one okay and then simply apply the partial derivative with respect to time to this vector here so you see this what was in front was just rewritten and here we have the derivative so derivative of the exponential function is the exponential function we have e to j k r times j is constant j and then we have a partial derivative of k r with respect to time and the same you do here, you get plus minus exponential function and then here you have j in front because it's constant and you have a partial derivative of this kr plus theta q with respect to time you get this, okay and then in the second step you multiply this row with this column so you get this, which is this term of course you can put this e to j kr in front and when you multiply e to minus jk r they will simply cancel out so there's this simplification but still so you get this term which comes from this one and then when this multiplies this also this will cancel out and you basically get this here or if you put two inside you get this one and now to calculate the to calculate very phase you multiply everything by minus j so minus j, j is 1 and you get the integral of these parentheses you get this one and then you split this into two integrals you have this one because dT delta T you get the integral of dKr and here you have 1 over 2 and you have just the integral of d theta cube so the first integral is the integral along the closed path of the differential of the scalar function, because kr is a scalar function, and because you calculate, because you come back to the same point, the value of this scalar function minus the value of the same point. So they basically cancel out. So the first integral is 0, and be careful because this is the angle, which is also a scalar. This integral is not 0y, because when the carrier goes along the circle, angle changes from 0 to 2 pi so although you start from 0 you don't add up with a 0 again with the same value because as you go around data increases to 2 pi so basically have integral from 0 to 2 pi so which is simply 2 pi divided by 2 pi that's it so you get the very faith in graphing not zero like in conventional semiconductors but pi okay so that was very easy to calculate right and now comes the question.
\fig{36}{Nanoelectronics of graphene and related 2D materials 2024}
This is a pi so what what's the consequence now of this and the consequence is the uh you can see the consequence if you look at the point of fall effect of course one to call effect is now lecture for itself, for which I have no time. And I hope you've helped somewhere in physics quantum Hall effect, maybe even in the previous part of the course if you took it. The point is I'm just going to go very very quickly through it and tell you that if you look at the so this is not really a crash course in quantum Hall effect this is really really compressed version so if you look at the integral of FDR along the closed path you know what is this this integral of the force along the closed path is work done along the closed path it actually corresponds to the particle energy along the closed path okay So basically that means if you calculate this integral along the closed path, it should correspond to a word, that is energy, right? And as you know from the, even the oldest quantum theory of Bohr, in which he explained the energy levels in hydrogen atom, when you have a circular motion, the energy which corresponds to this circular motion is quantized. That's how Bohr explained the spectrum of hydrogen, if you remember. that means that this value in quantum physics is quantized, and because the force is proportional to the momentum, it means that the closed line integral of the momentum dr must be quantized. And that's the simplest possible explanation I can give you for this. This is not really, you know, the main topic of this course, so I'm just giving you some sort of trivial explanation. that's why when you calculate this integral 1 over h bar times the integral of PBR along the closed path with the line integral along the closed path this value must be quantized okay and the quantization levels because this is by the way the angle the quantization levels are 2 pi times m m plus one over two. So for a moment don't look at this fly here. So if you look at the conventional semiconductors, it means that if you make a Hall bar and subject and measure a Hall effect, that is measure the Hall conductivity under the influence of the magnetic field where the carriers go in circular motion, you should get that the energy levels are proportional to this quantization here, n plus 1 over 2, where n is some integer. And that's why in quantum call effect, you get so-called Landau level proportional to n plus 1 over 2. Okay? That's all you need to know. I'm not going to go any deeper than this. However, you should also know that because of the circular motion carriers, there is also very phase involved, which I just calculated previously. And because this is the angle, this integral here divided by 8 bar is the angle, you should actually add also very phase, which is also an angle into the picture. But as I said, in conventional semiconductors, doesn't matter because this phi is very phase zero, and you end up that the energy, because you remember this is a energy this is proportional to energy energy is proportional to m plus one over two so these energy levels in such systems are called lambda level okay they correspond to like levels of energy hydrogen atom in simple board theory right so that's the same however in case of in the most general case you should also include this angle pi here and now you see the difference. In case of graphene, this is just this integral rewritten by writing down the expression for the momentum, but that's not important. You can totally neglect this. This is important because in graphene, phi is pi. If you look here, you have two pi divided by two, which is pi. So that means that this one over two and this phi in graphene will cancel out, and you end up that the energy in graphene, so the energy levels in such system if you make it out of graphene, should be proportional just to m, not m plus one over two. And that's why you end up with those quantized energy levels in graphene, lambda levels in graphene, are not proportional to m plus one over two, but proportional to m. Of course, you see also the square root that comes from the different band structures, because the band structure here is quadratic and in case of graphing is linear. If you try to repeat the derivation for quantum Hall effect, if you did it previously, and replace quadratic spectrum with linear, you will end up that lambda levels are proportional to the square root of n. And because n is integer, which could also be negative, you have to put also here absolute value. that's it. And now what does it mean? You see if you look at all other materials including bilayer graphene, so you should look at this inset here which is given for bilayer graphene but it's also valid for other semiconductors. If you look at the whole conductance sigma xy which is the conductance you know in the whole bar xy-conductance, xy-conductance, right? If you look at it, because of the lambda levels, which are proportional to n plus one over two, you will get that this is, you have, you have quantization of the call conductance, and the quants are integer. This is so-called integer quantum call effect. You see, you have one, level one, two, three, four, minus one, minus 2, minus 3, and so on. So this is the so-called integer quantile power factor, which you get in semiconductors and all other graphene derivatives, like bilayer, QL, and so on. The only material in which you don't get it like this is monolayer graphene, because you have a shift of the filling factor by 1 over 2. And because of the shift of the filling factor by one half in graphene you don't get integer but you get integer minus one half you get half integer so that's why in graphene if you repeat the same experiment this is now the domain plot here if you look at the levels they are not at integer but at half integers you see so it starts from one half and then you keep adding one so you get so called half integer quantum whole effect and that's a proof that you have grapheme basically the same group which published paper in 2004 in which they showed graphene, they actually published this paper in Nature a year later, and that's the paper for which they got the Nobel Prize, not the previous one. Because with this paper they managed to experimentally prove that the material they obtained by exfoliating graphene on a silicon dioxide substrate is indeed monolayer graphene because they got, they made a whole bar and measured the magnetic field they got the graph integer quantum core effect which is present only in graphene and nowhere else. Okay and it's also very interesting they then repeated this experiment two years later that you can actually see this effect even at room temperature because if you remember from solid state physics, you can't see this phenomenon at room temperature because the difference between lambda levels is simply too small with respect to thermal fluctuations. Because the thermal fluctuation energy, which is about 25 milli-electron volt, kVt over T, the Boltzmann constant times temperature divided by unit charge, it's 25 milli-electron volts. This energy is simply too large with respect to the difference between lambda levels, for that reason you cannot see this kind of quantization in at room temperature. However, in graphene you can actually see it, so what you see here is the measurement of 300 Kelvin in red, and you can still see the plateaus, so it's possible to see the quantization. Bear in mind that the scale is not the same. Here is E squared divided by H and here is 4 times this. Okay, that's why this appears at minus 2 and 2. These are the same levels are 1 over 2 and minus 1 over 2. Basically you can see it at 300 K. Why? Why in graphene it's possible to see at room temperature because if you look at the lambda levels in graphene as I said they are proportional to the square root of M by the way if you're interested I put here the expression for lambda levels you can calculate this by yourself if you know the quantum Hall effect you can take the bank structure of graphene and repeat the relation if you want a lot of work but you can do it basically these are the lambda levels you get for graphene. And now look what's the difference. The difference with respect to conventional semiconductors is that you have this square root. And because of the square root you actually see it. Why? Because if you remember, if you plot a square root of x, for instance, function. So this is function x and the square root looks like this. Here is 1. So the square root of x is actually larger than x for small x, right? That means that close to 0, square root of x, square root of n is larger than m and that means that here around 0 you can see here the lambda levels so m equals 0 is this level and this is now level for m equal 1 here should be m not n I made a mistake this is lambda level for 1 and this is lambda level for minus 1 because here you have a square root of m function this is rather large large. And that allows you to actually see the temperature. Of course, to get a large difference, you also need large B. So they had to use a very large magnetic field, almost 30 Tesla. That's not something you can get with a regular cryostat. You have to go to the high magnetic field lab, like in the mobile data search systems. And you can actually got as you can see here you can see even a temperature that's also like you can't do this with conventional semiconductors it's impossible.
\fig{37}{Nanoelectronics of graphene and related 2D materials 2024}
I hope this was interesting and now let's have a look at i'd like to mention just another interesting physical properties of graphene and that it's a very high strength. Now there is no much quantum theory involved but to understand this is very easy. Why? Because the graphene has a very strong column and bound thin plane and therefore it has a extremely high strength. This strength was measured like 16 years ago in this paper and it turned out to be enormous. it's about 42 breaking strength 42 newtons per meter which is enormous for a material which is just a monolayer and you see how it was measured they basically took a silicon wafer and drilled hole by reactive ion energy inside the wafer so these are the holes and then just placed graphene on top and then pressed it by the afm tip and measuring by the force pressing the graphene in in the hole, they could measure the breaking strength which comes from the pores when the finally graphing gets broken and also they measure the, this is young modulus, 1 terapascal is enormous. So you see the material is very strong. That's why people say that's the strongest material ever. And I have to say yes, why? Because this is the strongest material ever with respect to its thickness. So you should be very careful when you talk about the properties of graphene. People always say the best, I don't know, for this or that, but that mostly comes with because it's calculated with respect to its thickness, all I want to say. If you, for instance, when I organize this visit and you come to our lab in Como, we can show you graphene on our substrate. And if you just take like a plastic tweezer and just scratch it, you will destroy it. So how is this then the strongest material ever? well, its strongest with respect to its thickness. Thickness means if you will take any other material and thin it down to atomic monolayer, it will break before grafting that's it. As you know, what gives material strength is not only its column and bonds, not only chemical bonds and material, but actually defects. Because defects are those which determine eventually the strength of the material. And because graphene lattice is perfect, covalent, no defect, that's the reason why it's so strong. So, of course, if you take plastic freezer and heat this desk, you will probably destroy the freezer, not the desk. But you may think that this desk is stronger than graphene. Well, yes, in absolute terms. terms, but in relative terms, no, because if you would pin it down to a monolayer atomic, to the atomic monolayer, then graphing will be of course stronger. But that way you should always make a difference between the absolute property and the relative property with respect to the material thickness.
\fig{38}{Nanoelectronics of graphene and related 2D materials 2024}
And now, since we are finished with the pretty much physical properties of graphene, now we understand band structure, and we are about to start with electronics of graphene, I would like to make just one diversion, like maybe 20 minutes, not more, to explain you the properties of carbon nanotube. The reason why it's interesting to explain now is that we actually, hopefully, fully fully understood band structure of graphing and if you completely understand the band structure of graphing it's actually very easy to understand the physical properties of carbon nanotubes because properties of carbon nanotubes come directly from the properties of graphing and the reason for that is that tubes are obtained by rolling graphing sheet into a tube So, before I start with tubes, I would like just to, before I start explaining the properties of tubes, I would like just to point out the papers in which graphene was, sorry, which carbon nanotubes were for the first time demonstrated. I already told you there is still some confusion, because there was one very old Russian paper from 1950 something, I actually did this last time, in which they sort of discovered tubes, but typically people assume that they were discovered in 1991, mainly because the world was ready for nanotechnology this year. And the other one was 50 something, Nolan was talking basically about nanotechnology back then, and they published it, as I told you, in some obscure journal, so I don't want to get down to this discussion who was the first to discover it, but this is a paper which is usually referred to when people talk about true discovery of graphene, sorry, carbon nanotubes, because in this paper, these Japanese scientists in GIMA showed that there is actually another allotrope of carbon which looks like cylinder, because until that year, three allotropes were known, as we discussed at the beginning of the course. Graphite, diamond, and C60, or buckyballs. And then in 1991, he was looking at soot. Soot is what's left after burning materials. It is a carbon-based residue of burning. It would be chimney well, but they're not chimneys now. But if you have a fireplace with a chimney, a black material which fills everything with soot. I don't know Italian, but you know what I mean. Or if you burn the match, the black material which is left after burning that soot. So this scientist was looking at soot obtained from arc discharge. charge. And of course inside you get lots of garbage, basically lots of hydrocarbons, lots of amorphous carbon, things which are, you know, burned. But by looking at Zoot, he actually realized there is something which looks like a crystalline structure, a very ordered crystalline structure. And in this paper in 1991, he showed transmission electron images of carbon cylinder. So he basically realized by analyzing these structures that in this seemingly amorphous mass of burnt material you can find crystalline structure which is just in the shape of cylinder and fully made of carbon and these are the carbon nanotubes which is very interesting. So in this paper in 1991 he demonstrated so-called multi-walled carbon nanotubes which means cylindrical structure with several cylinders inside. Like this is like a so-called double-walled carbon nanotube. You have cylinder inside the cylinder, you can see here. And these are multi-walled carbon nanotubes with lots of cylinders, one inside the other. And then just two years later, he also demonstrated again that in the suit of arc discharge you can actually find also the single volt carbon nanotubes. So it's a cylinder entirely made of carbon and this one has only one cylinder so it's a single volt carbon nanotube. You see the size is extremely small. So why am I talking actually about this? Why am I talking about carbon nanotubes? Well the main reason I like to discuss carbon nanotubes briefly is because in contrast to graphene, carbon nanotubes can be semiconductor. It turns out that depending, and I'm going to demonstrate you this, because you can get carbon nanotube by rolling graphene sheet, depending on how you roll and connect the sheet, you can get tubes of different, that's called chirality, I will explain later what that means, and depending on the chirality of the tube you will get either metallic or semiconducting tubes. And if you can get semiconducting tubes that's great because that means you can actually make a transistor which can be turned off.
\fig{39}{Nanoelectronics of graphene and related 2D materials 2024}
And that's why people were super excited about carbon nanotubes since 1991 what did I say. for what was the year i think 1991 and 1993 right yes and that means that people immediately as soon as people realize that you can actually make you can get semiconducting nanotube that you can make a transistor and that seems like a nano scale a transistor because you've seen that a single-volt nanotube has a diameter of only one point something nanometer so it's really really a small object you can really make a small transistor and that's why these nanotubes have been investigated for the last more than 30 years also for applications in electronics however i have to say that initial excitement after the discovery was somehow damped by the fact that people realize that first synthesis of the tubes is not that trivial. Although you can easily get them in soup, the problem is if you use for instance arc discharge for the tube, you will get all sorts of nanotubes. Single-volt, multi-volt, semiconducting, metallic, you get a bunch of different tubes. And then if you consider their size, if you deposit them on on the substrate, how do you know which one is which? You have no way of telling whether this cube is metallic or it's semiconducting. That means you have to make the contact to figure out what is it. Is it just like a piece of metal or like a piece of semiconductor, which is a huge problem. And the second problem was that if you deposit them on the substrate, they basically get randomly distributed. Why? Because to extract them from arc discharge, we have to put salt in certain chemicals, do something with it. I mean, there is a purification method, I'm not going to go into detail. At the end, you get nanotubes dispersed in solution. And then you put solution on the wafer, you dry it, and you get a bunch of tubes, obviously, randomly distributed across the wafer and that's not the way to make electronic circuit yeah maybe if you connect this cube or that cube you get a semiconductor from conducting then you get the fpd but yeah i mean it's here and there but if you want to make a like a microprocessor with hundreds billions of transistors they have to be in exactly specific location that was a huge problem. So the first problem was the inability to distinguish which are semi-conducting, which metallic, and the second how to actually orderly distribute them on the substrate. So although people published interesting papers with carbon nanotubes, that is the huge problem. However, people never gave up of carbon nanotubes. Why? Because they also have a high mobility and that comes from the high mobility of the ducting. Although I have to say, high mobility, the highest mobility in tubes was measured in, as you can already guess, in metallic tubes. It's always like this. In semi-conducting tubes, the mobility is smaller but still high enough, still larger than that of silicon, and that's why people never gave up, gave up on nanotubes. And here I like to demonstrate one of the relatively recent, I mean, this is the paper from five years ago, in which people finally realized how to make something with tubes. So instead of, so there were two crucial advancements here. So the first crucial advancement was that purification of the tube would so much improve that now when you buy tubes, the producer can guarantee you the type of the tube, with the yield of 99\%. Basically, you can get a guarantee that 99\% of the tubes in solution are actually that type of the tube. For instance, semiconducting tubes of the certain parallel. The only problem is you get 1\% of something else. And that's not good because you'd like to make, again, complex circuits. You cannot have 1\% of transistors which don't work. you can't have 1\% of metallic channels, right? So that's why you have to use something which is also used by semiconductor industry, which is fault tolerant logic. You basically have to build a circuit in a way that you can tolerate certain percentage of faulty circuits. It's possible to do this. That's complicated. I'm not going to go into this. If you studied electronics, you probably know what I'm talking about. Faulty tolerant logic, right? but the other advancement was actually the idea to understand that instead of trying to precisely place tubes on a given place, which is never going to work if you do it from solution, instead, we should take those tubes and gently distribute them across the vapor until you get something which looks like this, like a homogeneous carbon nanotube film. Now this film is not very thick, because in most of the places, the thickness of the film is equal to the thickness of a single tube. Somewhere else, maybe you have two or three cubes, one on top of the other. So this looks basically like spaghetti, just pressed to be flat on the substrate. So what's the advantage of that? Basically what you get here is a thin semiconducting film. Here you have a 99\% of semiconducting cubes. If you put them like this on the wafer, you pretty much get continuous semiconducting film. You get some sort of semiconducting graphene. This is not truly graphene, of course, because you have bunch of cubes, which are individual objects just connected together. And you still have 1\% of unknown tubes, but still most of this film is semiconducting. And then you can treat it as any other basically film, thin film, thin semiconductor film. You just pattern transistors on top and you get semiconductor repetites. Okay, so that was the other basic idea, the other advancement that people realized, that forget about making individual tubes, you will never be able to do it like that, just make the continuous film and just pattern it on. Of course, this field, if you look at it, transistors made in this way, they don't have really the good properties as what you can get if you contact a single tube, but nevertheless, it's still good enough. So in this paper, what people realized was a 16-bit risk microprocessor. Of course, only about 15,000 transistors, but still a huge advancement if you look at what people did before, just contacting individual tubes and then trying to make some very simple logic circuit with the entire microprocessor. Not really with 100 billion of transistors, but nevertheless, that's a proof of principles. that's why the tubes are widely investigated now because this approach may actually help to build the complex circuits because in this field you can get actually mobility which is high enough. The only problem is of course if you look at this, here you can see the images so this is the die and then you can see of the larger and larger modification. When you come to individual transistor you see you have a multiple tubes so these yellow threads are tubes between source and drain and you see the transistor is rather long it's one micron it's not short but nevertheless this thing works you can't expect of course to get something like the US microprocessor from the first shot Also, when silicon was discovered, when the silicon transistor, actually when bipolar junction transistor was developed, it was not possible to make very complex circuits, so you should be very patient. You know, when Jack Kilby demonstrated the integrated circuit, you know Jack Kilby from Texas Instruments, he got, maybe it was 10 years ago, 15 years ago, Nobel Prize in Physics for the discovery of integrated circuits. he demonstrated first integrated circuit electrical engineer law. They didn't like it. Why? Because they said this is never going to work. Because at that time, and that was in 50s, so many many years after discovery of the bipolar junction transistor, still yield of working fats made of silicon was like 50\%. Basically you make two transistors, one works, another one doesn't. And then people, electrical engineers said, now imagine integrated circuits. Because when you make discrete components, we can measure all transistors, discard broken ones, and use only the ones which are good. But imagine now integrated circuits. You make two transistors, you integrate two transistors in a circuit. What's the probability of having circuits which work? If it's 1 over 2 for transistor then it's 1 over 4 for two transistors or 1 over 2 to the power of the number of transistors. So, as soon as the number of transistors in integrated circuit increases, the probability of working circuit drops basically to zero. That's why electrical engineers, many, didn't actually believe in integrated circuits. They thought it's better to just make discrete components and solder on the the PCB. But very soon, technology improved enough and it was possible to make interior circuits without the broken transistors. So never discard technology in advance. Anyway, so this is one of the interesting things I just wanted to show you regarding the nanotubes.
\fig{40}{Nanoelectronics of graphene and related 2D materials 2024}
And now how do we understand the properties of the tubes? How do we understand that some semiconducting, some are metallic. Well, to understand this we have to go back to graphene, because tubes are obtained by rolling graphene sheets. Now, of course, I have to tell you, when the tube is synthesized, when the carbon nanotubes are synthesized, they are not synthesized by rolling graphene sheets. You don't make graphene and then roll it into tubes. They simply grow like cylinder immediately, understand? But for the sake of understanding properties we are going to assume that we have a graphing sheet which is then the row. Okay, so how it's done? So coming back to graphing sheet, what you have here is graphing lattice that we investigated before. Here is the x y coordinate system we had before. Okay, here are our unit cell vectors in direct space a1 and a2. The only difference is that I didn't put orange and blue atoms because they're not important right now. So this is our graphing lattice. And now to understand properties of cubes we have to look at certain crystallographic direction in graphing lattice. So the first crystallographic direction I'd like to demonstrate to you is so-called zig-zag direction. If you draw graphing lattice like I did here, zig-zag direction is along the x axis. So this blue one, that's the zigzag. I think you understand already why the name zigzag, because it's simply zigzags, okay? Zigzag. Hence the name of the direction zigzag. Due to the hexagonal symmetry of graphing lattice, this direction exists every 60 degrees from this one. Why? Because if I just, if I rotate the lattice by 60 degrees, because it's hexagonal, I still get the same amount of steps. I will give you right now the number of the unknowns. Now wait, wait, we will come to that point. I'm just talking now about cryptographic directions and graphing. I will talk about youth later. We are still talking about graphing, okay? This is still graphing. So, if you rotate by 60 degrees, you get another zigzag. For instance, if you rotate 60 degrees clockwise, you will get again zigzag because of the symmetry of the line. If you rotate 60 degrees counterclockwise, you get another zigzag direction. So basically every 60 degrees you will get zigzag direction. Okay? Okay, that's zigzag. Then another important direction is direction which is exactly at the angle of half between two zigzags, which is called armchair direction, which is this one here, this red one you see. This direction here is called armchair and the only reason why it's called armchair is because this part here of the hexagon looks like part of the armchair which is drawn here. That's the best explanation I could come up with. So this armchair direction is exactly 30 degree between two zigzags. I have to say that in my image there is a kind of optical illusion you may think that this angle here is smaller than this and that's only because I used here to demonstrate armchair. If I draw it on the top you see then I think you understand that this is exactly in the middle. How can you actually calculate? Look at this hexagon here. You have this direction here and this direction here and there is a 60 degree and if you draw it here this must be exactly in the half, it's 30 degrees. And then again, you do the hexagonal symmetry of the lattice, every 60 degree you go from here there is another armchair direction so that means here there is an armchair exactly between these two and then if I keep going another 60 it's this one here because this is 30 30 30 that's exactly 90 with respect to for example which means that conclusion which is going to be very important to understand later to the tubes, for every zigzag direction there is another armchair which is perpendicular to it. Why? Because from one zigzag you go 60 degrees to the another zigzag and then another 30 for the armchair. You get exactly my idea. So for every zigzag there is a one armchair direction which is perpendicular to it and vice versa. For every armchair there is a zigzag perpendicular to it. These are the two special crystallographic directions in graphing lattice. In the most general case, of course, crystallographic direction doesn't have to be one of these two. Any other direction is simply called chiral. And chiral direction, for instance this one, is defined by the chiral vector ch, which is simply defined as the integer linear combination of these two lattice vectors. For instance, this kind of direction is defined by vector 4, 2. How do I know that this is really 4, 2 direction? That means 4a1 plus 2a2. Okay, let's go. 4a1. 1, 2, 3, 4, and then 2a2. 1, 2. That's it. That's vector 4-2. This vector points in the focal chiral direction 4-2. So that's another crystallographic direction, but because this vector is somewhere in between two neighboring zigzag and armchair directions, this is neither zigzag nor armchair. That's some chiral direction. That's it. Notice what are the chiral vectors of the zigzag and arbitrary directions. So, the chiral vector which corresponds to this zigzag direction, of course, points along this direction. So, what vector points along this direction? Any vector n0, you do it. If you take arbitrary number of A1 vectors, like 1, 2, 3, regardless of the number, and have 0 of A2, you will always get a vector which points along this direction. So the chiral vector of the zigzag direction, this blue one which I draw here, is N0. okay what's the parallel vector vector of the armchair direction this one this is any vector N N why look at the vector 1 1 a 1 plus a 2 is this vector so that's 1 1 if you now continue from this vector like another 1 1 vector this and this again the same so this will be 2, 2. This will be 3, 3, 4, 4 and so on. Understand? So that means that this untrained direction is simply described by any kind of vector n, n where n is arbitrary integer. And then for instance this one here, this zigzag which is now not important because we have the other one but okay this zigzag has a chiral vector zero n. why? because this is a1 this is a2 and this is 2a2 3a2 4 and so on so zero n yes so the chiral vector of this direction is zero n so you have a zero times a1 which is nothing plus n times a2 n times a2 is this why because this is a1 sorry what did I say a2 a2 2a2 3a2 4a2 so it's 0 n direction okay so I hope now you understood the directions in grabbing what is 0 n yes y n okay okay if you want to use M you can put here 0 M and here M M I just wanted to say integer but you're right doesn't matter okay I can correct do you want me to finish 45 minutes which should be in five minutes or to stop now because it's Can I continue for five minutes more? To have 45 minutes. Okay? Okay, so let me then finish this. Yes? Which one? The third one. Yeah, one which is n0. Yes? One is 0n. Yes? How about the third one? Try to calculate at home. Okay. So, I have to calculate at home. Okay. So, I have to calculate at home. Okay. So, I have to calculate at home. Okay. So, I have to calculate at home. Okay. calculate at home you go minus a2 plus a1 will give you this vector okay you go minus a2 so it's minus 1 plus a1 so it's 1 minus 1 2 minus 2 so it's n minus n actually n minus n must be the same number like here must be the same number so the person who complained who complained about n and u so what are you going to do here and an or MN because they must be the same you understand that's why I put and I agree here we can put zero and to be.\\
So I hope how you understand now you understand the directions in drop me what now it's time to make a nanotube okay so the nanotubes are defined by the corresponding chiral vector. So I'm going to construct you now I'm going to explain you how to construct carbon nanotubes for two. If someone tells you make me a carbon nanotube for two, this person tells you make me carbon nanotube which corresponds to the chiral vector for two. So what does it mean? Step number one you draw graphene lattice like I already have here step number two you draw the chiral vector for two this this is actually exactly this one we had here okay you draw the chiral vector for two okay clear step number two step number three take the scissors and cut graphene lattice perpendicularly to the chiral vector going at the beginning at the end so cut it along these orange lines you understand look at the beginning of the chiral vector and end cut graphene lattice perpendicularly to the chiral vector so here is the 90 degree angle going through the end of the vector and going through the beginning of the vector step number three throw away this thing that you cut out. So leave only thing which is inside the chiral vector. Final step to construct the tube. Roll it up now so that this point coincides with this one. That's carbon anodic 4-2. Do you understand? Wait, wait, wait. We will come to that. You just keep going in front.
\fig{41}{Nanoelectronics of graphene and related 2D materials 2024}
Let's construct zigzag carbon nanotubes. So here is the, for instance, let's construct zigzag carbon nanotube 5-0. So if someone tells you, construct me the carbon nanotube 5-0, it means use the chiral vector 5-0. The chiral vector 5-0 is this one. This is 5 times A1 plus 0 times A2. You see? 1, 2, 3, 4, 5. So this is the chiral vector 5, 0. Because chiral vector 5, 0 points in the direction zigzag, this tube is called zigzag carbonality. Okay? Let me know if you have a point of reference. Okay. So how do we construct? You remember what I told you. put the point at the beginning of the chiral vector at the end and now cut graphene lattice by scissors going perpendicularly on the chiral vector so that you pass through the point at the end of the vector so cut it like this and throw away everything outside ok and now in the last step what you do what you do you just, wait, this is too early to show, what you do you roll it up so that this point coincides with this so now when you roll it, what do you get? when you roll it, you get that the zigzag direction goes along the circumference of the tube and the tube direction is this one, perpendicular to the chiral vector which goes where? along the armchair direction. Why? Because for each zigzag, there is an armchair direction, I told you. And that's a... Okay, we are done. And that's a standard source of confusion with you. Because when someone tells you zigzag tube, it means the zigzag goes around certain parents, and the tube direction is actually armchair, not zigzag, you understand? Zigzag goes around, and armchair goes straight. And that's what I demonstrated here at the end.
\fig{42}{Nanoelectronics of graphene and related 2D materials 2024}
This is the end of the lecture. You can see one zigzag tube. Look. Zigzag is the blue one, you see? It zigzags along the circumference of the tube. So it goes around, cylinder. But the direction of the tube is this one. It's armchair.