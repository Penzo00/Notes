%\chapter{Introduction}
\fig{3}{Nanoelectronics of graphene and related 2D materials 2024}
We are going to start with the motivation. So why are we interested in those 2D materials, especially in electronics? What's the reason we are actually doing this course? Then after I hopefully convince you that they are very attractive materials for the future of electronics, then we are going to discuss the first of them, which is graphene, which is the first 2D material to be discovered. Then also other types of 2D materials like transition metal decalcogenides. And then we are going to discuss the physics of these materials, physics of electronic devices made of these 2D materials. And finally, and finally I'm going to show you some figures of merits of electronic circuits, how those materials fit into the modern electronics, how the electronic devices can be made out of them, and at the very end I'm going to give you just one lecture in nano fabrication of 2D circuits to understand how this is fabricated. By the way, I do research exactly on this, so I have a hands-on experience with these nanoelectronics of 2D materials. So what I'm going to show is basically what we do in my research group. Okay, so let's start first with the motivation, which is probably the most important part of the course. I hope to keep you interested in this. So, actually, before that, let me just explain what are the layered 2D materials we are going to discuss. So, what is graphene and related 2D materials? Let me just briefly explain to you what those materials are, and then I'm going to provide you a motivation. So, I already said before that graphene was the first 2D material to be discovered, and you may immediately ask me a question, how is this possible? Because you already learned from, for instance, semiconductor physics, and also probably from the previous part of the course on quantum devices, but there are actually 2D devices which have been discovered a long time ago, like for instance, two-dimensional electron gas or two-dimensional coal gas in a quantum well that is formed in aluminum gallium arsenide or silicon germanium cathode structure right you should know this probably from some of the previous courses so for instance if you look at the uh let me just delete this So let's say this is bulk aluminum gallium arsenide structure and you know that if you alternate layers here, you put for instance gallium arsenide at the top and then some aluminum gallium arsenide below and then gallium arsenide below that, you're going to form here a quantum well right you know this very well and in this quantum well you can get for instance electrons which are confined in this direction and that's what you call two-dimensional electron gas so this structure has been discovered basically in 18th long time ago and we call it a two-dimensional system but in reality that's not a two-dimensional system why try to take it out so try to take this two-dimensional electron gas out of this three-dimensional bulk structure. And that's as you can already guess not going to work. So if you try to take this plane of two-dimensional electron gas outside that's not going to work. So this two-dimensional electron gas is just a part of three-dimensional bulk structure and can't be taken out. That's impossible. Why? Because to form this two-dimensional electron gas you need a quantum well. to get quantum well you need a bunch of layers here you need a bunch of layers here of aluminum gallium arsenide and gallium arsenide to form the quantum well so this 2d electron gas does not exist without this bulk structure and therefore this is not a truly two-dimensional system because it cannot exist outside of this 3d bulk structure so that's one problem so then you may say okay but uh didn't we learn Did we learn when this structure is grown that it can be grown very precisely, layer by layer, right? Because you know that the typical method used to grow this is molecular beam epitaxy, which allows you to grow this structure really atomic layer by atomic layer. Cannot I take this atomic layer outside? Well, no. First, I already told you why not. because to get quantum value you need all those layers above and below so it's not going to work alone but there is another more serious problem if you try to take it out, even if you manage to take it out you have a huge problem, why? because each of these atomic layers which are grown one on top of each other are in this direction connected by very strong covalent bonds so you basically have to break those covalent bonds to take one, let's say, atomic layer outside and if you do this if you somehow manage to take one atomic layer outside to just get a two-dimensional plane like this of course first there is no electron gas here because the quantum well is gone quantum well is formed by these layers but the second problem are all those broken covalent bonds which are dangling outside of this plane they are by the way called dangling bonds and there by the way because they are not saturated they are empty of electrons because they're broken bonds they're extremely reactive which means if you take this layer outside of the material will immediately oxidize in air most likely if you do this in air for instance if you take silicon that may be the best example if you manage to take one layer of silicon outside the dangling bonds of silicon will immediately react with oxygen and you will basically get a layer of silicon dioxide, which is glass. That's insulator. That's not a semiconductor material. So obviously these structures are 2D but only within a bulk structure. And now we come to the materials we're going to investigate now. So I mentioned graphene is the first 2D material to be discovered. And to understand why graphene is really truly 2D material, you have to look first at graphite, which I put on my slide in the upper left corner. By the way, those slides you can download on VB. You can download them there, and then you can write on them if you want and so on. You will notice that I'm going to follow the slides, which you can download. There are lots of them. And the main reason for that is because if I try to write everything on chartboard, it's going to take forever, basically. So you will see there are lots of derivations in my slides. And that's because I put the entire derivations you can follow simply line by line, So we don't waste time on some basic math to do it on the chalkboard. So I hope you will not mind for me using the slides most of the time. I know the students prefer everything to be done with the chalkboard, but if you do it, it will take a really long time. Anyway, coming back to what you have here on this part of the slide. So you see, this is graphite. Graphite is also 3D material. However, graphite is made of planes. So these are two dimensional planes of carbon atoms, which are one atom deep. And in each plane, carbon atoms are arranged in hexagonal lattice. And those bonds inside each plane are very strong. Those are covalent bonds. These are, by the way, the strongest covalent bonds known to us. And the structure you see is layered. We have a layer upon layer. Each of these layers is what we call today graphene. And now coming back to the, for instance, structure grown by molecular green epitaxy, looks rather similar, right? So if you grow this structure, you can grow atomic layer by atomic layer. Here we also have two-dimensional layers of carbon atoms stacked on top of each other. However, there is one crucial difference between graphite and those semiconductors on the right. The crucial difference is that here in graphite there are no covalent bonds between the layers. That's the catch, that's the crucial difference between them. So if you have two layers, two two-dimensional sheets of carbon atoms, which we call graphene, so if you look at two graphene layers, there are no covalent bonds between. Basically what keeps graphite together is electrostatic force, very weak electrostatic force, force which we typically call wonder walls force but that's not covalent bonding and that means that those layers can very easily be taken out because the electrostatic force which keeps nothing layer to form graphite is very weak and that's the reason why you use pencils You all use electronic pens or pens? Does anyone have a pencil? You have it? Ok. Could you please demonstrate to your colleagues what happens if you take a pencil whose tip is made of graphite and you draw on the paper? Show them what happens. This happens. So that's actually exfoliation of graphene from graphite. What he's doing with the pencil. because these layers are so poorly bonded together, these are not coen bonds, when you take graphite and draw with graphite over the paper, you simply start shedding those, you simply start exfoliating those graphene layers from graphite, and the trace they leave is actually what you see on the paper. Basically what you have on the paper is a bunch of graphene layers just deposited on the paper as you go with the pencil over it and that's the reason you use the graphite pencils because graphite is very easy to exfoliate for the very same reason graphite is used for lubrication because it's based on exactly the same principle you get material which gets exfoliated and that makes a little bit of lubrication if you put it in in something that should be smooth. If you have two surfaces that should you would like to reduce, for instance, the friction between two surfaces. You can put graphite in between and because of the exfoliation, you will get a very good lubrication. So that's graphite. And then you can imagine if you if that's true, that means that you can actually take you can take graphene layers outside and that's the reason why graphene was the first truly 2d material to be discovered and the graphene is represented here so what you see here is the top view of one of these planes i don't know why this plane is drawn in this shape i mean it should be actually rectangular like this one but anyway I found this image in this form so I didn't want to change it so what you see here is the top view of one of these two dimensional planes of carbon atoms if you look at the top you see that carbon atoms are oriented in a hexagonal lattice so you have a perfect hexagon and that's graphene you may of course ask why graphene was discovered just in 2004 because the structure of graphite was known from before it was one from before that graphite is made of graphene layers wine moment extracted dropping from drop by before 2004 the main reason was that people didn't think that's possible and I'm going to show you later how it's possible how this can be done and what what are the arguments for the existence of graphene, because for instance some quantum theories predict that purely 2D material like graphene, when it's taken outside of the graphite, cannot exist. It will simply collapse. But luckily that's not true. It turned out that graphene is quite stable, and if you take one of these layers outside of the graphite, you get really graphene, which is the first 2D material to be discovered, and the first we're going to investigate in this course. Okay, so that's graphene. Now let me just give you a few, just to make a quick summary of the properties before I switch to other materials. So graphene, you can see here in the table, graphene is a semi-metal, which means that in graphene, conduction and valence band touch. So there is no band gap, so it is a semi-metal. so therefore you see the band gap which is denoted by EG is equal to zero you may immediately ask me why are we interested in material which is semi-metal for electronics because don't we need semiconductors for electronics and by the way why do we need semiconductors for electronics because mainly electronics today is digital and digital electronics require transistor to be possible to put in two highly distinctive states. State of high conductivity, when the voltage of the transistor is pretty much zero, when the transistor is highly conductive, and the high impedance state, state of extremely low conductivity, when the transistor is in the off state. However, in order to turn off the transistor, the material from which the transistor is made must be a semiconductor, must have a band gap. Because if you don't have a band gap, you can't actually turn off transistor. Why? Because it is the bank gap which provides you capability to turn off transistor because you get states actually get a lack of states because in the bank gap there are no electronic states so device can be turned off. I'll discuss this later in more detail this is just now the quick overview. So why are we then interested in graphene electronics? Well because of the second parameter which is the carrier mobility. the catch is that graphene has enormous carrier mobility uh so what is carrier mobility you know this very well uh by the way sometimes you will have a feeling i'm explaining something which is really trivial but please bear in mind that this is a quite mixed course so they have people both from physics electronics and also material science okay so not everyone is expert in everything right so when i start talking about some physics physicists will probably think wow come on this is boring i know this yes but maybe people from electronics are not fully aware of this and you know when i start talking about amplifiers then it's going to be the other way around probably then electronics people will say okay come on this is very easy right so anyway that's why sometimes i explain some things which uh seem basics but anyway so as you know very well so i I think this is known to everyone. So the drift velocity of a carrier in materials is proportional in the small electric field, is proportional to the electric field applies. And the constant of proportionality is called the carrier mobility. That's the value in low fields. And this is extremely important parameter. Why? Because the higher the carrier mobility, the larger the drift velocity on the same electric field. And that means that the carriers in material move faster. If carriers in material move faster, it means they can go from one side of the device to the other in a shorter period of time. Which means the device itself is faster. And that's why if you would like to have ultra-fast devices for very high frequencies, you need materials with enormous carrier mobility. So to give you some point of reference which you know very well, carrier mobility in silicon which dominates electronics right now is about 1500 square centimeter per volt second however this is in very low fields in higher fields because all devices must be able to work in high fields why because you apply large voltages on devices large i mean large voltages on the devices in high fields this mobility in silicon drops i'm talking about the n-type silicon drops to about 500 square centimeter per volt second so now compare 500 to 100 000 and this is room temperature mobility so that's enormous which means that these devices can be made very very fast by the way you may argue that seen even larger mobilities than that of graphene like in exactly aluminum gallium arsenide quantum wells you can go up to six million i think that's the current record but that's a cryogenic temperatures so unless you want to launch your tablet in space it's not going to work at room temperature so that's why at room temperature that's the highest room temperature carrier mobility this one of graphing which is about a hundred thousand square centimeter per volt second that means that the drift velocity in graphene at the same field is much much larger than that of silicone so interior you can make a much faster device with graphic and that compensates that compensates the life of the banger so you may ask me okay but hang on on. If I have a high mobility, okay, device is very fast, but if I can't turn it off, what's the point? Well, the point is then to use graphene transistors in applications in which you don't have to turn off the transistor. So we are talking about analog applications because there are lots of analog devices which do not require transistor to be turned off. And for that reason, graphene is foreseen to be used in analog rather than digital devices. And you will probably then say, okay, but come on, analog devices, you need them. Because you should understand that all your digital devices, they operate in the so-called base band. Base band is the band in which information is processed. So for instance the fastest microprocessor right now runs at about 6Gbps and you can push to, I've seen some people push this processor, it's one of the Intel i9 processors, they push it to 9Gbps but they reverse it to liquid helium. Cool but not really useful. Anyway, that's the baseband, that's the clock rate at which information is processed. you would like to be connected to the internet, right? And to be connected to the internet some information must be sent to the network and then transmitted to somewhere else. And for that you need analog devices. Because as you know, 5G is not in the base band. 5G is the frequencies above 25 gig. And if you go 6G, you could be even higher frequency. Why the devices are using such high frequency for exchanging information it's very simple the band is totally overcrowded with all sorts of signals right you have terrestrial tv radio stations wi-fi you can imagine all sorts of radiation and that's why the tendency is to go to the higher and higher bands which are free and for that you need high frequency devices so from the base band you need device capable of up converting base band into the high frequency band and that's then by up converting circuits or mixes and they are purely on the circuits and then also when the power is delivered to antenna to be transmitted over the network that's again analog power amplifier and then the information is reached to the network and then the network is for instance relates to the user but when it comes to the user you need the opposite process. You need to then down convert high frequency back to the base band. And down conversion is again done by the mixer. So they down convert high frequency and these are also on other devices. They might also amplify signal because maybe transmission to the network goes to the attenuation, especially if it goes to air, right? You know that the higher the frequency the higher sensitivity to alternation of the signal so basically you need lots of analog devices you can't operate mobile or anything without analog devices and that's why analog devices although they are not so widespread there are key components in communication without that all this digital would be able to work on the baseband and you would be able to transmit anything between the users okay so that's the motivation for analog devices currently The highest frequency devices are not made of silicon because That's very hard for silicon to do and if you have to go to really high frequencies and talking about hundreds of gigahertz and so on Then it's three files which are used the reason why because three files have much higher ability than silicon The reason why your laptop depends and everything are not made of three files is drugs. They are extremely expensive Like for instance, ingas, indium phosphide, aluminum, they are all very very expensive, especially indium. So by the way, the fastest semiconductor technology is based on indium phosphide, but it's extremely expensive not only because indium is a very expensive material, and by the way it's expensive because it's not abundant as silicon. As you know, silicon is the second most abundant element in earth crust after oxygen. So indium is not that abundant. So that's one reason why it's expensive. And indium phosphide is especially expensive because it's a very brittle material. So you can't make a wafer larger than this, like a hundred millimeter wafer is top. If you try to make a larger wafer, because you know that semiconductor devices are made than 300 millimeter wafers they will simply break and for that reason this technology is very expensive so that's why for instance graphene can try to compete in this analog arena and then finally the last data you have in the table is the thickness of graphene so you can already understand if this is just a one atom thick material then basically the thickness of this graphene layer should be pretty much the thickness of a carbon atom. However, that's not usually what's taken as the thickness of graphene. The thickness of graphene is typically taken to be the interlayer distance in graphite, which is 0.33 nanometers. It's very hard to measure the thickness of carbon atom. And then you start graphene in graphite, the layers are the distance of 0.33 nanometers. and that's typically taken to be the thickness of graphene monolayer, of one layer of graphene. Okay, so that's graphene. So now why graphene is so important? Well, graphene is extremely important because once people realized that graphene can be taken outside of the graphite without any problems, then what's stopping us from using other layered materials? because there are many other layered materials that exist and if you use them you can actually take those layers outside and get 2D materials out of bulk. So for instance the next one I put on my table is hexagonal boron nitride. Hexagonal boron nitride or for short HBN is the material which is extremely similar to graphene. Why? Because it's the only other material apart from graphene which has a monolayer of one atom tip. So it has a monolayer with atomic thickness. You will notice that all other 2D materials that we are going to discuss have layers which are actually few atom ticks. Only graphene and hexagonal boron nitrides have layers which are exactly one atom tip. The only difference between graphene and hexagonal boron nitride is that in HBM you have two atoms, boron and nitrogen. So that's what makes graphene the only monoatomic one-atom thick material. HBM is the other one-atom thick material, but it's not monoatomic because it's made of boron and nitrogen. the crystal structure of HBM is basically identical to that of graphene and graphite you indeed have a hexagonal structure you will only notice that type of the atom alternates as you go through the crystal structure of a monolayer HBM you have a nitrogen then boron, nitrogen, boron, nitrogen, boron and so on so they just alternate Okay, this HBN also exists in bulk. It's exactly the same as graphite. You just have those HBN monolayers stuck upon each other. We don't use it for writing. Why? Because, first of all, HBN is a synthetic material. It doesn't exist in nature. You can synthesize it, but you won't be able to find it in nature like graphite. As you know graphite was discovered somewhere in 15th century long time ago in England. While HBN is fully synthetic material and the reason you don't use it in pens is actually the fact that it's white. So it's not going to work. However, it has similar properties and you will notice in terms of the crystal structure. However, you will notice that in contrast to graphene, HBN is actually insulator. It doesn't conduct current and it has a very large band gap. It's about six electron volts, which is the reason I put here that mobility is zero because it doesn't have three carriers. And because of the similar structure, it has also very similar thickness because the layer, the interlayer distance between HBM, the interlayer distance in bulk HBM is 0.33 nanometers like in graphite. What it's used for? Well, HBN has its own applications, but it's very interesting in graphene electronics. Why? Because it seems like a perfect insulator for graphene. Why? Well, you know that if you would like to make a transistor, like a field effect transistor, you have to electrostatically isolate gate from semiconductor channels, right? And typically, the isolation which is used is oxide. Originally, semiconductor industry used silicon dioxide because it's very easy to form. if you have a piece of silicon you just oxidize you get silicon dioxide so that was the gate insulator in the past but now they switched to high k oxides because the the electric constant of silicon dioxide isn't enough so they switch now in the modern transistor since maybe like 15 years ago they switch to half near half unit dioxide but the problem is that all these oxides are pretty much amorphous material so if you have a crystalline semiconductor channel on top of each you put something which is amorphous it's not really nice and also it's a source of scattering for instance because you know that in oxides you can get lots of charge traps so it really makes lots of problems to make a good device so for instance making a good mosfet is half of the job is to make a very good oxide to be to avoid any sources of carrier scattering irregularities and that's why hdn is very interesting why because if you use a graphene as a semiconductor channel you can then put on top hbm as an insulator basically you have an insulator which is not not amorphous, but crystalline. This is fully crystalline structure. And that's why HBN is very interesting to be used in graphene applications with graphene electronics, because you basically get a fully crystalline insulator, which has the same pretty much lattice constant as graphene, which means there is no strain, there is nothing. It has a perfect crystalline structure doesn't have any charge traps it has a lattice constant similar to that of graphene they're very similar so basically there is no strain so it looks really like a perfect insulator and that's why i mentioned hdn here however i have to tell you that despite all this it's not really a perfect insulator why because six electron volts is not enough unfortunately if you want to fully turn off transistor okay graphing transistor can be turned off but anyway if you would like to really avoid any leakage current through the insulator you need a larger band gap because as you know one of the biggest sources of power dissipation in modern electronics is leakage current which leaks all over the place but in many cases through the oxide directly and that's a problem. And to avoid leakage current you need to have an insulator with a large band gap. So 6 electron volt is not sufficient enough for all applications. And now let me finally consider the other two 2D materials which are very interesting. So these two, molybdenum disulfide and tungsten diselenide. So this column here is molybdenum disulfide and tungsten diselenide. So these two materials belong to the group of transition metal decalcogenides or TMDs. So in these materials you have a transition metal, in first case molybdenum, in the second tungsten, connected to calcogen. So in the molybdenum disulfide is sulfur and in tungsten diselenide is selenium. Why these materials are interesting? Well first of all that's maybe not that important but molybdenum disulfide is the only other material apart from raphe which can be found in nature. Okay so it's not only synthetic you can find it in nature. Anyway so they are very interesting these two materials because they are semiconductors that's exactly what we need for electronics. So they're semiconductors and if you look at their band gap, they actually have a larger band gap than silicon. It's 1.9 and 1.6 electron volts. You know that silicon is 1.1 electron volts. That means that when you turn off transistors made of these materials, the off-current will be smaller than that of silicon because you have a larger band gap. Unfortunately, the price you pay for this is smaller mobility. I'm going to show you later when it comes to that point that unfortunately in nature there is a trade-off between the band gap and mobility. In other words, smaller the band gap, larger the mobility. Yeah, nature really doesn't like electronics. Anyway, so they do have a larger band gap, but also smaller mobility. You see, their mobility in both cases, in the best case, is about 150 square centimeters per volt second. And now you're going to immediately ask me, OK, but if it's so small, why it's interesting? Because silicon is, I already told you, it's like at least 500. So silicon should be way better than these materials. Well, yes and no. Silicon indeed has a higher mobility. But the problem is that thin silicon does not have high mobility. And I'm going to show you very soon when we switch to the following slides, that for the modern circuits you need semiconductor channels which are very thin. When the thickness of silicon channel drops below 2 nanometer, its carrier mobility in this channel suddenly drops down below 100 square centimeter per volt second due to the intense carrier scattering on surface roughness because of such thin channel. Basically, when you make silicon thinner than 2 nanometers, you end up with a lot of dangling bonds, right? I explain you what is it. And these dangling bonds are detrimental to the transport. Why? Because when the material is bulk, dangling bonds are far away. The surface is far away. You don't care. But when the material is very thin, just few nanometers, they become dominant in the transport because they simply represent a source of scattering. And that's a big issue with silicon. So that's why semiconductor industry seriously considers this material to be used as a replacement for silicon. Because now they need thin channels and unfortunately thin silicon channels are not good enough. But I will come to that in a moment. Regarding their structure, you will notice that I didn't put a 3D structure. I am going to show you later the 3D structure when we talk about these materials in more detail. But when you look from the side, you simply see the layers like in case of graphite. The only difference is that these layers are not monatomic because we have, of course, two types of atoms. And also their thickness is not one atom. They have a larger thickness than one atom. Why? Because here, if you look. This was this color. like greenish blue cyan. So this color here represents molybdenum atoms and yellow is sulfur. So basically each layer, monolayer of molybdenum disulfide is actually a three layer. You have three layers. You have sulfur, molybdenum and sulfur layer. Within these three layers, which we call monolayer molybdenum disulfide, the bonds are covalent. so they are very hard to break but between these three layer and the next three layer in the molybdenum disulfide there are no covalent bonds you have the same electrostatic you have the same electrostatic interaction like in graphite so basically this monolayer of NOS2 which is basically three layer you have sulfur molybdenum sulfur planes here the connections are covalent so this is hard to break but molybdenum disulfide bulk is this plus this plus this plus this and so on you understand so each of these so-called monolayers can be can very easily be taken away from from bulk and that's why for instance if you go to lobby you can find lubricants made of for the same reason that the graphite is good lubricant. And the same for tungsten diselenide and because these are, they have monolayers, they have actually three atomic planes, you see they are slightly thicker. Their thickness is about twice that of these two. So the main reason we are interested in TMDs is because these two materials, these two particular materials have a band gap so you can use them to make a semiconductor device. And by the way the TMDs are actually, that's a very large family of materials. There are hundreds of different types of TMDs, many not even discovered now, so who knows what's in store, right? Anyway, finally to finish the overview of these materials we're going to investigate, I like to point out another interesting thing, and that's that these materials, because they don't have a dangling bonds, they are therefore not reactive. Most of them are quite inert because it is the dangling bonds which introduces reactivity in the materials. And because of that, it turns out you can actually stack different materials, these kind of 2D materials on top of each other and create something that we call Wonderwall Scatter Structure. So you can basically put one material on top of the other and create material which doesn't exist in nature simply by stacking layers on top of each other. So for instance, what we typically do in my group is the following. We typically encapsulate graphene in HBN. So we make a stack in which we have our HBN at the bottom, then graphene in the middle, and then HBN at the top. Why? Because it turns out when you encapsulate graphene in HBN, you get this high mobility. Without this encapsulation, you cannot reach this high mobility. Why? Because graphene, as you already understand, is just a surface. But this surface must be physically placed on something. You cannot have it suspended in air. actually you can you can also make suspended devices where graphene is suspended for instance between two contacts then you can get also very high mobility but this is not really mechanically stable so at some point you have to put graphene on some surface right as soon as you put the graphene on surface it interacts with it how well basically if you put an oxide to use as a substrate substrate, then you get charged traps in the oxide and then you have a problem. You have a reduction of carrier scattering due to the charged traps in the oxide. And that's why it's better not to put graphene directly on oxide substrate, but to put it on HBM. HBM is also insulator, so it's not going to short circuit graphene device if you make it, but it will isolate graphene for detrimental impact of the insulating substrate. Why do you need HbN on top you need it because as you know from physics every surface exposed to air within a nanosecond get immediately covered by all gases which exist here in ambient unfortunately these junk gases we have in ambient so i'm talking about water vapor oxygen and so on which gets deposited on any surface within a nanosecond, introduces another source of scattering. That's one problem. Also introduces unwanted doping. I'm going to show you this later. So to prevent, for instance, graphene from the impact of the environment from ambient, you can then put HBN on top and basically encapsulate passivate graphene between two crystalline layers. That's one thing. And I'm going to show you later some much more complex devices we need with like I think the largest number of layers we put on a device would maybe 11 or something. For instance if you'd like to make a CMOS inverter you need an N and P type material. MOS2 is N type material, tungsten diselenide is P type material if it's T. So basically if you stack one on top of the other you can get CMOS. But the problem is of course that then in that case you need to insulated, so you need some HVN in between. Then you need gates to put, to be able to drive transistors. And the best way is to basically use graphene as a gate, for instance. And in that way, you can stack all these materials on top of each other and get something that we call Wonderwall's heterostructure, which is very interesting. Why? Because these kind of structures don't exist in nature. It is artificially made material in the You simply take material by material, put on top of each other, and you get a novel material. That's something I can show you if you come to our lab in Como, then we do research on this, we can show you how it's done. I know that some of you came to our lab recently, but I wasn't aware that you would come that day, and I think only a few of you came. But if you're interested, we can organize another visit just to dedicate it to 2D materials materials and we can show you how this is done okay okay fine so fine so this is the well short not really so short well anyway that's the overview of the materials we're going to discuss and now let me come to the most important point is why are why are we interested in these materials why do we care about graphing ms2 and all those materials well the main reason.\\
Why we're interested in this material is because of the device scaling. So now I put here in this presentation couple of slides trying to explain you why do we need 2D materials but before I go into those couple of slides let me just tell you in short just one sentence basically why do we need 2D materials. The point is that modern electronic devices are always scaled down. We try to make them as small as possible. Why? Quickly, if device is smaller then it's faster. Why? If the distance between source and drain is shorter, carrier needs shorter time to go from one side of the device to the other, okay, and that makes device faster. So the main reason why we like to make devices very small is because that's what makes them faster, okay. The other reason is that if devices are smaller you can put them more on a chip and that has two consequences it means you can get more functionality that means in the same surface area you can put more logic cores of a microprocessor for instance or if you look at the memory you can get a higher capacity if you can shrink devices by a factor of two that means that the memory chip you previously made with 16 gigabytes can be now 32 gigabytes because devices are smaller and you can do that more than chip. So the key issue in electronics is to shrink down the size of the devices and how is this related to 2D materials? So that's what I said in our explanation in one sentence that I will go a bit, expand it a bit more. So why is this important? How is this related to the materials? As you shrink the device size laterally, to put them more on a chip, in order to, so this is now a very simple explanation, I will give you a more detailed explanation later. As you shrink them laterally, to preserve the aspect ratio of the device, and therefore to get the same electric field and charge distribution, you should also shrink them vertically. That's it. As you shrink them laterally, you have to shrink them also vertically. that means you have to shrink the thickness of the semiconductor channel. And now the lateral size of the modern devices became so small that the transistor channel is approaching atomic thickness. And that's why we need 2D materials.
\fig{4}{Nanoelectronics of graphene and related 2D materials 2024}
The more detailed explanation why do we need these materials, right? So, I already mentioned devices are scaled down, and the best way to show it is to show a plot which describes Moore's Law, which is shown here. That's actually not a law, it's just an observation made by Gordon Moore. And just a bit of history, so Gordon Moore was the founder of Fairchild Semiconductors. So the time when he made this observation, that was in the 60s, he was looking at the data available at that time about the number of transistors per die which is the number of transistors per silicon chip and you see the first point he had was this one so that's a one transistor per chip that was the first integrated circuit made by jack kilby at texas instruments and then later on there are a couple of other data and by looking at those data Moore said it seems that number of transistors per die double every year and that observation turns out to be law. Why law? Because if you look at the development of semiconductor electronics it turns out that this is really true that actually this is a twice per year so increase of the number of transistors per die by a factor of two every year. Currently it's about slightly slower, it's about doubling of number of transistors per die every 18 to 24 months, so it's a bit slower. But nevertheless, that's geometric progression. Because you start with one transistor per chip and then you have two, and then you have four, and then you have eight, and as you understand, At the beginning that doesn't look big, but that escalates very soon because of the geometric progression. That means you have an exponential increase of the number of transistors per die with time. And the best way to plot it is then, because it's an exponential increase, to use semi-log plot. So that's why when this plot is made, you have here the number of transistors in logarithmic scale, and the year is given in linear scale. And obviously you see the function is in this scale, the function is linear, which means you have exponential increase of number of transistors per die. The data you see here is the data prior to 2010, and you can easily recognize those are Intel data. The reason why there are Intel data here is simply because Intel was the first one to make the microprocessor. So that was the 404, 4-bit microprocessor, long time ago. and it's actually related to Moore because he later founded, he was also founder of a company called Integrated Electronics which turned out to be then called Intel later. And as I said, just a bit of history, so you know that point junction transistor was discovered by Barden and Breton And William Shockley developed here a PM junction and bipolar junction transistor, which is more useful than the point junction transistor. But Shockley not only developed BJT, he also developed other types of devices like junction field effect transistor and also thyristor, which is irreplaceable in power electronics. and he decided to leave Bell Labs and found our company in a valley of city of San Francisco more mostly by beaches at that time to basically make a company and make millions out of his discoveries and I guess you know which valley is this now right so that's the Silicon Valley so he was the first one to build a semiconductor company in this valley which is now known as the Silicon Valley. Anyway, he had very talented engineers. One of them was Moore, but he was a very strange person, very hard to work with. And basically those people simply liked him because Shockley was totally obsessed with thyristor. Because today thyristor is used in power electronics. But the reason he was obsessed with thyristor rather than bipolar junction transistor was simply because thyristor is a bistable element. And that looks so exotic to him so he just wanted to commercialize thyristor without realizing that real money is in vjt the bipolar junction transistor so a couple of engineers together with moore left shockley and they founded this company called fairchild semiconductor which at that time in 60s was the largest semiconductor company in the world later on he and the same group of engineers that that were interesting left Fairchild because Fairchild didn't want to invest in MOSFET. Fairchild made millions in DJTs and they wanted to stay with something they know, which is bipolar junction transistor. And these guys realized that the future is MOSFET, not DJT. So they left Fairchild and then founded Intel. And that's why Intel developed the first microprocessor was actually developed by Intel. It's an interesting story why they developed it, but we don't have time for this. Anyway, what you see here is Intel data. Now you have other companies which make microprocessors, of course. But the reason I put Intel data is because they made the first microprocessor, 404, right? And you can see in this plot, it's really an exponential increase. Those numbers, 404, 80, and so on, these are microprocessors. What you see here is RAM. It's memory, random access memory, the capacity of random access memory. And that trend basically is, you can, this trend is even present today. I put here a newer plot than the one here. So it goes up to 2020 something, which is present day. And you can see really in the semi log plot that basically the number of transistors per die exponentially increase. And why does it increase exponentially? Well, because devices are made smaller and smaller, and as I told you, the reason for that is not only to increase functionality or capacity of memories, but also to make devices faster. And that brings us to 2D materials, because as I already told you, if you want to scale down device laterally, you have to also scale it down vertically. And if you look at the previous plot, you can see that this is unprecedented scaling because now device size, the transistor size in microprocessors reach nano scale. So you're talking about devices with a gate length of about sub 20 nanometers, which is extremely small. So the device is so small laterally that you understand that also the channel must be very, very thin. and I will explain very soon why. I mean, I already gave you one simple explanation, but I will explain why. So that's one reason why we need the 2D materials.
\fig{5}{Nanoelectronics of graphene and related 2D materials 2024}
However, there is another reason we need 2D materials, and that's related to this plot. This plot actually shows the breakdown of Moore's law, which was not visible in the previous plot. Why? If you look at this plot very closely, you can download this plot from this link. So take PDF, just click, you will get this plot. Here you have several properties of microprocessors. Here is the number of these values. And what's very interesting to look at is the number of transistors. So this is again semi-log plot. Okay, we have a logarithmic scale here, linear here. And you see this plot, the linear increase of the number of transistors per die. So now currently, 2024, we have basically about 100 billions of transistors per die, which is enormous. 100 billions of transistors per die. Anyway, you see, so the number of transistors per die increases, and you can immediately imagine if this number of transistors increases, then their lateral size also decreases, and therefore they must be faster and faster. However, look at the clock rate of the microprocessors. So at the beginning it was also exponential, the same as the number of transistors. So that looked great. However, somewhere around the year 2000, the Moore's law actually broke down. You see that the increase of the clock rate stopped and the clock frequency is pretty much fixed at about few gigahertz. And that's a big issue. So let me first explain you what happened before 1990. Before 1990, they were able to increase the clock rate simply by reducing the size of the transistor, because the transistors were fast. However, at year 2000, the power dissipation in transistors, in microprocessors, hit about 100 watts. So this is now this plot you see in red. And 100 watts dissipation of a silicon chip, which is approximately this size, size of course if you open a desktop you think microprocessor is big but what you see is just a fan it's a cooling system if you if you remove all this at the end you will see like a small square of this size now if you imagine if this surface area and it's 100 watts of power that's huge so that's why the clock rate basically stopped when the power dissipation hit hit 100 watts so first question why power dissipation increases with frequency there's the switching power because these transistors operate at this frequency and they basically switch all the time if you process digital information what you get switch which means transistor switch and for each switch for each switching you need certain power so the faster the operation the more the switching power and that's why when the clock rate hit about one gigahertz a year about 2000 the power hit 100 watts and it was not possible anymore to increase the clock rate significantly because that would increase the power and the processor we already burned down the consequence of that is that today transistors you have in microprocessors are actually very fast. If you take individual transistor out of a microprocessor, you would be able to run it at 100 GHz without any problems. However, microprocessor itself cannot run at 100 GHz because of the power dissipation. Because in a microprocessor you have so many transistors when you sum up all this power dissipation you get huge power which simply cannot be released from such small surface area. And that's the only reason why the clock rate is currently fixed at about a few gig and it simply stopped. It cannot go higher than that. It's impossible. Because otherwise you will burn down microprocessors. So before you were born, so before year 2000, the things were great. I remember this. In 1995, I bought a PC. It had a clock, 90 megabytes. It doesn't sound like much. But in five years, reached one gigabit. Because basically clock rate was doubling every year. It was a very exciting time to be alive. You know how people when they start talking, when I was young everything was better. Well, that's one of these stories. But now, after year 2000, basically you were born probably 2005 something like this, right? No? Okay, I don't know, I'm just guessing. Okay, maybe a bit earlier. around 2000 something like this yes it's over there so at that time then this stopped because of the power dissipation unfortunately so it cannot be increased and however if you look at the performance of the microprocessors you see that performance is still increasing so how did they manage to increase performance without drastically increasing the clock rate because Because previously, before year 2000, the easiest way to increase the processing power was just to double the clock rate. You get consistors operating at twice the frequency and the operation is twice faster. So basically you see that after year 2000 they realized that they want to increase performance because they cannot increase the clock so much, they have to basically increase the number of logic cores and that's why approximately until year 2000 there was only one logic core per microprocessor so all processors were just single core processors however after that they decided to increase the number of logic cores to basically introduce parallel computing because if you manage to compute two things at the same time in parallel you can basically increase the processing time however you will notice from this plot that increasing with the number of course does not increase really processing power because this what you see here is rule that's the processing power expressed through this parameter that it doesn't increase as fast as here why because I think it's obvious you cannot parallelize all operations that's the problem first if you write a code regular code this code cannot be running parallel unless it's optimized for parallel processing that's the first problem you cannot take a single thread core and put in a processor and think okay now it will be parallel no you still run on a single core unless you introduce multi-threading in your code so there are ways how this is done however this cannot be fully done because there are many operations which cannot be performed unless some other transformations are finished before then for instance if i would like to calculate something but any input parameter which comes from other calculation this thread cannot start before the previous one is finished you understand so that's why parallelization is not really a magic you cannot really increase it as much as you would be able to do it just by increasing the frequency but it still helped industry to make things faster even though even though as i said the clock rate is fixed at about typically now about five gig as i told you the fastest microprocessor today is six gig that's it at room temperature of course and then you have gamers which always try to overclock processors right and you know what they need for overclocking they need the new cooling system entirely new cooling system because they burn down the processor that's the problem okay so how to fix this problem permanently well the best way would be to be able to increase the clock rate not to get it saturated here but how can you increase the clock rate so the easiest way to increase the clock rate is to replace silicon with material of higher mobility. Why? Well, if you replace silicon with material of higher mobility, which is written here, so if you use material with larger carrier mobility, that means that everything you make out of this material is less resistant. Which means all devices have a smaller resistance or equivalent resistance if it's transistor. Why? Because the mobility is higher. And if the resistance is smaller, it means you can get the same drive current on the smaller voltage. And the smaller voltage means smaller dissipation, higher dissipation. So that's why one way to fix this problem is to replace silicon with material with larger mobility. Because that would allow us to increase the clock rate. Because that would reduce the power dissipation. However, as you see from the materials I showed you, they don't have a spectacularly large mobility. Which means basically they are going to be used in the future as mitigation technique. Because it turns out, as I told you, there is no magic material which has a good bank gap and very high mobility and it's cheap. Unfortunately it doesn't exist. It hasn't been found so far. So for the moment, in the future, probably somewhere here I'm going to show you soon, silicon will be replaced by 2D materials in order to mitigate the reduction of carrier of carrier mobility in silicon because if the silicon becomes too thin, mobility will drop so much that this will totally fail down, everything will fail down. And that's why 2D materials will be used in the future to replace silicon. That's the only reason. Of course it would be better if you find material with much higher mobility but we haven't found it yet. And we found that there are, as I told you, some 3-5s are good but they are very very expensive so not used for general purpose devices okay so that's one one reason why we need 2D materials.
\fig{6}{Nanoelectronics of graphene and related 2D materials 2024}
Of course if you look at mobility of graphene graphene looks like a perfect replacement because it has enormous mobility so graphene would be great but as I told you the problem with graphene is it doesn't have a bank yet so that means all the way graphene you you will be able to use very low voltage supply, because it has enormous mobility. The problem is you won't be able to turn off graphene transistors. And if you cannot turn off graphene transistors, then you have a static power dissipation. Because as you know in digital electronics, in a static state, steady state, when there is no switching, circuits are made in a way that there is no static power dissipation, there is no current which flows through the device. I will come to that later and explain you for instance CMOS inverter in which this is very easy to understand. But if you make a CMOS inverter out of graphene you wouldn't be able to turn off neither of the transistor in the inverter. And you will have a static power dissipation all the time. So basically what you gain by having material of higher mobility you lose by the fact that it doesn't have a band gate unfortunately. And that's why silicon in digital is going to be replaced by TNTs not by roughing and now since I mentioned a couple of times how small devices are because that's related to the channel thickness let me actually explain you how really small they are because now if you look at the PRs of big semiconductor companies they talk about 3 nanometer node and then 2 nanometer node and 1.8 nanometer node nodes so you think what are they making really transistors which are that small but unfortunately not because all these nodes you hear five nanometer three nanometer and so on they're basically just a sales pitch they have no any they're not based on any physical dimensions in transistors and that's a pity because now it's basically just wild west people are just throwing away some numbers which don't have any physical meaning however in the past again one of those stories in the past everything was better right in the past this uh node number was actually physically related to the gate length and it had a physical meaning so let me show you just this plot for a moment back in 2007 the the semiconductor companies released microprocessors in so-called 45 nanometer mode but this 45 nanometer had a physical meaning because at that time node was half of the gate pitch and the pitch is the period so it was the half of the gate period so what is it I put here at the top you can see one logic cell so this is not really important to understand how it works I just want to show you the geometry so basically in this logic cell you have semiconductor channels which run like this horizontally which means source is on this side the drain is on the other side okay and what goes across is gate so those are green bars that you see these are the gates which go on top of the channel now the gate has the gate length gate length is the most important parameter of transistor so for instance if you look at this transistor here here is the source for instance on this side drain is here and the distance between the source and drain is this length here length of this bar in horizontal direction and that's called gate length that's the most critical transistor parameter which basically defines the channel and gate size typically gate the channel and the gate have the same length because they are made by so-called self consistent self aligned, sorry, self aligned process in which you define source and drain island by using gate as a mask. So that's a long story. I'm not going to go into details right now. But the point is, that's the most important parameter of transistor. So this is the gate length that you see here. And then you hear you have several gates. So you see there is a certain period of the gate. So they're simply stacked like this one after the other with a certain period this is gate pitch and the gate pitch which is the period which is for instance distance from this edge to this edge is in this plot obviously equal twice the gate length because this gate length is very very small so we are talking about 20 nanometers so it's very hard to make and that means if you make the gate the next one most likely cannot be at a distance smaller than the gate length from the other one because that's basically the resolution limit of your exposure system okay so that's why the gate pitch which is the period of the gate is typically twice the gate length that's the smallest feature size you can get in a transistor okay so if the node is defined as half of the gate pitch and the gate pitch is twice the gate length then node simply means gate length so that's why in year 2007-8 when they developed 45 nanometer technology 45 nanometer node had a meaning because 45 nanometer was actually the gate length so that was the the size of the gate in the transistor and then in the next generation they developed so-called 32 nanometer node in which the gate length was indeed about 32 nanometers okay and then the next generation was 22 and then then everything broke down because at that point it was not possible anymore to shrink the size of the devices to really get gate length in 2011 at that size why well that's related to the problems in fabricating such small devices but before i explain you how they are made just quickly let me just explain you how these numbers are obtained do you know how these numbers are obtained 45 32 22 are those some random numbers or there is some connection between them so 32 nanometer node is the next generation after 45 nanometer node so the industry said okay because we are going to follow Moore's law let's then use the nodes which follow Moore's law and the Moore's law Paul tells you that every like 18 to 24 months, that's what's present, the number of transistors per time doubled. Okay, if the number of transistors double, it means their surface area must shrink by a factor of two. If the surface area shrinks by a factor of two, it means their lateral dimensions go down by square root of two. So if you take a calculator and divide 45 nanometers by square root of 2, you will get 32 nanometers. Okay? I think the easiest way to see it is here. After 40 nanometer node, the next one is square root of 2 is 1.4. 14 divided by 1.4 is 10. So the next node is 10 nanometer. Okay? So that's how you get. Divide everything by square root of 2. But somewhere here, this broke down. And why? Because at that time it was not possible to make transistor with a 22 nanometer gate line and that's related to the way how transistors are made. So just very quickly let me show you how is this done. So imagine this is a cross-section of the substrate on which you would like to make a device so how is this done you put somewhere here at the top a mask and the mask has some areas which are covered by chromium which is opaque material and this is sapphire which is transparent so from the top so you basically here you make transistors from the top I mean you make a layout of devices from the top you shine a laser light which comes from argon fluoride the Eximer laser at 122 nanometers radiation passes through sapphire here because it's transparent it doesn't pass through chromium and and basically here if you put the lens this will be focused and this will come here so this part will be exposed this part will be exposed and this one will not be exposed because the laser light is blocked so basically use the laser light to shine through the mask to get a picture of a circuit on a wafer okay that's how it's done and now you understand the problem if you use This is an X-ray laser at 192 nanometers. Physicist, question for you, what is the smallest feature size you can resolve? As you know in physics there is something which is called Rayleigh criterion, which tells you that in optical microscopy, but this is the same story, the smallest feature you can resolve is pretty much the same as the wavelength. Maybe you can push down the half of the wavelength if you know how to do it, It's really, really hard to go beyond half of the wavelength. And if you use a wavelength of 192 nanometers, obviously, how are you going to make something which is 22 nanometers in size? It's very hard to do. So that's why at that point here, basically, they couldn't shrink the gate much below like 30 nanometers. and all these numbers you see after this mode 22 nanometer they are basically just PR just publicity stunt by semiconductor companies although to be I mean to be fair they kept Moore's law at least in the fact that they managed to double the number of transistors so you see these numbers in yellow they show you increase of the number of transistors per die because what you have here is a transistor of density so this is the millions of transistor per square millimeter as a function of year hvm is high volume manufacturing so that's when the when the processor come to the market right so they managed to increase the number of devices on the chip by optimizations and also by increasing the chip size but they couldn't actually reduce the size below about 29 meters so even in present nodes so if you go now in the future this keeps going up the last time I updated my plot was at 20 20 I stopped at node 5 nanometer because really these numbers have no physical meaning anymore so this is just PR it's like when you buy I don't know graphics card RTX 3070 what does it mean 3070 It has no physical meaning, just a number. And then you buy a faster 149 TTI. What does it mean? You know it's faster than the previous one, NVIDIA card, but this number doesn't mean anything. And unfortunately, these nodes now have the same meaning. Actually, they have no meaning. Okay, so now they talk about a 3 nanometer node. That's the current node in production that will be released this year. Actually, it's already released. Intel released it already. TSM-C2. TSMC is by the way the largest semiconductor factory in the world. You know it's located in Taiwan. Taiwan semiconductor manufacturing company and they basically make processors for other companies like Apple. So they don't have their own stuff but they make for other companies. Samsung is another big player in this field. And that's why I didn't put here the other nodes three nanometer and 2 and 1.8 because they really have no physical meaning and they they're even changing the names of the nodes now what intel called back then 10 nanometer now they call it 7 nanometers because it turned out that at 10 nanometer node in 2018 they reached 100 millions of transistors per square millimeter and it turned out at the next seven nanometer node they had the same density of So they increase the number of transistors by increasing the value. And the more you go there, it's harder to find any reliable data. For instance, now it's impossible to find data for 3 nanometer technology. What's the exact transistor density? No company will use it. It's impossible to understand. So that's why I stopped updating this plot, because now they're basically stuck, and they're stuck with silicon, unfortunately. And now let me explain you why for small devices we actually need two-dimensional materials.
\fig{7}{Nanoelectronics of graphene and related 2D materials 2024}
I already explained this with the aspect ratio, right? But now I can give you a little bit better explanation. And that will be the way to introduce how transistors are made today. And you will understand that they are naturally progressing to two-dimensional materials. So what you see here in the top left is the conventional transistor you learned about in textbooks in electronics or solid state physics. So this is a MOSFET. So here we have a source on the left, a drain on the right. If this is for instance n-type transistor, this silicon body is p-type and source and drain islands are made of n-type. If this is an enhancement mode transistor, the standard MOSFET with a positive threshold that is used in digital electronics, then there is no channel between source and drain, you know this. And in order to get a channel between source and drain, you have to get inversion layer, which you create by applying the voltage between the gate and source. So when the voltage between the gate and source goes above the threshold voltage, you form inversion layer and then you can get conduction between source and drain. I'm going to discuss operation of transistors later in more detail because I'm going to introduce also graphene transistors and other types of transistors but now just remember that the operation is simply basically done by controlling the conductivity of the channel electrostatically by the gate. So you have a gate electrostatically insulated from the channel and by applying the voltage between gate and source you basically control the density of the carriers in the channel and by that you change the resistance of the channel that's it so the current which flows between source and drain depends not only on the voltage between train and source which will be a regular resistor but also the gate voltage because with the gate voltage you control conductivity of the channel and thus the resistance between source and drain so that's a standard method that you learned about now what's the problem when mosfet becomes very small when mosfet becomes very small source and drain come very close to each other and why is that a problem because that becomes the problem when they become very close to each other the drain then is capable of pulling some carriers from the source here in the area below the channel which is not controlled by the gate at all. And so that's the effect which comes from the fact that the transistor is very short, that its channel is very short, and that's called short channel effect. I have to tell you that short channel effect is a common name for many other effects which deteriorate transistor operation at very short gate lengths or channel lengths. I will talk about this later. But the main one, the main problem is that when source and drain are so close, drain is simply capable of pulling some carriers from the source not only here in the channel area which is fully controlled electrostatically by the gate but also here below below the channel where there is no full control of the gate so basically you end up with a transistor with a parasitic resistor connected in parallel because if this is a MOSFET M type for instance that looks like this right gate drain source what's the problem and you end up with this you end up with a transistor with a resistor in parallel why? because here the current of your transistor the drain current has two components one is the drain current which is controlled by the gate source voltage okay i will put also vds because current of course depends on the drain source voltage but on top of this you have this current which is flowing between source and drain and it's not controlled by the gate. So you basically just get Vds divided by the resistance this parasitic resistor you have which comes from the short channel effects. Is this good or bad? Well that's terrible because you have additional parasitic component in the transistor current which cannot be controlled by the gate source voltage. You lose transistor effect because transistor effect is capability of this device to have current electrostatically controlled by the gate. Here if you have a current flowing below which is not controlled by the gate you just have a resistor connected in parallel. So like between source and drain here we have a resistor connected. How do we see this in the curves of the transistor? What you see here on the right is one ID curve I picked up from some book. So this is some very long channel device if you look at the current it's very high current so it's most likely some power mosfet it doesn't matter i'm going to discuss more why it looks like this but the point is the key operation area of transistor is this one is the saturation in which drain current is fully controlled only by the gate source voltage so that's a long channel mosfet you see here in saturation you have the drain current which is controlled only by the parameter which is the gate source voltage it doesn't depend on the drain source voltage so this is where the transistor effect is at its strongest however if you introduce the transistor if you make a transistor which is very short you get extra current which flows here which is not of course controlled by the gate and then the idea curves look like this so the easiest way to see whether transistor is short or not just look at the output curves if you see that in saturation the curves are not horizontal sorry but slanted like this you know it's a short channel transistor it's a transistor in which we have a very strong short channel effects. And now, how is this related to 2D materials? So think now, how can you fix this problem? How can you fix the problem of the current which flows here, below the channels? How can you fix it? Well, the first attempt industry made was to make SOI transistors, which means silicon insulators. So they say like this, because the problematic current flows here below the channel, let's cut out this part of the semiconductor and put an insulator. Because if you put an insulator here, the current cannot flow here anymore. And you basically eliminated this parasitic current, which causes the problem. But by doing this, what you're actually doing, what are you doing? making the channel between source and drain very thin you're basically going into the direction of 2d materials because if the channel is thin and if there is nothing below the current cannot flow below the current can only flow here in the area which is very close to the gate and therefore fully controlled by the gate and the thinner the channel the better because then if If you make something which is even thinner than this, then you suppress even more current which tries to flow in this area which is not controlled by the gate. And that's why you need 2D materials, that's it. Because in 2D materials, you get the ultimate thin material. It is basically few atoms thick material. If you put gate on top, the current can only flow through material, cannot go anywhere else because it's so thin. therefore gate fully controls the current and you get full suppression of short channel effects of course you may ask me but okay so is this still used no this is not used because it's not good what's the problem here the problem here is that this oxide as you already understand it's amorphous material so this structure cannot be actually grown that's the problem so basically this structure is made by bonding silicon layers on oxide and this is very tough if this silicon is very thin that's basically a nightmare to do it in the silicon channel is very thin so that's why that's not used what it's used something which looks more complicated but it's actually easier to make so the transistors which are currently now on the market and basically starting starting from this 22 nanometer node I mentioned before are not made like this. So this textbook transistor doesn't exist anymore. No one makes it like this anymore. Now transistors that you have are made like FinFETs. So why they're called FinFETs? Because this is the cross sectional view of the structure which looks like this. So in this fin fat you have source on this side. This is drain. And the gate insulator and the gate metal are like this. So the gate stuck is on top. What you see on my image is the cross section of the fin. The reason why it's called fin because it looks like a fin. That's the last thing you would like to see when you swim in the water. that's what shark has on its back right that's the thing anyway in thin fat the semiconductor this is basically semiconductor channel standing upright the current flows perpendicularly like this from source to drain and it's controlled by the gate which is around the fin and now the question is but how does this suppress short channel effects well look at the picture if you look very close to my image you will notice some dots this thing here silicon pin is full of dots you know what those dots are silicon atoms so the thing itself is very narrow and because it's very narrow and surrounded by the gate carriers can only flow in the area basically fully controlled by the gate pretty much You understand? So this thing is like some 2D material standing upright like this although it's not really 2D but it's very thin. I mean unfortunately Intel did not release the scale on this plot but you can calculate the silicon atoms you know the constant you know the lattice constant of silicon so you can very easily calculate how thin is this right yes? Where's the channel in this? This is the channel. All of it? Yeah, yeah, exactly. Not just the edges? No, no, no. So all this is the silicon channel. And the current flows like this in this direction. And the gate is around. Okay? You understand? So that's how it's made. And now if you look, I updated my slides from last year, so I put here the output curves of the Intel 3 MOSFETs. Intel 3 is now a new name for 3 nanometer node. So Intel decided to be a bit honest and drop nanometer because it really, as I said, makes no meaning physical this number. So this is Intel 3 technology. You can see the output curves. They're not absolutely horizontal, but having very advanced fins, they actually not also very much slanted.
\fig{8}{Nanoelectronics of graphene and related 2D materials 2024}
So you see a very good suppression of short channel effects. Of course, the fin you see here is not this one. have it on the next one so put here let me show you all that i will show you so the one is this one actually this is the latest one and you can see how they progress so 22 nanometer was on the previous slide sorry was this one and then you can see here this is 14 10 and now intel. You see how nice fin is. Unfortunately again there is no scale. They released it without scale. I can't tell you the size. Here I managed to find the size. And you see in the 10 nanometer node, distance between two fins is 34 nanometers. So it's not 10 nanometers as you would expect, right? You see a very nice, very narrow, very homogeneous fins. But unfortunately even this is not going to be enough soon.\\
So, finally I'd like to discuss the top two images so that you can see the brief view of the structure, because this is taken, what you see here is a cross-section, but here you can see a view from the top of how it's made. If you remember the manual, this is what it looks like, it should be important. You can see it here, not from the side. So these are the pins, the pins come in this direction, and the gates go to the other side. The cable pin, and the gate goes around the pin. And so for instance, if you look at this structure here, this is the gate, this is the pin. So you have a source, for instance, on this side and a vein on the other side. Of course, the contents are not visible because in this image only the insurgents are made. In reality, all this space you would fill with dilactic, typical silicon dioxide, which is the octane-synced by oxidizing silicon, and then holes would be etched inside the silicon, so called valions, to come with a source of very metal contents and attach here and here, on these two sides of the beam. But here you can see the results of the metal content. However, it turns out that even these kind of structures are not sufficient, and that for further scaling, the structure should be changed. So what is the problem? So the problem is, if you look at the fin, this structure looks something like this. So you have a substrate fin, and then the gate which goes around, which I did not go here for simplicity. And now, similar to conventional MOSFETs from your textbooks, this field is a channel, and it's made, this is a, for instance, n-type MOSFET, it's made of two big people, three types of them, or even increasing, and then when the gate source voltage is applied, you have an inversion layer which is formed here and then as you apply a larger gate voltage of course the inversion layer becomes larger and eventually the entire fin becomes flat. But you see what are the problems. The first problem is, as I already told you before, fin has to be very thin in this direction. Why? Because if you make fin very thick, you won't be able to invert the entire fin, so you'll have a here area which is not inverted, and therefore from this area you can get, bearing in mind sources here, drainage on other side you can get carrier flow in this case electron flow through this area which is not controlled by the gate so that's a problem with your channel defense so that's why the fin must be very thin so you have you don't have this area but you can already understand that even if you make the fin very thin like here you have a problem here at the bottom because at the bottom you have area which is not controlled by the gate because the gate is here, okay, and for that reason through this area here you can get carrier flow from source to drain which is not controlled by the gate. And that means that the only way to fix this is not only to make fin very thin but also to get rid of this area demotor. In other way you have to find a way how to enclose entire fin with a gate to make so-called gate all around structure. However, such a very thin and totally surrounded by the gate structure is very hard to make upright.