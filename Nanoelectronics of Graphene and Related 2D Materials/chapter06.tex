\fig{64}{Nanoelectronics of graphene and related 2D materials 2024}
And so this is a kind of motivation for graphene, basically it means you have to be careful in which applications you are going to apply graphene because of its properties, some look attractive but actually just troublesome, Some are really good, so that's why you should find the application in which graphene can really shine. What I want to say, you cannot blindly take any circuit you want and just replace silicon with graphene. That's not going to work. You have to be careful what you're doing. And now it's time to discuss graphene transistors. Now, before this, let me just show one slide to discuss the carrier mobility of graphene, because I mentioned it many times, and that's basically the main advantage of graphene, apart from being atomic thick and flexible and transparent. So graphene indeed has the highest carrier mobility at room temperature. I put here one table for your reference, so you can actually get idea how graphene fits into the other materials. Those rows which are in orange, that's graphene, and if letters are in blue, that's cryogenic temperature, So it's not really for practical applications. So you should basically only look at this part of the table. You will notice that graphene appears in this table many times with many different values of carrier mobility and the main reason for this is the underlying substrate. Because graphene is just a 2D material, just a surface, its carrier mobility drastically depends on the substrate on which you placed it. And why? because the substrate is the main cause of carrier scattering in graphene. So if you have it on a very bad substrate, which is the easiest to make, that is silicon dioxide, then you're going to get maybe mobility up to 10,000. I have to say this is very... These are the references from which these data are taken. This is very optimistic, I would say. I never seen mobility this high on silicon dioxide. typically you can get maybe few thousand ten thousand is really hard to get on silicon dioxide because silicon dioxide it's a very good substrate in terms of cost it's cheap easy to make compatible with semiconductor industry right but it's full of charge traps so graphene is heavily influenced by everything in siu2 and that's why you can get maybe up to few thousand but even few 2000 is larger than that of silicon, you see n-type silicon is 1500, as I said that's bulk, high fields 500, the less than 2 nanometer it drops below 100. You see the other semiconductors like MOS2 we are going to discuss later, it's not, maybe I should update this table because approximately the larger the mobility in MOS2 is about 150, so that's approximately the highest you can get and you see like also you have, of course this doesn't sound like a lot, but I already told you when you go to thin channels, it's going to be larger than that of silicon. And then as a reference, I also put n-type germanium, which has a high mobility, you know this, then silicon also was used in the first bi-power junction transistors, but it's not so used now simply because I already explained you, it has a water soluble, It has a water soluble oxide and it's very expensive material, I mean much more expensive than silicon. And then you see, for instance, graphene, if you place it on a ferroelectric substrate, you get like 70,000, that's a special substrate which is used to suppress, so in this substrate the carrier scattering on charged traps is suppressed, so that's why you get the relatively large mobility 70,000. You can go, for instance, if you, I think the best way to make graphene, okay, you can suspend graphene, although this is low temperature data, you can get about 100, 150,000 even at room temperature if you suspend graphene. Suspended graphene means it doesn't touch anything, so you have two contacts and graphene is for instance placed on top so it doesn't touch, it's not in contact with the substrate, it's just suspended in the air and if it's suspended in air then there is no scattering on the substrate because it doesn't touch anything but this is not mechanically stable structure so I didn't bother putting here room temperature data. Of course the best is if you encapsulate graphene in HBN, so you put in HBN sandwich and that really gives you very high mobility in excess of 100,000. So here is a paper which claim to, so this is taken from this paper, which claimed that at room temperature it can go up to 350,000 square centimeter per volt second. I would say that's a bit exaggerated maybe because first they did not encapsulate in HBN, they used HBN, put graphene and tungsten diselenate on top, so it's a bit different type of heterostructure, but nevertheless they measure this very high value, but this paper still stayed in archive, it's not yet published, it's been in archive I think few years now, it could be that it was not possible to reproduce this result, that's my guess, I don't know. Anyway, what I can assure you that you can get around 100, 150 thousand without any problem, so these values here, that's without any problem of room temperature. In our lab we can routinely get 50 thousand without basically taking much care if you're encapsulate in HBN. So this is really, really high and it's very good for applications. Of course, I also put here, you can see the bandgap of the materials. You will of course notice that the larger the bandgap, if you look at the MOS2, the larger the bandgap, the smaller the mobility. So that's what we discussed before and it's valid basically for all materials and you also have in the table three files, I put only one ingus, you see 15,000 very high mobility and bang gap is 0.75 which is not bad, that's very close to silicon. So, that's a very also good semiconductor, but unfortunately it's very expensive, it has indium, like indium phosphide too. So that's it, you get some idea about the, about the mobilities in graphene and now it's time to turn into electronics to discuss the graphene transistors and how they work and how the transistors properties change because as you understand graphene has totally different band structures, so let's discuss this.
\fig{65}{Nanoelectronics of graphene and related 2D materials 2024}
Now in order to simplify things I'm going to first discuss graphene resistor which is just graphene sheet, so here is like a graphene sheet with two contacts. So it's a typical resistor, just two contacts. I call them source and drain because I'm going to use them later to make a transistor but at the moment you can call them as you wish, so because this is not a transistor, this is just a resistor, you see the gate is missing, the third controlling contact is missing. So we have two contacts, the graphene and everything placed on insulating substrate, of course why, of course because you don't want to short circuit the device, so that's why everything must be placed on something which is insulating and as I said before the typical substrate is either SiO2 which is easy or if you want to make a high performance device you make it on HBN. If you ask me why don't we always make it on HBN and encapsulate it, well there are two reasons. The first reason is that HBN substrates are not really available on the large scale right now. Basically what you have to do is to use a silicon substrate and then somehow deposit a large-scale HBN on top to have a isolation from the substrate, but CVD grown HBN is kind of still lacking. I mean people can grow it, you can get nice crystals, but on the wafer scale it's a bit hard to get it right now. So that's why even now you will find many devices simply made on SRU2 because that's easiest and the fastest way to do it. Anyway, you need insulating substrate, you have these two and now what do we get here? First I have to tell you I'm going to use one approximation which neglects all parasitic resistances in this circuit which exist, I mentioned them before, so if you for instance contact, if you connect these two contacts to a voltage generator, you will realize that the resistance of this device is not only given by the resistance of graphene sheet between source and drain, but there are some resistances in series which are parasitic and I'm going to neglect for the moment, which are, first you have serious resistances of the contact, of course, because this is not super, these are not superconductors, these are regular contact, you have some resistance there and then on top of that you have, we already discussed, because we have a contact here between the metal and graphene, you have then the contact resistance, because whenever you have a contact between two different materials due to different band structure, there is something which is called contact resistance. you have to be very careful because depending on the type of the materials in the band structure you can also get a Schottky diode for instance, you can get a Schottky contact I told you this before, luckily because graphene doesn't have a band gap in most of the cases when you make a contact with metal also without band gap you end up basically with only contact luckily so there is no Schottky contact and that's good because otherwise you will get two Schottky diodes back to back and they will block device because however you pass the current, one diode will be reversely biased and it will block the current. So luckily we don't have this here. So what we have are just some resistances, serious resistance of the source, then we have a contact resistance between source and graphene, then we have resistance which we would like really to discuss which is resistance of graphene sheet between source and drain and then again here we have a contact resistance between drain and graphene and finally serious resistance of the drain contact. Those parasitic resistances will be neglected in analysis, why? Because we can simply model transistor by investigating transistor without them and then just adding those in series, that's it. That's easiest way to do it, so I'm going to neglect them. So what I'm going to look at now is just resistance between source and drain. So the first question is, let's assume zero Kelvin, let's assume zero Kelvin, if it's zero Kelvin then we know that the band structure looks like this, we have a Fermi level exactly at the Dirac point and now the question is, if I apply a voltage on this structure am I going to get any current or no? What do you think? So we have zero Kelvin, Fermi level exactly at the Dirac point, which means there are no electrons in the conduction band, there are no holes in the valence band, even on top of this density of states is zero at the Dirac point. So if I apply some voltage, am I going to get any current or no? you think? Well, first what voltage did I apply? So, that's the first thing you have to consider because when you apply a voltage at the ends of the conductor the Fermi level splits by the E times V, where V is the applied voltage, right? So, that means you have a conduction not only at the Fermi level but in a range around the Fermi level and that means you will actually have some conduction even at zero Kelvin. In my initial analysis I'm going to assume that VDS is negligibly small because otherwise I have to consider potential drop along the channel which I'm going to consider later. So now I'm just trying to do a very simplified picture in which applied voltage between drain and source actually between the two contacts is negligibly small. Basically it means I'm looking only at the transport at the Fermi level. Okay, do I have current or no? Well, if that's all true you would say no, because if I have a transport only at the Fermi level then there should be no current because density of state is zero and there are no electrons and holes, forget it, no current. Actually in reality there is current, there is current because graphene has one strange property and and that however you make it, it's never absolutely homogeneous and that means the following, that this band structure that you see here is actually invalid, it's invalid in the sense that as you go along graphene flake in some places band structure will move slightly down, So this band structure will move slightly down so that the Fermi level ends up in this area in the conduction band and you have some electrons. So what you see here is the charge density in graphene looking from the top and these reddish yellowish areas, these are the areas in which the band structure somehow moved down and Fermi level ended up in the conduction band and also in some other areas the band structure will slightly move up and then in that case the Fermi level ends up in the valence band and you have some holes which is here denoted by a blue color and the black color is the region in which you really have a Fermi level exactly the Dirac point, it's a region which is transition between these two. So that means that however you make graphene you always have some charge carriers in it even at zero Kelvin, because if the Fermi level goes in the conduction on valence band, you have states there, because as you remember, DOS is not zero there, DOS is zero only the Dirac point. And this phenomenon is called, people call it electron hole puddles, which means you have some puddles of electrons and holes across graphene and that's a universal property of graphene. like this, you can't avoid it. And there is a theory which tries to explain conductivity which stems from this and it's related to the conductors quantum and so on, but I'm not going to go into, if you're interested more you can look at this nature physics paper. I'm going to skip it because I really don't need to go into this detail. So basically ideal graphene is ideal in this sense, it's a kind of mixed conductor even at zero Kelvin and that means that even at zero Kelvin you have some conductivity, its conductivity is not zero, that's it. So even at zero Kelvin you have some conductivity and then of course if you consider device at room temperature then you have thermal excitations around Fermi level which means that even without electron hole puddles you will have conductivity at final temperature, because even if you have this situation at, for instance, room temperature, without any electron hole puddles, you know that some electrons, due to thermal excitations, will jump from the valence band to the conduction band, and you will simply have both electrons and holes, so you will have a conduction. So that's why in realistic graphing, that's why in realistic graphing you always have a current, whether the voltage is negligibly small, whether it's zero Kelvin or not, you always have the current. However, this electron hole puddles is rather hard to model and I'm simply going to, I mean rather hard to go into theory of this, I'm going to skip it and I'm simply going to either neglect it when it's not important or simply to model it with some finite conductivity at Dirac point, that's it. I'm going to assume that conductivity is finite at the Dirac point regardless of the temperature. Okay, so I hope this is clear, but then there is another issue and that's the issue of doping, ambient doping, because unfortunately graphene doesn't have bulk, so it's impossible to protect it if it's exposed like this, it's impossible to protect it from the environment. You can protect it from the environment if you encapsulate it in HBN, So then it's protecting both from the substrate and the environment and in that case Fermi level in graphene stays at the Dirac point However, if you make a conventional simpler device, which is not encapsulated something like this what I put here Due to the exposure to ambient you will have some ambient impurities attached to graphene by simple process of adhesion I mean you simply have the not, what did I say, adhesion, absorption, because I already told you that if you take a clean surface out of vacuum and expose to ambient, what you're going to get immediately is the coverage, because all these molecules which fly around will simply stick to the surface and you will get monolayer coverage within a nanosecond, which means you will get some ambient impurities. if graphene is exposed like this, you will have some ambient impurities attached to graphene. I denoted those ambient impurities with these six green impurities. And why are they important? They are important because they actually work like acceptors in semiconductors. In other words, those impurities are going to trap some electrons from graphene and basically leave behind holes. So, what you end up with are immobile impurities, which are negatively charged because they took some electrons from graphene and left cold behind in graphene. It's like acceptor doping in semiconductors, basically identical thing. And what that means, that means that actually in graphene, we have certain impurity charge, which is denoted here by Qin, which is the charge of these six holes we have in graphene, which comes from ambient impurity doping. And that also means if we have six holes in graphene, like these six holes, that the band structure doesn't stay like this anymore, but that means that graphene is a P-type semiconductor, and that means that in real graphene, the band structure is actually moved up so that the Fermi level is somewhere in the valence band, okay. I hope you didn't go to the other building first. By the way, the message I sent yesterday via VB hasn't arrived yet, you know. The message I sent in the evening was through the online system, not through VB. so don't be surprised if you tomorrow receive message that tomorrow's lecture will be 9.02 because this VBIP is really useless when it comes to messaging. Anyway that means that we have a Fermi level inside the valence band in other words these six holes that you see here are actually holes which are here okay So the band structure moved up so that you have here six empty states and these six empty states here and these six holes here, that's what you get. So that's actually real graphene, you always have some doping, there is always some doping. You are not going to have it if you encapsulate it in HBN, then there is no doping and the Fermi level stays here but in reality you always have some doping and that means graphene is a p-type conductor and now finally what I said okay, so if I apply some voltage I'm going to use the convention used for the n-type MOSFET, so I'm going to assume that the contact which is at the higher potential is drained and the contact at the lower potential is source So if I apply the positive drain source voltage, I'm going to get a current which flows like this. Typically, you know, you can ground one point in the circuit as a reference point. Typically, with n-type FETs, you ground source, but this is really relevant. And if you don't like this ground, you can remove it and just put it, but it's not going to change anything, of course. So if you apply a drain source voltage, I'm going, and I'm stressing enough, negligibly small for the moment, I'm going to show you later what happens if the drain source voltage is not negligibly small, but anyway, if you apply some small positive drain source voltage, you're going to get current which flows in this direction, let's call it drain current, even though this is not still transistor, and then how do you measure resistance? simply Vds divided by Id will give you resistance R, so resistance of this device will be applied drain source voltage divided by the drain current, right, but now the question is okay, that's how to measure it, but how can we calculate this, can we somehow calculate conductivity or resistance of graphene, well of course the answer is yes.
\fig{66}{Nanoelectronics of graphene and related 2D materials 2024}
And I'm going to show you know how to do it but this is very easy if you study physics then this is easy for you I mean I remember I read some comments on the course you know at the end of the course you can give your comments about me what you think about me and the lectures and some physics guy complained oh so why are you deriving all this well I mean I'm deriving all this because not all of you here are physicists, okay. So, please be tolerant to non-physics people, be inclusive, okay. So, I have to go through this derivation, even though maybe for you it's really trivial. But I'm going to go rather quickly because it's really easy. So, I'm just going to point out the main differences with respect to semiconductors. Okay, so in order to come with resistance or conductance of material, what do we need? We have to know geometry of the material, of course, and we have to know the conductivity. To know conductivity, we need to know what? Mobility, let's assume we know it, we've seen it in the table, and what do we need else? Concentration. We need concentration of carriers. And to calculate concentration of carriers, you start from charge. So what's the total charge in the previous system? What's the total charge in the previous system? You simply say the total charge in the previous system is your elementary charge, which is E, electron charge, right? Times D, because this is a 2D material, To get number of carriers, we need to multiply the surface area by concentration, right? So that's why you have A, which is the surface area of graphene and then finally multiplied by concentration and concentration is P minus N, P is the concentration of holes and N is of electrons, Y minus, because I'm calculating charge, so the charge of holes is positive, comes with a positive sign and the charge of electrons is negative so comes with negative signs. Okay, so let me now calculate quickly the concentration of electrons. So how do you get the concentration of electrons? So basically you say, if you remember, that's the reason why I derived expression for DOS. If you remember, I showed you that DOS times DE is the number of total number of carrier states at energy E in the energy interval DE at energy E, you remember this. So, that means if I multiply DOS with DE, so if I take DOS and DE, I'm going to get how many carrier states there are in the interval DE around energy E, but that's just the number of carrier states. How many carriers do I actually have there depends on their probability distribution, which depends on the temperature, which means that this number must be multiplied by probability of having a carrier at that state and that probability is given by Fermi Dirac distribution function, which is denoted by Fn and it's depicted here. So Fn of E is the Fermi Dirac distribution function, which gives you probability of finding carriers of given energy E, right, and I think you all know this, so this is the, at the top, okay, I put a dose for reference, you know those very well, that's what I derived, and now here is the Fermi Dirac distribution function as a function of energy, you know that the Fermi level probability is one half, and then here if you go well below Fermi level for the electrons probability is 1 and then it goes down and goes 0 here ok, you know this, that's the Fermi-Dirac distribution function and now you basically have to calculate this integral, that's it so how do you do it first you take the expression for those which I did, ok why it's integrated from 0 to infinity why it's integrated from 0 to infinity because look this is a graphing band structure, right and where do we have electrons here? In the conduction band and the conduction band is for energies from zero to infinity and that's why this integral you sum up all this from zero to infinity okay and then when you calculate you put here a dose of E and now this is difference with respect to conventional semiconductors because this density of states look like this that's what I derived before and then you put the Fermi Dirac distribution function which is given by this and then well if you sort it out if you put constants in front you get expression like this which gives you the concentration and now of course the tiny problem is as you know this integral cannot be solved in closed form, this is Fermi Dirac integral, so if you can just simplify it, if you introduce these substitutions, bear in mind this is integral with respect to E, so you integrate with respect to T, which is E over kB T, kB is Boltzmann constant, T is the absolute temperature and the integral is going to be a function of Fermi level, which is introduced here, so it's used here in this substitution X is EF over KBT. So with these substitutions you can easily see at home you're going to get this integral here and this integral here is the Fermi Dirac integral of the first order, the one which unfortunately cannot be calculated in closed terms, so it's this integral here, that's it. I put here some properties of this integral which we are going to need later. So this fermi Dirac integral of the first order is actually the most general is the first order integral of the most general integral which is of the nth order which is written here so if you put here n equal one you will get the one that we need right and then you see the difference is that in the most general case you have t to the power of n I think you know this and this gamma that's the gamma function you know this, gamma function is integral from 0 to infinity of t to x minus 1 e to minus t dt, okay, which for integer is calculated by a factorial but that's not really important you know that's gamma function and one property we are going to need is that if you take the first derivative of this integral you get actually the integral of the lower order by one and the only one which can actually be calculated in closed form is when n is equal zero because then this is one and that's the integral which can be calculated which turn into this logarithm of one plus e to x, okay, that's it. This last one is just the expression which is not really useful, this is just express this integral y of poly, I think it's called polylogarithm function but that's not really important you can skip it anyway basically that's the expression you get for the concentration and now if you would like to plot it I'm going to show you how it looks but basically the only way to plot this is to use some software in which you can define Fermi Dirac integral and they're just just plot this okay I'm going to comment what do we get out of of this very soon. And finally, it's quite often that people use expressions at zero Kelvin, why? Because you see the expressions of finite temperature get involved, this Fermi-Dirac integral, so it's kind of complicated to understand what's going on and it's much easier to look at this situation at zero Kelvin because typically all these integrals become very easy to solve at zero Kelvin, then things become very simple and the main reason is that if you go to zero Kelvin, the Fermi Dirac distribution function becomes just a step function like this. You just get one below Fermi level and you get zero above for electrons of course. And then this is very easy to solve. If you have this, then the entire integral that you had here at the beginning turns into this very simple expression because this turns into a Heaviside function basically, you know what is Heaviside function, step function which is, well let me introduce these two function I'm going to use them later so that there are no surprises. So the Heaviside step function you know this very, oh yeah I don't have a chalk, anyway let me just write down, you can write it, so the heavyset is this, so it's 0 for x negative, 1 for x positive, that's it, that's the heavyset step function and another function I'm going to use simply to simplify this expression is related to Heaviside function looks like this minus 1 1 x and this function is sine x, it's a sine of x, so it's 1 for positive x minus 1 for negative x, so we're going to use them to simplify these expressions. So, now if you look at the situation at 0 Kelvin things are much easier because the Fermi-Dirac distribution function is either 1 or 0, 1 or 0, right, nothing in between and that makes things really easy to do because, which means that this function here is, this function here fn of E is either 1 or 0 and when it's 1 you can basically just write it down like this because it's 1, you don't have anything here and you see now concentration is multiplied by the Heaviside function of EFy, that's easy to understand, because if EF is negative, so have a look at the band structure, if EF is negative, which is here, which is the valence band, how many electrons do we have if EF is negative? So this is the band structure, this is zero, right? And if EF is negative here, and we are at zero Kelvin, how many electrons do we have? We don't have any because conduction band is completely empty and that's why expression is multiplied by Heaviside function and when the EF is positive, this function is one, so then we have some integral. So when the EF is positive, then we have some integral, right? But from where? from 0 to EF because only there we have electrons right from 0 to EF that's why it's written here from 0 to EF and you can also see it here if you move EF to be positive on this side this function will go like this and the only place where it's not 0 it's in the interval from 0 to EF exactly as I put here on the chuck board and that's why this integral goes from zero to EF and then because the DOS is just a linear function of energy integral is a square function and that's why you get here the square dependence on the Fermi level and this is the final expression for the concentration as a function of the Fermi level and now this expression is very easy to understand it basically tells you if the Fermi level is negative so if the Fermi level is in the valence band there are no electrons, concentration is zero, and if Fermi level is positive, so if the Fermi level is inside the conduction band, then the concentration depends like a square function of the Fermi level. The larger the Fermi level, the more electrons there are. That's it. It goes with a square. Why? Because dos is a linear function of energy in case of graphene. That's it. so that's very easy to understand in contrast to this function with Fermi-Dirac integral and that's why you will often see when people discuss graphene they basically mostly use expressions of zero Kelvin and then just assume that at higher temperature there are more carriers due to thermal excitation because this is much easier to understand than this Fermi-Dirac integral and then you can repeat everything as a homework please do it at home, repeat everything for holes, so the difference is now for holes when you calculate integral you have to take the expression for the DOS, okay the expression is the same but what you have to change is the of course the Fermi Dirac distribution function because it must be done, it must be the function for holes which is here plotted in blue, okay so it's this one and then basically you get symmetric expressions so when you solve them please do it at home I'm going to skip this you're going to get very very similar expressions you see in case of holes basically if you compare for instance expression for the holes and electrons you see the only difference is the sign here you have X here you have minus X and here you have EF and here you have minus EF basically just the sign is different, why? Because the band structure of graphene is symmetric, so what happens above zero, what happens above zero is symmetric to what happens below zero and that's why you have only sign change. For instance, let me just discuss the final expression of zero Kelvin, if you look at the concentration of holes, you have heavyside of minus EF, which means that when EF is positive, this is zero, because then this is negative number and heavyside function of negative number is zero. And why when EF is positive, we get zero? Because when the Fermi level is positive and the Fermi level is inside the conduction band and there are no holes, and that's why it's zero. and when the EF is negative like here, when it's in the valence band, then minus EF is positive which is 1 and you have a square dependence of EF which goes from 0 to negative values. So again, quadratical increases as you go from 0 down, concentration of holes quadratical increases with the magnitude of EF, or EF squared, the same. okay so now we know how to calculate concentrations if we know concentrations we can now calculate everything else so the first thing I like to calculate just to write down the expression is for the for the surface charge density surface charge density simply total charge divided by surface area a so it's this expression here and now if you have expressions for P and then you can easily find out that this is the surface charge density at finite temperature and at zero Kelvin is given by this expression here. I'm going to comment this very soon, I will make the plots which is much easier to understand and finally what we were discussing, so what is the resistance of graphene, so when I apply a drain source voltage I get some current, so how much is Vds divided by ID this is R and it's calculated here here I'm just going to write it down via conductance because it's easier to calculate conductance first so resistance is one over conductance okay and the conductance itself is equal to bear in mind this is 2d material so it's with over length multiplying the conductivity because it's a 2d material because there is no cross-section okay with over length multiplying conductivity and as you know there is a limited number of letters in Greek alphabet so to avoid the confusion between the surface charge density and conductivity I put here Sigma CH where CH stands for the channel so this is conductivity of the channel so not the surface charge density in the channel but conductivity in the channel unfortunately the same symbol as for the surface charge density, what can I do? anyway so this is W over L so with over length multiplying the as you know standard expression for conductivity NMU times PA P mu P okay and then you put the expressions we got you get something like this or at zero Kelvin you get something like this and I repeat again this was all derived assuming negligibly small drain source voltage which means the Fermi level is everywhere the same because you know when you apply the voltage there is a shift of the Fermi level by EV times DS and now we can use these expressions to okay this is not important.
\fig{67}{Nanoelectronics of graphene and related 2D materials 2024}
We can use these expressions to actually plot and show what I was talking about. So let's plot first concentration of holes. So the blue curve is always value at zero Kelvin, the orange is at finite temperature as a function of the Fermi level. So what do we have? If the Fermi level is positive, concentration of holes at zero Kelvin is zero. Why? Because if the Fermi level is positive, it's in conduction band and zero Kelvin there are no holes. So that's why we get here. If Fermi level goes down from zero, it becomes more and more negative. So if it goes deeper and deeper into the valence band, then concentration of holes increases quadratically with EF. That's what I showed you. The only difference, if you look at the finite temperature, you get this Fermi-Dirac integral. Basically, the only difference is the entire plot is kind of shifted up, why? Because due to the thermal excitations there are always carriers, so you get something like this. So let me, I think you know this, let me just quickly explain here and then I will skip it for the electrons. So basically if you are at a finite temperature, even if Fermi level is in the conduction band, there is still possibility to have some holes, you see, concentration of course is not zero, why? if for instance Fermi level is here but temperature is finite then you can get thermal excitations right so basically you have the Fermi Dirac distribution function which tells you there is a possibility still to have cold here you are a detail of the Fermi Dirac distribution function you have thermal excitations and therefore you can have some colds you know this if you plot like here the distribution function you will understand because it looks like this right I just plotted vertically because it's finite temperature even if Fermi level is in the conduction band you can get some electrons jumping from valence to the conduction band and you have some cold so that's why at finite temperature the entire plot goes up so that's why I said people usually look what happens at zero Kelvin then they say okay if the temperature is finite they have extra carriers so just shift everything up of course dependence is not really quadratic but it's not really linear translation up but you've got the point okay and then for the electrons is the other way around for the electrons at zero Kelvin of course if the Fermi level is now negative, the Fermi level is in balance band, there are no electrons, thermal excitation is impossible, you have here. If the Fermi level goes from zero up, the concentration quadratically goes up and if the temperature is finite, again due to the presence of thermal excitations, we have a higher concentration of electrons. So even if the Fermi level is here, you can have some transitions and therefore you can get even for negative EF, you can get the concentration. Okay, and then the surface charge density, if you remember surface charge density was proportional to P minus N. Now you have these plots, you just get P minus N, you get something like this. It's interesting to note that even at finite temperature at zero you get surface charge density equal to zero y because although at finite temperature you have both electrons and holes due to the symmetry of the band structure of graphene you have the same number. So the holes compensate electrons you get zero again even at finite temperature. And then of course if you go to a positive EF, when the Fermi level goes into conduction band you have more and more electrons, so these two curves become more and more negative, this is surfer charge density, right? And if the, if Fermi level goes to negative in balance band then they get a positive more and more because you have more and more holes. And finally conductance, this is interesting conductance because resistance is one over conductance so let's have a look at conductance. So the conductance at zero Kelvin is given by this. So you see there is no conductance at zero. Why? Because if you go back and if you assume that mobility of electrons and holes are the same, you can take mobility in front in the expression for conductivity and you will get inside parentheses only N plus P and because both N and P are zero at zero Kelvin, this is the conductance you get at zero Kelvin. And now one correction, this does not include electron hole puddles, okay, because this is correct but this is without electron hole puddles. Due to the presence of electron hole puddles which are not included in this simple theory I gave you, this doesn't look like this. In reality this blue curve is shifted up, okay, it goes up. Why? Simply because electro hole puddles provide carriers even at Dirac point. So basically in reality this is going to look like this. The entire curve is shifted up because of the electro hole puddles. okay and then at the so the blue curve you just get by summing up this for the holes and this for the electrons okay if you have electron hole puddles even at zero Kelvin this goes up and if you have a finite temperature goes even more up because you're summing up this and this and of course conductance is higher so So basically you can see that at the Dirac point conductance is at the minimum. Why? Because the Fermi level is in the region of the smallest density of states and as the Fermi level goes away, either if it goes into the conduction band to the right or valence band to the left, conductance goes up because we have more and more carriers and of course conductivity goes up and therefore conductance goes up, which means that resistance which is one over this, will have a maximum at zero Kelvin. So if you plot resistance as a function of the Fermi level, you will get the curve with the maximum, which looks like this, for instance. So this would be resistance. It's not really symmetric, unfortunately. Sorry about this. So here is the maximum, and then it goes like this. and in a perfect case without electron hole puddles this maximum zero kelvin would go there where would go to infinity because the zero kelvin without electron hole puddles conductance is zero so resistance goes to infinity but it's never infinity because of the electron hole puddles there is conductivity and it's finite okay and if the temperature is so this is this is like zero Kelvin and if the temperature is finite then conductance is larger which means resistance is smaller it looks like this okay fine.
\fig{68}{Nanoelectronics of graphene and related 2D materials 2024}
So now we understood the graphene resistor we know approximately how it works and now it's time to look at really what we are looking for in this course and that's graphene transistor so how to make now transistor out of this resistor well the idea is the same as for the MOSFET. Basically the only thing you have to do is to add the third contact exactly between source and drain which is going to be used to control the charge density in the channel. To do so you have to make a contact which is isolated from the channel so not only contact like source and drain but the contact which sits on top of an insulator and as depicted here you see we have a gate which is metal could be made of the same metal our source and drain for instance, but there is a thin insulating layer between the gate metal and the channel and that creates so-called gate stack. So basically the main idea behind the transistor is the following and I think you know this very well. By changing the potential of the gate, you change the voltage on the gate capacitor the gate capacitor is created by gate, insulator and the channel. So this is something which is called MOS. In the semiconductor industry this is MOS capacitor because you have a metal which is one plate of the capacitor then O is the oxide which is typical insulator like in this case you can put an insulator which is typically oxide and then below is the semiconductor channel which is the second plate of the capacitor. So the basic idea is there is a capacitor here. If I apply a voltage on capacitor, I change the charge of the capacitor. If I change the charge of the capacitor, it means I change the charge in the semiconductor channel. But by changing the charge in the semiconductor channel, we change its conductivity, right? The more charge there is, the higher the conductivity. And it basically gives you a simple way how to control the resistance of the channel by changing the voltage on the capacitor, which is done by the changing the gate voltage, actually potential or voltage between the gates, so we'll come to that in a moment. Okay, so it's a very simple idea and typically in some books you see this kind of approximate of the transistor operation. So you have a carrier flow between source and drain which is a water flow between source and drain and inside in the middle you have something which can control conductivity of the channel. So how much current can pass between source and drain that's it and that's done by the gate which does not leak you know there is no water going in this direction so so it's electrostatically isolated from the channel. And now let me just show you the symbol for this transistor before I explain how it works. So we have again a resistor between drain and source like before. I put a drain source voltage positive, I get a drain current which flows like this and you can calculate the drain current as always as the drain source voltage divided by the drain current. The only catch here is that this resistance is controlled by the gate. That's the point. The resistance is controlled by the gate. In contrast to resistor, in which this resistance does not depend on gate, because there is no gate, here depends on the gate voltage. That's it. Because by changing the voltage on capacitor, we change the charge density in the channel and therefore conductivity of the channel. And in that way, for a given VDS, you have different drain current depending on the applied gate source voltage. Okay, so how does this work? Now to understand in more detail, let me just go from the beginning and now you will understand why I did this resistor. You should understand that if you now put a gate on top of the, if you put a gate stack on top of the channel, you will still pretty much have the same doping as you had without gate. Why? Because those ambient impurities which are absorbed, so it's a physical absorption, which are physically absorbed on graphene will stay there, so even if you put the gate on top that won't change very much. Which means that even if you put gate at the top this is still p-type material which for simplicity I took here to have the same six holes as we had before. So So it's a p-type material and the band structure looks like this and if the gate source voltage is zero, that is if the voltage on the capacitor is zero, there is no induced charge in capacitor so we still have only these six codes. And now let me explain what I mean by the gate source voltage. First of all, in this analysis again, I'm going to assume that the voltage between drain and source is negligibly small. pretty much equal to zero. Why? Because I said already if the drain is at the higher potential of the source then there is a potential distribution along the channel. I would like to avoid this for the moment, I will explain you later what happens when you have it, but now I don't want to talk about this because it just complicates things for no reason. Okay, so I'm going to assume that the drain source voltage is negligibly small and that means if the drain source voltage is negligibly small it means that basically the drain is at the same potential as the source which means that the entire channel is at the source or drain potential call it as you like let's call it source potential and in my plot here in my schematic I ground its source this is typical convention used in common source circuits but it's not really important as I said you can remove this ground it's not important and now I have three contacts of course I apply voltage between drain and source to get a current which flows in this direction but how do I actually control state of the channel well state of the channel you control by applying the gate potential but calculate it with respect to what so the easiest way to understand is to assume that there is a gate source voltage connected here so between gate and source we have a voltage and that gate source voltage will totally control the carriers in the channel why because if the drain source voltage is negligibly small then the entire channel is a source potential which means that the gate source voltage is not only voltage between the gate and the source but it's also voltage between the gate and any point in the channel because the channel is a desource potential okay which means if you look at this gate capacitor here we have a gate one plate we have insulator that's capacitor insulator and then we have a graphene channel which is the other plate of the capacitor that means that this gate capacitor has a voltage gate source because the channel is at the source potential at every point. So Vgs, if you put a voltage generator here between gate and source, it's going to be the voltage on the gate capacitor. Fine. When the voltage Vgs is equal to zero, how much charge is induced in the capacitor? Zero, because you know capacitance expression. So the charge induced in capacitor Q is capacitance times the applied voltage. If the applied voltage is zero, there is no induced charge, which basically means the only charge you have in the channel is what comes from doping, which has nothing to do with this capacitor and that's this situation here. So this is what you have for the Vgs zero. But what then happens if I apply some gate source voltage? So if I apply non-zero gate source voltage, let's assume I apply positive gate source voltage, like convention that is typically used for n-type transistors. If you apply a positive gate source voltage, that means the following, gate source voltage is the voltage on the capacitor, which means the positive charge induced on the capacitor is what? of the capacitor, which is CG, times the applied voltage, so this is QG. Because the positive voltage is applied between the gate and source, it means that this positive charge, QG, is induced on the positive plate of the capacitor, which is gate. But that means also there is a negative charge on the other plate, and the other plate is graphene sheet, which means that I have an opposite charge minus QG induced in semiconducting channel in general if it's MOSFET but here is graphene channel. And now you see what happens. We originally had six holes but now by applying positive gate source voltage we induced three electrons. These three electrons will cancel out three holes and we end up with only three holes. So what happened to this transistor? It turned out now to be less conductive. Why? Because originally had six holes but now we cancelled out three holes by three electrons which were reduced by capacitor, we ended up with half the number of holes from the beginning, only three, between basically this transistor channel is now twice more resistive or the conductance is half of what it was before. So that's how you control the resistance of the channel, very simple. That means that the total charge in the channel, which is very important, we are going to need it for later, the total charge in the channel, let's denote it by Q, is the original charge Q imp, which was induced by impurities plus extra charge of the electrons, which is minus Qg. So it's Q imp minus Q G, which is Q imp minus C G V G S. That's the charge we have here. C G is the capacitance of this capacitor. I will show you later the expression. It's very simple. Like this is a parallel plate capacitor, right? But let's keep this for a moment. What's more important to understand, okay, but what happens to the band structure of graphene inside the channel? What happens to it? Well, now when you apply the positive voltage, it goes down. Why it goes down? Because we have fewer holes. Here originally we had six holes, but now we have only three. If we have only three, it means the band structure must go down so that the Fermi level is closer to the Dirac point and we have fewer holes in the channel. So that means if you count the number of holes here, empty states here, should be six when the Vgs is zero and now if you count it should be only three which means three electrons filled three whole states here and we ended up with only three whole states here and that's how you control the current simply by changing the voltage between the gate and source you change the amount of charge induced in the channel that changes conductivity of the channel and therefore the current for a given VDS. Okay. So if that's clear, then it's time to plot the transfer characteristic of a GFET.
\fig{69}{Nanoelectronics of graphene and related 2D materials 2024}
GFET is simply graphene field effect transistor. So I'm going to use this throughout the course. GFET, graphene field effect transistor. That's what we just discussed. So let's plot the transfer characteristic of a GFET. So what is transfer characteristic? Ah, I didn't say one thing before. the symbol of a transistor I put you see it's a kind of symmetric it doesn't have an arrow because you know if you have an n-type transistor I can draw it I have check now to make a shortcut so this would be n-type transistor right drain source gate and the P type would be like this, source drain gate, it's like this and the current always flows from top to the bottom and the N type flows from drain to source in P type flows from source to drain you know this but so there is an arrow which always points out at the source here there is no arrow why? Because this transistor is symmetric, it has a symmetric band structure and it can be N or P type depending on the position of the Fermi level, where the Fermi level is located and because of that I simply put without any error. Although you notice that the convention I use is the one used for N type transistor when the current flows from drain to source because I have to call both contacts somehow so I use convention for the n-type okay so what is now transfer characteristic of GFET well this is simply IV characteristic of a transistor but the catch is here that this characteristic is much more complicated than that of a resistor because if you have a resistor you know very well what is the only IV characteristic of a resistor is this. So current as a function of Vds is given by this and the slope of this curve is one over resistance, that's it. But this is not a resistor. Why? Because it has three contacts. So basically you have to plot the current, which current first question, because there are also two currents here. One is the drain current, another current is I haven't mentioned so far but it in theory it exists and that's the gate current. So when the current flows between drain and source, that's the current we call the drain current but you can have actually current flowing through the gate, do we have a gate current or not? In this simple analysis no, why? Because I'm looking still at DC, I assume that all voltages applied on the device are DC, there is no time change and because there is no time change this is a pure DC regime which means that the gate current is zero. Why? Because this is a capacitor. You know that capacitor current at steady state DC is zero. You get capacitor current only when you change voltage, so the capacitor charges or discharges, so in dynamic regime there is a capacitor current, but then once steady state DC is established, there is no capacitance current to the capacitor. And that's why the gate current at the moment is irrelevant. that's why at the moment gate current is totally relevant because this is steady state DC the voltages I apply are constant or if they change, they change very slowly so it's like DC and that's the reason I assume that drain current flows from drain to source because in the presence of the gate current source and drain current would be different right so if you for instance orient a gate current to get in then if you apply the current Kirchhoff's law on the entire transistor you will get that the drain current plus gate current is equal to the source current they're not the same but in my case they are the same because there is no gate current we have a capacitor in DC okay so that's why at the beginning I'm plotting only the drain current the only current in this transistor but I have two voltages I I have voltage between drain and source, which is the voltage across the channel, and I have also voltage between gate and source, which controls the conductivity of the channel. In theory, you should plot now a 3D plot to make ID as a function of VGS and VDS, and that would be very ugly because, as you know, 3D plots are very hard to understand. So instead of plotting 3D plots, what's typically done in electronics, you plot two separate plots. you plot first drain current as a function of one of the voltages when the other one is fixed kept as a parameter and then you swap the voltages you plot again drain current now the function of the other voltage when the first one is fixed okay so basically the transfer characteristic is the one which is obtained when you plot the drain current as a function of the gate source voltage when the drain source voltage is fixed. That's so-called transfer characteristic. Why it's called transfer? Because it shows you the transfer effect from input to output. Because the gate is, gate source is the input, drain source is the output and it shows you how the input voltage, voltage between gate and source influences the output current, the drain current. Hence the transfer, the transfer characteristic. Why am I plotting this one first? Because in my case I assume that the drain source voltage is negligibly small pretty much zero so fixed and if the drain source is fixed to value close to zero then the only voltage I can change is the gate source and that's why I'm going to plot first the transfer characteristic of the GFET which means ID as a function of VGS when the VDS is approximately zero it's kept as a parameter which is equal to zero. Okay, let's do it. So let's start from the Vgs equals zero. When the Vgs is equal zero we have situation that we discussed already at the beginning, when in the channel there are only six holes and because there are six holes we have certain conductivity and therefore conductance and therefore some drain current when we apply the drain source voltage. So basically, if you calculate, if you remember the resistance, actually conductance, if you remember conductance I calculated at the beginning by using those formulas, you take this conductance, multiply it by this Vds, which is negligibly small, and you get some, of course, negligibly small drain current, but the current which is here. so that's why for VGS0 we have current here and then what happens if we apply positive gate source voltage, let's say I apply gate source voltage which is this one it's not denoted on my plot but it's value here, maybe I should put it so what happens if I apply this value, I showed you that if you apply a gate source voltage what happens, you induce electrons in the channel which means the band structure goes down and the Fermi level comes closer to the Dirac point, right? The consequence of that was that we had three holes instead of six, which means conductivity dropped down, conductance dropped down, and the drain current, which is conductance times the drain source voltage, drops down. So at this value of the gate source voltage, when this is the band structure, the current must be smaller than here. and if you remember what I told you if this was really six holes and this was three holes it means that this current here must be twice this current here six over three is two okay but let's keep increasing Vgs if I keep increasing Vgs what's going to happen I will induce even more electrons I will cancel out even more holes and the current will keep dropping down until I induce six electrons. For instance, assuming this happens at this voltage here, this is the situation at which I induced six electrons. If I induce six electrons and I had six holes, I end up with zero carriers in the channel, which means I must hit the minimum conductivity of the channel because channel it's not zero conductive you remember I showed you the plots but we have a zero conductivity that is conductance drops to minimum you remember I made a plot and that happens here that happens when at the Dirac point that means that because there are no carriers in the channel it means that Fermi level hit the Dirac point and the only reason we have a finite conductance and therefore finite current not zero current is because of one electron hole puddles which you can't get rid of and two because of thermal excitations. you get something like this. You get that the transfer curve for a Vgs smaller than this value is like this. This is p-type conduction of this GFET p-type because we have a hole conduction. But what happens now if I keep increasing the gate source voltage above this value what's going to happen? I'm going to induce even more electrons let's assume nine electrons. Six electrons will cancel six holes I had originally, I end up with three electrons, so the channel is conductive again and if I have three electrons and the mobility of electrons and holes is the same, I should get the same current as here. So basically it happens this, current goes up and I end up with some electrons in the channel because if you keep increasing VGS, the band structure moves further down below the Dirac point, the Fermi level ends up in the conduction band and you have now electrons, let's assume three electrons, so that means that this current must be equal to this. If you keep further increasing, you're pushing band structure more down, let's assume we induced 12 electrons, six electrons cancels originally six holes, you end up with six electrons, the same you should get the same current as here when you had six holes basically this is the transfer characteristic of GFET here we have n-type conduction is this clear it will be even clearer in the next plot when I show you the charge distribution yes because of the initial doping you remember graphene resistor I explained at the beginning because this doping stays in the channel even if you place gate on the top has nothing to do with the gate capacitor that's extra doping which comes from ambient and it gets trapped by the gate and it stays there okay so that's why we start from this level not from your right if for a local quick exercise assuming this is GFET encapsulated in HBN how this would look like if just n-type basically if you start from here and just going up because if this would be if this were GFET encapsulating in HBN then the starting point for VGS would be this Fermi level exactly the Dirac point so we start from the minimum conductivity and for the positive VGS we go here which means if would like to get p-type conduction you have to go with the negative Vgs okay that would be the transfer characteristic of of encapsulated transistor clear depends on the application if you want to get only n type conduction then yes this is better because you get like an n-type transistor n-type MOSFET for a positive VGS you get current we just increases although it doesn't start from zero unfortunately but I will come to comparison later you will see when I compare both MOSFET and GFET so this shift to the right is because of this shift of the Fermi level here, that's what you exactly correctly pointed out and now we can calculate how much is the this voltage, gate source voltage at which Fermi level hits the Dirac point, that voltage is called Dirac voltage, it's typical in physics denoted by V0 but I'm going to denote it by the threshold voltage of the transistor because of the equivalence with the MOSFET, because when the gate source voltage of an MOSFET hits threshold voltage you hit the minimum conductivity state you turn off transistor right so for this reason I'm going to denote it by Vth but in physics typically people denote it by V0 but it doesn't matter this voltage we have the Fermi level at the Dirac point which means that the amount of the total charge in the channel should be zero from this input theory right. So you simply say if this is the expression for the charge in the channel which I gave you in the previous slide then at this voltage this should be equal to zero this is when we have six holes cancelled by six induced electrons right. how much is impurity charge in encapsulated transistor? 0, how much is the threshold voltage there for? 0, so you get this, the threshold voltage is at 0 the curve is shifted to the left with the minimum at 0.
\fig{70}{Nanoelectronics of graphene and related 2D materials 2024}
And now I'm going to repeat the whole story involving charge density to actually draw what I just told you Basically this is repetition of the previous slide, the only reason I'm going to do it because it's necessary to understand exactly what happens before I introduce finite Vds, which changes potential across the channel, that's the reason. So I have to now repeat, basically this slide is repetition of the previous one. Okay, break. So let's now plot the surface charge density in the channel and the potential distribution so that we are ready to understand what happens with the transfer characteristic when the drain source voltage is not negligibly small. So let's start from the beginning. What you have here is now the surface charge density as a function of x, x points from source to drain. I'm going to assume that it's zero on this side of the channel and L on this side and this L is, if you look very closely, that's actually gate length. Notice that there is a difference between the channel length and gate length because the channel length is from this edge of the source to this edge of the drain and the gate length is the length of the gate. In other words, the gate length L is actually smaller than the channel length. in a good transistor this difference must be minimized because otherwise you get parasitic serious resistance for instance if you look closely here you will see that this distance here is not zero this distance is denoted by LA A stands for access so these are the access parts of the channel to the gate or that's where the source and drain access channel as you like So, it's called excess and that excess length between gate and drain and the same length between source and gate must be as small as possible, hopefully zero. Why? Because otherwise apart from the channel which is gated, part of the channel whose charge density can be controlled by the gate, you have also then two parts of the channel which are not controlled by the gate, which means you have fixed resistances. In other words, that you just add up to the serious resistances in the source and the drain which are not gated and basically just deteriorate transistor properties because what you would like to have device in which current is fully controlled by the gate, but if you have these too many of these serious resistances then you have parasitics which is not gated and that's really not good and that's why in good transistor this is minimized. You may wonder how then in a real transistor this is eliminated, for instance in the transistor of this type, for instance when we make a GFET, basically the only way to eliminate is to have a very good alignment so that the gate really fits between source and drain by leaving here tiny, tiny, tiny gaps. You cannot really put in this, at least in this configuration, you cannot put gaps to be zero because then you will have a short circuit between the gate and the source and drain. In devices in which we have to really eliminate the gap, then we overlap source and drain with the gate, but that's because only we use aluminum for the gate, which oxidizes in air and forms alumina on the surface, which serves as an insulator then between the gate and overlapping source and drain contact. In semiconductor industry, if you look at the conventional transistors, back then when they made the first semiconductor transistors, it was really very hard to fit. So if this is gate and this is, let's say, source and this is drain, it was very hard to fit gate really between the source and drain islands. And then Italian physicists developed method, which is called self-aligned method, in which actually gate was used as a mask for the implantation to create source and drain. Because if you bombard this with ions to implant, to get N plus regions, if you previously make a gate, this gate will actually prevent implantation here and preserve the channel. and you basically get a self-aligned process in which you just make a gate and then just implant everywhere. So there are tricks how this can be avoided, that's the point, and it must be minimized, this axis length. And for that reason, for simplicity, I'm going to assume that this axis length is pretty much zero, that is much smaller than the gate length, and for that reason I'm going to use for L interchangeably, I'm going to call it either gate length or the channel length because I'm going to assume this is very good transistor with access length totally eliminated. Okay, so here we have the surface charge density along X, so along the channel, so this is the direction of the channel and the value at VGS equals zero, also we have VDS zero at the beginning as I already discussed, so the value at VGS equals zero is what? Because the only carriers in the channel we have are impurities, we know that impurity charge I showed you in the previous slide, impurity charge can be calculated as CGVth, right, so you can write down Q impurity is CG times Vth and then surface charge density of these charges is when you divide this by the surface area of the gate right of the channel but that's the same as the surface area of the gate it is this right and then you have to look at the expression for the CG, so now it's time to tell what the CG is, so CG is the gate capacitance, okay, it's sometimes called geometrical gate capacitance because it depends on geometry of the gate and this capacitor, okay, it's not written here, this capacitance is what, this is, we have a parallel plate capacitor so this capacitance CG is the surface area of the gate which is the same as the surface area of the channel as I said so it's a times epsilon of what of the insulator let's call it epsilon ox because typical insulator is oxide if it's HBN that's not oxide but because in semiconductor industry usually this notation is used I'm going to use it all the insulator doesn't have to be an oxide okay and divided by the thickness of the oxide T ox and then when you divide CG by a you get epsilon ox over T ox and that specific gate capacitance this is the gate capacitance per unit surface area which is denoted here by C ox okay so now we know epsilon ox is the epsilon of this insulator T ox is this thickness and a is the surface area of the gate which is if this is the gate okay I have a picture here I didn't see okay so epsilon the relative sorry not that for the epsilon ox is the the electric, it's a channel permittivity, okay, let's say it like this, is a channel permittivity which is equal to relative electric constant multiplied, of course, by the vacuum permittivity epsilon zero and divided by T ox, T ox is the thickness of the insulator, okay, and A, which is the surface area of the gate, is equal to the gate length times the gate width which is the same as the channel width and channel length, okay. As I said I'm going to interchangeably use for W channel width or gate width and for L channel length or gate length. Now in applications this COX must be as large as possible, why? what's better if the gate capacitance is smaller or larger? yes you have more control over the carriers in the channel because the larger the capacitance more carriers you can induce of the same gate source voltage and that's why in a good transistor this gate capacitance must be as large as possible. Of course when I say as large as possible I mean C-Ox because A is fixed by geometry so basically what you can really change is C-Ox. It means that in ultra high performance transistors you have to have very large dielectric constant and very thin insulator so that's why now in semiconductor industry as insulator hafnium is used or hafnium oxide HF02 which has a dielectric constant about 20 and the typical thickness is close to 1 nanometer slightly more than 1 nanometer originally industry used silicon dioxide because if you use the silicon dioxide as the insulator then you can use polysilicon as the gate and that's very comfy because then everything is made of silicon basically you have oxidized silicon which is the insulator and you have polysilicon as the gate and that was extremely comfy because you stay with the same materials like silicon you just oxidized to get oxide and also polysilicon was great because depending on the doping of the polysilicon you can change the work function of polysilicon basically controlling the threshold voltage of the transistor to be exactly what you need. However this Kashi period of industry has gone has gone been lost why they cannot use anymore these materials why because for high performance transistor you need the large co2 so basically it turns out that in order to make high co2 for high performance transistors the thickness of silicon dioxide should drop well below one nanometer and then you have a huge problem with the tunneling which means that then you have a huge gate current even in DC, because when you apply a gate source voltage you get a huge gate current tunneling because of very thin, very thin insulator. And in order to fix the problem they had to switch to high K-oxide like Hafnir, because if you increase relative electric constant then you can also increase thickness to get the same C-ox. by increasing T ox you suppress tunneling current and that's why Hafnir is now used and the problem with Hafnir is first that's not silicon-based material so it's not simple to introduce it in the process and the top of all that it's not at all compatible with polysilicon. So now the gates are made of metal, so the transistors now have a metal gate like titanium for instance and then you have half Nia's the insulator that's the only way okay anyway the point is remember C ox must be as large as possible for a good transistor okay and now let's what we have here so finally impurity charge density Sigma imp is CJ divided by a times V th in the absence of the gate source voltage and CG over a as I already wrote here is C ox so there you go, impurity charge density is C ox Vth, then you plot it as a function of x, that's a constant y, because we assume channel with homogeneous properties, which means the density of holes does not change across the channel, it's the same everywhere and that's why this value is valid for all x, that means that these six holes are homogeneously distributed along the channel and we have a constant sigma imp which is given by this simple formula in the absence of Vgs, okay fine there's a surface charge density in the channel then potential distribution across the channel luckily in this case they're investigating situation in which Vds is approximately zero which means that the drain potential which is fixed by the metal contact of the drain and source potential fixed by the metal contact of the source are the same. So this source potential on this side is the same as the drain potential on this side and that basically means the channel is, the channel potential is basically fixed by the two potentials of the source and drain which are the same, it means that the channel is equipotential, it's the same potential as source and drain. Conveniently you can put Vs, source potential to be zero, which means you ground the source of the transistor but that's not required at all as I told you you can skip this zero everything you'll still hold okay that's the channel potential then we have the voltage what's important to understand we have a voltage on the gate capacitor okay and I'll be very careful what's the voltage on the gate capacitor because the gate capacitor is capacitor form with gate insulator channel voltage on the gate capacitor is the voltage between the gate and the channel because gate is one plate channel is the other the voltage on the gate capacitor is voltage between the gate and the channel which is simply potential difference between the gate and the channel okay luckily in our case this is equal just to Vgs right why because in our case channel is equi potential which means that the channel potential is the same as the source potential okay and for that reason V channel is V source is the source to get Vg minus Vs which is Vgs how much is Vgs right now right now we have Vgs0, is this, is 0. So, this is 0, this is the voltage on the gate capacitor, okay. And then let's discuss finally the surface charge density in these points here along the transfer curve. So, in the first one we have this situation at the top, of course, we have surface charge density already plotted at the top that's when the Vgs is equal to zero. Now notice by Sigma of X I'm going to denote surface charge density in the channel which is equal to Sigma imp when Vgs is zero because there is no charge currently induced by the gate so they are the same remember what I wrote you before Q total charge in the channel is impurity charge minus Cg Bg channel right minus voltage on the gate capacitor which is voltage between the gate and the channel okay so I am going to define by Sigma surface charge density in the channel which is obviously Q divided by A so it's Q imp divided by A which is Sigma impurity minus CJ divided by A is C ox times voltage between the gate and the channel so that's what's plotted here that's what's plotted here of course in our simple case because in our simple case channel is equipotential this is Vgs okay because the channel is a disorbed potential so you get that this is Sigma in minus C ox Vgs and in our initial case plotted here when the Vgs is equal to zero you have just Sigma in you basically have that this plot is this that's this point here that's point That's what we had at the beginning, I hope this is clear. And now the reason I introduced this sigma is because it's going to change when we apply some voltage, right? So what's going to happen? If we apply voltage, that means if you apply some gate potential, if you for instance connect the gate and source via generator and apply some voltage between the gate and source, then the voltage between the gate and the channel is going to change, but because the channel stays at the source potential, it's fixed, voltage between the gate and the channel is still just Vgs as I said, but if Vgs is not zero, we get here voltage on the capacitor, which is equal to Vgs, right. So originally we had Vgs zero and now as we increase, as we go to the right, as we increase the gate source voltage, this level which is the voltage on the capacitor just goes up, in all points just goes up. Equally because the channel is equipotential, so all points in the channel are the source potential between voltage, between the gate and the channel which is voltage on the capacitor is Vgs and as you increase Vgs simply goes up, it's this level here and then what happens as you increase Vgs you remember, as you increase Vgs which is voltage on capacitor you induce electrons and that means the surface charge density in the channel goes down. Okay, so for instance if we induce three holes in the channel, if you remember, we are going, sorry, if we increase Vgs at this point we induce three electrons, so So we originally had six holes, we induced three electrons, we end up with only three holes. Assuming again everything is homogeneous, those holes are again homogeneously distributed across the channel and we have a now smaller surface charge density which corresponds to these three remaining holes located here. if you want to really calculate mathematically this you can do use this expression which comes from what I derived here because Sigma imp is C ox Vth because Sigma imp is C ox Vth you can take C ox in front and you get Vth minus Vgs that's what written here and now as Vgs increases Sigma decreases we inducing electrons canceling holes here we induced the electrons we cancel three holes and up with only three holes and we have a lower surface charge density and that corresponds to 0.2 and then if we further increase and apply gate source voltage equal to the Dirac voltage Vth then we know that at this point we induced six electrons we cancelled all six holes and that means that happens when the Vgs is equal to Vth when the surface charge density then is zero and then in point three we have a zero surface charge density which is what you have here okay and then if you further increase if you go to point four then at that point we induced the nine electrons in the channel minus six holes that gives you three electrons you basically have the same conductivity at point two because you have three electrons instead of three holes but the supercharged density now goes negative because you have electrons which you can get also from here if the Vgs is larger than Vth this number is going to be negative and then finally at 0.5 we induced 12 electrons minus 6 holes we end up with 6 electrons so 0.5 is equivalent to 0.1 we have the same conductivity right the same number of carries just its negative because we have electrons, that's it. So that's what you already understood from the previous slide. You will understand very soon why I draw all this because when the drain is not of the same potential of the source we are going to have some big change here and then situation will start from this and then that's why I would like to discuss it. Anyway finally just to give you a hint about the symmetry of this curve so far I assume the curve is symmetric this transfer curve is symmetric and that would really be the case of HBN if you have a device in which a graphene channel is encapsulated in HBN this curve not only has a minimum of zero but it's fully symmetric like a P branch is identical to the electron M branch however if you put the device on silicon dioxide then it's not symmetric anymore I have to tell you that in reality electron currents are going to be smaller in other words at point four you will get a smaller current than compared to point compared to point two and smaller current at point five compared to point one why why the current is smaller if charge density is the same That means that in graphene on SiO2 something else is different. Density is the same, so what's then the difference? The only remaining thing is mobility, which means that if you place graphene on silicon dioxide, most likely electromobility is going to be smaller than whole mobility despite symmetric band structure. I'm going to explain why and because of that conductivity here is going to be smaller, so that's why I wrote here that these points 4 and 5 should be below point 2 and 1. In other words, it means if you plot it, actually when you measure the transfer characteristic of such graphene transistor, you get something like this. you see here the current is smaller for the same difference with respect to the you see with respect to the so this part of the p branch is steeper than this one here so the slope here of the p branch magnitude of the slope is higher than that of the electron and that's the consequence of this asymmetry in carrier mobility and now you can immediately ask me how is this possible because graphene has symmetric band structure so how come electromobility is smaller well that happens due to the scattering which was not introduced in previous considerations the problem is that because of this very strange wave function of carriers in graphene it turns out if the carrier spends more time along close to the carrier center then the scattering rate is higher and if you have a graphene on top of SiO2 the carrier scatters the carrier sorry the scattering centers are positive charge inside the SiO2 so that means if you have a hole, hole will scatter like this on a positive charge it will be repelled and the electron will change trajectory like this basically electron spends more time because of the attraction than hole which is repelled in conventional semiconductors that doesn't matter whether it goes like this or like this you get the same scattering rate but in graphene because of this strange wave function scattering rate for electrons is higher and therefore mobility is actually smaller which is the reason why these currents are smaller okay but that's something you can see only in practice when you measure transistor you see that n branch of the curve is somehow with a smaller slope than the p branch, okay. Only on SIU2. If you have only on substrate in which you have positive scattering centers. In HBN you don't see it because HBN does not have scattering centers for graphene, so you get a fully symmetric curve like this one. You can also get it for other reasons, you have to be careful because this is a very simple theory. I can show you some transistors in which you have encapsulated graphene yet the curve is asymmetric, but that comes from contact because if you make ultra-short transistor, very short, then you have a problem that contacts are very close to each other and if you use a the gold contacts which are usually best for graphing to minimize the contact resistance, gold will slightly dope channel next to the contact into the p-type. So you basically get some sort of pnp structure. So the channel is n-type for instance, but next to the contact you have a channel which is p and because you have pnp you have a higher scattering than if you have a p-branch which PPP okay and the consequence is again a smaller electron current so there are many reasons in practice that could reduce mobility which are not really included in this theory but what I'm telling you now is enough to understand the operation of device okay fine and now before I show you what happens with what happens when we apply a finite VGS I would like to discuss one point which was actually wrong I said something and no one on this slide had objection which one what's actually wrong here in this slide well if you remember I said when the VGS increases the band structure, okay you cannot see it here maybe we should go to the, so many animations, we should go to slide before this one, you see I said when the VGS increases the band structure moves down right, so now have a look at this slide VGS increases band structure moves down potential in the channel stays where it is. Do you understand now the problem?
\fig{71}{Nanoelectronics of graphene and related 2D materials 2024}
So if the band structure goes down that means potential energy of the carriers changes right but the potential energy of the carrier in the channel is given by the channel potential so how come channel potential doesn't change but the band structure changes, how is this possible? The answer, this is not possible. In other words, in reality the situation looks like this, if the band structure goes down by some value and if the electron charge is minus C, then this value which goes, by which the band structure goes down, I can denote by minus E V channel source and the V channel source is the voltage in the channel voltage between the channel and the source. Now this makes sense because if the carrier is in the channel that's the potential energy of the carrier in the channel when you multiply this by the carrier charge which is minus E you get by how much the band structure goes down you understand in other words what I want to tell you is this band structure cannot go down unless the channel potential goes up because the band structure goes down by the potential energy which changes so that means that the voltage between the channel and the source goes up as the band structure goes down so as the voltage between the gate and source is increased the voltage between the channel and the source in the channel also goes up and now this may introduce some confusion from the physics point of view this must be clear right because when you take the potential when you take potential and multiply it by the charge you get the potential energy so these two must fit together what is not clear is the following I said this is capacitor right have you ever seen this phenomenon in a regular capacitor so let me tell you what I'm talking about so let me now draw the regular capacitor to understand this point so here is the capacitor gate channel okay this capacitor has two contacts gate and let's say this contact is sore okay I apply a voltage on capacitor VGS what I say is that the potential of this plate does not stay at the source potential but actually goes up this one potential will go to the gate potential this is clear but why the potential of the channel will go up, it must go up otherwise you don't have the band structure moving down because that's the basic physics, you take potential energy, calculate as charge times the electric potential right, but how come you don't see this in a regular capacitor, so imagine this is now a regular capacitor, this is now metal capacitor, what I'm telling you is that when apply a voltage on metal capacitor this plate does not stay at this potential but slightly goes up, have you ever heard about this? Never, what you know from the metal capacitor that you learned maybe during your first year is that this potential will be equal to this, that this potential doesn't go, basically this potential as it goes up draws this one slightly above, so why do you see it here and you don't see it in metal capacitors, so that's the answer. Why do you see it here and not in metal capacitors? Why in metal capacitors this doesn't happen and in this capacitor it happens? Actually it happens only in, it happens also in metal capacitors but you don't see it because of the huge density of states in metals which make this potential increase negligible. this is now going to be clear, physically what does it mean charging capacitor? When you apply a voltage on capacitor you are charging it by this much, right? The band structure moves down and this here which is now denoted by QG that's the charge that these are the electrons which filled states in the valence band and it can be calculated through capacitance relationship right that's the induced electrons that we induced by applying a voltage on capacitor and now the difference in quantum systems like graphene that is in low dimensional systems the density of states is rather small which means in in order to get charge that you pump inside like three electrons or six electrons, because of the small density of state, you have to push band structure a lot down, which is the reason that this eV channel source is not negligible, which is the reason that this voltage will be observable here. While in metals, they have such huge density of state that you basically don't have to push band structure down much to get the same amount of charge. You have to just negligibly small push band structure down. In metals, because of the enormous density of state, this V channel source is pretty much zero, almost zero. It's negligibly small. And that's why in metals you don't see this. You don't see this, that the channel is not of the same potential of the source. it's negligible you can't see but in graphene especially in graphene because it has a zero density of states here you have to push band structure a lot down to get the charge pump defined by the capacitance relationship right so those three electrons I mentioned six electrons and so on in order to get them to fill this state you have to move band structure down a lot and that's why this This is visible in low dimensional systems. This is the reason why capacitance, which I'm going to introduce soon to model this phenomenon is called quantum capacitance, because basically this effect is visible only in quantum systems, because only in quantum systems you have a low density of state. In systems with high density of state you don't see it like metal. But you can see it in MOSFET channel by the way, it's possible to see if you have a poor silicon channel with very low density of states, I mean if the channel is very poor you may actually see it there, but it's mostly observable in quantum systems. And how do you model this? Well modeling is quite trivial, basically the modeling only, the only thing you have to do in modeling is to understand that in reality channel is not at the same potential as the source, that's it, that's the only thing, you have to understand the channel is not at the same potential of the source and to calculate this mysterious quantum capacitance you basically just apply voltage Kirchhoff's law, that's it, there is nothing quantum about it in terms of derivation, you just use conventional voltage Kirchhoff's law which tells you what? It tells you that the voltage you apply on the device, which is the voltage between the gate and source, is actually voltage between the channel and the source and voltage between the gate and the channel, that's the voltage kickoff law, which is written here. Voltage between gate and source, which is the only voltage we can control because we have access only to the gate and the source. This voltage is this voltage, voltage between the channel and the source plus the voltage between the gate and the channel. And now here is the main difference, the main difference is that the voltage on capacitor, pay attention now, is not Vgs, the voltage on the capacitor is what I already told you, it's the voltage between the gate and the channel actually, because the channel is the other plate of the capacitor, not a source, source is not of the same potential as the channel. The rest of the derivation comes from here and it's based on this understanding that voltage between the gate and the channel is the voltage of the capacitor then capacitance relation is valid for this voltage not voltage between gate and source there is no capacitance between gate and source it's only between gate and the channel and because of that you use capacitance relationship for the voltage between the gate and the channel you know that voltage on capacitor is the accumulated charge which is QG divided by capacitance CG that's it understand this however before I now show you what the consequence of this I have to tell you that this relationship here is actually used typically in its differential form it simply means you take differential of it to define quantum capacitance I can actually use the original relationship and divide and derived from their so expression for the so-called static quantum capacitance. Capacitance in DC but this capacitance in DC is not that important, what's more important was the capacitance in AC and in AC you are interested in changes in the device when you change the voltage, that's AC and for that reason instead of using static relationship which is DC, the one I wrote at the beginning, I will just take differential of it because differential indicates change that's what we have in AC regime because when I apply AC voltage on this device I would like to know equivalent scheme of this device and the equivalent scheme is going to involve some capacitance and that's why I need capacitance is in AC not in DC okay because capacitance is in DC are open and you can solve circuit without them but here I would like to know what's going on in AC so then what you do you just take differential okay this is very simple dvgs a dv channel s plus you take differential of this dqg divided by CG, CG is geometrical gate capacitance which is constant so when you take derivative you just get this expression here and now you divide everything by dqg you get this expression. Here on the right hand side we have 1 over Cg which is the 1 over the gate capacitance, these two expressions have dimension of voltage over capacitance and therefore the entire equation can be simply written that dvgs over dqg is 1 over C where C is the total device capacitance or capacitance between gate and source, why? Because look at the expression, this capacitance tells us by how much charge in this system changes if we change the voltage between the gate and the source. So this expression here models the capacitance between gate and source, which is simply denoted as C. So 1 over C is 1 over CQ and this 1 over CQ is the quantum capacitance, that's something you don't have in metals and because of that in metals you get that 1 over C is equal to 1 over CG, that is capacitance of the device is simply geometrical gate capacitance. In quantum systems you have one extra capacitance here, which shows you by how much the potential in the channel changes to get change of the charge dQg. Why in metals this does not exist? Because how much is in metals dV channel source? By how much you have to change band structure in metal to get Qg, by nothing basically, because we're in normal density of state. Because of that in metals this dV channel source is zero, because in metals band structure basically does not move down at all because of the high density of state. So in metals there is no this term or you can say that in metals quantum capacitance is how much? Infinite. But in metals there is no this term. In this system we have to take it into account. In other words, you see now what we can do. If this is gate, the contact we have access to, and this is source, another contact we have access to, we know that in between we have the channel because this is source, this is gate and here is channel in between. So here is channel in between, this expression here is what? That is the equivalent capacitance of two capacitors connected in series, right? So you can say that the capacitance between these two contacts is simply series connection of capacitance of capacitors connected in series which are, which have capacitance is CG, geometrical gate capacitance and CQ which is the quantum capacitor, okay. In metals CQ is infinite and you know that in series connection capacitance is always dominated by the smaller one, so if this is infinite it means it's gone, in other words it was in this circuit, it's all short circuit, not there. In metals you don't have it, but here in graphene and other 2D systems you have it because of the limited density of state. So you see this quantum capacitance is nothing mysterious, it just comes from voltage Kirchhoff's law and understanding that the band structure must move down by charge times potential electric potential to get the potential energy, that's it.