\fig{9}{Nanoelectronics of graphene and related 2D materials 2024}
And that's why now industry goes downright. So now they're going to make it horizontal and what we're going to discuss now are so-called nanosheets. This is the latest development, but we most likely will reach next year. That's that's a prediction 2025. So here I put the evolution of the MOSFET so you can understand how the technology is progressing. So what you see here on the left hand side, that's your own good textbook MOSFET. Actually there are two MOSFETs. So here is the three-type MOSFET, here is the n-type. We just put P and N next to each other because this is typically we used to make a CMOS inverter, which is a typical structure used to benchmark all the pieces of the power device. So this is the so-called Wiener device, that is the one you can see on the next books, because the silicon channel is here and here, so this is the area controlled by the gate, which is, as you understand now, not so really good when you make very small devices. And then the next development, the next development was to make pins. So now you have pins, so this is the basically anti-consistor made of two pins. So you have like two pins in power that make a single transistor, and you have the same here for the three-type. Why, by the way, why there are two pins, why not just a single pin? Well, typically because the pin has to be very thin, so the current which flows to the pin, sorry, to the pin is not extremely large. And that's why it's better to have more things to identify. Why? Because as you know, for instance, if you make any logic gate, this logic gate does not work alone. There is always another logic gate connected to its output. And that means that the currents, the on currents of the previous logic stage, is used to charge and discharge the gate capacitance to the following stage. and if you want your audit to work at high frequency, this on current must be very large. You can quickly charge and discharge the capacitance of the power station. And that's how you can have a thin pin with a current which is not very large, because this is a thin pin, and you need a few more pins to add to get a current high enough that can exist. Okay, so that's the reason why we have here two pins for one transistor and two pins for another one. Okay, so this is clear, that's me in fact, that's what I was explaining to you. And now, the final thing, the next development is to not make them upright, but basically make them fit. So now again, if you look at this, this is the cross-section, so the current again flows perpendicularly to the edge, so that means that the source is, for instance, on this side, and probably it can be inside or behind. And what you see here are nanosheets. There's literally a nanosheet or a nanoribbon, as the Intel calls them now, which are stacked one on top of the other. So first, why are they stacked one on top of the other? Because if you look at the nanosheets, they are rather small, so you get a large current. You need to stack several of them to make a single transistor. basically if you look at the fin, if you cut fin like this you get nano sheets so this is one nano sheet okay and because this is this even has a smaller cross-sectional area than the fin you need to put many more nano sheets to get similar current so basically in 3D this nano sheet transistor looks like this, you simply have a nano sheet which is a semiconductor channel which is very thin so you're talking about 5 nanometers okay the current flows in this direction because for instance sources so for instance this is the n-type transistor then the current flows in the opposite direction of course the electrons which flow from source to drain and the current will flow from drain to source and then of course where is the gate and I look very broadly at the gate, so then the reason why the structure is made horizontally, the gate is all around, you see? So the entire thing, sorry, the entire membership is encapsulated by the gate. So that's why the gate has a full control over the carriers in the channel, because they have no place to escape in the area which is outside of the gate control, like here. Here was a problem with fins, there, it's nowhere, just have a very thin chip. Then of course they stack several nanos on top of each other to connect them together so you get all these nanosheets that you have in the single transistor that you can't charge enough to operate the high frequency. Then next to it you have a three-type transistor and then the nanosheet CMOS. That's what you bring to CKR with the next year. someone's trying to enter the lesson sorry? someone's trying to enter the lesson hmm okay I suppose it's clear now I think first you should have the technical skills and then you can guess okay So that's how we make nanosheets consistent. And now you want to understand where you see it going. Where this is going. Where you think you're going in relation to the materials. Because if you'd like to have a full control over the carriers, you have to make this nanosheet as constant as possible. Because the thinner the sheet, the closer electrons are to the carriers, to the gate. So they are at least fully controlled. And that's when this thickness drops to 2 nanometers, you have a problem with silicon. Because then mobility of silicon rapidly drops down and then you have to use other materials. Because silicon is not going to be able to sustain this. So this is probably going to happen soon. I'm going to discuss very soon this. And then finally the next improvement, so after this, which this technology is probably going to appear on the market next year. After that, now they're going to know who the vertical segment is. Basically, in order to crash more devices on a chip, because it's very hard to make that lateral smaller, we're going to talk about that in a moment, we're going to put them on top of the other. Because if you already stack chips on top of each other to make multiple chips for a single transistor, why don't you then stack all of them on top of each other? First p-type and then n-type. So basically we put a p-type transistor at the bottom, n-type transistor at the top, and we get the CMOS inverter which is vertically integrated. It's not any more or less vertically integrated, but perpendicularly, vertically. And that saves the space. Because this CMOS inverter takes half of the space in this one. we can put more components in the chip, we can double the number of quality cores, for instance, in the micro-cores system. And that's going to appear, that will be the next generation, I will give you approximately the timeline soon, and this is done simply to put more devices in the chip, because it's very hard to scale them down longer. And now I can maybe finally address this issue about scaling down where is the bottleneck now. I explained to you the last time that the bottleneck recently was the wavelength of the argon fluoride exomer laser, which was used to expose the structure to the mass, and the problem was the desolator had a frequency of 100 nanometers, and they managed to make structures like 30 nanometers release, not by breaking the law of physics, but by using very strong resolution enhancement techniques. That's basically a lecture on it's own. I used to give lectures on this, how this is a traditional enhancement technology work. But we don't have time now for this. That's basically a total overhaul of the entire exposure system, how to make something of the 30 nanometer scale with a very little company in mind. But for the total scaling, they realized that that's not going to work. and they have to reduce the, obviously the only way to reduce the structure size is to reduce the wavelength of the laser, right? If you have a shorter wavelength, you can make smaller structures. There are some ideas to use cryptofluoride as a emulator, because it has a lower wavelength, but that would be a big thing to get out, because also, now, from this, and the wavelength of electromagnetic radiation drops to 159 meters, this radiation gets absorbed by oxygen and everything, including air. And that's a conflict that is totally inevitable, because you have to put the entire exposure system in vacuum, understand, because when you're doing it in air, the radiation will get absorbed by air and so on. And for that reason, because of this huge problem, they decided to drop a crypton fluoride laser, which kept around 150 nanometers away. they decided to go directly to XGIM, after a while, to photograph these. Art standard, this is basically using the X-Folding structures by using soft X-rays, because the wavelength is only 30.6 nanometers, so that's the current technology. This technology has been under development for the last, like, 30 years, because everyone understood, all better 30 years ago, that that's the way to go, you have to really reduce the wavelength, but the problem was that, because of the absorption of the radiation, that's very powerful too. And just a few recently, first the proper extortion machines were made. Let me just give you a few examples of the problem. The first problem is of course air, the radiation gets absorbed by the air, so the entire exposure system can't be played in vacuum. I mean that's not a problem, but it just complicates everything because the system you put away for a few calopsies and so on has been vacuumed, so that's the first complication. And the second complication is optics itself. Why? Because the standard optics, you put a mask made of sapphire, which probably you've seen, I told you already, and you shine a laser on top. But if you have an ultra-short wavelength, like that of a big ultraviolet, it can be totally absorbed by any optics you put, by the mask, by the lens, whatever you use is going to be absorbed. And that's why they had to actually develop instead of transmission optics, they had to develop reflection optics. Which means that instead of using for instance a lens to focus the radiation, you have a mirror which does this. Because if you let radiation pass through the lens alone, it has to be reflected from the mirror. And considering the wavelength is only 30.6 kilometers, you can already guess what's in the mirror. It's a private question. That's why it's been developed recently, basically, and it's extremely expensive. There is only one company in the world that can actually make this exposure machine. It's a Dutch company, ASM Health. They actually make these big exposure machines for CMOS and Royal Photography, for all these big sensors and other components. One machine like this, just for the exposure, costs about 20 billion euros. It's an extremely expensive thing. And because of all these problems, you basically saw X-rays that just were introduced like in the last 5 years. Also, what's the source? That's another interesting point. What's the source of this radiation? How do you get it from the X-rays? What's the source? Basically, it's plasma. You have a highly ionized plasma, which decays, and then you get a broadband radiation, and then from this broadband radiation, you use these reflectors to filter out 30.6 nanometers. I mean, it's really four other levels of conventional optical technology. And that's not it, because we really talk about really scale-down structures. Now the industry is considering virtual integration to be able to increase the number of devices. And I want to show you that this is not just some story like this, that's actually happening on this peak. If you look on the left, so that's a demonstrator four years ago done by Intel for this multi-production at the total demonstrator. So they made, you see, the entire CMOS using nanosheets, transistors. So as you can see, this cross-section, there you have three MOS transistors at the bottom, and N MOS transistors at the top. These are the nanosheets, the channels, and you see they're surrounded by the gate stack. So you have a gate inflator, a gate metal, and then and then everything is inside the, inside the, uh, intoiter. By the way, this SPI, if you can see here, that's also intoiter. SPI stands for shell attention. But that's not important. And, by the way, why do you always put, why do you always put NMOS on top of PMOS? Why not the other way around? There's a very simple reason, I don't know. In silicon, electrons have higher mobility than electrons. And that means if you want to make an N and T MOS transistor in the same drive pattern, you have to make T MOS transistor channels wider and bigger, so that you compensate for smaller mobility in areas of T type C. And when you make these structures, it's not so not so visible from these images, but usually they go like this. That's why you typically put three modes, and that's why you typically put three modes on the bottom and one on the top, to get the same part at the same voltage. So of course this was done in silicon, so that's demonstrated in silicon, and for instance And this year, PSM-C, the largest semiconductor plant in the world, they actually demonstrated gate-overarm structure, not with silicon, with actually molybdenum disulfide. So you look at this paper, published this year, you can see, of course you don't have working on staining here, but you can see gate-overarm beds, and inside they use molybdenum disulfide rather than silicon. So you see, if a big component is, is, is based in time, to the material, then there is something. That's why the student team is going to run out of steam soon, and they have to switch to new materials, and they're always testing the level of lignin and disulfide very seriously. And I'm going to show you the ID first, they're actually quite good. We're quite good and the expectations are that in maybe 5 to 10 years this material will replace silicon, that silicon won't be able to work at small thicknesses. You can see here, these are just a signature of materials here, so you see in the center there is a molybdenum and also sulfur, so that's a signature of molybdenum disulfide. And then around, there is a signature of K, that's because the gate insulator is K, I would like to say, it's not an A insulator. And then finally, all over the structure, they came and used the method they used for gates. So that's the point. The point is, you know, yet another technology goes about something which is going to be used who knows when. you know, like, people don't know what they're using. Like, it seems like 50 years ago, they keep doubting when you get into the 20 years. I hope that this is not going to be the case. Although silicon was quite resilient, people used to discuss the thing that silicon is going to run out of steam long time ago, but silicon is still a dying force of the electron. A little later, I'm quite certain that something will be fixed. So we have to replace the material.\\
Anyway, I hope that was the introduction, that you hopefully like this. The basic feature of NORM in general way is for us to have multiple different... Because the RMS is something that the current control should be very small. A transistor must have, you know, a transistor must have very high form of ratio, very large difference between on and off current, to be able to have a distribution of the two states, period. But also, at the same time, not only this difference must be large, the on current must be large. Because if the on current is small, like I said at the beginning, you don't hear problems, so the difference is small. Then you have a small current to charge and discharge the capacitors of the fault towards the state. That means that we will reduce the bandwidth. Do you understand? If we have a man-line current, it's going to take forever. It's going to take some milliseconds to charge a discharge capacitor. But we need to charge this charge in less than a second. I need one half second to at least appear for the operation. That's the problem. So that's how you can stack several of them, connect them parallel to get reasonably large time to get used to the cost of operation of the product. Okay? Is it clear now? Fine.
\fig{10}{Nanoelectronics of graphene and related 2D materials 2024}
And if you would like to look at the future, there is no better place than to look at this table. This is a very cryptic table. So you see here in the table you can download from... I took the link, no? Well, just type in Google IRDS 2023. IRDS is International Roadmaster of Devices and Systems. So IRPX is a sort of consortium made of the biggest semiconductor companies like Intel, TSMC, Samsung and so on. So once per year they sit together and try to foresee the future, to understand what the technology is doing, what kind of transistor we want to be making in like 3, 5, 10, 20 years from now. This may sound strange because they are competitors, why do they sit together and try to foresee standard for countering this, but you have to understand that this must be done simply because they make chips, but they don't make tools for making chips. And the tool makers must know where the industry is going to understand what kind of food they have to develop. So that's why they've been banking about extreme ultraviolet since, I don't know, the 90s? And that's why companies like ASML managed with a huge investment of all these companies, of course, to develop extreme ultraviolet light tools. And that's why this is a very nice look into the future. That's what I think today is going to have in the future. So what I downloaded here is a table from, so just type in Google IRDS 10.0.3, you will get lots of documents. So, events 1, that is called more bool. So, you have more bool, you have more than bool, and you have beyond CMOS. So, the first part here, more bool, simply means how to get more out of bool flow. Another word, how to use present, present to reveal the core chip to continue device case. That's it. How to keep going in bool flow, we will pick up today. That's more bool. Then you have a modern rule. Modern rule means do more what the rule block allows you, not what the rule block provides you. In the sense, I mean, some sense of the modern rule, but it's actually quite simple. It simply means how to integrate different functionalities into the same thing. So think about a microprocessor with graphics card integrated inside, or maybe find chip interrelated inside of some central interleaved something. We're talking about systems on chip, as I was saying. So more than two is systems on chip. So how the existing chips can work functionality. And the final one, we all see most of the total exotic technology, which means the breakdown of the current technology and think about something like blue sky, quantum computing. I mean some of you took previous courses, so we are going to see quantum computing more because you can solve some exotic problems but not the real life problems. Some people are very skeptical, some are not. The real problem is that the current quantum computers you need cryogenic temperatures. That's what you're looking at here also. You're looking at it here. That's not the real problem. I mean, I've also discovered, because a lot of people discover something, normally they get the wrong, proven wrong for like 10 years. I think the best example, so that's why you should never predict anything about semiconductor technology. I think the best example is in one of the bookings, Semiconductor Physics Introduction, And we offer at least one of the editions of this, . We wanted to point out that we should never describe anything regarding semiconductor performance in the unknown. So here's an example of a native Indian American chief who said something in 19th century, like people will never invent anything better for communication than a small signal. So I mean, it was . so that's why let's not make any predictions in that direction anyway you can have in this table more work so this is the latest edition you can see what I told you maybe it's very hard to see but if you open the file you can you can zoom in probably this is the most interesting row to you so this is the gate length ahp means high performance computing the one So, one below 8 means high-performance computing. So, let's talk about high-performance computing, which means when you need a raw processing power. And if you look here, 2023, that's basically still this year, 2024, that's why it's in red. They say you're going to use inkpads, that's true. But then you see next here, it's written FGAA, it stands for water that will be all around. This is just paint all around. That's the way it should show. These are the nanosheep consistors. It's very hard to maybe see from this picture, but if you do it in your knowledge, you'll see you actually have a couple of nanosheeps coming like this, and paint surrounding them. So that's these nanochip consistors we paint all around. And then if you go further, this is like 2028, they're still expecting to use the same technology, nanochip, but then you see 2031, like 7 years from now, they're going to keep stacking them together. So this CPAT is complementary factor, that means n-type on top of 3-type. You should understand that this technique is not new to you, so that's why I push seven years from now. Because, I mean, try to imagine how to make crystalline sheets, because, I mean, I'm not going to mention about crystalline material, surrounding polymorphous diatomite stack. That sounds totally insane, I mean, you can't grow this. So basically what they do, they go, silicon, germanium, silicon, silicon, germanium, and you circle the main one, this artificial layer, and you end up with an energy that's suspended. The insorcery, the wind is suspended. And then you fill this area around the gate side. So it's really hard to do. And then you put in the water. So it's really a very long thing, and that's why it's very hard to do it. But anyway, here you can see the future. What's interesting, you can see here they expect next year which have a gate length of 14 nanometers, we are going to see whether this is true or not it's very hard to say because they don't read any data right now, so I can't tell you what's actually the gate length but they expect next year to be 14, and then here in 2028 it's 12, and you see here all this in red means double it means to make something that this year is going to be with doubles Because the 1,200 meters that's below the measurement of 30.6 degrees of salt, it's going to be hard. And it's going to be hard to alter those devices because they are so short that it's questionable whether we'll be able to sustain suppression of such an audience.
\fig{11}{Nanoelectronics of graphene and related 2D materials 2024}
So I hope you understood this introduction. Why you need the materials. materials because that's a major revolution in semiconductor technology. I mean, no one could talk about these materials if silicon were able to sustain further scale. Unfortunately, silicon can't sustain any materials which are not going to get their carrying mobility significantly reduced after forming hitness. And the best candidate for this are coating materials because they are inherently too weak and they don't for the reduction of the chemical reduction into their . That's their data. They can hold that in . The information is with surface and stuff. So I hope the introduction was clear. And now let's start with 2D materials. The first 2D material we're going to discuss is rockin'. The main reason why is because the first one to be discussed, the one which actually ignited all this evolution of 2D materials because the discovery of the ground we showed that we can actually take any layer of material and take normal layers out of it and try to see what is the property of this material and then we can exploit it for some applications. So let's start with graphing. Grappling of course is made of carbon. So let's consider carbon to be ice or silicon. So let's consider silicon. What makes silicon such excellent semiconductor? Because if you understand what makes silicon such an excellent semiconductor, you understand where you should look at it. But let's first discuss silicon. Why silicon is so good? Well, there are a couple of reasons for this. I listed them here. The first one is it has a four-barrel electronics. Why? Because if material has a four-barrel electronics, then it's very easy to talk with. You can easily, very easily make an item with that machine. Of course, that's not prerequisite to help for product selection, for the shops alone. This is kind of the complicated thing. The other one is, of course, when you make your devices, you want to make them cheap. You don't want your smartphone to cost $100,000. If it was $100,000, you would be able to buy more and not take a billion. that ordinary people would be able to buy. That means that the technology must be there to cheat. Technology to be there to cheat, must materially cheat. Materially to be cheap means it must be abundant. It must be everywhere, not already available. And silicon is luckily one of those materials. Because if you look at the silicon content in the first class, as I already told you last time, in the second just after optimum, it's 27\%. So that's why silicon is good, because it's everywhere. They do not have any sand, because sand is silicon-based. You can get it from sand, if you want. So that's it. And the third one is to have material which is easy to process. Because even if material is cheap, maybe processing is expensive. Luckily, silicon processing is cheap. And the main reason for that is that the insulators and you use silicon by a CPU to oxidize the silicon. To put the oxidized silicon, you get silicon dioxide glass, which is a very good insulator. And it's insulator with the in the past, as a gate insulator. Now it's not doing that anymore. It's going to decay too long. About 3.9 is not enough for short-term advertising. So now we have to reach a decade of 2022. Depends on the company. and other new data, so it's just educating, yes, it's about . And all these three make a good semiconductor. So now if you would like to find the alternative, then the best place to look for alternative is the same column of the periodic system, because you expect to have . If you look at the column of the big siliconies, you have a semiconductor domain, a rate, but not so great, why? because the remaining of course, as you know, was used in the plastic semiconductor industry actually the prototype of all the junction transistors were made of germanium and silicon but there are a couple of problems with germanium first, it's a relatively expensive material it's not abundant in silicon that's problem number one of course, it has more evidence than silicon that's understandable And so the first problem is the bundles, it's not the bundle, so it's a much more expensive silicon. And the second problem is the oxides. Because germanium oxide is water soluble. Which means if you make an insulator out of germanium and expose it to air, it's going to dissolve into the presence of water. So you can't make a step of some particular oxygen source. Most of it. The tinid oxide will be water soluble. So that's why germanium is only readily used. Germanium, by the way, has a higher degree of silicon, but because of these problems, it's really readily used. It's been used, for instance, in 30... 45... ... It's been used, but not as a channel. They used it to plant germanium within source and drain islands, to strain silicon channels and increase the degree. This is not very interesting application of germanium in semiconductor technology. But it's not used anymore. It's too complicated and it's very complicated. It's just a bunch of this and this. So basically germanium, yes, but you see there are problems. Then what are the others? Okay, phenolitis, forget about it, these are the methods. And then basically the only candidate is carbon. Of course, there are other semiconductors, but I'm talking now about monatomic semiconductors. Because you have like Brown, Morrison, Eichner, Hildesheim, and Schwalm, Ingalls. They're all semiconductors, very good semiconductors, but they're not the only one. Monatomic, I'm talking about monatomic semiconductors. So basically the only one which left is carbon. And he used carbon. And until recently, the answer would be not why. Because until recently, only two elements of carbon were known. And none of them was suitable for electronics. The first was diamond, useless, because it's an incubator, and not because of the price. By the way, the problem in all of the fact is expensive than just the artificially inflated price. Because diamond is extremely useless material, it can be used in applications. Maybe for new things, if you want to build something very fast, you can buy it from there. Or maybe for something you ask, you can also use other things. So not applications, but applications. And that's why it's so expensive, because those companies convince all of you, all these people, to buy a diamond ring. And this is a sign of love and so on. There is no reason why it's expensive. Pretty much it was not expensive. And in ICO, very soon we will be able to gradually calm down, and now you can get synthesized diamonds of pretty good budget quality and quality of life in some countries. ICO prices have been recently not too positive, but it's okay. But anyway, it's interesting. Because they are talking here about trying to single-agency computing And the only other allotrop is graphite. And graphite unfortunately is a metal, so we can't use it. So it seemed like a close case for carbon.
\fig{12}{Nanoelectronics of graphene and related 2D materials 2024}
But then the new allotrop of carbon were recently found. And now each one was the first to be found, that is now I can't say disputed. but basically all these three allotopes you see here, graphene, macchi balls, and carbon nanotubes, they all come from graphene. Oh, sorry, I forgot to say graphene. So graphene is basically the main constituents of these allotopes you see here, because if you touch graphene in a specific way and then you roll, you will get a macchi ball. If you roll it, you will get a carbon nanotube. If you stand it vertically, you will get a drop-mite. The one I said is not really useful because it's a metal. So which of these algorithms was the one who discovered it first? Of course it was drop-mite itself. Drop-mite itself was discovered in the 16th century. I told you it was discovered in the UK, actually in England. and they first thought it's a coal, so that the burning can heat up, and it turned out that people didn't want to burn at all, people got too angry, it looked like a useless coal because people didn't want to burn, and that's why they made a reason why it's used today in factory computing. Because for instance, if you would like to evaporate some materials to deposit on semiconductor You can use a crucible-strained drop pipe, which you put metal into the white metal, and then when you heat up the metal, the operatives of the drop pipe will just stay there. So that's one, for instance, another application of drop pipes. It doesn't burn heat, but they couldn't burn it. And then they realized that when they channel drop pipes, their cancer goes to the red light, They realized that, for the reason I already explained, your exfoliation, leaving the ground clean is fine, you get a white face, they realized that ground light can be used to mark peaks. And then, late in the 16th century, the first application, the real application of ground light was to mark sheeps. Because in England and Wales, they had lots of sheeps, and you couldn't understand which sheep is yours, and which sheep is from some house. and then use graphite to mark the sheet. That was the first application of graphite. But then later on, English people understood the objective properties of graphite, and it was used to make the best camballs. It was to get a camball, you have to melt iron, for instance, but container in which you melt iron, it will have a severe impact on the quality of the ball. And if the ball is rough, then we will shoot it from the camera, because of the roughness, we have lots of airbag friction, and it will not fly far away. However, they realized that if they put in this container, the ball, if you cover it with graphite, and then melt the ball, as we already discussed, it won't make the bird come in, it won't make the bird, and it will make a very nice, smooth ball. And that's why the English Royal gave you the 5 Cs, because they're able to shoot the balls the longer distance they can touch the guns. And that gives you critical advantage of the C, because if a ship can shoot them another long before the other one can respond, they're good. So that's why the Royal gave you the 5 Cs, because they have a draft pipe, and make them draft pipe with a strategic material and they will not reach any export of draft pipe from from the island of the British Isles. So the first one to be discovered then, which one was the second one to be discovered? As we know today, I only told you the structure, graphite is known for like, I think, more than 60, 70 years, and it was known already back then that graphite is made of these graphing graphene sheets that are strained together to make graphite. But actually people didn't think that you can actually start to extract monolayer graphene out of graphite. So the first one to be discovered after graphite, so the first, actually the third algorithm to be discovered after graphite and dynamo was where, well this is now this building, where there are a lot of tubes of ductibles. The ducky bolts were officially discovered in 1985, there's a paper about this, and we were the guys who got the full scientific approval for us for this discovery. That was a big news, because that was the final amount of carbon after the final and it looks extremely exciting because we had the arrangement of carbon atoms like a football, like a CCCT, we can run with football, like using football games, like couch, right? We have hexagons and pentagons, run with the same structure, but unfortunately this material, although so exciting, doesn't have many applications, because it's not possible with a small atomic inside ball of carbon, Not in one, not in two, and also because it's relatively expensive to synthesize. You can't synthesize it in huge quantities cheaply. So it's a very nice from a scientific point of view, but from the information point of view, not much. And then this is not disputed whether actually nanonautics were discovered first in 1952, because in 1952, a group of Russian scientists published a paper. In some period, the Russian Journal, in which they show transmission electron images, transmission electron microscopy images of carbon nanoparticles. Something which really looks like carbon nanoparticles, like cylinders made of carbon nanoparticles. And we could use published in this experience journal, and we could use the 1952, and we could call it old, no one made a mention, basically, of arms of people who did it. And that's why we will... That's why you will find that people usually say that they've been discovered in 1991. Okay, I'll take a break. So that's why it's usually assumed carbon-monoxide was discovered in 1991. Because of that, people were already ready for nanotechnology, in 1991. And this was published in Nature, and it was discovered by a Japanese scientist, Jim Muffin. And he showed very clean images, first of the multi-world nanomaterials, and then later on, of the zero world. And it's very interesting that also like C60, you can actually get nanotubes out of graphene, just a graphene sheet, depending on the way you roll it, you can make a cube, you can get cubes of different sizes. Now there is a question someone asked me, whether C60 has the same properties of graphene, no, it doesn't have the same properties, because it's basically a molecule. It's a molecule, and also you can get it by cutting out ground-in sheets. It doesn't have a fully-fixed algorithm structure, because it's got hexagonal syntax. And of course, C60 is not the only one that has C70 in its arm. There's a natural C60, because that's the one we just discovered. However, the properties of tubes are very similar to that of graphene. C20 could be used to get tubes by rolling graphene into a tube. And I will talk about properties of tubes later on, because first I'm going to discuss the physical properties of gravity, and then you will see that if you understand the physical properties of gravity, it's very easy to understand the carbon-amperes. It's easy to get this automomiter. So then the carbon-amperes, let's say, in 1991, and then in 2004, in this vector in science, it was published, the discovery of gravity, which is kind of strange if you think that graphing was the last, the last power to be discovered, although all these power-tops are based on graphing as a whole, I mean, CCC and Filt and Grappite, they all come from graphing, and graphing was the last one to be discovered. It was kind of discovered before in 1962, but again, people didn't pay much attention to it. I will actually show you the result. And then this discovery in 2024 was important because not only it was kind of demonstrated, but also the physical properties were revealed, which hadn't been done before. And that's why people typically assume that graphene was discovered in 2004, although I don't know if people have seen it before. So not only this paper in 1962, which I'm going to discuss later, but also, for instance, if people grew other materials, like semiconductors, it happens if you grow a new semiconductor that in case you have a problem with the chamber, you get some organic contamination. In many cases, it turns out that this organic contamination was aggregation of carbon from the surface of the semiconductor. It turned out to be .. I think people see it before but didn't pay attention. For instance, in that case, the person is studied as contamination. No one actually looked closely what is that contamination. And you get this kind of growth, typically, on catalytic metals, for instance, nickel, copper. You get it from basically, you heat up catalytic metal to high temperature. And to all of it, whatever organic thing you can think of, it most likely decomposed and deported from the metal surface of the ground. It's interesting. Of course, in most of the cases, multi-layer dropping, but even in some cases, you get a monolayer dropping. And that's why I put this paper here, because it's very interesting that in 1962, I would say, this scientist at home, if the scientist actually named dropping, dropping, programming, but we gave this same sheet of carbon atoms. But basically the same experiment was repeated in 2009 and published in Nature and Art and Technology, a very prestigious journal, but basically what they've done was pretty much use these in 1962. It's very, very similar. And of course the reason why you're discussing graphene is because you're going to see it the properties and possible . That is why we are going to start with the first 2D material.
\fig{13}{Nanoelectronics of graphene and related 2D materials 2024}
And now let me address this issue I mentioned. How come it was the last one to be discovered? It was the last one to be discovered, mainly because people didn't look for it. And the reason people mainly didn't look for it, because there were papers and books in quantum physics explain that strictly two-dimensional materials can't exist. And they put here one paper which does some calculations here. So, long story short, you know that atoms in crystal lattice, if you look at the atoms in crystal lattice, they are not fixed in position, but they find the oscillation. The higher the temperature, the larger the amplitude of oscillation. One could visit this model by Thomas. virtual particles which come as a composition of those vibrations in others. Well, in the simple harmonic approximation, if you apply this theory on a strictly 2D material, a material which is just one atom, it turns out that when you calculate the amplitude of oscillations of atoms in a strictly 2D crystal, it turns out that its amplitude, under any finite temperature, is larger than the atomic distance. And that means the crystal is was also kind of a good experiment because with the development of semiconductor technology people got a kind of tool capable of growing very thin films. So thin film grow very thick base of semiconductor technology. So when people grew thin films, they knew there was one very interesting property of thin films. that, you know that every material has a certain melting point, so if you heat material about this melting point temperature, material simply starts evaporating, becomes under liquid and then starts evaporating but the crystal structure is not, because you basically have both oscillations I mentioned so much the crystal structure collapses, you get too much liquid then it's like that will get us when we keep the material on top of that. So, when people found out that the T-in-Q make material thinner and thinner and thinner, which you can do by growing T-in-Q, at some point, that is below certain thickness of the T-in-Q, the melting point starts dropping, dropping down. So the melting temperature is getting smaller and smaller. and then if you make that circulation so people did the plots like this so they put the melting temperature as a function of the thickness so this is thickness of the field right they got a plot which looks like this so typically it's like this so the melting point temperature is dependent on the thickness but below certain thickness okay it's not that sharp it goes like this below certain thickness the thinner the material the smaller the melting temperature and then if in tempo it extrapolates this comes here which means that if you make material which is so thin that it becomes 2d the melting temperature is zero kelvin which means material is unstable on any finite temperature is going to collapse. So that kind of confirmed this theory. And that leads to people being moved to the materials. They didn't think that strictly 2D materials can exist.
\fig{14}{Nanoelectronics of graphene and related 2D materials 2024}
However, there were signs that that's not actually true. And this is this paper I mentioned before, 1962. So in this paper, of course, we have the main author, not the only one. In this paper, this scientist investigated something we call today liquid phase exfoliation. So he tried to exfoliate the athlete from the athlete. You can think that this is not so hard to do because all that you do is write your pen seal, right? So that you write your pen seal, you exfoliate the athlete from the athlete. And the liquid vapor solution is very similar, the only difference is that the vapor takes the drop light, putting some liquid, like some solvent or something, and even pure water, and then just some evaporation. So I'm going to very strong alpha sum, and due to the oscillations, due to my alpha sum, the drop light is going to simply get expunged into the dropping shape. Because I know very well that the bonds between the dropping shapes, the drop light and the so this is one of the most important, which is very big, and they can be very easily stored. However, people tried to do this, of course, but didn't come up with anything useful, mainly because graphing sheets are made of carbon stores, and the carbon is known to be hydrophobic material. Which means if you suspended the wrapping sheets into liquid, they would try to aggregate together, because they had to flow. They don't like to stay in strong fluid. And that's why if you try to exfoliate, they're not the same. They are like, you use this first material, but this material will quickly aggregate back into some... But, not really graphite, we're not going to be speculating this again, but something like this, where you take a piece of paper and stand it and that's what you're going to do. However, what he did, this man, he serialized this problem of chemical density, so he said, okay, why don't I first oxidize graphite? Because if I oxidize graphite, this is the balance of oxygen, I'm going to get a hydrophilic material. And hydrophilic material is going to be then exfoliating, and this hydrophilic, the flakes are going to stay away from each other. That's what you need. If you basically took the piece of graphite, oxidize it, of course, graphite oxide is not a certain type of cell. A new oxidizer, a photoexoid oxidizer, is involved in graphite, it's a rather interesting theory, right? So how do you oxidize graphite? We have to subject graphite to various compounds, like sulfuric acid. The process is called the Huffer process, that's kind of cool, right? And then by oxidation, by using very various compounds, you get oxidized graphite into graphite oxide, which is basically this graphite decorated with lots of oxygen and hydroxy groups. In some ways. However, because of the presence of these groups, this material needs hydrophilic, so if you put in a liquid and then exfoliate by ultrasound, those sheets are going to stay away from each other, they're not going to aggregate. And you basically get the sheets of gravity outside. And then, this is graphene oxide. If you then take some very strong reduction agent and reduce graphene oxide, you will reduce it back into graphene. So that's how you can get graphene sheets isolated from graphite, for a very complex chemical process. That's what we want in this case. And then the result was scanned by the transmission electron microscope, and based on the contrast, So you can see here some sheets of graphene, of course, it's random. Why? Because this is done in a big picture, you cannot expect that sheets stay like this. But you can see them here based on the contrast. It was possible to calculate the thickness of these sheets, and you see lots of sheets. Lots of sheets have a thickness in the range between here and 4 angstroms. Do you remember I told you that the typical thickness for graphing is taken to be 0.33 nanometers? That's 3.3 nanometers. This is very, very similar to this. So basically this is kind of discovery graphing. However, it wasn't considered as such. Why? Mainly because it was just this image without anything else. There was not a real proof that this is indeed graphing. And on top of that, it was just a picture. There was no any idea, there was no any mentioning of possible physical properties of this material. Any further information.
\fig{15}{Nanoelectronics of graphene and related 2D materials 2024}
And that slide paper from 2004, which is the 50 years, which is the image from this science paper from 2004, is typically assumed to be the one which kind of was the one which announced discovery of graphene because in this paper scientists managed to place graphene on silicon ducts and substrate not only that to make electrical content to it and basically measure some ID curves and that was a kind of more important because you could get the idea of all the physical properties of the material However, I have to say that these scientists, of course, they got the Nobel Prize in 2010 to do the one with the paper, but not for this paper, because this paper was disputed at the beginning. They actually couldn't publish it, the first published picture couldn't be published, because many scientists who reviewed the paper were very skeptical. The main reason was that although they claimed that this is like monolayer, they had actually the paper is monolayer graphene, because they did not have a typical proof that this is really monolayer graphene. So what you see here is the atomic force microscopy image, which is used to measure the high profile of the structures, and then you can probably argue, okay, if this is AFM image, atomic force microscopy image, didn't they then prove prove that this is graphene by measuring the thickness by the element. The answer is no. They couldn't. And for that I have to explain you what happens when you put graphene on a substrate. So first, what is atomic force microscope? So atomic force microscope is a simple system in which which you have a tip, which I'm going to talk about tapping mode, but there is also a combating mode that's on the left, a taping mode, in which it oscillates at a certain frequency above the surface, typically 200 kHz, so it oscillates very fast, but the amplitude of oscillation of the tip depends on its interaction with the surface, and the interaction simply depends on the distance between the tip and the surface. surface. So basically tip goes around, taps like this on the surface and based on the interaction you get amplitude which depends on the distance between the tip and the surface and you get information about topography. So you can measure this, for instance you can measure how much is this thickness. I will typically use lowercase e for thickness. You can measure how much is this thickness. You can basically get a hydropile on the surface. is what you have here. So you see this dark area, that's silicon dioxide substrate, and then this is the first step up, this is monolayer graphene and then this is the second step up and so on. So now, why they couldn't prove they have a graphene on the surface? Well, you may think that if you put graphene on a surface like this, so let's assume this is graphene sheet, you should be able to measure here, like here, thickness difference of 0.32 nanometers right and that would be really proved that this is monolayer graphene however that's not what you measure by AFM because in reality the structure doesn't look like this the problem is that you do measurement in air so this is substrate and this is substrate and this is graphene the problem is that as As soon as you explode, I told you already, any surface where we see now, say, we get a monolayer of junk which floats in air, deposit it on the surface. We get all these water layers, all the molecules which are in the air, and they have a thickness of about, let's say, 0.9 nanometers. So this is, of course, not monolayer, 0.9 nanometer is not monolayer, it's a load of junk. I just said you get one layer within one second, but if it stays longer, this layer will grow to a height of 0.91 nanometer. That's what we call depth gradient. So this is basically garbage. This is garbage on the surface. And then when you deposit graphene, you put graphene on this garbage. Graphene has a very large surface area, so it cannot penetrate the layer. It basically sinks on it. And then when you take the tip, you should understand, point 33 is here, right? When you take the tip, the tip unfortunately penetrates this garbage layer, because it oscillates on the surface and these are just some gases moving, they cannot stop moving. And basically what we measure by AFM is the height difference from here to here. So from the top of grafting sheet to the substrate. And because of that, instead of 0.3 nanometer, you get 1.2 nanometer. If it's 1.2 nanometer, you can't claim this is monoware grafting, because it could be two or four monowares of grafting on top of the other. And that's why they couldn't actually prove that this is monoware grafting. And although we know now today that this was indeed monoware grafting, they couldn't make such claim in the paper. And basically the paper for which they got the monolayer drawings was not that number, but the next one I'm going to talk about is in which they definitely proved that they have monolayer graphing in power because they deposited such structural silicon dioxide so they made a profile and measured the control effect. And what they got was a top integer control effect. We get top integer control effect only in monolayer graphing, not in other materials. In other materials it's only an integer form-control effect, including binary and three-layer graphing and also graphing. Okay? So you can't get an integer form-control effect in other materials. And that was the final proof that they really had graphing. And I think that was the main reason why this second paper was taken to be like a justification for dealing with that. And of course, it is a deserved result for a lot of the products. Because they are not only to extract this mysterious monolayer graphene, but to actually show its physical properties. That is very important. And then later on, it turned out that people were totally wrong about the graphene, actually 2D materials, that not that they, because people thought they can't exist, it turned not only can be deposited on solid substrate, but you can basically deposit that on everything. You can even suspend the way graphene is going to work. And that's what you see from this paper here in 2007. This is the first paper in which the transmission of electron microscopy into a graphene was taken. What you see here is the model of graphene, so this is the load of the information. I will show you later on the presentation. We should be positive on the copper, on the copper template which is used for transmission electron microscopy. Because you all know the transmission electron microscopy, you scan the beam sample, qualify energy electron beam, so basically the beam passes through the beam sample, against the transmission electron microscopy. You collect it here, you get some information of the center. And you see here the graph will be positive on this copper grid, so this is a dark stripe that's copper, so that opens fire because it's copper is thick. And you see it's suspended, it's basically stable in air, because this will be positive in air, and also under the conditions of scanning, it's basically like a scanning plan, like We have to have a pair of scanning chamber, otherwise you can get scattered locale. And then it turned out it's also stable in liquids. You can also put it in liquids. You can also put it in polymers. So this was the Samson paper. You see these are all very old papers, very old. It was like between 15 and 20 years. Because these are the first papers we demonstrated, you can first isolate the polymers, and then you can put them in the cell. So this is the first paper we demonstrated. It was like between 15 and 20 years. Because these are the first papers we're going to be demonstrating. You can first isolate it, you can move it whatever you want, you can even put it in the liquid. This paper is interesting because here they pioneered the way how you can do a batch transfer of gravity after it's gone certain surface. Okay, so that was proved that this is really a stable material. And by the way, before I continue, how do they actually activate? How do they actually activate the gun? You won't believe it, it's a very simple method. They use poch-tape. So you take a piece of gun-pipe, and press poch-tape against, you rip it off, and what happens is, because of the very poor one-to-one interaction, you will detach lots of gun-in-place from the tape. And then if you, or a paper tape, hold it like this, and do this couple of times, you'll keep exfoliating this thin material. At some point, you press this against the substrate, rip it off again, you'll get lots of garbage on the substrate, but somewhere in this garbage, you'll find monolayer garbage. And that's how they discovered it. It's also interesting how they managed to find it, when they were talking about this label. Today, so it's basically, you could say like, not on price or something. No. I have to say, they did not invent this method, this is even more interesting. Because if you... Does anyone have a free scanning panel in microscopy? So scanning panel in microscopy is a microscope with a tip similar to this. The only difference is that the tip does not oscillate, it just moves about the surface, you apply the voltage on the tip and measure the tunneling current because the tip is not in direct contact with the substrate, you measure the tunneling current between the tip and the surface, you can measure the tunneling current because this distance is very small, and you know tunneling current exponentially depends on the distance between the objects like if you have a potential barrier right and that means by measuring tunneling current you're basically measuring the topography of the object so how do you calibrate scanning tunneling microscope? the small problem with STN is that the scanning tunneling microscope requires conductive surface because if the surface is not conductive the tip is going to crash away Because how the procedure works, you lower down the tip and monitor the current. At some point you will see the current, which means the surface is closed. But if the surface is insulating, you're going to crash the tip on the surface because there will be no tunneling current. So that's why STN can work only with conductive software. And for that reason, for the very same reason, people who work with STN, they use graphite for calibration. Because the structure of graphite is very well known. If you look at the graphite like this, you look at graphite sheets, you know what you're going to see. You're going to see hexagons made of carbon atoms. So that's why graphite is used for exploration. But because of this job, it has been possibly done in circles. You know what that can do? It can remove the vibration. You know that it can. put a scotch tape on the graphite, rip it off, and get the flat surface of the graphite, and put this in SPF. They take the tape and throw it in the river. But there you go, you're not afraid to go into the river. So, these people from Manchester realized that actually they can find psychic physics in this scotch tape, but that's not what they discovered. That's the method we use even today, if you come to my lab, like my students usually do, to show you. But we don't use anymore, we rarely anymore use this scope shape now, because as you can guess, the scope shape gives us a very small resolution layer, lots of blue. If you press again the sub shape, you get also lots of blue residue. So what we use today is a so-called Dyson tape, which is popularly called a blue tape, that's the color. It's a tape made by a Japanese company, NITTO, which is one of the sponsors of NITTO's to be final, so you can take place soon in this terrain, right? They're going to support the energy. Okay. So that's the company. So they make this tape for vaporizing. When you have a vapor, you want to rise it into the chips. The vapor should be steeped on something so that when you dive the chip, they don't fly away. And this company made this nice, smooth, vapor-dicing tape, which picks material very well, but doesn't get so much glue. And it's perfect for exfoliating the material.
\fig{16}{Nanoelectronics of graphene and related 2D materials 2024}
Okay. And now let's answer the question which you will probably now immediately ask. Okay. If you are convinced now with this experiment that your acne is stable, you can do it with whatever you want, basically. How do we explain this theoretically? Because here it has lots of these particular quantities. Well, there is also a theoretical explanation for this. One, of course, is that this story I explained to you about poems and so on, that's based on harmonic approximation, so that could be the first source of error in this story. The second and more significant source of error in this investigation was that people really investigated perfectly quite soothing materials. we can only understand that even though graphing is just one atom deep, you cannot get it perfectly flat, that is possible. On whatever supersquare it's already suspending, it's going to get some small corrugation. It can't be here perfectly flat like a mathematical plane. And it turns out, if you take into calculation those corrugations, Then, some of this material turns out to be . And I just want to put here, I summarize the explanation you can find here, the single interaction between these types of hormones, which stimulates the material due to its expansion into the engine, because of the correlation. That's the only thing you have to know. There's a little too much for this course, I think this is quite enough. And to really prove that there are correlations in Gramping, I think you see what I promised you, the high resolution TMA from Gramping sheets from the previous paper. You can look at it here. So here is Gramping sheet, this part of it is Gramping sheet. You can see the scale, it's one nanometer. You can see those nice shapes here. These are the hexagons of Gramping crystal bodies. But if you look very closely, you will see that in some areas you can very nicely see the hexagons, and in some, not so much. Why? Because of the corrugation. Because if the sheet is standing exactly 90 degrees with respect to the incoming beam, actually, instead of where it is, then you see the hexagons. But if you have a corrugation, the beam doesn't pass on the 90 degree angle, and you don't see the hexagons. and that's why it looks like this. So this is also experimental evidence that this is indeed the prurigation used in humans because it exists. Okay? Why? So I hope that's now clear, this introduction to graphene, but now it's time to look at its typical properties basically the right of theory of government.
\fig{17}{Nanoelectronics of graphene and related 2D materials 2024}
What's very interesting when it comes to graphene is that each band structure, that's what I'm going to do now, can actually be derived with the paper. Because it's a 2D material, its shape variation can be actually solved with the paper. You don't need to solve it medically, but for other semiconductors, actually it's the same method, it doesn't matter. and the concept of the very interesting of the playground for physics. What's also going to be very interesting, I'm just going to come down to now, before I start this long derivation, is that what we're going to get is a very interesting event structure. In contrast to other semiconductors, which have a very advanced structure, indeed the energy is proportional to the square of the momentum similar to classical physics although here the momentum is actually you know the quantum momentum not the Fourier or Rayleigh function in case of graphene the energy is going to be proportional to the momentum without square it's going to be a linear function of the momentum This is very very interesting because the particles we are speculating are relativistic. I am going to show you, the fan structure we are going to get is responsible for relativistic particles. So basically it means that particles in Jalpin decade are relativistic particles. The reason I'm telling you this now is to keep you motivated to look at the remaining part of this language. So let's try to show this. Now before I show it, I just have to give you some quick introduction and wrapping before we start to be writing the recreation. The first thing I'd like to discuss is just the physical structure, this is the carbon alloys we have, and why it's like this, and then I'm going to solve it. So in order to understand the physical orientation of carbon atoms in graphene sheets, in monohedral graphene, we have to start with carbon, because material is made of carbon. So this is the carbon atom in a ground state, that's what you know. It has six electrons, so one S state is totally filled, and we have four valence electrons in the second state. The QS is totally filled, 2PX, half-filled, 2PY, half-filled, and TZ totally empty. So that's the carbon in ground state. Carbon in ground state rarely makes chemical compounds. There are just very few chemical compounds where we reach drop units in the ground state, like metal-influenced CH2, that's one of them, but it's very hard to find. What usually happens with carbon is that it goes excited, and by excited they give the form of these two 2s electrons, get out of the 2s states, and fill up the 3s state. So basically you get four balanced electrons which are distributed like this. You get one in 2s state, one in 2px, one in 2py, and one in 2pz. And now comes hybridization. Depending how they are hybridized, you can get different types of chemical compounds. So what does hybridization mean? it means that the carbon makes a chemical compound. It doesn't go like this. Actually, some of these electrons mix together and make hydra orbitals, and those hydra orbitals are then used to make column bonds to other materials. The fact that it has four valence electrons allows three types of magnetization. The first is SPG. SPG hybridization in which all four valence electrons are mixed and yet the four hydrographical needs that they are mixed up together highly, but all four are identical. Despite the fact that one electron comes from 2s and the other three come from So please take a drop, mix, and you get identical chloro-hydroformic oils. What crystallizes in this hybridization known as S-P3? Diamond, for instance, the graph, diamond. Diamond, or it raises net A, CH4. So how does it form? We have four hydrocarbidors. Bear in mind they are full of electrons, which means they tell each other that the geometric orientation in space in space is the one in which those orbitals are far as possible from each other. That means they are oriented towards the corners of our tetrahedron. And that's the crystal structure of an ion, or for instance methane, CH4. However, graphene doesn't do this one. graphene is based on sp2 in which only three electrons are used to make highly orbitals so sp2 indicates one electron from the s and two from the t states spg was one from n and k from t okay while the fourth electron is just three it's not involved in following bonds so basically it looks like this i didn't write I think you could have bought from SPG because that's not really important right now. So this is how the orbitals move for SPG2. You have a, so this is the first orbital you see, this double lobe. So each orbital has a big lobe and on the opposite side a small lobe. The big lobe comes from E and the small from S, or other way around, I don't know. Basically they are fixed up. And this is one hydrido-orbit for one electron. And then you have these two for the other and these for the third one. So how they are lifted in space? Again, these orbitals, because they are full of electrons, they try to go parallel to each other. But because there are only three of them, the geometrical shape under which they are the forward-folded of these ones is in plane, counter-terminal degree, outside the sphere. That's the radiation for a specific radiation. And that's the reason why you have to use this here. Because these three carbon atoms, these three carbon, there we go, because we should like the four carbon atoms, are distributed in plane with 120 degrees. So now you can imagine, if you try to, if this carbon makes a connection with other carbon atoms, so now this is in different scale, so this two carbon atom is now this this one here, you see, these are the three orbitals of this carbon atom, where it makes connection with the other carbon atoms, because they are in the same hydroperglytization state, they also have three hydrocarbidyls also in plane, where they start joining particles to form covalent bonds, which is very strong, and one has a sigma bond, which is the strongest to us, they, because of that, stay all in plane. So you basically get the carbon atoms get connected together, stay in plane all the time, because of the 120 degree angle between them, all fixable atoms. That's the reason why graphene is so immaterial, and that's the reason why it's crystallizes its fixable. Okay? So each two orbitals of neighboring atoms join together and form very strong . While this poor p-z, well, it's . But also, there's also sp-monitor innovation, where only one s and one p are . And by the way, that's the reason why you have so many organic compounds. because organic compounds are made of carbon, because carbon is so versatile, you can get so many different fertilization types it makes so many compounds, that's the reason why basically organic life on our planet is based on carbon because the variety of compounds that carbon is made of The reason for this is that carbon is the smallest atom with more valence electrons. Carbon is the smallest atom with more valence electrons. And because of that, because these electrons are closest to the atomic core, these more valence electrons, both forms a valence atom. Which gives the ability to organic compounds. And that's the reason for such widespread organic valence in carbon. But that's not the issue here. The issue is to understand that it is graphing, when there's an in-plane structure which is hexagonal. So then, what the lattice looks like, so I took this image from this website. So here is the lattice of graphing, and now you can understand, it looks a bit complicated, but it's actually simple. What you see here, just look at the orange color, it's not the other color. so what you see in orange, so for instance here, here and here, these are the magnets like the fabric orbitals, and for instance this orbital and this orbital join together and make a signal bond, a covalent bond between this atom and this atom and what you see here is the in-plane hexagonal crystal structure of grani which is also called something called like a honeycomb, because it looks like a, you know, when you open the, the egg hide, as you see, there's a very patient called the beans made in the honeycomb, or chicken wire, that's what it's called, chicken wire, how the wire is used to pass the chickens. It also has several other things that's how people call it. So these are the signal bonds, okay, ok, covalent bonds which are in plane. That means that the gravity is actually very stable, because all carbon atoms are connected by very strong covalent synatoms. That gives us a very high strength of gravity in plane. But what about this last, the fourth electron in p-z state which is not part of any covalent bond? Well, this electron is fully bound to the atom, and when the crystal lattice is formed, this fourth electron can see it basically move around. That's what gives Galtea its good conductivity, both electrical and thermal, this small electron. Because it's not bound, it can freely move around, and it's responsible for very good electrical and thermal properties. That's a free option. And these are the so-called high bands, high bonds, sorry. So this electron is located here, for instance close to this atom, it is this light bulb here, which are vertical with respect to the, again because of the electrostatic repulsion, it's vertical with respect to the plane of graphene, and it's used for what? It's used to basically establish connection between the next graphene sheet in the atom. but because of very cool connection, graphing can be very easily exploding you can very easily break graphite crystal in vertical direction you can very easily separate graphing sheets because of these very thick high bonds like this fully bond electron-TZ state but it's very hard to break in thick length because of very strong power and long distance of graphing okay, and now based on this We can already guess the event structure of graphene, carbon. In those, if you look at the event structure, you should basically just look at the, this cup field is at stake, because the all electronic properties of material will come actually from this element, not from single bonds, because they don't contribute to the elements move around crystal atoms. And then if you look at the PZ state, you have a single state, an atom, which is half-filled. If atoms join crystals, what's going to happen to this state? Due to Pauli exclusion principle, the state must split, let's call it sub-states, quantum states to the number equal to the number of atoms. Which means that half-filled state will result in half-filled bands. Which basically means that at zero Kelvin, we are going to have one energy band, which is full of electrons, and another energy band, which is totally empty. Because we have a half-filled state, the ground is not one. We are going to get one band totally filled and one band totally empty. That means that the pure calorie formula must be somewhere between these two bands. The one below is going to be a valence band, totally filled with electrons. The one above is going to be a conduction band, totally empty. The only thing which we don't know yet is what the band gap is and if there is a band gap in this particular depth. people dead, but we've got to tell them. Okay, great. Thank you. I'm trying to do a little bit of a Thank you. Thank you. Thank you. Thank you. Thank you. Thank you. Thank you. in case of dropping, induction and various bands also have different names they're just called tie bands because you start from the P state which is half field and then because those half field P state field rise to this half-filled tie bonds, then those bands are also sometimes called tie bands. So you have the upper tie band, which is typically denoted by tie star, and then the conduction band, which is completely empty if you're a cabin. And then you have a lower tie band, which is denoted by tie, tie star, and that corresponds to the entire field of valence bands. I could use my concept, but it's not working. It's a drastic change. No, no, no, it's pretty good. The column is the one that we... Yeah, it's all on the other end. Yeah, yeah, yeah. It's the one which can make really strong electrical connection. This is just how roots, how the roots can... How you can just move it around without making any strong in the column. Okay? Okay. Fine.
\fig{18}{Nanoelectronics of graphene and related 2D materials 2024}
Let's finally now find the band structure of the lattice. In order to find the band structure of the lattice, we have to start with this crystal structure. So you see here, it's just one part of the crystal structure of the lattice. I just took, in this case, I think 44 atoms. Okay. And of course, this lattice should extend in both directions in my image. And so, so my image is the plane of the lattice. I'm going to introduce the XY system of course in plane like this, like this for example, axis Y is perpendicular. And then the first line is defined the unit cell. So the unit cell is this parallelogram, which is shaded here in blue. Okay, and the unit cell vectors are these two, A1 and A2. From this you can already understand that the Latte structure is complex, because there are two atoms per unit cell. You have a four times water of the blue atom, which is one atom, plus you have this orange one, which is inside. So it's a complex crystal structure, and because of that, usually the lattice of graphene is split into two sub-ortices. So for instance, if you look at my image, I split it in the following way. If you look at this blue atom, you can get all other blue atoms by translating this blue atom via linear integer combination of the unit cell vectors. That's how you get two blue atoms. However, integer linear combination of the this blue atom cannot give you this large atom inside because that would require a vector which is smaller than these two unit cell vectors. That's why I'm going to take that all blue two atoms for, let's call it blue sublattice, okay, there's one sublattice, and then all orange atoms for another blue sublattice, let's call it orange sublattice. So I'm going to call blue sublattice A and the orange one B, okay, which is written here. And then of course if you start from this orange atom and translate it by intergenic recombination of the unit selectors, you can get all of other orange atoms. It's very interesting to note, and you will very quickly, is that if you go around the hexagon, the atom is always under made, it's blue-orange, blue-orange, it's never twice blue or twice orange, which means that whichever atom you look at, it's always surrounded by k nearest neighbors, which are from the opposite sublattice. So for instance, if you look at this atom of sublattice or lattice A, this blue atom, you see, it's surrounded by three other atoms, which belong to the opposite sublattice, sublattice piece. That's going to be important now for the derivation. Now, as I already told you, you have all the derivations on my slides, so basically I'm going to skip most of the math, because it's very clear, and it's going to explain to the most important part of the standard variation of the equation itself, and you have it inside the linear of the graph. Let me give you an example. So here I wrote the expressions for the unit cells I took today, 1 and k2. We took their points, and we then need the variation of each column. How can you derive it? Well, that's very easy. I can just show you one, you can put each other, but this is very easy. So, for instance, if you look at the unit cell vector A1, A1 points along the axis that has only number 0 on x component. Okay, this is clear. And how much is the x component? It's expressed via A, where A is carbon-carbon distance in these lines. If you have in mind, whenever you look in a book about graphene or carbon nanotubes or even graphite, you will see that everything is scaled with respect to A. You have to look very carefully what this is. In my case, A is simply carbon-carbon distance or Cc distance. In some books, they claim that this A is magnitude of vector A1 or A2. In that case, the vector A1 would be simply A0. However, in my case, A is not the magnitude of this vector, but the carbon-carbon distance. Which means that, because this angle is 120 degrees, and this is 90 degrees, we have a 30 degree here, so the cosine 30 times this plus the same value will give you the magnitude of a1 because cosine 30 degrees is so much, it's square root of 3 divided by 2 multiplied by the root of 3 square root of 3 so that's why the magnitude of the vector a1 is a times square root of 3 if you assume that A is the carbon carbon distance. In some books, I've already told you, A is the magnitude of the unit cell diameter, to be very careful. There is a difference of square root of 3. And that shows you what's going to happen on the exam. If you have, for the exam, questions, questions which require two vectors, please, there are equal. I really want to see that from the paper, you would have to inspect the ray boundary center, and what it is, how centered it is, 3 is 3, so 3 divided by 2 times 2, and you like to see the variation. If you don't like it, just write the expression. So, whatever. That is, okay, you have to get all the points in the exam, you should, it doesn't mean very easy, but if you want to get all the points, you need to write everything you have on the slide. So don't just write the final expression. It will be like, I keep repeating this, and then the examiner is someone who just put the final expression how to move. They just move it like that. Anyway, now from this model, so you see the vector A2 in this one is a bit more complicated because it's neither along X and Y. Anyway, the derivation is quite trivial. These are the initial vectors. And now in order to derive the event structure of graphing, or actually in order to solve we are going to use the approach that is quite common in solid state basis and that is the way we are going to apply the time finding approach. Basically we are going to consider, if we look at the graph in lattice, we have this free electron which comes from the pZ state, we are going to assume that this free electron basically can jump only between the US neighbors, which makes sense, that's quite good. because of course it is much more probable for the electron to move from one atom to the neighbor than to the second neighbor. It's much more probable than when it moves first to the next neighbor, then the next neighbor and so on. So that's why this approach is quite common in solid state physics and it allows to very easily, very easily solve the Schrodinger equation. how, I don't think that's the topic of the talk. So basically, what we do, I think that if you study physics, you know this very well, and you might think, well, why are we talking about this? But if we have to do it, we could also get people to put electronic gears on, I'm sure. Whether electronic people are better at this approach or not, so let me do it quickly and I hope, I mean, nevertheless, it's a nice weekend. So let's apply it to the other thing. So how is it done? In order to solve the problem, we first start with isolated atoms. So for instance, if you look at one blue atom, one is A, here it is, and look only at the blue atom, which is isolated, so far away from anything. There is no gravity of it. You have only atom A. In that case, you can find, for instance, if you have an electron, so for instance, if you put this blue atom at the origin, and then if you have an electron with the position vector r, that means that the electron is here, this is the origin, this is the position vector, the state of the electron is the vicinity of the isolated atom A, and here we have some weight function, which is called the orbital atomic function, which can be found, it's not so hard, and the expression, I'm just going to write down the expression for this weight function like this. So this is a weight function, we describe the state of the electron from this atom, electron is 5 a of r. So 5 is the name of the wave function, a in y is the state of the electron next to the atom a, and you can see that you have to make your own atom in this system that I have here. Now I have r in the position vector of the atom, that's it. This convention test simply means it's actually a vector of home, nothing else. It's that's it. Fine, that's easy. But then, bear in mind, we have also atoms B. So if you look only at the atom B, far away from anything, so I go with the atom B, you can basically repeat the same story. Say, okay, if I have atom B at the origin, then if I have an electron on the position vector r, because the atom B is at the origin, then this electron will you get some big function that's denoted by pi B of r. So the same as before, it just means that a, because this is a B atom, not a. That's easy. And now, finally, we should understand that we can have a good thing, because at some point, we have to go back to . We cannot have all . We cannot have all atoms in the . Obviously, we can put one of them in the other. So that's why we need the expression where the atom is not at the origin. And that's given here. So as you see here in the third sketch, it is finally some atom A, new atom A, which is not placed at origin. So the origin is somewhere else, it's here. The origin of the coordinate system is here. rAi is the position vector of atom A. Why the position vector is denoted by rAi? because this one is denoted by A i. But why the atom is denoted by A i? Because, because at some point we have to go back to the other one. We don't have only one atom A throughout the atoms. We have plenty of them. And this index i that you see here basically counts the A atoms. So very simple. Look at the A atoms in my original graphic analysis. Let's assume that the atom with index 1 is this one here. one here. So this is A1. And let's count them. A2, A3, A4, A5, A6, A7, and so on. If you count, you can see I have 22 blue outputs. That means in my particular case, index i I runs from 1 to 22. This is the first one, this is the second, this is the 27th one. Okay? I have 22 new outputs, and this index I runs from 1 to 22, depending on the output. Okay? Fine. If the electron is here, then the position vector of the electron is r, okay? But the wave function phi a is not anymore phi a of r. What you need is to put a vector coefficient vector with respect to the other, which is this vector here, not here. But this vector now is minus this plus this, so it's r minus r a r. That's it. So this is now in a isolated out of a i, where the origin is outside of the unknown, that's the wave function we described in state of the element. Now when I'm talking about this atomic orbital, we wrote, it turns out, a very good approximation it is assumed that if you look at all these wave functions of L atoms, around, isolated all these atoms here, they form a complete set. Which means that any other vector, any other wave function, can be represented as a linear combination of these vectors, of these wave functions. Okay? And I'm going to also assume, typically, this is that therefore they're all of top-down. So that means that those wave functions are chosen such that they're all of top-down, which means if you make a scalar product between two of them, you get zero. Otherwise, you have a scalar product that you can prove the same way as the two of them are correct. Great. And I'm also going to assume, for simplicity, that they are all non-narrative. that means if you have a scalar product, if you have a scalar product, you will be doing that with wave functions and vectors, the result is 1. So that means I'm not wrong. And based on this approximation, it simply means that whatever wave function I have, even the one I'm looking for, which is the wave function of the electron inside the entire counting of this, this wave function can then be represented simply as the linear combination of these vectors here, these wave functions, okay? which is written here. Looks a bit complicated but it looks very familiar and it's quite simple. So what we should look first. So let's first have a look at the first sum. First sum corresponds to the linear combination of these wave functions around atlas A. You see this third here, if you look very closely in the sum, this this term here is actually this wave function here. So this sum runs for all atoms A, which means from 1 to 22. 1 is here, 2 is here, 3 is here, 4 here, 22 is here. Okay? Understand? What's in front of the wave function? It's coefficient of the linear combination. The of the linear combination, which is in front of the wave function, should be some function of the quantum number k. So what is the quantum number? Quantum number is something to describe the state of the interaction. Because this is a two-dimensional lattice, there are two independent quantum numbers, they are typically put together into a single vector called wave vector which has kx and ky coordinates. These are the quantum numbers kx and ky. This k vector you can see here and of course when you look at this coefficient to complicate it but if you don't like it you can just write down that this is the sum i goes from 1 to 22 so we are going along the entire lattice a of some coefficient of the function of the k vector times the pi a of r minus r a like this that's the first sum so this is sum coefficient of the function of the k vector which contains two k numbers kx and ky so this k k, this is just the vector, kxky, there are two components, because it's a 2D system. In this case, there is a 2D system. But, you can of course write it down like this, but then it would be much harder to be right than what I'm going to do, so I'm going to overreapply something which is wrong from solid state, which is that's block theorem, which tells you that this constant, u2d, translates of symmetry of the lattice can be written in this form. So here was some constant, and constant it doesn't depend on the position vector of course because this coefficient here does not depend on the position vector. So here was some number which is a function of k, which is denoted by the salient of k, times you have a work phase platform, which which is A2JKRI. That's this. J is the unit, the imaginary unit, for square root of minus one. The only reason I use J not I because this is a course in electronics, it doesn't look like this at all. And if you want to get electronics, I is used for electrical current, so electrical engineers prefer to use J for the imaginary unit. So avoid these . You don't like . So James, the imaginary, you leave k to k. And you also get here the scalar product. You leave k vector in this position vector of the element k. That comes from both theories because of the translation symmetry. I'm going to skip this part because this is something that we should have learned before. If you didn't, you know, you probably just accepted that it is . Let's remember quantum translation of signatures. When you translate atom A by any integer linear combination of the unit cell vectors, you get the same atom. Okay? Fine. So that's the first term. But apart from atoms A, we also have atoms B, orange. We have to add another term, if you have a linear combination with respect to all atoms in the matrix B. so you put here pi d, this is now the corresponding wave function of an atom b which is not not origin, actually atom pi, so it's pi d of r minus r bi, where r bi is the position vector of atom vi, and i counts now the atom, it's not the case. So in my case, i in this second the second sum also goes from 1 to 22, because if you look at my units, there are also 22 orange numbers orange genomics. The only difference is that now this is b1, this is b2, b3, 4, 5, 6, and this is b22. Ok? And that's the only difference. And then again, you have the coefficients in linear combinations written using block theory, in the same spot. Ok, that's it. and the only thing to simplify right now is you can say that this coefficient psi a does not depend on the particular atom, why? because of the translation of C to G, because if you assume that this lattice is very large, like in A, then you translate by units of atoms, you get the same lattice, right? so that's why this coefficient psi a should not depend on i, it should not depend on the other and it can be taken therefore in front of the sun the only difference between the second and the first line is that in the second line i put this psi a of k and psi d of k in front of the sun, that's it and now Now, what I wrote here, that the wave function of electron in graphene lattice, which I represented simply as a linear combination of individual atomic orbital functions, because this one is higher space. So every vector, every function can be written as a linear combination, including this one, which is psi of k of r, psi of k comma r, which is the wave function of electron in Rami 1's trigonometric series. I think that's good. So this is the wave function I'm looking for. That's the wave function of the electron in Rami 1's diagram, because that is the Zolida combination of the four-month-old one.
\fig{19}{Nanoelectronics of graphene and related 2D materials 2024}
Okay, and if that's really the wave function of an electron in gravity lattice, the one which comes from p-z state and moves around, then this function should obey Schrodinger's equation, and Schrodinger's equation is written here. That's Hamiltonian acting on the wave function equal to the energy multiplying the same wave function. So that's the ideal equation of any quantum system. We know now the wave function and the relative linear combination. The only thing you have to do is to take this wave function and put it here in this ideal equation. And this is what you get. Now to understand what you get, probably the easiest way, because I don't know if you remember the previous expression, avoid going to the previous slide, have a look what you have here in this parenthesis. What you have here, that is actually the weight function that I wrote in the previous slide. This long expression, the long curve, the linear combination. That's written here, you see? And if you look at the right hand side of the eigen equation, this weight function is just multiplied by e. So here we see E times the wave function. Okay, that's it. The top part is a bit more complicated because we have in the top part, the left-hand side of the equation, we have a Hamiltonian acting on this wave function. And I'll tell you what happens if you put a Hamiltonian to the right of this wave function. So basically you take this and you drop it and put Hamiltonian to the left. What's going to happen? You have Hamiltonian here to the left. Here you have some coefficient which is a function of the k vector. Hamiltonian does not act on it because Hamiltonian acts on something that depends on the coefficient vector r, not k. If you look at that, the Hamiltonian operator can be placed after this psi A of k. That means that we have a Hamiltonian here in front of the sun. Hamiltonian can get inside the sun because Hamiltonian is the multi-patch of the number of the output. There is a way to say that. So basically we put Hamiltonian inside. And then if you look very closely, you have a position vector here, but this is the position vector that fits the atom, not the position vector of the electron, which is r. And therefore, the Newtonian cannot add to d2. It's also constant at the point with the Newtonian. And that's why, finally, the Newtonian comes here, and it stops here, cannot go more after, because here we have a position vector r. So basically, you write down this expression here at the top, and put the Newtonian here in front of the wave function, this is the difference of the position vector, you have to move more, this is the difference you can put inside the sum. And then you do the same on the other part of the wave function, have your tonium go over the first constant, get into the sum, over the exponential factor and then you get stuck before the wave function, you get this. Okay? So how do you solve this in physics? you solve it by helping going this from the left by this you basically now say in order to solve this I'm going to use orthogonality of the atomic orbitals I know that they are orthogonal that means if I apply a scalar product of them I should get either one or 0 to 10 to the power of 0 and that's why I'm putting here what I did here now I picked up one of the other say so I picked up one specific out of a I don't know I didn't put any index but let's say 17th article so this is specific out of a with index i equal 17 for simplicity I didn't put any index but remember that here I have index which is constant which is fixed so I picked up one specific column A I took its wave function and made bra out of cap how do you do it you just hold it and take the convex modulate and that is going to be grounded. That's what we need to create the scalar product of the scale. To get a ground. So now I'm going to multiply this entire equation from the left by this ground. But bear in mind this is one specific outcome here. And that's going to simplify things a lot. Look what's going to happen now. When you multiply. So let's start from the first line you multiply here from the left okay this is the constant thing know it from no problem so this bra can come here and now bear in mind because this is bra of one specific algorithm doesn't depend on index r this is a 7b it doesn't depend on r okay so that means you can put this bra also inside the sum here and then here this is just scalar exponential term you can put it after it so that's why you basically end up with this you see You have bra just to the left of your Newtonian Then here you have ket Then you do the same with the other side of the first line Okay, the bra you're gonna end up here for the same reason jumps over the first constant gets into the sum jumps over the second constant and then it goes in front of a Newtonian, which is here, you see? So this line here, you get from the first one, simply inserting here and here, this ground. And then you do the same in the second line. In the second line it's a bit easier because there is no Newtonian, but the principle is the same. You put it here, jumps over energy, because energy is also just a function of k, gets into the parentheses jumps over this constant gets into some jumps over this it ends up just to the left of the wave function and you do the same with the other circle and now you understand what it is so now we can very easily also turn the sequence by this equation how let's start with the last term here so what do we have here look very closely Here we have a scalar product between the wave function of one specific atom A and any other atom B. Because here I runs from 1 to 22. So this is a scalar product between the function of atom A17 and B1, B2, B3, and B2. How much is this scalar product? Wave functions are orthogonal. And the scalar product is one only if they are identical, if not, the product is zero. So this first term goes away, zero, one. That's the case. What about this one here? Well, in this case, it's a bit more tricky, it's trickier, but not so much. Because again, we have a scalar problem, we did an atom A and all the functions of all other atoms A. So basically this term here, which is going to be written here, this term here I can't stop screaming So this term here, sorry for this, this one will be over here, this term here is going to be zero unless index i corresponds to the picked up out of a understand because unless the index i in this sum is equal to 17 which is the index of my output a which I picked I always have a scalar problem between two different function so I get zero again so the only non-zero term is when the index i corresponds to the output a in my particular case index 17 for instance that out of this entire sum you have only one element which is not zero and that's the element when i is equal to 17 which means when sorry sorry wrong one which will look here when r a i is equal to r a when out of a i is identical to 1 only then this is not zero so only when the vector rAi is equal to rA, this is not 0 so, and what's the result? when the vector rAi is equal to vector rA I get a scalar product of the same two wave functions the result is 1 and the only thing which comes out is this term for rAi equal rA multiplied by this here And that's what you have here in the bottom. So you have energy, you have psi-y, and you finally have just this term, that's r-a-i-z-2-r-a, that's it. So you see the second line, done. We did it very quickly. Now the first one is a bit more complicated. So now have a look at the first line, which is this one here. okay so what do we have here here in both cases we don't have a scalar program we have a special program in which we can do toning in the new rule and that is matrix transition element from one state to the other that matrix transition element is right copying from LH1 from one atom to the other. So that's what we have here. Now, if you remember, for instance if you look at this term here, if you remember what I told you, we are going to assume a tight binding approximation in which copying is possible only between the nearest neighbors. So, if you look here at this copying, here we have a copying between one specific atom A, and all other atoms A i. So now, in the case where atom A i is not equal to atom A, how much is this topping? It's proportional to the probability of this event, and how much it is? Look at the crystal structure. If this is atom A, topping is possible only between the nearest neighbors. What are the nearest neighbors of atom A? Atoms B. So that means, if atoms A and A are two different atoms, hopefully it's impossible. Because atom A has three nearest neighbors, which are all atoms B. That's why I put this picture here. This is blue atom A with the position vector rA. Its nearest neighbors are atoms B, not A. this jump is impossible, which means that this probability is equal to zero, which means that this transition matrix element is equal to zero, because atom A does not have neighbors from the same sub-lapis. So the only non-zero-terms is here, when atom AI is actually atom A, because then you have popping from atom A to atom A which basically is not popping, it's into the energy of the electron proportional to the energy of the electron outside A ok and how much is this? this is a constant and does this constant depend on the arc of A, which arc of you pick? because the symmetry of the arc does not depend on any arc, so it's a constant which is constant for all arcs in the line of A And that's why it's very convenient to pick it up. Do I have a freedom to pick an energy constant? Yes and no. Why? Because the energy has two components. One is potential energy, which is calculated with respect to certain reference energy. Right? So you always have a freedom to pick out the reference value with respect to which all energies are calculated. And for that reason, to further simplify my derivation, I'm going to assume that this energy, which is called the side energy, or energy of electron outside A, is simply euclidic. And because of that, the last term that was here, which you get when the atom Ai is equal to the atom A, is also zero. I introduced it by convention to calculate all energy values with respect to that one and because of that this is the right part of the cycle which is zero. So you see this first term is entirely gone. The only thing which is left to discuss is the last one in which we have a jumping between atom A and four other atoms B and now we understand this is actually the only term which is not zero. This is the only non-zero term in the first line. Why? Because here we have electron jumps from atom A to all other atom B. If you look at the picture I put, obviously this is zero, unless we are talking about these three nearest neighbors, these three nearest B atom neighbors of atom A. So when this index i, this one here, corresponds to one of these three p-atoms, this is not zero, and this basically corresponds in proportion to the probability of this time. For all other atoms v, this is zero because they're not the nearest neighbors. And then to further simplify things, I'm going now to count atoms v like this. If this is my chosen atom A, I'm going to call this atom B1 with a position vector RB1, this one B2 with a position vector RB2, and this BG with a position vector RB3. Which means that this entire sum here can simply be rewritten again with bottom limit, with one limit, which limit that index i amounts from 1 to 3. Because this is non-zero only for the nearest neighbor jumps. the electron jumps from atom A to one of these three nearest neighbors with indices 1, 2, and 3. Atoms B1, B2, and B3. Okay, so basically here finally, we will get here at the end, is just to just rewrite the previous expression, but to just leave index i to run from 1 to 3, for the three nearest neighbors of atom B. Okay, and that's how you get this one. And now finally, it's a Ferguson communication, you see the Ferguson communication. So let's rewrite this, what we got. So what we got, you remember, this was zero, we could enter the reference line up. Basically what we have on the right hand side is this one, which is written here. And on the left hand side we have just this expression here, which is written here. And now finally, have a look at this here. This term which is proportional to the probability of jumping, right? That's the energy, that's the energy which corresponds to the jump from atom A to one of the three atoms B. Now this energy, does it depend on atom B? Does it make any difference if the atom in the electron jumps from A to B1 or A to B2 or A to B3? Does it make any difference? No. Why? Because if you rotate the crystal structure by a complement, you get the same crystal structure. We created it because of the synergy. All these three energies must be the same. Okay? And that's why these energies are constant. unfortunately it cannot be zero, because we already took this to be zero, we took the sine energy to be zero so now we cannot any more pick up the number and that must be then some numerical value which is calculated to that we compute zero and it turns out to be minus 2.8 electronvolts so for that reason this energy is denoted by minus gamma where gamma, so for hopping energy is Why this energy is negative? Why it's minus gamma? Why it's not just gamma? Because if gamma were to be positive, then the jumps would be prohibitive. If gamma were to be positive, then the jumps would be prohibitive, because they would increase the energy of the system, and that would mean that the material is an electrolyte. but we know that graphene is a conductor, because its conductor jumps not favorable, not forbidden and for that reason the energy, the copying energy, the energy associated with the jumps, or copying must reduce the energy of the system so it must be negative that's why people measure it, they got minus 2.8 like a whole, because you are being in the the complex. If it were insulated, then it would be positive, okay? And this is just like a value from delta, like, seconds and calculations. So that's why this is minus gamma, and finally, you just get Psi e multiplied by sum you have this exponential factor and just scalar minus gamma equal to this.
\fig{20}{Nanoelectronics of graphene and related 2D materials 2024}
You get a very simple eigen equation at the end which is written here at the top this is what you get and now the question can I find now energy out of this eigen equation I find the band structure of r-e from this higher equation here. Can I do it? No. Why? Because this is a system 2 by 1. I have two unknowns and only one equation. Look closely. You may think we have three unknowns. Psi b, psi a, and e of k. But no, we have only two. Why? Because you can divide everything by, let's say, psi a, and you can get here psi e over psi e. That's one unknown value, and then the other one is the energy e. So basically we have two unknown values. The first one is the ratio of these two coefficients, the second is the energy, so we cannot solve for energy. We change the e in another equation. Now before I show you, I want to get another equation, but I would just tell you how to get it. To get it into perspective, and then we will continue. So how are we going to get the missing equation? Remember what I've done? How did I come up to this point? What you can do now is to repeat everything from the beginning, just multiplying by pi d of r minus rv. So now you can pick up some specific atom d to its bra, and then multiply everything from the left and repeat everything you've done okay? and that will give you the second iteration so then you get system 2 by 2 which you can solve at one time and I'll be going through six languages and you can actually do it through the paper you can find the exact bank structure of the problem but before I do this some boring math most of which I can give to you that you can just not do anyway, here what you can do at home, which is written on this slide, you take this and try now to simplify it because this will be very important for the product derivation, so before we come back to not implement with another function what you do is to first get rid of this vector rbi which appears in this sum, I remember you it comes from here and to do so, ok we are done almost, to do so you have to express vector rbi all three vectors as a function of vector i and that's very easy, look for for instance vector RB1. RB1 is RA plus vector U1 which I defined as the vector which points up within two cardinality. RB2 is then RA plus U2 which is now this vector. Okay? And RB3 is RA plus U3 vector. So you can write down RDI is equal to RA plus U1. It's very simple. You can write this. And that's actually written here, you see? RBI is replaced by RA equals BI.\\
Okay, sorry guys for the delay. So people online, do you still hear me? Is it okay? Yes. Thank you. okay so we found out yesterday that any of these vectors of the position vectors of the B atoms can be written as RA plus corresponding UI vector and then if you use this expression here and put inside here for this RBI you're going to get the, sorry, you're going to get this expression here, this is just rbi replaced by ra plus ui, okay, and now when you split this exponential term into e to jk ra times e to jk ui, this first term e to jk ra can be cancelled out with this one, because the first term doesn't depends on index i, can be taken in front of the sum and simply cancels out with the one on the other side of the equation. And in that way, you just get this. So it's without this term and this term, you get this. And now you see here we got some sort of eigen equation. We have this sum inside, which is obviously some complex function of the wave vector k. That's it. and if you would like to make it in a more compact form then you just write down that this sum is as I said some function complex function of the k vector just f of k and finally that gives you the first eigen equation which is minus gamma of Psi b times f of k which is this side is equal to e times Psi of a that's it and as I already told you this cannot be sold right now because you have two unknowns one is the energy E and another one is Psi B over Psi A okay so for that reason we have to keep going but before we do this let me just expand this expression for this function f of K because as you can already expect this function is going to be very important and that's what I've done here so this is very simple math I can leave this to you for a homework you just expand the sum so you take this is e to j k u1 plus e to j k u2 plus e to j k u3 right I'm going to just calculate the first one just to understand what we are doing and then you can do the rest so the first term is e to j k u1 u1 is this a is this is kxky the scalar product between u1 and k is simply a times ky because we have a zero here right so that's what you get e to j a ky that's the first one and then the second one is this one that's when you take the scalar product between u2 and k in the exponent and the third one is with u3 that's how you get this and then just to simplify this a little bit more you can take this term in front of the in front of these two because if you look closely you have the same term here okay you take it in front and what you have here is two times cosine of this and then when you expand this and this complex number you're going to get this that's it do it at homework for homework it's not not something I'm going to waste time to waste time now so basically you get as I already told you a complex function of the k vector you have kx and ky and this is a complex function okay we are going to come back to this function later because currently as I said we can't solve this eigen equation so we have to keep going.