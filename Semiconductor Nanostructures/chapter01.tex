\chapter{Lecture 1}
\section{Introduction to the "Semiconductor Nanostructures" Lectures}
\fig{1}{CourseOverview.pdf}
Good morning. My name is Giovanni Isella. We'll be teaching this lecture on semiconductor nanostructures. So some of you are taking it as a second part of the integrated course, some others just these five credits. So before actually starting our lectures, I would like to give you an overview of what we'll be studying and looking at in the following two months essentially. So the structure of the lecture is actually can be divided into two main subsection. So the first part we will overview Bikes and Conduct. So we will recover some of the concepts we have seen in solid state physics, but actually focused on semiconductor structure. And in particular we will look at different ways of calculating the bus structure. We will also use some end-on code. In the last year I was using mainly MATLAB, so I still have this code. This year I would like to go to Python and Jupyter, Jupyter, as a tool for calculating the type mildly, the K.P method, the band structure of some semiconductor. And then we will also see how we can simplify and grasp the main characteristic of a semiconductor looking at the concept of effectiveness. Then the remaining part of the lecture will be actually dedicated to nanostructures. So in nanostructures we essentially have three main ingredients. So first of all we typically need to alloy different semiconductor because let's say I want to form, to confine an electron or a hole in a nano size dimensional space, I need to have a semiconductor with a larger band gap typically and one with a smaller band gap. So and this is very often done allowing different semiconductor. So once we are able to create this nanostructure quantum confinement effects enters in our picture. So we will revise some of the main concepts like the first example you can do in quantum mechanics which is a particle in a box but we will see that it's quite different to having let's say in an ideal space where I only have a square potential two areas in the conductor where I try to form this structure where every each one of the components let's say the barrier and the well, have their own symmetry, effective masses and in general, band dispersion. And as the last topic, we will see strain effects because if I'm mixing different semiconductors, I mean there are some special cases where I can form alloys that have exactly the same, let's say, lattice parameter. Otherwise, I typically have strain effects. So I can have a compressive region or a concise region and we will see how this impacts on the impact of this on the band structure which we can also use to obtain beneficial effects. So many devices nowadays exploit strain engineering to actually improve the physical properties of semiconductors.
\fig{2}{CourseOverview.pdf}
So in the first part as I was telling you we will look at different semiconductor properties. Here I put some in the prototypical band structure. So we have gallium arsenide, which is a direct band gap semiconductor. We have silicon and germanium, which are indirect band gap semiconductors. So for example, in the case of gallium arsenide, the minimum of the conduction band is at the gamma point at the middle of the briluant zone. So this means basically that they have one state for the moment we represent with this sphere where the electron can sit in the conduction band. Silicose is an indirect band gap semiconductor and the minimum is here, this point, along this delta direction. And this is no longer a single minimum. We have a valid degeneracy. We have six minima which are represented by these pockets of electrons and in germanium it's again in the other bandgas in the conductor but this time the band gap in heat is here and so essentially now the degeneracy and the symmetry of these states changes. So in terms of conduction this will matter a lot. For example, a hot topic regarding the degeneracy of valleys in silicon is the formation of qubits in silicon quantum well where I would like actually to have only one value active. So how to remove this degeneracy so that I can put the electron in a quantum state that I can reproducibly obtain every time. And so for general, things are similar but also different because the symmetry is different. So when you look at this landscape, different landscape that we can have in different semiconductor. And then we will use a lot of the concept of effective mass.\\
So typically you see the effective mass in electronics lectures where it's related to the effect of an external force, the electric field typically. So for example, the drift velocity will depend on the effective mass of the charge I'm considering. But we will see also that actually this enters in the carrier density and also, for example, in quantum confinement energy. And here in all these examples I did, the effective mass is a number, it's a single value, physical quantity. But we receive it, especially in a semiconductor like silicon and germanium, where we have valid generancy, we essentially have different effective masses. So effective mass is indeed a tensor that we need to consider in its full complexity in some cases.
\fig{3}{CourseOverview.pdf}
So we will then look at what happens when I do an alloy. So both from the physical point of view, let's say, so this is a trace diagram and we use an easy case of silicon and germanium where we can obtain essentially any alloy we want in this perfectly random and substitutional way. And so we will see how this having different germanium content for example impacts on the band structure. So in this plot here you see the calculation of it, get the band structure of silicon, okay, that of germanium and also of some alloy. So you see that you can have an and tune your bud structure from one kind to the other allowing two components in this case of the semiconductor alloys of course you can have also three or four components.\\
Then we will look at once we know how to make a noise we can make nanostructure as I was telling you. So we can make an athero interface where I put close together semiconductor of different bandgaps and we will see how to calculate this band alignment and actually you can have three different kinds of band alignments. So this is called type 1. The labeling is not very imaginative, let's say. So you can have type 1, type 2, and type 3, basically depending on where the electron will hold will sit in this heterostructure. So in type 1, you will have electron and holes in the same material. In type 2, you will have electron in one material and also in the other one. And in type three, you will actually have the balance band of one material is in contact with the conduction band of the other material.
\fig{4}{CourseOverview.pdf}
We will then look at how we can actually treat quantum confinement. So which is the mathematical tool or theoretical tool that we can use to understand the properties of the quantum well or a quantum dot or a quantum wire. And what we will use is this envelope function approximation. So we will see what happens when I apply this to a two dimensional, one dimensional, zero dimensional structure. And we will also say make some examples of how we can use quantum confinement. So in particular, I would like you to show three different devices.\\
So the first one is an electro-optical modulator, which modulates light as a function of an external bias, which exploits what we call the quantum confined Stark effect. So Stark effect, you know it in atomic physics, is essentially what happens to energy levels of an atom when you apply an electric field. And this is kind of an artificial atom, so a quantum well, and and we apply an electric field F and we will see if this can be used to modulate light. The second example with regards how we can make so-called split gates quantum dot because actually a lot of the example we will see we will based on the fact that I confined electron using two different material with two different energy gaps but another possibility is what you can see here so if I apply some potential between two gates I can form an electrostatic potential inside my my structure and form a quantum well or the quantum confinement potential that I protected with my gates. And this is what is used for example in spin units where I'm able to essentially trap single electrons in a smaller nanometer size space and then for example control and read the spin of these electrons. And the other device I would like to show you, which is still based on quantum rel, if you see something on the quantum cascade matrix.
\fig{5}{CourseOverview.pdf}
So the last topic will be dedicated to strain. So essentially when I grow material with a certain lattice parameter on another one with a different lattice parameter, I will have a strain which can be either compressive, in this case, where say the upper epi layer, let's call it, has a larger lattice parameter than the substrate. So you will obtain a compressive strain or you can have it the other way around. So the AP layer is smaller lattice parameter and you can have it in size too. And we will also see that there are methods which are actually do not imply using different lattice parameter but external stressor also mechanical stressor to tune the properties of the semiconductor.\\
So also here I would like to make some examples of application. So we'll take the first one from the electronics. So in 2008 Intel introduced strain in their CMOS line so and actually you can see here but this is the NMOS so the part of the CMOS chooses electrons wherever is inside strain applied on the channel so this is essentially what you see here this is the gate stack and these are source and drain and so this is essentially channel but electron in this case are flowing so the channel here is set under the size strain, while here for PMOS, for holes, is set under compressive. So we will see why this is done and why this is beneficial. And the second example is here, is why, for example, strain is beneficial also for lasers. So you can have a quantum well laser, which is really good because you can find electron and holes in the same limited space so you reduce the threshold current because you have more probability for stimulated emission to set on. But actually if you use a strained quantum well this is even better than BDC but this is related to the fact that they can tune the bound structure, and in our case, we will see this, you can reduce the effective masses, for example, of holes.

\section{Formation of Semiconducting Materials, Bandgap vs. Lattice Parameter}
\fig{1}{Survey of semiconductors properties.pdf}
I would like first to give you an overview of which are the semiconductor materials and alloys in a periodic table and these are hundreds of materials so we are typically typically think about the silicon or gallium arsenide but actually there is a whole zoology of different materials and new materials are coming like many transition matrix metal decalcogenides, these 2D materials are also semiconductors. So I would like first to give you a very simple physical picture why some materials are semiconducting and why some other are not. And then we will go through a more quantitative analysis. So the final goal is being able to calculate in the tight binding approach the bind structure of a real semiconductor like silicon or germanium or gallium arsenide or indium gallium arsenide. And we will reach this goal by steps. So we will start with a very simple model where essentially we forget about the crystal structure of a semiconductor periodicity. We will just look at kind of a semiconductor molecule, so which are the relevant orbitals and electronic states involved in the formation of the bond. Then we will add periodicity, let's say, in one dimension. So we will consider one-dimensional chain. And only at the end, we will put the full complexity of the system considering the real structure of a semiconductor. So in this first set of slides I would like to first to give you a very simple explanation and overview of why and when we form semiconductors and which are the main classes of semiconductor that we can encounter in the technology.\\
So this is the zoom of the periodic table which is centered around group 4 and you see here already you have silicon which is a a most relevant semiconductor, it's actually one of the most abundant elements on the earth. Germenium has been less common but it's also a semiconductor. And actually you can form also alloy of silicon germenium. You can also form alloy of silicon germanium. You can also form a alloy of silicon and carbon. Silicon carbide is actually a very important material for power electronics. So in power electronics you need material which can sustain a lot of power, so which means high current and high voltages. So for high voltages you need a high band gap, otherwise you get tunneling of carriers. And for high current you need a very good power dissipation and silicon carbine has these properties and it's I mean actually Italy is quite quite well placed in this technology because in Catania there is a production site of ST microelectronics on the CNR which developed this material in the past ten years. But then again if you move one step to the left and one step to the right you have group 3 and group 5 and these alloys are all semiconductor so more or less whatever material you pick up on the third column and you mix it with perfectly 50\% mix with the material from the column number five, you get a free five semiconductor. So, for example, gallium arsenide is a semiconductor, gallium phosphide, indium arsenide, and so on. And you can also form a ternary alloy where essentially for example you take two different elements from one column so let's say this indium gallium arsenide you have indium gallium mixed with arsenic so again you have 50 percent of material from group three and 50\% of material from group 5, but you can mix up for example the one from group 3. You can do the same also with of course group 5 and then you can also have quaternary alloys where you mix up different elements, but in all these semiconductors the strict rule is that you have 50\% material from group 4 from group 3 for it and 50\% material from group 5 okay you can play this game once more so go one step more to the right and one step more to the left. And so you have these two, six elements. And also these materials are semiconductors, so carbon telluride, zinc telluride, and also here you can make different alloys and form a semiconductor. So what I put here in this bar below this section of the prionic table is the atomic configuration of the outer shell, of the valence shell of the prionic table is the atomic configuration of the outer shell of the valence shell of the original atom. So in group four, you essentially have two electrons in the S state and two electrons in the P state. Okay, so S2P2p2 now if we look at 3 5 on average we still have two electron in the S state and three and and two electron in the P state so in a 50 percent of alloy you get the same thing and if we look at the 2,6 material, again, on average, we always have this S2P2 configuration. So let's say we can guess that forming or not semiconductor material has to do with this specific configuration.
\fig{2}{Survey of semiconductors properties.pdf}
And actually, what happens is the following. So let's say we states and six electrons in the P states. So P states are simply degenerate, so we can think about Px, Py, and Pz orbitals. Each one of them can host two electrons. So the total number is actually eight. So what happens is that if I take my central atom, let's say a silicon atom in this structure, and I coordinate it with four neighbor atoms. I can form covalent bonds. And so each one of this central atom will get the four additional electrons he needs to reach this octet configuration. So he has four electrons already. And then he will get another four from the labor atoms. So, the S2P2 structure and the four-fold coordination that we have in the structure explains why I form this bond. Because essentially, I can reach an atomic configuration which is stable, filling completely the atomic shell reaching this magic number 8. And this is actually what happens in the semiconductor we have seen. So if you notice this different structure, so this is what we call the diamond structure, because actually carbon, you know, has many polymorphic phases and one is diamond and in diamond you have this kind of structure which is an FCC with a double base where each atom has a tetradonal coordination with four neighbors and so you recognize that this structure is essentially made out of this building block. So we can imagine this as a legal block where we have one atom coordinating with four and I can arrange it in this crystal structure. I can do the same with three five semiconductors. So the structure looks very similar. The only difference is that now let's say I have two different atomic species. So the one would be, for example, gallium. The violet one would be gallium and the orange one would be, for example, arsenic. But again, I will form, I will fulfill my rule or filling the octet and having a tetragonal coordinator and this structure is called zinc blend so zinc selenide is this zinc blend mineral which gives a name to this structure. So typically most 3-5 materials crystallize in this zinc blend form. Then we go to the 2-6, it's a very easy mistake in the slides. So this is 2-6 and a lot of them, not all of them, crystallize in this bullcite structure. So bullcite is zinc sulfide mineral. And you see that this is an hexagonal with a body center. But again, you can recognize this tetragonal coordination. So it's a different way of arranging this Lego block. But still, the basic rule, I want the central atom to coordinate with four neighbor atoms to get the four extra electrons to reach the octet and close the atomic shells. Of course, I mean this attack that group 4, 3, five and two six crystallizing leaf structure is not completely true. For example, you see the zinc blend is already in two six material, but it crystallizes in the three five, in the zinc blend structure. And actually, in some cases, there is a competition between these two mechanisms. So under certain condition, for example, when I grow a nanowire, I can obtain garium arsenide in a woodside structure. And also recently there have been demonstration of gemanium grown in a woodside structure. So in an out of equilibrium condition in a, say, epitaxial deposition system, you can also force your system to adapt to a different structure. But in the general rule is this one. We need this tetragonal coordination.\\
So, so far, I think I've convinced you why we form this material, why we have a strong bond and we can actually, this material exists in reality, I don't have gallium and arsenic evaporating away, but they stay together. And let's say the simple explanation is that in this way we reach a stable atomic configuration. The next step is why these materials are semiconductor because I can also form a stable structure and have a semi metal or a metal. So let's try to understand again using this simple arithmetic let's say of counting states why this material is semiconductor. So let's start with a simple example which is the formation of hydrogen molecule okay I know that if I have so the hydrogen is the simplest atom I can think of I have the two s orbitals which are separated in atomic hydrogen. And when I put them together, the orbitals overlap. So I have a deformation of this kind of bonding and antibonding state. And I form the H2 molecule. So let's, again, do the math and the arithmetics about how many states and how many electrons we have. So here in each molecule of hydrogen, I have two states and one electron. Okay. So when I found the molecule molecule the number of states cannot be disappear in H2 I will still have four states and I will have two electrons now so what will happen is that essentially two electrons will go in this bonding state and that will have an empty anti-bonding state. So let's make now the same kind of calculation for a semiconductor. So let's imagine that I want to form this semiconductor molecule. Also in this case I have two S states and six P states. Okay? So Px, P, Py, and Pz multiplied by spin degeneracy. And I have, so for a total of eight states, and I have four electrons, okay, which are essentially the S2P2 electrons that we were discussing before. So now, if I merge my system, I will have that of these four electrons. I will get four more electrons from the covalent bond. And so we'll be able to fill this eight bonding state and leave empty the eight anti bonding state so if we use let's say this schematics and take a similar kind of reasoning that you did here. So when my S state get in contact with the S state of the other four electrons, of the other four atoms, I get an S bonding and S antibonding state, which is split, and I have three P states that do the same. So I get free bonding and free antibonding state. And now I need to accommodate eight electrons. So two will be here. And then I will have other six electrons here. So one, two, three, and one, two, three. And so just as in the case of the hydrogen molecule, I will have that this bonding states will be filled and this anti-bonding state will be empty And there will be separated by some Energy gap and so this is exactly the definition of semiconductor I have a completely filled valence band a completely empty conduction band and I need some energy to go from one to the other so also from the say this simple arithmetic of atomic orbitals in a very simple way explains why these materials stay together and why they are semiconductor and not metals for example okay because by this the trigonal coordination I can basically fill the eight bonding state and leave empty the eight anti bonding state which in a semiconductor pitch in a solid state crystal lattice pitches become essentially the balance band and the conduction band of our material.
\fig{3}{Survey of semiconductors properties.pdf}
So looking at this, now that we understood why this material are semiconductor, we can look at some systematic behavior that we have in the periodic table when we look at the particular class of semiconductor. So the first group of semiconductor we consider is group four. Okay and here if we go from the top of group four downwards We have carbon which in a diamond-like FCC structure has a very large gap so it's typically not considered a semiconductor even though in recent there is a lot of technological effort in trying to actually make it a usable material because it has an exceptionally high mobility, exceptionally high thermal conductivity, so it's a dear material for power electronics. Actually, I remember a paper by William Shockley from the 50s where he was saying, okay, silicon is good. We will make electronics on this. Silicon carbine would be even better because it's a higher mobility and higher, let's say, power dissipation capabilities. But the ideal material would be diamond. It's very difficult to obtain in a, it's difficult in nature. This is why it's so rare because you need high temperature, high pressure. So if you want to do a planar in a reactor, it's also kind of challenging. But there is work going on and there is people starting to make, for example, diodes in gallium arsenide. Then we go down, we have silicon, we have germanium, and we have that alpha-thin. So, thin has different polymorphs and one alpha-thin is the one with this diamond lattice. The common, let's say, metallic thing that we typically use is called beta-thene and is an hexagonal as an hexagonal structure okay and this alpha-thene is stable down to 17 degrees C and then you have a transformation to the beta team. So it's solid state phase change and you go to the beta team. So we notice a systematic behavior in this material. We can see that the gap decreases monotonously as we go down group four and at the same time the lattice parameter increases. So first thing we notice is that the lattice parameter increases as I go, well let's say, decreases as I go down the periodic table while the band gap, let's say, increases. Actually, alpha-T is a semi-metal with a negative band gap. So we have a small overlap. So this minus 0.41 means that we have the conduction band and then the valence band that overlap by this quantity here 0.41. Okay? So can we kind of grasp from our structure, from our picture of atomic orbitals why this happened. So let's have a look at why we have the systematic dependence of the lattice parameter and gap of the two materials. So we can actually understand this if we look at this plot.\\
So here I just put in a graphical way the dependence of the band from the lattice parameter. And this is just a plot of the orbital wave function for the hydrogen atom for the different orbitals. So this is 2s and 2p, this is 3s and 2p, and this is 4s and 4p. So these are the orbitals that form the bond in diamond, in silicon for example, and in germanium. So what you see here is that the spreading of the orbital is actually much smaller than in 2s when in 2p. Okay, and what I put here is actually this dash line are the essentially lattice parameter expressed as a ratio of the ball radius. Okay, so it means if you take the ball radius and you multiply it by this Or if you take this number, which is approximately five by the ball radius you get the lattice parameter of Carbon, you know, diamond. Okay? So we see that when we have two atoms which are the distance, which is roughly the lattice parameter, I will have here the 2s and 2p orbitals of the other atoms who have a very strong overlap, and this strong overlap takes place only where the atom is very close because the orbital atomic orbitals are not spreading that much and so this means that if I have a strong overlap I will have a larger splitting between the bonding and anti-bonding and I will also have a very stable bond which is also testified by the very high melting temperature. So the melting temperature is an indication of how strong is the bonding between atoms. So in carbon, the atom can form only if I go very close between, if I put the two carbon atoms very close in diamond, it can form only if I go very close. And I will have a very strong bond and also a very large separation between bonding and anti-bonding level. So now if we go to silicon, essentially the other two states here are already filled, are kind of core level in our case. And so we see that the spreading of the wave function is, they are different from zero, larger distance from the nucleus. Those are also less intense because the area below one of these curves that needs to be one. So here it is completely empty, let's say the 2p, while here the 3p has already some field part. So we will have a weaker organ that takes place at the larger distances and even more this will happen in Germany. So this means that essentially this relation between band gap and lattice parameter can very clearly be explained by this simple model. So I need an overlap of orbitals and this overlap takes place in larger distances if I have a larger atom, but it will be weaker because the orbital are more spread out in space and partly occupied by core level electrons.
\fig{4}{Survey of semiconductors properties.pdf}
Actually, our model of this formation of bonding and anti-bonding level can also give some insights which we will see better in the future about the nature of the conduction band and valence band. Because what I was telling you is that we have bonding S-like states bonding P like states which are completely filled when I have this octet that could be the octet. And then they have bonding and that bonding level which are completely empty. So what we can say is that this structure of the valence band will be very similar in also the conductor. And actually if I go back to our to my introduction here I have a plot of the band structure of valueium arsenide, silicon and germanium. And you see that if you look at the valence band, it looks more or less the same for all materials. And actually we have one state here below that we will see it's coming from the s bonding and three states here on top and this these are coming from the P3 PXPY and this is essentially identical first glance why the conduction band is quite. So if you look at the lower bands in silicon germany, gallium arsenide, they have substantial differences. And so this is also, this is also explainable with our molecular, let's say, molecular approach to semiconductor. Because if I consider how this splitting between bonding and that bonding level depends on distance, we will see in the next lecture probably that this is a linearly dependence on one over the square of the lattice parameter. So it means that if I go in this direction, I have a small lattice parameter. And if I go in this direction, I have a larger lattice parameter. So what happens is that I can have, let's say, a different configuration. So I can have a situation where essentially also the first level of the conduction band is P-type. And this is, for example, a case of silicon. And actually here, a better example would be another example, besides Germania, would be gallium arsenide. Or we can have a situation where the bottom of the conduction band is S-type. And this partial beam already gives an insight of why the valence band is always the same in a semiconductor while the conduction band is quite different from one semiconductor to the other. Because depending on the lattice parameter, I can go to a situation where I have P-antibonding leather and silicone is actually quite a unique case. Aluminium arsenic is also a similar case. And then cases where instead the bottom bottom the conduction band is styled.\\
So let's now move to the case of 3-5. Let's see if we can find also some systematic behavior here. As I was telling you before, so 3-5 crystallizes a lot of them in this zinc blend structure, where they have a central atom which is different from the neighboring one. And here what happens essentially that on top of the covalent bonding, I will also have an ionic component. Okay, because let's say when this is gallium, when gallium will kind of take an extra electron, it will get negatively charged, while arsenic, which gives away an extra electron, it will get negatively charged while arsenic, which gives away an extra electron, will be positively charged. So in this case, the bonding is not just... So we still have this octet rule to fulfill, so there is again a covenant component, but on top of this there is also an ionic component, which makes the bond stronger. And we can actually notice this behavior, the contribution of the ionic component, by looking at this line of a periodic table.
\fig{5}{Survey of semiconductors properties.pdf}
So here I have Germenium where I put the lattice parameter and the energy gap. And then I have gallium arsenide, which essentially has the same lattice parameter because the orbitals involved are the same. So the same kind of reasoning that we did here for Germenium with 4s and 4p orbitals will apply also to gallium arsenide. So say the strength and the overlap will take place more or less at the same distance. So you see that the lattice parameter is the same. And then we have this Ziggs selenide, 2Ziggs, where again the lattice parameter is essentially not changed, but we see a very strong variation in the energy gap. And so, this variation in the energy gap is actually, so the extra energy is due to the formation of this ionic component. So once I form the covalent bond, I also have ionization of the two elements and this gives extra energy so separates even more the bonding to antibonding state. So in our case, separates the valence band from the conduction band. So here you see a very clear example where essentially not varying the lattice parameter we have an increase in the mass structure.\\
We can also now look within let's say the group 4 element and look at what happens when I alloy gallium with phosphorus, arsenic and antimony or do it the other way around alloy aluminum gallium and indium with arsenic so what we notice here is that if we move let's say in this direction so we keep the group field element constant and we change the group 5 element you see that the band gap is a slowly decreasing and the same thing happens here when I have is I keep constant the arsenic cotton and I reduce the aluminum gallium and in you and they go down also in this case again get is decreases how can we now in this case let's say the covalent the ionic component is the same because I always have but what is changing now is the covalent behavior because now essentially I'm mixing one element which has say for example this kind of configuration with another one which comes from the upper position in the group in the same group which has for example this configuration and so I have kind of an average I mean I improve the overlap between these two as I would have between two elements from the same with the same and principal quantum number okay so here again we can see it very clearly that as let's say the atomic principal quantum number of the element of the group three or five decreases, I have larger band gaps because the overlap is better. Here the ionic component is the same, it's a covalent which is changing. It's exactly the opposite or the dual case of what was happening here where we have the same covalent component and the ionic contribution was changing.
\fig{6}{Survey of semiconductors properties.pdf}
So just a few words about this wood side structures in the conductor. So we will not actually address the in our lecture a lot with this kind of material, but they actually very important. So gallium nitride is a large band gap semiconductor, which crystallizes in this wood side phase and it's basic material for making LEDs white LEDs are obtained from gallium nitride and also for again for power electronics because all white band gap semiconductors means essentially that they can sustain higher voltages and so deliver more power so for example in cars or more electro technical application they are extremely important.
\fig{7}{Survey of semiconductors properties.pdf}
As i was telling you so there are also other class of material no sorry this is a just a just a variation in general overview of this two six, which are also called monochalcogenides, because calcogenes are elements of the group six of the periodic table. And you can see that you can have a larger variation of band-gap and some of them are zinc blend and some of them have the wood side structure, a lot of attention on transition metal decalcogenides. So in this case, actually, we have one metal, typically a transition metal, which is mixed with two atoms which are calcogens. So. So for selenium tellurium. And this is kind of a demonstration of every, uh, let's say rule as is exception. So here our computation of the orbital doesn't work anymore, but this material are also not three-dimensional materials. So they form a single layer which comes in many different polymorphs. So the most used are these two H and one T. And you see that they are not as monatomic as graphene, in sense of graphene all the atom carbon atom lay on the same plane here we have say three different planes but still they are layered material so you can form a single monolayer will be a perfectly stable system and also in this case you can have different band gaps depending also on the phase you use and this is kind of an exception to our general rule so you can have also semiconductor material which do not strictly follow our simple arithmetic rule that we use.
\fig{8}{Survey of semiconductors properties.pdf}
So to conclude I would like to show you this plot that we will use many times in our lectures and it's quite popular you can find it in different flavors where you see the dependence of the band gap and from the lattice parameter and here we have a different color coding different kind of material so let's start from the red which is a group 4 so you see that you have we have this general rule that we also that as you increase a lattice constant you decrease the gap. Here I've had the team as a negative material. Then we have this violet, sorry, the green first, which are three five. Here, I mean, the plot is a bit more crowded, but still we see this general trend that as band gap increases, the lattice parameter decreases. And all these lines means that I can form an alloy and mix up the semiconductor and obtain band gap, which is intermediate between this. The next class of material is 2,6. So we first noticed that all the, this, uh, uh, uh, points are above the green one. So because the ionic component is larger, so the energy gap is larger. And also here we can see this general trend of the critical. And the last set of material are essentially this nitrite, which are in principle 3-5 but since nitrogen is very strong electronegativity they behave quite differently and they are typically in the wood side structure and the name was a lot of freedom in deciding which kind of material we'd like to have for a certain application or just to exploit interesting physics effect.\\
But in reality, actually you cannot get into, if you want to make a device or even to have to have a sample to measure in your experimental apparatus, you need some substrate, some mechanical stable object that you can use for fabrication and to measure. And actually the number of substrate, which the kind of substrate which are available today is not infinite. so you cannot have a substrate of indium gallium timonide okay so you need to grow this as a reproduction layer on top of some available substrate so in some cases these are just very expensive to produce so there's no real need to invest time and money in this in some other cases they simply don't fall so you don't have a stable phase in a crystal or in a stable method to form a crystal in this video so here just to give you an idea I I've listed some kind of substrate. So of course, the easiest substrate to obtain is our silicon wafers. So which are disks of different dimensions. So over the evolution of microelectronics evolved in size. So now we are at this 300 millimeters, so it's like a pizza. And of course if you increase the size of your substrate it means that if you run a process you get more devices out. So with a smaller size you need to run every step of photography, etching, deposition and so on more times to get the same number of devices. Just to give you an idea, a relatively simple consumer device electronics, it's more or less requires different, of the order of 400, 800 different steps of etching, cleaning and so on. So industry is evolving towards 300. 450 for the moment is not, I mean, it's a bit demonstrated, but it's still not on the market, let's say. And the cost of one of these wafer is between 200 and 800 dollars, depending on the doping level of the amount of oxygen in water. So another going down the, let's say, this list, we have germany wafers. So why do we need germany wafers the main application is solar cells for space application so we will look at it into this so satellites are typically powered up with solar cells which are multi-junction solar cell so we have essentially three cell one on top of the other and the first cell is made of germany and then on top of this you see here if I draw a line on the lattice but longer the lattice parameter of geminium I very close to gallium arsenide and aluminum gallium arsenide so you can form you can have different band gaps one on top of the other with the same lightest parameter, and this allows you to get high efficiency cells. So what you typically buy is around 28\% efficiency, and in laboratories people went up to like 48 or so. And so now the cost you see is similar to the one of silicon but with much more size only 100 mm. So this is great. Then you can have valium arsenide substrate which are used for making laser for example and here let's say the price is typically the factor of 10 more than that of silicon and again size is 100 or 200 millimeter maximum another substrate which is used for free files is in duene phosphide. So again, let's go back here. So this is gallium arsenide and this is indium phosphide. So more or less every other semiconductor in the 3-5 family is grown on these two kinds of substrate. So you can, for example, decide to grow lattice matched, so grow all these different alloys, or control and put substrate in your material and move around in this table. But this is essentially what is currently done in research and also in industry. Another relevant substrate now is silicon carbide as I was telling you before there is demand of this material for power electronics and here the cost for 150 millimeter wave is again pretty high so everything is much higher than silicon and for example in gallium nitride it's which is one for used for LEDs is grown on sapphire which is a crystal structure of alumina and this is actually a bit cheaper and this is also why the LEDs are not mean you can find it anywhere And this is kind of a lucky configuration because a gallium nitride, it's a horrible material from the positive view of structure, the one you have in LEDs. It's plenty of defects, but these defects are electrically and optically inactive. So they don't actually have a strong impact on the possibility to use this material. So even if the quality of the substrate and the quality of the layer is not, it's very far from silicon, they are still usable. So this plot calls for, actually there is a lot of effort to actually try to use silicon as much as possible. So many groups in the world are looking for strategies for example to grow gallium arsenide on silicon in such a way that you can combine the optical properties of gallium arsenide so it's essentially having a laser with the cheapness and all the other good things that silicon have the fact that you can have make electronics so that stable oxide you have you can do very easily and so on okay so I think now we can make a break and then we as I was telling you the plot of this lecture is giving first a very qualitative overview and then mass introducing the quantitative approach.

\section{The Linear Combination of Atomic Orbitals Method}
So we will start the first step towards understanding and calculating the full body structure so we were talking about bonding and anti-bonding levels and we will see how we calculate this in a molecule and then extend this to a priodal crystal. So we can restart. So we start with a simple problem.
\fig{1}{LCAO.pdf}
Let's say many of you will say again the atomic molecule, hydrogen molecule, but it's actually, we will use this simple example to introduce some mathematical method that you use also in more complex system. So we will start by studying the problem of this linear combination of atomic orbital. And actually to make the thing a bit more, let's say, interesting, we will not really use atoms. But we imagine that we already know how to make a quantum well. And so we know that if I make a confining potential like this, I can have a confined level. So the first one will look like this. For example, the second one will look like this. And so the symmetry of this state has a very close connection with the two orbitals involved in the formation of a semiconductor. So this is the first level is an even state just like s orbital. So typically we will use to schematize the charge distribution in charge orbital just like sphere and then all the same sign. So we say it's positive density, positive density of electrons. While the second state of this quantum well is odd. Well, of course, we are talking about symmetry around the central axis of this quantum well. And this will be exactly what happens. This is an S-like orbital. And this is exactly what happens in p orbitals. When we plot, let's say, the angular component of a wave function, we know that we can have these two lobes, and one is positive and the other is negative. So these two states will actually have the same symmetry of the atomic state. So what we are going to say can apply to a molecule, but also to, for example, quantum wells made in a later structure.\\
So the system we are going to address is actually this one. Let's imagine that I have two quantum wells. One will be the left quantum well and the other one will be the right quantum well. So we will describe, say, take the energy scale and we have zero at the top of our well so the shape of this potential will be described by something like V left and this one will be V right and of course this is just the same potential say translated by some quantities so if I now take this central axis here and I say that this quantum well is displaced by quantity a and this one is displaced by quantities minus a so the left potential is some kind of square potential displaced by a quantity and the right one will be the same displaced by minus a. Actually it's more convenient if we call this separation A. And so this will be half of the separation between the quantum line. So this will be a more convenient way of plotting this. So our starting point is that, so what we already know about our problem is that we suppose we have already solved the problem for this for an isolated quantum well. So we have already solved this Hamiltonian, let's say T plus V times some wave function is equal to the energy of this wave function times the wave function itself. So by T, I mean the kinetic energy operator. No reason how to, so it's going to be something like this. We're not going to, we're not going to end the derivation, so this will just be the kinetic energy. And then V would be the potential I was discussing before, so confining potential. So I already know which is the wave function, which are the, say, solution of this Hamiltonian wave function and the energy level. So let's say if we consider the first state, this will be an S-like wave function and that will be, this will have an energy, epsilon S. So from this starting starting point I want to solve a different problem which is the following.
\fig{2}{LCAO.pdf}
What happens when these two quantum wells are no longer separated by the interact. So I take essentially one quantum well and I put it close to the other. So this A distance is now reduced. And what does it mean close? It means that they have some kind of interaction. So if I plot again the two-way function, then we kind of overlap. So one-way function of the original atom will fill the other potential and the other, and the presence of the other electron in the other world. So this is What we are going to address this is Our same problem But we want to solve. The guess which is down within this linear combination of atomic orbital is that I can find the solution as a linear combination of the solution of individual orbitals. So, let's go see the solution and this will be a combination of some some coefficient, complex coefficient a s of the sorry a l from left quantum well of the solution of the left quantum well, unperturbed, and the right component times the right quantum well. So we know that this initial guess, the solution of the isolated quantum well, they are just some real function, for example for the s states, which is translated by some quantity in the original pictures. Okay, so our known quantities are essentially the energy and the state of the original atoms. And it doesn't matter how many atoms I stick together. So here I use two, but essentially the orbital will be the same and it will be only displaced from one to the other. So what we are going to do now is essentially solve the problem for this electron, where now the Hamiltonian is given by the kinetic energy terms plus the sum of the two potential because now the electron will fill both quantum wells. And I want to try to find a solution in terms of this wave function which I have obtained from the problem without say the perturbation with one without one quantum mechanics we are used to superposition because the Schrodinger equation is a linear equation. So if I have a solution that I combine linearly together, I will have another solution. So this is a mathematical statement behind this concept of superposition of state. That I can have an electron which is dead or alive at the same at the same time just as a Schrodinger's guess here we are doing something different because these are not the solution of the same Hamiltonian I'm combining the same orbitals or two different atoms so it's kind of out of a mathematical foundation of quantum mechanics. So because one thing would be to combine linearly, for example, these two levels. Okay, if I make a combination of these two levels, which are inside the same Hamiltonian, I will get the solution for the isolated quantum web, which is a mixture of these two states. Now I'm essentially all adding the same orbital of different atoms and so which is a physical inside behind this? So physical gas it's very simple. I want to form a molecule and I think that this molecule can form only so I have two ions. This molecule can form only if I put some electrostatic glue, electrical glue, so I have some probability that the electron sits between the two atoms and this can both together the ions of the hydrogen or the two quantum waves. So it's a very simple classical picture. And so I say, I need to superpose the two-way function. I have the probability of finding the electron there. So it's not a good starting point there to say, because it is completely in contradiction with what we typically use in quantum mechanics. But let's go further and see what happens, let's say. So now, so my goal is what? It's finding, essentially, this value, the new energy level, which I formed when I put together the two quantum wells and also these two coefficients. So essentially one, this one is the eigenvalue of my problem and this will form the eigenvector of my problem. Okay, a simple way to obtain this would be the following. So I have, I can essentially take this equation and multiply it from the left hand side with one of the two base.\\
So if I make it in an integral formalism, I take the complex coordinate of my left quantum well and I multiply it by this. In space, in our case it's a one dimension, so it's called just VR. It's considered just just a one-dimensional problem, I will have an algebraic equation where these two coefficients will appear so to make the notation a bit simpler we will use a Dirac notation so instead of writing this in terms of integral we will just say I will take the bra of this. And the cat of this one. Okay? And here I will do the same And then of course I can do the same thing just with the other orbital of the right-hand quantum well. And this will give me essentially, as we will see later, now it's not easy easy to see but we will get two algebraic equations that we can solve. So let's have a look at the first one, right? So we concentrate on this and we will see that here we have due to this multiplication we have different terms coming out. So first term will be this one. Okay, so this is what I get if I multiply VL times this times this. Okay, now this is relatively easy to calculate. So Al is a number and when I apply the kinetic energy operator times the left quantum well potential to the left quantum well operator I just get the energy the eigenvalue of the isolated problem epsilon s right and so now these two the function are exactly the same they are normalized and so I simply get this a L multiplied by the energy level of the isolated problem okay now the second term would be this one so let's underline it in the orange. And now essentially I'm multiplying this right-hand potential for the left-hand function okay so here we can very simply get out the coefficient and let's see which is the physical meaning of this integration. So what I'm doing now is the following. I'm taking, so if I draw again the two quite well, I'm considering the overlap between this orbital, fl, this orbital, the right, and this potential. So this integral defines a quantity, so it will also be a number because it's an integral which is called beta no sorry sorry this is again left so i'm considering the overlap between these two quantities. Okay, so the left wave function and the right quantum wave. Sorry, I was mistaken here. So this number, beta, is defined like this. So it's a negative value of this and it's called on-site overlap so why on-site Because we are taking the wave function of the same side, on the same left quantum web. Why overlap? Because we are looking at the overlap with the other quantum web. Why we put this minus here? This is simply to have beta as a positive value, because we notice that in this case, in the case of S orbital, so this wave function is always positive and as we set the energy to zero, this confining potential is negative and so this quantity here will be negative and to make it then more easy to handle later we put this minus sign so beta will be a positive number and this is called all-site overlap so this is with our second term let's number them this will be one two Now we add an additional term So again here we have the left wave function And now let's make it purple. We are multiplying these two times this okay and so we have t plus b right a right phi right so again we can take this out.
\fig{3}{LCAO.pdf}
And now if I multiply the kinetic energy plus the right-hand quantum well potential for the right-hand quantum well wave function, I get the eigenvalue for the isolated problem again. So I get epsilon x epsilon s times phi right. So we get AR epsilon s and then here I will simply get the overlap between the two wave function so this parameter here it's indicated typically with the letter lambda alpha sorry and it's called the non-orthogonality factor. What does it mean? It means that if we had that this two-way function were completely orthogonal, this would be zero. If I had the same wave function, this would be one. And the physical picture is the following. So we are simply overlapping the integral from the wave function of one contour well to the wave function of the other contour well. Okay? So if we had started with an orthogonal base of wave function, this would be exactly zero. In this case, it's typically a small quantity, but it's not zero because the base we choose was kind of weird, as I was telling you. We are not using the following, strictly following the indication of quantum mechanics. So now we are left with the last term which implies the multiplication of B left with this one.\\
And if you try to make all the computation, you will see that in this way we have taken into account all possible combinations. So we have this, we have this and we have this. Okay? Again we just take out this complex coefficient. So this is the wave function. And we look at the physical picture behind this equation. So again we have our two contour wells and now what we are doing is overlapping the left wave function with the left contour well and with the right wave function. And this is called, this term is called overlap. Integral. It's indicated with, really indicated with the Greek letter gamma and again we have a minus sign here to make this quantity positive at least in the case of s orbitals gamma would be larger than zero. Okay? So this means that now our left hand side of the equation can be simply written down like this.
\fig{4}{LCAO.pdf}
I will have the coefficient for the left quantum well multiplied by the epsilon s this was term one then i will have minus a left beta and this was term two then Then I will have plus AR epsilon alpha and this was term number three and then we have minus AR gamma and this was term number four. So this is essentially in synthesis. This part here of our equation left hand side. Now we need to understand what's happening to the right hand one. Okay. So let's make it here so the right hand side was this times the energy times this plus this so this is a bit easier to understand so here we just have a times the energy times this and this is of course one because it's the same with function and the other part would be ar times the energy and then we get phi left phi right and this is again alpha the non-orthogonality factor okay so here we will have something like this so if we collect everything we say on one side we will have that the coefficient multiplying epsilon a L are following its alpha epsilon minus e minus gamma and all the coefficient multiplying ar will be epsilon minus beta minus e equal to zero so i just moved everything on the right hand side actually no let's uh that's sorry let's make it this is a bit easier let's not do this let's just collect them in this way.\\
And if we do the same for the other let's say equation when we multiply here by the right-hand wave function you will get the following OK. So this is now our algebraic system of equation. But in this simple case of where I have only two terms interacting can be solved. And so you can find Al and Al as a function of this parameter that you can calculate because they all depend on the isolated problem quantities. So, epsilon, gamma, beta, alpha, we can all calculate them if we know the original problem. If we use this approach that we think we already know the solution of the isolated problem. But actually what we will do now is make a mathematical trick that allows us to make the system much simpler and also to exploit this more complicated problem where we have for example an infinite chain of atoms or 3D crystals.\\
So what I would like to show you is that if I set alpha to zero, if I find a way to eliminate all the terms in alpha, so if wave functions are truly orthogonal, what happens is that essentially my system will become like this. So this is alpha. And this is now beautiful because this is now an eigenvalue, eigenvector problem. So you see I can make basically this my eigenvalue and aL and aL would be eigenvector. Why the thing just above was just some JPEG system of equation with no way of solving it, casting it on it, but using the same matrix formula is not making value and having vector. So is there a way that you can use to set this alpha to zero? Actually there is, it's called loading autoconalization And this is just a case, a simple case of our two orbitals, which I'll show you. So you can basically start from the two S wave functions, the original one. So the orange is from the left quantum well, the green from the right quantum well. And you can mix them up in this way. So you can calculate alpha, your non-hortogonality factor, and put it in this pre-factor and you create a wave function which is this one, which you see is now a bit different. And this negative part is just what you need to make the two-way function. Now, in the case of the two quantum well, we lose the symmetry of the system because now these things are no longer even or odd. But if you imagine a crystal, you will maintain also the symmetry because you will have left and right will be always indistinguishable. So this is a visualization of this expression for so L would be the left one to L and L are the right one and you get this and if you apply you can do this for any refunctions so this is the case of p orbitals and now you see that the two things so now this is already positive and negative but now you see that the peak value of the wave function is a bit different in one case from the other and you will get so from the original wave function you get something which preserve the same symmetry but it's a total and so it means that we can basically use the eigenvalue and eigenvector approach it actually means more because if I knew from the beginning that I could use this phonality approximation is not in method I could have written the Hamiltonian immediately in a matrix formulas which means that I could have written down a matrix where a column was left and right orbital.\\
The row was also labeled on the left and right orbital. And then here in this matrix I only had essentially the different elements that I get in this matrix crossing columns and rows and this would give me essentially exactly this coefficient. Okay, so if H is T plus D left plus V right and you apply this matrix formula is for solving the Schrodinger equation, you will end up in this matrix, which is our Hamiltonian matrix. Okay? We will make an exercise later to show that this is true, but you can also try and just fill this and calculate this and you will end up with this equation, with this matrix. So how we complete our problem? So it means that if I multiply this matrix by my eigenvector, which is AR and AL, I will get the eigenvalue multiplied by the same A vector. So from now on, probably we will meet in the future, we will directly use this kind of formulas. So when we know that basis is orthogonal, we will use all the approximation we will make. We typically deal with using a limited base of wave functions, so in this case with two, you can use more. But then, so if I use, let's say, when we look at the type binding, we will have six different states that we combine, and we will have a six by six matrix, but we basically can fill the element of this matrix simply applying this matrix formulae and then solve the eigenvalue, eigenvector problem. Okay? So let's do it now for this simple case.
\fig{5}{LCAO.pdf}
So to find the eigenvalue, which is the energy A in our case, I just need to calculate the determinant of this and set it to zero. And in this way I will find E, the energy, and when I know E I can calculate the value of a L and AR from the different states so let's do this so this will be simply epsilon minus beta minus E square minus gamma square equal to zero. And so I will have epsilon minus beta minus E equal to plus minus gamma if I take the square root on both sides. And so I will get that E is equal to epsilon minus beta minus plus gamma So I can draw a An energy diagram similar to the one we have shown before so from the isolated quantum well i had a certain energy epsilon and now when i couple the two quantum well i will get that this energy becomes epsilon minus beta and then I will have plus minus gamma. Okay, so and these are the two bonding, the anti-bonding level, which are separated by quantity, which is equal to 2 gamma.\\
So, now that we know the, let's say, eigenvalue, we can find the eigenfactor. So, for example, if this energy will be epsilon minus beta plus gamma, and if you replace this in this original system of equation, of algebraic equation, you get as a solution that in this case a left needs to be equal to minus a right. And then you can normalize the wave function and find the actual value of a L and a R. So it would be 1 divided by the root square of 2 and minus 1 divided by the root square of 2 because the overall probability needs to be 1 as you know very well. And if you replace E minus beta minus gamma, which is a lower bonding level, you will simply get that Al is equal to plus Ar. So they are the same. Okay? So it means essentially that in one case, so this will be the E equal to epsilon minus beta minus gamma, I simply have to add the two wave functions. So the left and right will be added so we have this kind of overlap with some probability of finding the electrons in between and this is the say electrostatic glue we were discussing before.
\fig{6}{LCAO.pdf}
And the anti-bonding level since we a left so let's recall this is a case where as a L is equal to a right in the anti-bonding case one is opposite than the other and so you see that we have this node and they kind of go to zero at this point. Okay? So this is a very synthesis of what we have done.\\
Let's discuss a bit about the sign of our overlap integrals. So we have seen that the overlap integral is actually the same, more relevant quantities we have calculated, because the on-site integral, $\beta$, is only shifting the energy level level but is not giving splitting, which is what we are interested in. Interesting physical phenomena arises because we have bonding and anti-bonding of balance and conduction and energy gap. So in our case, I showed you that gamma was positive in the case of our orbitals. And the reason was that it's shown here because it's essentially an overlap between positive will function, the positive will function, and negative confiding potential. But actually, for example example if I consider the next energy level so if I overlap p-type with function you know important in semiconductor I would get the opposite behavior the gamma is actually negative okay and this can be shown here because now we have the overlap between let's say the negative part of one but also partially the positive part of the other so Sorry. Sorry. Okay. So here you see that here we have this sign change. So this produces a change also in gamma. And the other thing I want to point out because we will use it later is that if we generally look at the overlap between s and p orbitals, the absolute value of the overlap is stronger in s orbital than in p orbitals again due to this change of size. So this is actually the reason why in this plot that we were discussing before.
\fig{7}{LCAO.pdf}
\fig{8}{LCAO.pdf}
\fig{9}{LCAO.pdf}
\fig{10}{LCAO.pdf}
\fig{11}{LCAO.pdf}
\fig{12}{LCAO.pdf}
Okay, so this is essentially two times gamma is a separation between bonding and anti-bonding and this is why in this drawing we will consider assuming that the way this bonding and anti-bonding states diverge from the isolated state it's larger in S than in P because of the better overlap that we can kind of understand looking at this picture here okay so I will have to but maybe it's now it's a bit too late I would like next lecture we will address the problem where we have frequent wells. Now what I wanted to just show you is why this kind of calculation is interesting. So you know then we will make an assumption also for several. Because actually, so we as I told you at the beginning we are moving on two different, let's say, scenarios. One is we assume that we are talking about atoms, SMP orbitals, but we can also apply the same kind of thing to artificial atoms which are nanostructures like quantum well. And this artificial structure are actually used for example in devices like far infrared photo detectors. So quantum well infrared photo detectors.\\
Using this matrix approach we are calculating the solution of our problem and the second one is to show you what happens when I have more and more quantum-well atoms interacting and as I anticipated last week, this is used for some design of devices exploiting nanostructures structure, like we discussed the quantum well inference of the quantum cascade-laser. So we probably wanted to address its portfolio. Let's imagine that I have three quantum wells. They are nominally identical and I call them one, the center one, two, and three, one of the three sides. So our Hamiltonian is essentially given by, as in the case of two quantum waves, we have kinetic energy operator and then we will have the overlap of three different quantum well confining potentials so if we again set to zero the energy level here i will have the square well for quantum well number one number two and number three and again my starting point is that i suppose that I know the solution for the isolated problem. So if I take only one two well number one, I know that this will give me a solution and we will consider for example the first state, the S-like wave function is, and this will give me an energy of the isolated part moving. Now if we follow the same approach we use over two quantum wells, it means that we are looking for a solution, let's call it psi, which is just the superposition with different coefficients of the wave function located in the three quantum well. So we'll have a coefficient a1 which multiplies the wave function centered in quadrant one plus a2 phi2 plus a3 phi3 and as last week these are essentially the same wave function just translated so let's imagine that is zero our system and we have a distance a and a half in one direction a and a half in the other direction we say phi 2 will simply be phi 1 translated by which is minus a l and wave function number three is essentially the same but in the other direction okay Okay, now since we learned from the past lecture that I can use loading orthogonalization Now I will assume right from the beginning that these three wave functions are orthogonal so they are not exactly the original wave function, they have been modified similarly to what we did, what I was showing you for the case of the 2.12 in such a way that the overlap is zero so the so-called non-autogonality factor is really zero and then I will make two additional hypotheses with our simplified resolution the first one is that i consider only near neighbor interaction say here i have in this problem we have two kind of entities that say the wave function and the potential of the quantum map so to have an overlap integral whether it be the all-side overlap integral, or properly say overlap integral, I need this free angle to be near-neighbor, otherwise I will consider them zero. I will consider only near-neighbor interaction and later on to power simplify the solution I will also consider the beta what we call the on-site integral to be negligible okay so again since we have already done the exercise step by step in the last lecture now we will go straight ahead and write down the Hamiltonian in a matrix format so I will take I will build up a matrix where every column is leveled after one wave function every row is also labeled as one with function and so if I consider my complete Hamiltonian which is this one the one where I have the one where I have all the quantum well and kinetic energy each element of this matrix will be essentially obtained like this and so on and so forth this way will be all the other contributions. Then I can find the eigenvalue of this matrix so the eigenvalue will give me the energy of the data I obtained with the linear combination of atomic orbitals making these three quantum wells interacting. When I know the eigenvalue I can also get the eigenvector, so the values of the three coefficients a1, a2 and a3. have an idea of the shape of my final wheel function. So let's start to do this and for example I start by calculating this element so this will be 1T plus V1 times this plus this V2 times 1 sorry plus Psi 1 V3 Psi 1 okay so the first term is very simple to get is essentially the energy of the isolated system so this is just epsilon. The second one let's look at the shape it takes so I have the overlap between essentially this wave function and quantum wave number two so this is basically what we call the on-site overlap so at the same orbital sitting on the same side interacting with some potential sitting on different side these are near neighbors so the wave function and the quantum weather near neighbors so this term is not zero for the moment and so this will essentially be minus beta the on-site integral and the second term will look exactly the same because it's essentially the same kind of overlap but we've got well number three which is the one in our left hand side okay so this this first term can be written as epsilon minus beta which is minus two okay so i would So we first write down the matrix in the, all the terms and then we make, to simplify calculation of the determinant we will set beta to zero. For the moment we calculated properly. So let's move on now to the second term, say this one. here I need to multiply phi1 times T times the Hamiltonian times phi2 okay so I can get one term which is following one term which is following the last term that will be this one So the first one is what in the last lecture we called the overlapping integral because it's essentially superposition of this wave function with the wave function here times the overlap times this quadrant here so it's what we call minus gamma Here we essentially have epsilon times this because the kinetic energy operator plus the quantum well of the confining protection of quantum well number two and the wave function of quantum well number two is essentially the solution of the isolated problem. So since we use the loading orthogonalization this is now zero. The second term is actually a mixture of functions sitting on different, on three different quantum So we have one wave function of quantum well one, the wave function of quantum well two, and the potential of quantum well three. So we did the hypothesis that we consider only near-neighborly interaction, this is not the case because we have wave function with different elements of this integral of this matrix element are sitting on all the three quantum well. So we set this to zero. so this will be minus gamma again since quantum well number the problem is symmetric so if I do the same kind of reasoning for this matrix element I will get exactly the same result I will also get minus gamma okay so now let's say this The matrix needs to be symmetric, Hermitian, and so I start to calculate this value because this one will be essentially minus gamma and this one will also be minus gamma due to the fact that we are calculating in Hamiltonian and it needs to be an Hermitian quantity and we are assuming that these are real quantities because our energy is essential. simply just copy them on the other side of the diagram. So now the next term is this one which is essentially the following. Okay so this will be something like this. So again, the first step is just the energy of the isolated problem. The second term is again an on-site integral, so we have the same wave function and the different potential so we call this minus beta and the third one would also be an on-site integral but actually we have again elements which are not near-linear So 3 and 2 are separated by 1, so I set this to 0. So my next element is essentially epsilon minus beta, and for symmetry, if you do the calculation for this element here, you would get again epsilon minus beta. So now the only left term is this one. So let's calculate 2 and 3. So we can calculate it like this. And now this is again epsilon times the two-way function, which are orthogonal, so this is 0. And the other two terms are again 0 because we neglect, we decide not to take into account no near neighbor interaction. So you see again you have 1, 2, 3, 2, 2 and 3. Every time, in all this integral, I have no near neighbor interaction. So it is also 0. So essentially, our term here is 0. OK. And so essentially I got a matrix that you can see in the slides. So now, essentially, I need to calculate the determinant of this matrix here. But to make it a bit simple now we really set beta to zero. We make these hypotheses that beta is negligible. Of course, in principle you can do in the calculation also with beta it's just a bit more complicated to do so let's slip it by a bit and set it to zero. so we obtain this determinant. yeah okay well E is our eigenvalue so the energy of our system of equivalent So if you set this determinants to zero, you obtain something like this. which means that I have one solution which is, so I can collect essentially this So one solution is that the energy is the same as one of the isolated problems. And the other solution is essentially that the energy E capital is epsilon plus minus the root square of 2 gamma. So we have three levels which are essentially spaced. one is in the middle and the other one is 2 square of 2 gamma above and the other one is below so if we put that let's say in a order we already say that when I have s-type orbital gamma is a positive value. So this energy levels are essentially one is placed at epsilon minus root square root gamma. The other one will be at epsilon and the other one will be at epsilon plus root square root gamma. So now I don't go through all the calculations but essentially you can replace this value of E in this original eigenvector eigenvalue problem which essentially was the following Okay. so if you now replace for E these three values that we found for every energy level, every eigenvalue, you will find your eigenvector. So for the fundamental state, the one where capital A is equal to epsilon minus 2 gamma, get the following. You get that a2 is equal to a3 and a1 is 2 square 2 a2. Then of course if you want to find the actual values you need to impose the normalization condition. So the other condition is that a1 squared, a2 squared, a3 squared is equal to 1. We are just interested in the relative variation. So for the moment, we simply forget about this. So how this wave function will look like? So if I plot here the free quantum wave, I will have one state here, another one here, and these are exactly the same height, A2 is equal to A3, and in the middle I will have the wave function which is multiplied by 1.4, so we'll have something more like this. So the first quantum profile level will have this kind of shape, the wave function. Now if I look at the other, the second eigenvector, where this is epsilon, you replace the values and you essentially get that a2 is equal to minus a3 and that a1 is equal to 0. Okay? So again I can make a hand drawing of how the wave function will look like. So I will have something like this. And then something like this. Because A2 and A3 are one the opposite of the other and A1 is equal. Then if I look again at the last solution. so this will be epsilon plus root square root square two gamma what you get is that again a1 is equal sorry a2 is equal to a3 this system is symmetric I mean the a2 and a3 can only be or equal or one the opposite of the other so the function must be even or odd there is no other possibility and a1 in this case is minus root square 2 a2 and so the shape of our wave function will be like this, we will have these two which are equal and this one that will be negative, okay? So just to show you how it works, even with a simple exercise, with all the approximation we did with the pencil paper I here would like to compare the results that you get actually solving this problem numerically so this is the solution of the numerical software of Poisson Schrodinger equation this is not the wave function it's the square root of the wave function so it's a square away from the probability of finding the electron in some places, but we immediately noticed that our simple solution has some strong similarities with this. So here we see that we have essentially three levels which are basically spaced equally. and this one is more or less the original one it is not really the case because this solution I did not consider beta to be zero so when I solve numerically I have all the possible solutions but you see that we have equal space level and if you look at the shape of this with functions so the plume line is a fundamental level, the one with the lowest energy So you see that it's very similar to what we plot. The green one is the second level, so you see that in the middle is zero. And the last one is the third level, and you see that here we have a point where the wave function goes to zero as we obtain in our calculation. And based on this kind of reasoning, we can say that when I have two quantum wells interacting, I have two levels forming. When I have three quantum wells, I have three levels. If I have, for example, seven quantum wells, I will have seven levels. And so on. And then we form this mini-band. of course if I now move from quantum well to the actual atom with the crystal lattice what happens when I have an infinite number of atoms? I would have an infinite number of levels forming the band that we are looking for but actually to address this problem we will need to we have extra information that we can get and which is the fact that we know that the solution for a crystal lattice needs to fulfill a block theorem so we know that the solution needs to be in a form like this where we have this phase factor and this is the block function which is periodic in the lattice. So the next step would be to leave this linear combination of atomic orbital and move to what is actually called tight binding approach, which is essentially the same kind of fundamental idea but taken into account system periodicity. So we know that we have a periodic system and we use this information. So what we will do today is exactly this. So we will extend this linear combination of atomic orbitals in a periodic system. And we will see that already with this simple model, when we have a one-dimensional chain of atoms, we can understand something about vast structure of actual semiconductor. So for example, we can understand from the curvature of the band if the orbital giving rise to this particular band is of s or p type And we will also make other considerations regarding the overlap integral and its relation to the band dispersion and also to the interatomic distance.
\fig{13}{LCAO.pdf}
So this is a band alignment that you have in these kind of devices where you have a series of quantum well, and some are thinner and some are larger. So you can imagine that the electron will sit in the deep level, which is the one that you find in the larger. But this quantum well, which are smaller and coupled one with the other, will actually form several levels. So we have seen that if I couple two quantum wells, I get, where is it? I get two states. You can imagine that if I couple three quantum wells, I will have three equations, I will have three states. If I couple seven quantum wells I will have seven states. So these states will form a mini band, what we call a mini band. So it's a band similar to the one of the crystal but of course with a limited number of states because we have a limited number of orbiters contributing. So in this these devices, this quantum-wide infrared photo detector are used to detect infrared light. Okay, so wavelength of the order of 10 microns, so energies of the order of hundreds of millivolts. And in this case, instead of using a semiconductor with a very small band gap, which becomes too similar to, let's say, a metal, so it's difficult to apply. You have very large current, it's difficult to obtain P-N junction. What the people do here, it's essentially using only electron confined within the same band. So this would be, for example, the conduction band of gallium arsenide and you make this nanostructure. So you can have this transition now, it's of the order of 100 mEW, for example, due to the far-infrared absorption. So you need the electron to stay here. When the photon comes, you need to transport them outside your contacts. How do you make transport? You make a conduction band of states which overlaps and cross all the structure. And you obtain this by this several closed the space quantum way and you form a mini band just in the way that we have seen today. So in this way you are able to extract an electron and bring it to the contact when you apply an electric field. This is a kind of slightly different design which is so this one is called bound mini band. So we go from the bound state, which is this one to this mini band state. This one is called bound to continue. And essentially here we just make the structure in such a way that my photon of the right energy is able to take the electron out and then it can flow in the conduction band. So this is a use of super lattices, so artificial sequence of atoms for absorbing light.\\
Another application is the quantum cascade laser where you use the same thing to emit light. So a quantum cascade laser is typically made of two main parts. This part here, we will look at it later in our lecture, is where actually the lasing action takes place. So you have the transition of an electrode from this state to this one, and you can making a photon. And then you have this level which serves to depopulate the second laser level. So you know that you need at least three energy levels to make a laser because to take population inversion either you put a lot of electrons here or you put very little electrons here. So in this case, this case is used to depopulate this population. Why is called cascade? Because essentially the same electron which undergoes this relative transition is then injected in the next stage. And so again, it falls down this step and again can emit another photon. So this is why it's called cascade laser because the same electron in principle emits a number of photons, which is the number of periods in my system. Now, everything here is quite delicate because you need to inject the electron exactly in this state and to extract it exactly from the lower one. How do you do it? You do it through this here we could injector. It's actually again multiple quantum way, which are very narrow. So may interact one of the other. And so I have a mini band, a mini band, which brings the electrons on the lower level of stage one. to the upper level of stage two and so on. So, and you see that, I mean, basically this is a linear combination of atomic orbital applied not to atomic orbitals, but to quantum world orbitals.