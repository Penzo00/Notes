\chapter{Lecture 3}
\fig{1}{spin.pdf}
So from the atomic physics, we know that spin-orbit interaction is essentially a relativistic effect. So the electron moving around the nucleus fills an electric field, essentially, which is related to its orbital motion, which is described by the quantum operator L. Sorry, So it fills a magnetic field, and this magnetic field interacts with the electron spin. So in atomic physics, this is described by this very simple Hamiltonian, where I have a, we know the energy of a dipole, magnetic dipole in a magnetic field is essentially be proportional to B dot the magnetic moment. Here we have something very similar where E comes from the orbital motion. So it's a relativistic effect of the orbital motion. Now in a solid, things are a bit more complicated because the angular momentum is no longer a good quantum number. So you know that in physics and particularly quantum mechanics, conservation laws are associated to symmetries. So in a Coulombian potential from an atom we have a perfect symmetry, circular symmetry, so L, the angular momentum is a good quantum number. In a solid actually this is no longer the case. So even if I am at the gamma point, which is the point where we have the highest symmetry, I should actually treat the problem in a cubic symmetry. However, if we make the approximation that this cubic symmetry can be considered also spherical, we can actually treat the problem relatively easily and you will see that we will get a lot of information also in this way. So we will use this, the same Hamiltonian, let's say, which is valid for atoms, also in a crystal. So this is called the spherical approximation.
\fig{2}{spin.pdf}
So essentially now we want to add to our picture this spin-orbit interaction, which will be proportional to some constant, which for example, contains the atomic number. So lambda is proportional to the atomic number for power. So we expect that this effect is stronger in heavier atoms and weaker in lighter atoms. And then we will have this dot product between L and S, L is here. angular momentum, atomic angular momentum and the spin. So we know that, how do we calculate this? So we can define the total angular momentum, which is essentially just the addition of these two contributions, orbit and spin, and if we take now We calculate the square of the modulus of this angular momentum, quarter angular momentum, we get something like this. we get this expression from which we can calculate L dot S just as one half J square minus L square minus S square. okay so if we apply now this to our valence band we know that L squared is essentially this expression, okay, where L is the orbital momentum quantum number which in the P state is is equal to 1. And for example, for s state, this will be equal to zero. So for s state, we won't have any spin orbit interactions. This is why when I was previously commenting these pictures, I will focus on the difference in the balance band, but not in the conduction band. So the conduction band, we don't have spin-orbit effects because there is no orbital angular momentum. So we perfectly know now that this state, this curvature indicates that here we essentially have only s orbital, so no angular momentum. So this is why we made all this discussion just for the case of the valence-bound state. Then we have that S squared will have a similar dependence. And we know that j square will have also a similar expression. And since we are combining l states and spin, the values that j can take are only a given subset. So J can be equal to L plus S, which in our case means 3.5, or L minus S, which means 1.5. And then this means that this J equal to 3.5 can have four different values of JZ. j z, so the projection along the quantization axis, which can be this, while in this case j one and a half so l minus s can take only two values which are one half and minus one half okay so let's go back to our Hamiltonian so we say that it is given by this expression and what you can do if you make this calculation you simply need to replace the expression with just pound. So this is capital J, and this is the J quantum number. Okay? and in this expression essentially L is always equal to 1, can only be 1, we are discussing about P states, and S can only be equal to 1.5.
\fig{3}{spin.pdf}
So essentially I will get some states which corresponds to J equal to 3.5 and some states which correspond to J equal to 1.5. And the difference between these two energy levels, so if you make HSO for the case of J equal to 3.5 minus HSO for J equal to 1.5, So I don't make the calculation, but essentially you get this value, 3.5 lambda h bar squared. So it means that we have a separation between states having j equal to 3.5 and j equal to 1.5, which is given, activated by this spin-orbit interaction. So this is actually the splitting that we were observing before. So what we have calculated now is essentially this quantity, 10 to 0. So in the use of spin-orbit interaction Essentially we moved from this picture where all the states are degenerate in gamma to a situation where we partially removed these degenerates. So we'll have essentially the states corresponding to j equal to three and a half at the top of the valence band and the states corresponding to J equal to 1 half separated by an energy delta 0. And this energy will depend on spin-orbit interaction. As in atomic physics we say that we know that this lambda factor depends on the force power from the atomic number. So we expect the attack to be larger in heavier semiconductor than in lighter one. So let's go back again to our let's say prototypical case. So we see in gallium arsenide we have this splitting. In germanium, the splitting is very similar because essentially all the average atomic weight is the same. So gallium is just germanium with one electron less, arsenic is germanium with one electron more. So average we expect is very similar in orbit interaction. If we go and look at the case of silicon, on this scale, we don't see any splitting. So spin-orbit interaction is much, much weaker because silicon is a very light element compared to germanium or gallium-1-synon. So if you go to tin, for example, we will have a very strong spin-orbit interaction. Okay, so our picture led us to following the conclusion for the Vanes band. we have a band which is on top here, a band which is below here, and they're separated by this spin-orbit interaction. And this state will correspond to J equal to one half, and this point will correspond to J equal to three and a half. So this also explains why this upper band will be four times degenerate, and this lower band will be only two times degenerate. Now in degeneracy I'm also putting spin, because essentially this three and a half value of J can take four different values of JZ. So as we said before, JZ can take four different projections along the quantization axis, so this will give our four degeneracy, while in this case JZ can take only two different projections along the quantization axis. Now I think we can make a break and then we will dig further into the relevance of the projection of the total angular movement. Okay, so we introduced this new term in the Hamiltonian of the valence band and explain why we don't need it for the conduction band. And we see that essentially now we have a splitting of energy between states with a total angular momentum of three and a half and the state with a total angular momentum of one and a half. Now I would like to do more and actually not just the energy dispersion but on the orbitals entering the state. So we move let's say from the eigenvalue to the eigenvector. And essentially, so far we discussed our tight binding approach using for example px orbital and now we need to add spin degree of freedom. So one reasonable candidate to describe this state would be to add to this px orbital spin up or spin down state to describe it. Actually this is not possible because we see that essentially we need a state which has a given value of j and also states which have a given value of jz. And if we look at how these px orbitals are constructed, we need to back to the fact that the solution of angular momentum operator are the spherical function. So px is actually a linear combination with this pre-factor of a spherical function with L equal to 1 and also M equal to 1. So just to be a bit more clear, so the solution of the L-square operator are the spherical functions which are leveled with two quantum numbers. So one One is the it's associated to the value of L square and the other one is associated to the projection along some direction. So the exorbital are a combination of state. that we have the same value of L but different values of M. So this leads to the situation that if I add now the value of spin, for example spin up, this state will be, let's say, a solution we will possess a good quantum number L, the spin, but not the total angular momentum, because these are two different projections, let's say. So to actually describe this state here, the one with different value of jz, I need a different combination of orbitals. So the derivation is actually quite... it's not straightforward, let's say. So as we did for the case of the Hamiltonian, I will show you the solution of this problem, and then we will try to get some physical insights from this solution. let's say that the procedure to obtain this is called the derivation of a Klebsch-Gordan coefficient. Okay, so the point is the following. We know that we have state with certain number, value of j, and we want them also to be, to have j z as a good quantum number. And this is again obtained by combining the different orbitals that we have.
\fig{4}{spin.pdf}
So let's write down the solution for the first set of states, both with j equal to three and a half and j z equal to plus three and a half. And these states can be written down as follows. Okay? And the state with the same value of j but the opposite projection of jz will essentially have this kind of expression. Okay? So this state will have all the ingredients we need. So we have a given value of L, which is always 1, given value of S, which is always 1 up, and then we also have J and JZ, well defined, which are essentially the values that I'm writing here. Okay? So we can then plot down This other state, in here the correct combination is the following. And We get the other two sets of states. So again, here you can notice that these states, they don't have the well-defined SZ. So the projection of the spin is not up or down, because they're the combination of up or down. Because again, the general idea is not to have states with the well-defined projection of the orbital angular momentum of the spin, but just the well-defined value of the total angular momentum and projection of the total angular momentum, okay? And the last set of states will be essentially this one. And again, here you see that we have a mixture of up and down states. So again, I haven't found yet the way to put it in a simple way, but essentially the correspondence between this state and the band dispersion is the one given in this table. So you have to take it as it is, but we will try to draw some conclusions and see what happens from this delegation. So essentially, what we typically call the heavy hole band, heavy means that the adaptive mass is larger and so we have a larger radius of curvature. It's essentially composed of states which has J equal to three and a half and JZ equal to plus minus three and a half. And one thing we can notice from the expression I wrote down here is that essentially these heavy-on states do not contain a component of the PZ direction. So if I choose Z as a quantization axis, say, I only have Px and Py orbitals. So we can say no p orbital along the quantization direction. Okay? The second band, which is degenerate with this one, gamma, is what you call the light holes, because the radius of curvature is smaller, so the effective mass is smaller. And actually, it corresponds to states where J equal to 3.5 and JZ is plus minus 1.5. And these states do contain the z-orbiter, and actually, it is actually, let's say, it has a relatively strong weight. So we have this factor of two here. And eventually, we have this band, which is separated by delta zero, the spin-off interaction energy that we calculated before, which is called split-off band, because it's splitted on the two original ones. And it has also similar properties, similar to the one of the light-on. And so this is essentially a summary of the expression. We have the I don't know them by heart, and you don't know, you need to know them by heart with expression, it will be meaningless. But you need to remember that, for example, that you will not need to actually make any effort because you make a lot of examples, and you will essentially, you can use a lot of effect that these states are actually not containing p-z orbitals while the others are dominated by these other orbitals. So this is a two dimensional picture of what happens in the balance band. And we say we obtain this in this spherical approximation. So assuming that actually that the potential seen by an electron in the center of the It's spherical, but we know it's actually, this is not the case, this is cubic. So if you do a more appropriate calculation.
\fig{5}{spin.pdf}
And you calculate, for example, the isoenergy surfaces of different semiconductors, you will see that they are not sphere, okay? And if we move from, you see this is the case of silicon. So silicon is extremely weird because the spin-oblique interaction is very small. So we see that we have what we call the heavy or valley sort of cube, but actually if you make a cross-section, if you cut it, you will see that this cube is actually has this shape. And then this, if you go to the light-on and ebio band, you have a very complicated symmetry. And it's also strongly dependent on the energy. So isoenergy, I can take an energy of one micro-electron volt, one milli-electron volt, one volt, and we get different isoenergy. So if you move, stay close to the center of the brillouin zone, the light-on and ebio looks, a split of band looks more like this. So we take a expression which is less strange. No, actually, sorry. So this is the evolution at different energies, okay? Of the first band. Sorry, it was mistaken. So here it's the, so this is the, let's say, heavy old band for silicon. And these are all heavy old bands, but at different iso-energetic energies. energy. So you see that in silicon you have this kind of star-like behavior, while in gallium arsenide and in inium etymolide you get a cubic shape of the isoenergetic surface. So it means that in any direction you will have a different effective mass actually, okay?
\fig{6}{spin.pdf}
So what we did in this lecture was to actually be able to label the generic structure of the semiconductor according to this table, which is summarizing the main different behavior that I can find in different semiconductors. So what we understood is that I can have a conduction band that can be either direct or indirect. from what we said at the beginning of this lecture, we understood that if I had a direct gap minimum, this is typically the symmetry of S orbitals, so it's even, while for example if I move along the delta direction of silicon, I have a mixture of S and P orbitals. This is what we see by decomposing the SP-3 matrix. And then We understood that actually the valence band is typically made of these three bands which we call the heavy or the light or the split off and from the point of view of the orbital component, they are essentially decomposable in this way. Here you can find essentially the decomposition also in terms of So this first wave function here is the spherical harmonic. While this is the same wave function of Taylor as the superposition of Px and P1. okay so this is a general behavior that the semiconductor can have now we will briefly go through some let's say prototypical semiconductor and see which is the difference between one and the other okay so first of all a bit of geography of a brino and zone of a FCC in the top of FCC crystal, so zinc blend and diamond. So we typically, so gamma as you know is the middle, the center of the brilliant zone, and then we typically use Latin letters to indicate points in the brilliant zone, so except gamma is the only exception, the same. So L is actually the point here along the one one one one one direction of the green zone so it's a point at the limit of the border of the green zone in the one more one direction why the one one one direction itself takes the name of a greek letter and its capital lambda The other direction we will look at is the other point, is the x point, which is at the limit of the brillouin zone in the and all equivalent direction. And the direction is indicated by the Greek letter delta. So we will mainly focus on these points.
\fig{7}{spin.pdf}
So silicon, if we look at the band structure, we see that they are not to scale. It's more for didactical purposes. It's a semiconductor with a minimum of a conduction band. It's not at the border or the center of the Brillouin zone, but it's in between the x point, which is this one, and the gamma point. So this is gamma, and this is x-point. So this direction is called delta, and this minimum here, which is called delta valley, is essentially more or less at 85\% of the Brillouin zone along the delta direction. So it's very close to the x point, but it's not at the x point. Why this is important? Because this has an impact on the degeneracy of this state. So delta is 1, 0, 0 direction. We have six of this direction in the cube, like the six phases of the cube. So this means that I have six of these delta valleys, which are all energetically equivalent, but are in different positions of the conduction band of silicon. So the way we will use to plot this valley is drawing these ellipses, which are ellipsoids, which are essentially the isoenergetic surfaces of this state. So in silicon you have six of these and you have the six-volt degeneracy, which is important for example in conduction. You can imagine that that from one electron moving from one valley to the other means essentially a process which requires zero energy because they are all equivalent and only momentum transfer. So it's a very efficient scattering mechanism. So for example, this can be a limit for MOSFET using silicon. Or on the other scale of devices, say if you want to build a quantum computer isolating a qubit, you'd say, isolating a single electron in silicon, well the problem is that now you have six of these states where this electron can sit, so this is already complicated your life if you want to address a single spin, a single state in the semiconductor. So another indirect gap semiconductor from group 4 is germanium and in this case you see that the minimum is along the SDL point, so along the 1, 1, 1 direction, so what we call the capital lambda direction. Okay, and then here you have all the other trends drawn, so this is just a minimum band here. So which is the degeneracy of this state? So again if we plot this ellipsoid along the one-on-one direction, we count eight of these states. But actually, what happens here is that since the minimum is exactly at the border of the Brillo and Zorn, we should only count half of them. So meaning that, let's say, this part, the state, is actually corresponding to the other side of the Brillo and Zorn. So the fact that the minimum happens really at the border of the Brillo end zone gives the fact that we have 8-fold degeneracy but only 4-fold degeneracy because half of the state of this valley here is essentially on the other side of the Brillo end zone. So this is why in Germania we have only 4 L-valleys. Okay, another characteristic of the medium that we will look at is the fact that, so the energy gap, the minimum energy gap is around 0.66 electron volt here. But actually, we have another gap which is only a few, 840 milli-electron volt above, which is a direct one. So for example, you can immediately understand that if I shine light on a piece of germanium, I will have a strong absorption from this transition, which is direct, the very weak absorption of this transition, which is indirect. And 0.8 electron volt corresponds exactly to 1.5 micrometer, which is an important wavelength for telecom applications. So, for example, gemium is used as a material that we can form detectors for telecom applications. Let's take another example.
\fig{8}{spin.pdf}
So this is gallium arsenide. Again, this is not to scale. And this is a simpler material in the sense that the So, the band now is direct, so we have the meaning of the gamma point, so we detect any degeneracy, we have only one state. Of course, the interesting thing in this plot is that we have also, let's say, the other values which do not disappear. For example, in a high power laser, if I want to have a high power, it means I need to have to put a lot of electrons in my valley and so we believe that at some point I can have also electrons in the other indirect valley and they are not participating to the laser action but they are simply increasing the threshold current and giving losses for flutary absorption. we will also in the future look not only at the say minimum energy gap but also the one close by especially if the energy difference is not very large and one thing you can notice here is what i was pointing out to you before uh when the case of silicon is not written in this plot but essentially this split of band in silicon is only 40 milli-electron voltage well as i was mentioning in germanium in 0.3 i know it's written here 0.29 0.044 was written, 0.044 and this is again of the order of 300, so as we were mentioning before. Another example that I would like to show to you is aluminum arsenide. So aluminum arsenide is a 3-5 material, but it's an indirect gap. So the gap is essentially between this state and now this is really the x-point, it's not the delta direction. So in silicon we have something like this, okay? So the minimum is here. In aluminum arsenide, we have exactly the minimum at the border of the brillouin zone. So for example, this reduces the degeneracy for the reason I was showing you before in the case of adjiminium. So we have here now that we don't have the six-fold degeneracy like in silico, but it's only three because half of the states are, let's say, outside the brillouin zone. So we would count them twice.
\fig{9}{spin.pdf}
3-5 semiconductor, sorry, 2-6 semiconductor are in most cases direct gap material. So we know the example of carbon-methyluride. And they are relatively strong spin-on interactions, so we see that we have the direct gap here. Just as an example, a final example, a material which is in the crystallizes in the wood side structure. So it's no longer cubic, it's an hexagonal symmetry. So it is a material with a very large diary gap. And here we see the interesting thing is that moving away from the cubic symmetry to the the hexagonal one, we completely remove the degeneracy of the balance band. So also at gamma, the central point of the green windsor, the heavy holes and light holes are separated in energy. And as I was mentioning before, this is a material at the base of white LEDs. So in 2014, the Nobel Prize was awarded to three scientists that actually were able to synthesize this material because it's not easy to grow, it doesn't have a proper substrate. And nowadays it's used also for power electronics. For example, my computer charger here, it's made with gallium nitride, because having a very large band gap can give a lot of power heat up without still maintaining this operational convention. Okay. So I had to introduce the structure of the valence band without a lot of explanation. Now we will try to use this expression and understand why I wanted you to know different combination of orbitals entering the balance band. And one thing where this combination of orbitals is important is the study of optical transition selection rules.
\fig{1}{sel.pdf}
So you have seen some of this thing already before, but not all of you have studied engineering physics, so in some sense we heard it for the second time. But I would like to add something new, which I think you didn't hear before, and how we show you how the symmetry and the combination of the state in the balance band can actually allow to inject spin polarized electron in a semiconductor. So there is a field of spintronics which doesn't include magnetic materials, where the spin is induced externally, for example, by using optical absorption with circular polarized light. So we go step by step. First we will see the impact of symmetry of the valence and conduction band state of selection rules, you will see what happens in the case of linearly polarized light, and then you will see what happens in the case of circularly polarized light, how we can use this to inject spin. Okay. So the starting point for discussing optical absorption is of course the famous golden rule, so which gives us the probability that the transition between an initial and final states take place in the dipole approximation. So we write down the complete expression. So which are the ingredients in this expression? So of course this delta direct function just tells us that we need to conserve energy. So the difference between the initial and final state is the photon energy. This is, let's say, pre-factor, where essentially E0 is the amplitude of the electric field. the main ingredient we will focus on is of course this matrix which will tell us how strong the transition is and if it's possible or if it is possible or not. So f will of course be the final state. P is a polarization vector of the electric field. E is a momentum operator. And I will be the initial state. Okay. So, in the case of, let's say, a crystal, we know that the final and initial state will share a common, let's say, formal aspect. So, they are, they need to fulfill block theory, if I'm considering a crystal. So there will both be functions like this. So the final state will have This kind of expression and the initial state we'll have this kind of expression.
\fig{2}{sel.pdf}
So, if we now want to calculate this matrix element, we can take out this polarization vector from this expression. So let's infinite this. Well, I will integrate over the whole space, okay? So this integral, because of this operator, will be split into two different integrals. in one case I will derivate this exponential part, in the other case I will derivate the other block function part. So we get something like this. So if I make this derivation, I will get essentially Ki, h bar Ki, which multiplies again this. This is essentially a vector now, I want to highlight it now because we have this dot product, so of course also before everything here was a vector, say. And this is, let's say, the first integral, and the second one will be the following. Okay, where essentially I'm derivating now the poetic part of the block function. So this first part of the integral, the one where I derivative the wave phase component of the block function is zero. Why it is zero? Because essentially we have this multiplication factor which I can take out of this integral And then I just add the overlap between two wave functions, the final and the initial wave function, which are the solution of the same Schrodinger equation problem, the solution in our case of crystal lattice. So they are two non-degenerate states, so essentially they are orthogonal. So this term is actually zero. So we see that this wave component, this phase component of the block fraction does not really enter into the selection rule calculation. So we are left with this other expression. now we can take the hard way and say let's consider a generic expression between any two points of the balance of conduction band which means that kf and ki are different it means that i need to add a phonon also in my picture the third particle giving the momentum to cluster from one state to the other we can take it a way which is a bit easier and we consider only direct transition. So kf is equal to ki, and if we do this, so this term here, this multiplication is essentially one. So we are left with the fact that our probability of having the transition or having it or not at all, it's given only by this time where we have essentially the block function of the final state, of course I can replace one state with the other because then we take the square of this matrix element, and the momentum operator applied to the other state. So what we will use now, we will make this step forward and consider for example that the final state, if we are dealing about the direct band minimum, is an S state, and the initial state comes from the conduction band, balance band, sorry, and can be heavy-old or a light-old or a split-off state. And we will see what happens in these three different cases, okay? So let's consider the case of a heavy-old. So we want to have a state like this, and then we have the momentum operator and then we have our Avio state which we can write down now in the using the Clebsch-Gordan coefficient okay now we we need to fix one thing here because here I have a spin, while here I didn't put any spin level. So putting up or down is equivalent. I mean if I put the spin down here, what happens is that this momentum operator is not going to act in any way to the spin, and so at some point we overlap between spin up and spin down state. So this transition is essentially forbidden. So the only possible case I can have is that I will go from a spin up state in the avial band to spin up state in the conduction band. Because the other one, this matrix operator is not going to change the spin part of the wave function, so there will be a torsion. Okay? So let's look in more detail how this matrix element is formed. So it's actually made up of different contributions. so let's look at what happens if I multiply my s state spin up and then I take the x component of my momentum operator so I do minus i h bar partial derivative with respect to x and then I need to derivate the px component or the py component. Okay? So here, let's go back to our simple picture of our withdrawal s and p like states, like just even or odd So S1 will be an even state, and the Px orbital will actually be the odd state, but the derivation along x will change the parity of my system. system. If you go, let's say, dipole approximation with this operator is substituted by R, the dipole distance between the two atoms, so you will have that this is another odd function. And so you see that this term is not actually zero, it's different from zero. So if I derivate the x-orbital along px, I go from an odd function to an even one, so the overlap between these two is not zero. So to make it shorter, in many books you find written something like that. px here means the px orbital, and px here is the x component of the p of the momentum operator. So it's rubber. Since one of the textbooks I recommend uses this notation, I wrote it down, but actually let's say we will call this term something like this and this matrix element is called PCD P momentum operator between conduction and valence span. Okay? So it means that this first term will look like this. We have eX, let's say, X component of the polarization vector, which multiplies minus 1 over square root of 2, PCV. Then I have an additional term, which is what happens when I multiply. I take the derivative of along X of the Y orbital. And in this case, we need to essentially notice the following, that if I take and I draw, let's say, an orbital like this, this orbital is ordered in this direction, but is actually even in this direction. So if I consider the symmetry along the direction perpendicular to the orbital itself, it's no longer odd, but it's even. And so this means that when I apply the partial derivative of x to py, I'm not going to change the symmetry. And so it means also that all the terms like this are actually zero. okay so it means that this momentum operator with actor and we've got no zero attempts only when operating the direction of the order time to see okay And so, eventually, my matrix element, in this case, so let's summarize it like this.
\fig{3}{sel.pdf}
So if I'm looking at the transition between the heavy oil and the S state of the balance the conduction band so this is a balance band and this is a conduction band my momentum operator will look like this will be minus root square of two to BCV, which is coming with this. Ex times this, plus Ey times this. Okay, because also, when I go to the radiation of the y direction, I will only have this step. if something is missing, which is this. And so, the first conclusion we can have is that if there is an EZ component in the light engine of my sample this will not be able to excite every old transition so every transition will not take place to a transition where the direction of electric field is the same direction of the quantization axis that I have chosen for to solve my problem. okay now since we need to basically consider the square of this matrix element. I need to take the square of these two, add up this and this. So I get for the every-all s-state transition, the square of my matrix element is essentially one half pcb square plus one half pcb square which gives pcb square. Let me check that this is correct. No, it's not correct, sorry. Let's keep it separated. So if I have a, I will have X polarized light, matrix element, which is one and a half BCD square and similarly for y polarized light I will have a similar absorption strength and for Z polarized light I will have zero. Okay? So now we can quickly do the same thing for the transition between the light hole and S state. So this will be much faster because we understand now how how to work with this matrix element. So the matrix element will be S times this, and now the catch coordinate coefficient is the following. Ok? Now again, I've left this space blank, but let's say if I put the spin up state, I will have a transition between these two states and the spin up, but if I put the spin down, I will have it on the other side. So eventually, if I don't care the spin that I inject in the S state, I will in any case have all transitions, because here I can have both. So the final state will have different speed, but again the transition will take place between the valence and conduction band. So I leave it in this indeterminate form, let's say, up or down, but in any case I have both possibilities to put an electron in the conduction band, so total probability of the transition will be the same. So what does it mean now that the x polarization will give me what? PCD between s and px multiplied by 1 over root square of 6 and I need to take the square so this will be 1 6 PCD square. y polarization will give me the same 1 6 pcd square and the z polarization will give me four divided by six pcb square which means two-third pcb square okay so in this way i can essentially draw a table like this. Well here I have a different polarization and the two different transition.
\fig{4}{sel.pdf}
\fig{5}{sel.pdf}
So for each polarization between every hole, as I told you, I have one half and one half and zero. And in the other case, I have one sixth, one sixth, and two thirds, okay? Now we need to make a check about what we derived here, because essentially the direction of quantization for defining the total angular momentum of every hole is completely arbitrary. So if I rotate my system, I work and choose a different one of the main axes, I shouldn't get any difference in my experiment. So the total absorption between valence band and conduction band needs to be independent from the direction I've chosen arbitrarily for quantization axis. And this is actually the case because if you add up the total strength, so 1 half plus 1 half, and you add up these two things, you exactly obtain the same value. So we at least checked that in all cases the total strength is the same.
\fig{6}{sel.pdf}

\fig{7}{sel.pdf}

\fig{8}{sel.pdf}

\fig{9}{sel.pdf}

\fig{10}{sel.pdf}

\fig{11}{sel.pdf}