\chapter{Lecture 2}
\section{Tight Binding 1D}
\fig{1}{TightBinding1DChain.pdf}
\fig{2}{TightBinding1DChain.pdf}
\fig{3}{TightBinding1DChain.pdf}
\fig{4}{TightBinding1DChain.pdf}
So let's start to address this new problem. So the problem we want to address is the following. Let's assume that I have a one-dimensional chain of atoms, which are spaced by a certain distance a, which would be the period of our lattice. So the potential of this system can be written down as simply as a superposition of the coulomb potential of the different atoms. So I can write it down as a summation over, say, this lattice periodicity r, which goes in principle to minus infinity to plus infinity over the atomic potential of a single atom displaced by r. So graphically this means that I am assuming that my potential is a superposition of this different atomic, atomically confined potential. Now I will simply extract from this row the central atom Of course the choice is completely arbitrary So here I will have plus r direction and here minus r And I will write down the potential as following superposition of the atomic potential of the central atom. So this would be V atomic R, which is because in this case of the central atom, R is equal to zero. And the superposition of all the remaining atoms, all the other atoms in the chain, but for the same term. So we have this other atom here, which will give the protection of this. So I will call this potential simply as delta U this would be the summation over R different from 0 over the remaining atomic potential so essentially my Hamiltonian the Hamiltonian of my crystal will be simply given by these three terms that we'll have the kinetic energy T plus the atomic potential of the central atom plus delta O okay so summation of these two elements is essentially our crystal potential now in the same way as I did with the FCIO approach I write down the solution of my problem, let's call it the wave function, sorry again I will assume that I know the solution of the isolated problem so for example I know the first wave function is the energy of S orbital and I Let me write down my solution, Psi as the summation over R minus infinite plus infinite of these different orbitals Okay, where R is essentially running over just the lattice site Well, r is a continuous variable and now here I need to put coefficients as I did before different weights of this wave function and I will show you later that the only possible coefficient which which fills the block theorem that allows this function so the final solution need to satisfy block theorem and this happens only if this coefficient takes this form okay it's phase factor running just over the lattice side so this is slightly different from the usual block function. Well, this r is a continuous variable, so here this is just taking the value of the position of the lattice site. as I will show you later if we do this we have essentially the tower solution translated by a lattice parameter r is equal to this which is essentially block theorem so and this happens only if I write it down in this way if I use this coefficient this will leave it for later to do as we did for the crystal potential that was divided in two contributions, one from the central atom and everything else we will do the same for our wave function psi so we will write it as the atomic function of the central atom So when capital R is equal to zero, this face term becomes one, plus the summation over all the left sides but the center one of the stem. Okay? So now our problem can be written down as follows. Our Schrodinger equation is the kinetic energy plus the atomic potential of the central atom plus delta U, which multiplies the wave function of the central atom plus the summation over the other wave function and this needs to be equal to the energy of my system in our case this will be the band structure of our system times the wave function which is this one Okay, so this is the solution we are looking for. And as we did in the last lecture, we simply multiply this equation by this wave action. . on both sides of the trillion. ok, so let's see which kind of terms arise from this integral and different multiplications So first we can multiply phi s times this times this. So this is very easy. So let's call this term 1. It will be... And this is essentially the energy of the isolated problem. with our epsilon. Then we have a second term where we multiply the gain by s r by delta delta u and this is again an on-site integral because essentially I multiplied the wave function sitting on this central atom by the contribution of all the other atoms by the protobium potential but in the original problem it was just the quantum well on the left hand for example now it's a summation of all the different atomic atomic moment of tension so we will call this again minus beta so the next term I need to multiply this time space So let's call it number three. Ok? Justo? Should be? So this I essentially take out the summation. and now we have a set of integral like this one and again now I will rely on the fact that I can make this wave function octagonal so if I use again the loading orthogonalization method all these wave functions are octagonal and t plus v is an Hermitian operator So if an operator is Hermitian, it is Hermitian in any base which is also orthogonal So this is essentially the base orthogonalizing VAT is this one, the original problem, the isolated atom problem But now if I make the matrix corresponding to this Hamiltonian in a different base if this base is also top-on-up the Hamiltonian will still be a medium so it means that essentially I can do this write down this element in this way which is just complex conjugate so these two will be equal Now this is clearly just epsilon times this, and so this is essentially zero because we function R of tau. So this term number 3 will be 0. So now we are left with the last term and I need to multiply this times this Okay, so it's number four. Maybe it's the other way around. It should work. Except that the world could be a mess. I'm wrong. That may be the same. Okay. So the last term is for me. Again, we can take out the summation from the integral. and we can see that this is now just the summation of a series of overlap intervals so I have a wave function of my central atom times this retorchic protection of the central atom so the sum of the coulomb interaction, coulomb potential, and all the other atomic sides multiplied by the corresponding wave function so we can essentially call this gamma and it will depend on R so we will have an overlap for the near neighbor, an overlap for the second near, and an overlap for the third near, and so on And of course as r increases, this gamma r is going to be smaller and smaller because they have less and less overlap. Okay? Actually this will be minus gamma as we did in the past, we always put this minus sign here. So this means that our thing, if we go back to our Schrodinger equation, this is the left-hand side of the equation, it would be given by the following quantities. be epsilon minus beta minus the summation over all the atomic sides d to the power zero of this gamma r. Okay, now the left hand side was the following. so let's also calculate it so I will have this times this times this this is easy, this is just the energy E capital and then I will have this times this times this one and we use the loading orthogonalization method all these overlap integrals are zero so essentially the left hand side the right hand side sorry this is following so we immediately get essentially without any further calculation we get our band dispersion so this will be given by epsilon minus beta minus the summation of this over the So now to write this expression down in a more understandable way, we could see the just near-neighbor interaction which means that in this summation we only consider r equal to plus a, one lattice parameter on the right hand and r equal to minus t from our central atom we just move one step to the right and one step to the left so energy becomes epsilon minus beta minus Of course, since our system is symmetric, gamma A and gamma minus A are the same value. okay so we just call it gamma or if you want to call it gamma a you do the same for both sides of the crystal because being a crystal is symmetrical now we can do the following and write this two complex exponential simply as minus two gamma cosine k a okay so eventually we get this midband dispersion so let's compare this with some proper band structure calculations.
\fig{5}{TightBinding1DChain.pdf}
Now for example let's look at the band structure of silicon and gemelium. And for example if you look here this bottom band we say should come from the s-orbital and it actually shows a value which is similar to the cosine of k ok so this, let's say, resembles cos k okay moreover if we make a step further and we make a this approximation so we consider k a for zero so i go towards k pointing at the gamma point so under this approximation i have a the cosine of kA becomes 1 minus 1 half kA squared. So around gamma my bend dispersion takes a parabolic shape and the curvature of this parabola will essentially depends on the sign of gamma. So if gamma is positive, I will have a smiling parabola with a curvature pointing upwards, and if gamma is negative, I will have a sad parabola with a curvature pointing downwards. As we discussed in the last lecture, gamma is actually positive for s states and t is negative for p states so again if I take my very simple calculation and I compare it with this more complicated calculation of mass structure of silicon and germanium as example I see that here I have a positive curvature so I can immediately tell that this band will come from s orbital so it is formed mainly by the overlap of these s orbitals Instead, if I look at the top of the balance band, I have a negative curvature, and so I can immediately tell that these bands are coming from P states. So the balance band comes from the superposition of P states. Then if I look at the conduction band, I can see that there is a kind of difference between these two semiconductors. for G-manion around gamma we have again positive curvature which means that again I have S-state while for silicon I have two bands essentially touching and one is again G-like and the other one S-like And this explains the kind of plotting that we were using in the past lecture. Exactly this one. So you remember that when I have the overlap of this kind of semiconductor molecule and they have the overlap between S state and P state. Each one of them forms a bonding and anti-bonding state, but then if I try to fit with electrons in bonding state, I have two electrons here and six electrons here, so if you have four electrons, I am available for forming one. And so we say, okay, in those little electrons, the bulk of the band is banded to P-type, And now we can recognize it in the band structure by the sine of the curvature. Because a negative curvature at gamma means I'm coming from the B body. Instead, then we pointed out that you can have, depending on the lattice parameter, two different structures of the conduction band. So in one case, in the case of geranium, the conduction band was also S-like at gamma. This is what we found today. While in the cynical, it was . And this is what we found today. So now we can better understand this plot using a simple 1D atomic chip. Another kind of conclusion we can get from our model is the following.
\fig{6}{TightBinding1DChain.pdf}
So we have seen that around gamma, so when k is equal to zero, our band dispersion is essentially the following. It's gamma k a squared. this is what we derived here k2, a2 so this is essentially a parabolic behavior that we can discuss in the framework of an effective mass approximation So you know that whenever I have a minima or a maxima in the band structure, I can write it down as if this was the energy of the electron, of an electron that doesn't have the electron free mass but the effective mass. okay and so based on this I can essentially write down that gamma if I simplify these two points must be equal to h bar squared divided by 2 m star a a square and so I get the gamma should scale as 1 over a square which is Okay. So choose scale like 1 over a square, which is what we use in this plot. And again, we can try to be quite bold and compare our super simple calculation with some experimental data. And what I did here, I just dropped a band gap, which we associated to the difference the balance of the conduction band over 1 over a squared and you can see that for elements belonging to the same category, so group 4 or group 5, we actually get a very nice agreement between this result and the reality and experiment. So even though our 1D chain was a problem that was extremely simple, we only considered one orbital, we considered one atomic chain, we got immediately, we got some results that can actually help us in reading more complicated calculations. So the first result is essentially that we can actually understand the origin of the state giving rise to a certain band by looking at the curvature around the counterpoint and you will see that this is extremely relevant for example for optical transition you know that in optical transition I need to change the parity between the initial and final state in the dipole approximation so it means that this optical transition between the balance band and the conduction band will be relatively strong because they go from the P to S while this one from this state to the The other one, the other P-state will be instead weak because I have the same parity. Ideally it would be zero, if I should run away from gamma, this approximation is phenomenal. It would be bad mixing, so I would have some optical transition where that would be weak. And the second result we obtain is this scaling factor with the overlap integral with the inverse of the square of the lattice parameter. So the stronger is the overlap parameter. The closer are the atoms, the larger is this value, so the stronger is the overlap parameter. So, everything was, we get this nice agreement between our simple model and some experimental tests but there is this so called elephant in the room, a big problem that we are not looking at and actually some of you spotted this problem and the problem is a problem, for example you can see it from here.
\fig{7}{TightBinding1DChain.pdf}
Our calculation can be easily moved from a one-dimensional chain to a three-dimensional situation. So I can, of course, instead of considering capital R just a number, have it as a vector, and now we get exactly the same kind of solution. So the only point is that even when I go to near-near calculation this R should run on three different vectors and for example I can consider in an FCC lattice which is similar to the model of silicon conductor I can consider these three different linear neighbors and if you work out all the math you get an equation like this one which is plotted here so the result of our s-like in our combination of s orbitals in a lattice which is fcc will be like this and this is pretty much similar to what we found with this lower path So if you compare the point that the critical point indicated there is not exactly the same but if you go from L to gamma you go down like this, then you go to gamma up to X and say this one and so on. But the main problem with the elephant in the room is that in this way we have a path. We do not have a path gap. we have one band and the width of this gap is essentially proportional to 2 times gamma so we give an indication of how strong is this overlap integral but we don't have a band-band So in my previous consideration also on this plot, I was actually a bit fishy with you because sometimes I was... so gamma is essentially this value, but sometimes I was confusing it with this other value which is the difference between the balance and the conduction value because this is a value. So we have something which is not clear and it's actually an important problem. This way we got bands, so a band structure but we don't have a semiconductor, we don't have separation with the balance band which is completely filled and the conduction band which is completely empty. And in the first lecture I was pointing out how relevant is the tetragonal coordination. So the fact that the central atom needs to coordinate with four neighbors to get four extra electrons to fill the atomic shell and this gives a condition where I completely fill my balance band level and I have a completely active conduction band that is to say I have a simple combative and even in the 3D structure that I have here this technique is missing so this is a simple FCC lattice but it's not a diamond lattice which is an FCC with a double base so we need to actually address the problem with the right structure to get the right answer to get the not just the band dispersion but also the band gap and so this is what will be done in the next set of slides.

\section{Tight Binding 3D}
\fig{1}{TightBinding3DChain.pdf}
\fig{2}{TightBinding3DChain.pdf}
So essentially the problem we want to solve is this one. It's a crystal where I have the central atom here I use a zinc blender structure to distinguish better the two atomic sides So for example this can be gallium arsenide, so cation is the one from group 3. This can be gallium, as I said in our example, and the anion will be arsenic. Of course, in the case of silicon or germanium or dynamo itself, this would be in the same atomic species, silicon or germanium or carbon or aluminum. Okay, so now we need to take into account two more things as compared to our previous example. The first thing we want to take into account is we want to really consider the diamond or zinc lambda lattice. which means that we want to properly take into account coordination, the target coordination. And the second point is that we have seen that this of having 4 extra electrons filling the atomic shells requires that we take into account all the orbitals which are playing along in this case. So we need to consider S and P orbitals. While in all the examples before we were considering one orbital at a time, we were considering a chain of s orbitals and we say the curvature is upward if you do the same example with the chain of p orbitals you get downward curvature but we never had them all at the same time now we will need to do this okay so again let's write down which is the Hamiltonian and which is our solution So we have our crystal lattice. and we want to find the solution of psi to this problem now we will actually no longer really distinguish what's inside HC before we consider it as a superposition of the atomic one, of the central atom and the perturbation terms in this case we may not need to go into this detail so this will just be full crystal of the atomic one and we will concentrate on the shape this wave action has to take. So as before, this will need to be the summation of a different atomic site of a wave function. Let's call it Ki and it will indicate A which stands for the anion. So these are the atoms sitting in the anionic position. position. For example, this one. So again, this will be the same wave function translated in the position of the data. So this is a vector, both things are vector. From now on I will not add this extra aside. know we are considering the full problem in a free space and then I will need so this R will run through all the crystal at a I will need to do the same on the atoms you know the orbitals refunction will sit on the cation side and there will be displayed not only by capital R but also by d which is for example this vector here okay so t is a vector which for example can take this value it's It's along the 1, 1, 1 direction, the diagonal of our cube and its dimension is a quarter of the lattice parameter. OK? And so this displacement by a factor t will also need to be added here. Okay, so writing down a function like this, I've taken into account the first requirement, that is to say, we need to consider a double base, so we have now two orbitals, and these two atoms stay in a very specific geometric condition, which is essentially they are displaced by this term D, which is the separation between the two atoms in the unit cell. So the second requirement was that I need to have all the orbitals involved. So essentially each one of these orbital kA will not just be given by, it will not be an s orbital or a p orbital, but it will be the summation over a different orbital. so with a certain component of the S wave function of atom A a certain component of PZ wave function of atom A and the same will be for p and for p and the wave function sitting on the cationic side will be something similar so we'll call this coefficient Cs and so this way I will consider S of eta network and free different p of eta. So this is also the reason why in this model we are using this code SP3 type binding. Okay, because I consider the free field the answer is 1. So this is the full problem. Now we can make the following simplification. Let's use only the near neighbor interaction. This means that in my psi solution, my summation will run only through the different d vectors connecting one central atom to the other. So for example, let's take this as a central atom. Of course, taking one or the other, it doesn't make any difference. So interaction, my summation will run over this value, this value, this value, this value. So we have four different vector, d1, d2, d3, all with the same modules, which is a4, pointing in different directions in real space and the summation will be just over this 4 because these are the near and equal to my system and so my solution will be essentially like this I will like for example the Q A function which is the summation over the different parameters and then I will sum over d taking this value d1 d2 d3 and d4 and I will have something like this in phase term multiplied by the other combination of atomic ornithol or the other element of our central bed structure. Okay? Okay, so this is what is summarized here. I think I used some lightly different, you know, we'll correct it later, but this is essentially what we call KIN. which is the simulation of orbitals and now what I can do is write down again the Hamiltonian of my system in the matrix formulation okay and in this case so I have essentially four orbitals for each atomic side one s and two p I have two atomic sides I will end up with an 8x8 matrix so essentially I can write down something like this that we check to put it in the same in the right order so we have a matrix where one column will be labeled as Phi A s so this is the s-orbital sitting on the atomic side of the anion then we have the one sitting on the C atomic site then I will have the PZ orbital sitting on the atomic site A the PZ orbital sitting on the atomic site C and so on for all the different PY and PZ So 1, 2, 3, 4, these are 8 elements, so we have 8 rows and 8 columns, and I can write down the matrix. Now we will not do it step by step, I will show you how this matrix looks like, and we and we will try to understand the difference between the two.
\fig{3}{TightBinding3DChain.pdf}
\fig{4}{TightBinding3DChain.pdf}
First we point out which are these three different d values So if the epicentrum is in this direction 1, 2, 3, 4 our extension is indicated by this different combination of the previous two so d1 is the diagonal pointing out in this direction d4 is the same just with negative x and y properly you get out these are values for different d vector which are running running on our solution. So the matrix you get looks like this. So let's try to understand which is the meaning of the different elements. So let's start from the one I have highlighted in black. You see this from the diagonal of our matrix. So this means that we are taking the superposition of our crystal Hamiltonian with the same orbital. So, this third term will be something like P s star of crystal Hamiltonian, like this. So this will be the first element. So this is essentially really a value, a single number, putting together what we call the energy of the original atomic site and some kind of whole site integral. So the physical meaning is something like epsilon minus beta. But as you can see, we won't really care about separating these two parts. Because what actually happens is that these different constants that you have in this matrix, they are not obtained from a numerical calculation, but they are fitted to some extent because essentially if you count the number of different overlap integrals that comes out of these matrix there are only nine values so you need nine constant that you can get for example from absorption measurements or protective mass measurement and then and you can fit your calculation to this nine constant and calculate your field band structure so we have nine in the case of the zinc blend structure where we have two different atomic sites if we consider a group four element where the onion and cattle are actually the same element we get only six this number, this parameter that we need. So the physical meaning of this diagonal curve is essentially this. It's what we called in the previous hour epsilon, the value of the amplitude of the atomic orbital plus some on-site energy. Okay, let's now look at the out of diagonal gaps. You see that you have many zeros and we always like to have zero value because you have to use the right terminating basis to solve the problem. And you see that this zero, which I highlighted in yellow here, takes place whenever you have the matrix element between one orbital fitting on atom A, for example, and different orbitals fitting on the same atom. Okay? So every time we have this, we have zero. How does it count? So let's consider for example the first case we have a VAS star times the crystal Hamiltonian times VAPZ for example. And let's imagine how this overlap of different quantities would look. So let's consider our central atom. We know that this crystal Hamiltonian is for sure symmetric because we have a crystal. So in 1D, let's say, it will look like this, the full crystal Hamiltonian. And now we are taking the overlap between S-orbital, which is a function, which is always positive, and the p-orbita which is that is a function which is equal to the s-orbita and the function which is ordered which is the p-orbiter so you can see that now we have something which is either which is a protection which multiplies something which is even, which is odd. So it's clear that the integral must be zero. And so this is the case for all the zero-tel certified in this matrix. So it's essentially different orbitals sitting on the same atomic site. Now let's have a look at all of the remaining elements. I've highlighted a few of them. So this one, this one, this one and this one. One thing about this element is that they are all multiplied by a factor G. Which can be G0, G1, G2, G4. G4, so you can take four different values. So this G times is essentially given by the possible phase combination I can have with the different vector d1 to d4. So G0 corresponds to the situation where I have always plus sign, g1 we have plus plus minus minus, g2 plus minus plus minus, and g3 plus minus minus plus. OK, so why do we need this factor? Let's, for example, take into account the first overlap integral here, which is VSSg0. So you see, this is the overlap between the atomic s-level cyclinome atom A and the atomic s-level cyclinome atom C. So to visualize this overlap, I have drawn here these green boards. These are essentially the s-orbitas, which are even, so they are always positive. And now I need to consider the overlap between this one and this one, this one and this one, this one and this one, this one and this one. You see that we all have the same kind of overlap. It will be four times the same thing. And in fact, this term here is multiplied by Jit zero, where all these different contributions are counted on the positive side. so I have this base term because we have a different location space but they do all overlap in the same way okay this is different from what happens in this overlapping integral so this overlapping integral is the integral between the s orbital on atom a and on p Here I draw Pz for example, three orbital for an atom C. So here the orbital will look like this. I have an even function in the middle and an odd function in the bottom. Now you see that the overlap between this central atom and one is identical to the one I had before. There is no difference. I always have this green positive lobe pointing downwards while atom 2 and 3 They have a red lobe, the negative one pointing upwards So it is clear that the overlap between these two will have opposite sign as compared to the one between these two So if you look now at what G3 is is G3 comes positively out of 1 and 4, these two, and negatively out of 2 and 3. And so we get that essentially we just need this number and we just change in the appropriate way the sign by adding the correct phase factor. so if I then use a px or py orbital, I will have a different value of g because the overlap will be different but I can still use the same value of this overlap integral which is essentially the one between this and this and just change the sign and the phase using the correct value of g. okay let's consider the other two relevant cases one is this vz-z so I'm considering the overlap between vz orbiters of course if I do with x or y it's exactly the same if I rotate the crystal I get the same situation. Here I get at least a positive negative node overlaps with a positive one of item y one and four and then the opposite happens that they have positive node here overlapping with two and three. So this means that when I add up the contribution of this four over an up integral but they are all counter-savvy, all at the same side. So again I have G0, which is the same function, same positive or negative contribution in this case, to all the others. Last example, we have BZy. So I consider the BZ orbital and the y orbital. Here we see that we have G1, which means positive, positive, negative, negative. So 1 and 2 give the same contribution because they have the red lobe overlapping and the green lobe overlapping. And 4 and 3 give the opposite because they have the opposite lobes overlapping. So it's a lot of work to derive the matrix, but you see that eventually it will depends only on a limited number of overlap integrals and also on just this phase factor so once I know the matrix here I can solve this eigenvalue eigenvector problem for every value of k point. So for every value of k, g will take a different value and that will have a different solution and this will be the energy of a different value at this k level. And so this can be done numerically very efficiently.
\fig{5}{TightBinding3DChain.pdf}
So these are the numbers taken from this the paper that proposed this method first, from Charlie Cohen. So you have the six elements that you need to solve the problem for group 4 elements, or dihedral, silicon, and germanium. And actually, since we are not interested in absolute values of energy, here, this value is 0, these numbers become 5. So you have 1, 55, because I'm not interested in absolute energy values, but in relative values. And this is the same for 3, 5, and 2, 6 elements. And in that paper, there are a few lists of elements that you can use. So let's see how this works. And now let me just try this. So this last year I was using the, let's say, to solve this Hamiltonian. And this year I moved to Python, in particular to Python, which is open source. I will give you an indication. I think you don't even need to install it, you can run it from the server. And the thing I like is that you can actually, say, put together elements of code. So we have this block where you can run block by block the code which is written in and also you can have a text with a LaTeX capability which is also nice. So this is essentially the LaTeX that you can derive from this Charlie Cohn's paper. So we have an 8x8 matrix. actually this program was written by Rick Muller I have taken it from the slides to Rick and essentially you have here so you need to run the code essentially one by one So these are the values of the relevant materials. This is one from the Charlie Cohen paper. In this block essentially build up your Hamiltonian. So you see these are the g factor. They are divided in real and imaginary parts for computational reasons. then these are the different elements of your Hamiltonian and eventually you define a function which calculates the uniformity and at the end you can plot your bands. So essentially you need to simply use this band operation. n is the number of k points and material is the material among this list so for example this is gemenium if i put five points So if I put five points I get a rather even bed structure with only a few key points. So if I put more, I get a more smooth bed structure. Okay? And so here you can see that if we have an 8x8 matrix, we should have eight beds. So we have 1, 2, 3, 4, here it means that essentially the bands are degenerate. So 1, 2, 3, 4, 5, 6, 7, and then there is 8 bands which is degenerate. So we have 8 by 8. And of course you can do the same for different material. So let's compare our calculation with some refined experimental data. So we go back to the presentation. And this is our Chanco and Lott compared to a more complicated calculation which is of self-motivation. And we can see that the balance band is actually corresponding to what we get with more complicated methods. even with a simple computer, let's say, open software, you can get the balance band to happen. Why? If we look at the conduction band, this is already completely different from this one. And in fact, the original paper was with the type-binding calculation of the balance band of diamond and zinc-blende crystals. because essentially what happens is that using only s and p orbital is not enough to get a sufficient description of the conduction band so it means that you need to put more orbitals so professional theorists they add d orbitals So in this case, we have five orbitals with a kind of more complicated symmetry. And so it went down with an 8 plus 5 matrix system.
\fig{6}{TightBinding3DChain.pdf}
The simpler approach is this one, what is called S between S star. Well essentially you see that we have the same element as before, as A, as C, Px, Py, Pz, A, Px, Py, Pz, C. And we have two, they fake S orbital, which are added just with the purpose of simulating the effect of the d orbital. So these are not real atomic orbital, because the next one will be the d. but it's more a calculation tool. So I have these two, S star A and S star C, of course, on the row and on the column. And in this case, I only get 10 by 10 maths instead of 13 by 14 that I would get by adding the t. And also, the shape of this overlapped integral is much simpler because we have a spherical distribution of charge or not. the weird angular distribution that we have in the other test. And so again you can use this S-star type binding method and do your calculation. And here, so these are the different values that you now need. so you need a bit more numbers for the different values. These are corrections from the authors, so they re-edited the paper a few weeks later, just by drawing a line, adding the new values for the right overlap integral. And this is what you get in the US. Maybe next time we will play a bit more with our code, the second application how to imagine a story.
\fig{7}{TightBinding3DChain.pdf}
\fig{8}{TightBinding3DChain.pdf}
And now you can see that the conduction band is described properly so the bottom of the conduction band is this light here and now we get exactly the same result as more complicated material and this is true for silicon of course for germanium and gallium arsenide. Okay? Let me continue next time because I want you to wrap up and give you a little more insight into this. What I wanted to point out is essentially that, what I want to point out is that this that this type finding approach is an approximation, is an approximated method. It's not the so-called ab initio method, where I only put in the orbitals of the original atoms and get out the lattice parameter, and then get it so. But the power of this technique is that essentially, with a few parameters, so basically you can count them on hand, you get the full band structure. So you can calculate the band structure any point of the river and so on, and so then with a bit of coding you can get effective math by fitting the curvature of the bend and so on. And these parameters are typically not calculated from the original approach of the lab. So these d are fitted with for example more experimental data or nowadays with more complex calculations. For example you can use that's the functional theory which works efficiently if I calculate the energy band at some symmetry point of the Brimelian zone but it will be very complicated if I want to calculate the full path structure in every point of the Brimelian zone so you can run DFT calculation to get the value of the band of the energy band different gamma, x, this relevant point you see here, so L, gamma, x and so on and you use this value to fit your experiment with your formula integral and your matrix and then of course you can assume the effective masses and so on. But once you have done this, you have access to the full equivalence. As a matter of fact, this kind of approach is used, for example, in modeling tools used by electronic engineers. The thing that you have a material that is changing, you know, you have in your device, you have a different alloy and you want to calculate for every point, which is the effective master of the curvature of the bandwidth so we take into account for example the mobile behavior you can run this as you see in my computer in a few seconds you can imagine in a proper local station and it's very easy to do it.\\
Okay, so let's try to recap what we did in these first few hours of lectures. So as I told you, the idea was going step by step. So we started with a very naive idea of why a material is a semiconductor or not, and we counted the electron in the valence band and find out that if I consider the hybridization essentially the overlap between, not the hybridization, the overlap between S and P states, I will have just the right number of field valence band state and empty conduction band state. So we associated this overlap of orbitas to the formation of a band gap essentially. we did this in a more qualitative way with just two atoms. We moved to a chain of atoms, so we put in the idea of a lattice of a periodic repetition of atomic size, and we already got some insights in the structure of real semiconductors, so So for example we associated the curvature of the valence band and conduction band state to, say, the overlap of S or P states. So we recommended this figure here. essentially to really get not just a band flow, distribution in energy and k-space of electronic states, we actually need to put in the real structure of a semiconductor and bring back this idea of four-fold coordination, so use a diamond or a zinc blender structure. And so this brought us to two different level of approximation, let's say. In one case we use this sp3 model which essentially uses s and p orbitals on the different atomic side, on the two atomic side of forming the base of the diamond or zinc-blende structure and we found that this is actually good enough to describe properly the dispersion of of the valence band but it actually gave a very bad result when I look at the conduction band. So we added two more orbitals, so this means also more say overlap integral, and we introduced this sp3 s star Hamiltonian which is essentially the one taken here from the paper by by the book. OK? And just one of the message was also to check that mean why at the time when this theory was developed, it was a job of theorists, essentially. Now, actually, with a common personal computer, we can easily solve the eigenvector and eigenvalue of this matrix and so find the bind structure. So we like to use the Python code that we introduced last week to do some example and compare the structure difference in conductors. So let me now share. Okay. Okay? So, as I said, this is the first example which was done in the order code that is essentially visible in this different cells, and we basically, so we have all the matrix, the different elements, eigenvalue and then I think here as I was saying the eight states of respect to each band and this is actually some of them are at least partially degenerate so you can really mount uh for sure but all the eight so today we will use the other approach of this paper by Vogue. Essentially we are businessmen. And here to make clear everything coding is added in this .tp file. So, stab and you open it. It's very open here. you will find a different function and the addition of a different material. So it's handy to deal with. For example if you tb.txt, tb means the critical function file, we can put for example silicon, so we have a different overlapping needed for the calculation. And then, the first number, the total number of n, so number one is n, and the other number is a point. So if I plot this with only a point, You will see that I have very dense structures. It really doesn't matter to me. It's okay. So we say that we have maximum answer. And so this will be the full result of our SP3 calculation. In here some generate 10 bands. However, what happens is that initial S star has been to get the description bottom of the conduction band. You cannot really rely on the upper bands. But with these devices, what is relevant is to adjust the top of the balance band to the conduction band. So it is actually more meaningful to adjust the first five bands. actually this band here, this one which is actually 2, you can see the degeneracy here, we are at 4 and with 5 we get the... with 5 I get the show of C. So you see for example, the unicorn has a minimal energy gap state, so the top part and this in the delta direction. So you can do the same with the E-Dialect to find that the E-Dialect gaps with the minimum is 0, or with the Luminomars E-Dialect, which is also E-Dialect, so the minimum of the energy is in this point. interesting thing it's also compared in the same band for a semiconductor. So here actually sub routine band point the number of bands is down to different it doesn't start from one band, but it goes to nine. So now the correction band is actually down. So for example here I calculate the band structure of two bands of silicon, which can be K. So diam, then we can do it for silicon, for germanium and for carbon. down as we did in the photo, you see, but if the behavior is physical by our tranquility. And then I just plot this different value. Run Excel, you will get a plot, see that diem, and one is all the valence bands are colloids, So the top is right to this energy. So this means the energy difference between the conduction pad and this line, row line. So this gives you an idea of the value of the energy gap. It's the minimum. It means that it's . So, Carburet, it's an indirect gas, so the energy gap is 5.7. Then we have Silicon, an indirect gap. We have Germenium, which is an indirect gap, but in a zone, so the minimum of Silicon is along this delta direction, which is in the L direction. we have thin, or do it's actually semi-minimum, negative band, so it's minimum. Minimum, where we have the minimum, and again, actually, okay. And another thing we can do, we can do the same. Now, do the other, well, use the other in the first. Kind of moving along. one column, we move along. So we take gemenium, galactomine, then we go and gallium arsenide, we bring one step to the left, zinc selenide, two steps to the left. So and we recommend, essentially you have this parameter, but you have and so that is actually easy because we have an ionic bond. So if you do this you recover the band gap of one of the tate, one of gallium arsenide, 2.5, one of zinc ID is 0.5 or so. We recover. And also like this general feature that we will further probably You see, as the band reduces, also in the curvature, this band is actually effective mass is lower. The lower is the band, the lower is the action band. But we will see actually the case experimentally. You have other regularities in the conduction. As a final thing you can use a very simple script to calculate the band depth. I made the case of the type. It's a nice to go from an insulator, a band-gap, and a material with a band-gap. Run this code, number, extract. Okay. Okay. So let's go back to our slides. So beside this, the analytical numerical approach that we did with this code, we can also understand something, say, analytically. So here, we also already commented it in the last lecture, this is a comparison between our calculations and one done with this code, and in more complex approaches for calculation, this is essentially our tight binding. sp3 star and this is a pseudo potential method. Okay, and we see that now we have a nice agreement between the behavior of both in the balance band and the bottom of the conduction band. And this is the same thing for germanium, and we have the same thing for gallium arsenide. We can also understand something by looking analytically at our solution. So now I go back to the sp3 approach, so there is no s star. You see we only have eight states, eight columns and eight rows in our Hamiltonian. And we can, I mean, understand something also on the eigenvector. So far we discussed just about the eigenvalues of the bank structure, but also we can grasp some physical intuition of what's happening to the eigenvector. So which means is given a certain state, which is a combination of the different orbitals forming this state. So the value of the different a and c coefficient that we use to derive, to solve our problem. So this is what happens to our sp3 matrix if I go, if I set k to zero. So you remember in the original matrix, we had a lot of this g factor here. This g1, g0 and so on, which are actually dependent on the vector of reciprocal space. So this is actually how and where the k vector, the reciprocal space vector, enters in our calculation, because eventually we get energy for every k point, and k enters only through this g phase factor in our matrix. So if I pick up a specific point in the Brillouin zone, I can write down this matrix for this specific point. So if you're doing for k equals zero, you essentially get that a lot of GAMs goes to zero because essentially only zero is different from zero in this case. And we can see now that our six by six, sorry, eight by eight matrix decompose in four matrices, each one is only two by two. So what does it mean that I have isolated this? for example, submatrix here, it means that the eigenvalue, sorry, the eigenvector associated to this matrix will contain only these states. So they will be made up only of A, of the s-orbital on the anion-cricutian side. So the p states will not enter in these energy values, okay? So this means that here I have a two by two matrix, which is similar to the one we got for the linear combination of atomic orbital. And so this means that some of the solution of the eight energy levels that I calculated solving the full matrix, two of them will be made up only of S state. state. So we'll have, say, bonding and antibonding band position at the gamma point, which is made only of s orbitals. So it will maintain, let's say, the symmetry and properties of this s orbital. And in the same way, the fact that I decompose here three different sub-matrixes, each one containing only px, pi, or pz, means that I will have three other couples of bonding an antibonding state which are made only of Px or Py or Pz orbital and the other states we know is entering the general solution. And so this is what happens again at the gamma point. So this is one of the S bonding state, this will be the sum of the P bonding state and this would be an S-antibonding state in the case of Dijon-Marcenai.
\fig{9}{TightBinding3DChain.pdf}
Now we can take another, make a thing a bit more complicated, and we decide that we do not really take one point in the reciprocal space, but we take one direction, which is this delta direction. So delta is essentially the direction along the It's a 0, 0, 1, whatever axis you can choose, 1, 0, 0, or 0, 1, 0. In the case of 0, 0, 1, so if you replace this value in your g function, you see that g1 and g2 become 0, and you only have g0 and g2. So now the H solution decomposed into two sub-matrices, so each one will give four energies, and the corresponding eigenvector. So this means that this first matrix you see will contain only S and Pz orbitals. So if I move along the delta direction, I will have a bonding and anti-bonding state, which is made up only of s and pz orbitals. And z is the direction that we choose to work with. So if you do the same for the 010 direction, you would have a combination of s and py orbitals. and if you take the 1, 0, 0, you will get the combination of S and Px orbital. Now you cannot say get this information from here, but actually what happens is, for example, that one of these bands is actually the bottom of the conduction band. So in the case of silicon, this band here is essentially the combination of s and p orbital. And which orbital, z, x, or y, depends on the direction I take. So if I move in the z direction, this would be pz. if I move in the y direction, it will be py and so on. Okay? While the other state, the other four states will be formed by bonding and anti-bonding states, which means only p-orbital, px and py, where we need two orbitals. Okay? And this will be essentially, mainly, one of the state is the balance band of our semiconductor. Okay. So now what I would like to do, so we use this S star approach to better describe the conduction band, I would like to focus a bit more on the structure of a valence band. And for doing this, let's consider in more detail the overlap integrals of a valence band. So one of these overlap integrals, what we call the ZZ here is essentially given by the superposition of two PZ orbitals which are displaced in this way. So this direction is a 1, 1, 1, or minus 1, 1, or whatever, in one of the four directions of the coordination. And actually, I can actually decompose this z orbital, this overlap integral, into different overlap integrals, which are similar to the ones that we obtain in chemistry. So in chemistry, when we have the superposition of two p orbitals this way, we call it pi overlap, and we have the formation to overlap in this way. This is for example the case of the state of in graphene, so where we have a non-saturated bond sticking out of the graphene plane and they form these pi bonds. And we can see that we can also have a contribution from what we call a sigma bond, which is actually where the two atom forms overlap in this way. And what we can see is that the discussion we did of the sign of gamma, so gamma is now essentially our BZZ, the discussion we did in the case of a single atomic chain, chain it's actually a bit more complicated because we see that for the Sigma bond we actually have that gamma is a negative value okay because we have that if I add here the perturbation potential the same I will have an even function, the perturbation was more like this, this is even, but now I have a superposition with two functions which gives a negative overlap and so I get the gamma becomes negative. But the pi component is actually, so let's call this gamma Sigma the pi component is actually positive because I have overlap between those of the p orbital of the same sign in so our nomenclature of time binding this gamma is actually called VPP sigma or VPP phi. Okay? And so our general VZ structure is actually a combination of these two. So why we have this trigonometric dependence of the square of cosine? Why they are overlapping this way? Well actually the answer is very simple, because essentially I'm doing the following. I'm taking this orbital, and I'm projecting it along this direction, so I need to multiply by cos theta. And then I am also projecting this in this direction, so I'm also multiplying it by cos theta. make the overlap integral I will have let's call this preset that the beat in P Sigma will be something like this is that costita times our say perturbation potential times again these z cos theta so this is how I have this square of cos theta in this overlap and to get actually the other component the pi component I need to project not in this direction but essentially in this other direction so instead of cosine I will have a sign. So it's a straightforward derivation. So this is just to tell you that in the overlap of the p states is actually a combination of these two different terms with different signs and different let's say physical meaning. We will see that this is extremely relevant when we will deal about strain in semiconductor, but this will happen with one month or so. But for the moment.
\fig{10}{TightBinding3DChain.pdf}
I would like to use this proposition to introduce a very simple model which describes the dispersion of the balanced band, which is fully qualitative, but is actually very helpful in understanding the band dispersion in the balanced band. So we will use it several times. So we will describe some of the properties of the balanced band, which is essentially the band dispersion and as a consequence, the effective mass, as if our crystal was just a cubic lattice of Px, Py, and Pz on them. So we have this simplified configuration. The main ingredient here is that, as in the more complicated case of the diamond lattice, Here we will have a combination of sigma overlaps and pi overlaps. So the main physical ingredients are both in this model. So we can consider some pi overlap, sorry, some sigma overlap. And then also some pi overlap. Now, so before, in the previous picture, I was actually using the red and green to indicate positive and negative quantities. This is no longer the case. So here I'm just, I don't really care about the sign of each node, but I'm only distinguishing orbiters in different directions. So let's try to understand why, if I move away from gamma, I have a certain dispersion of bands. so let's consider this frame of reference and let's say I want to move an electron in the z direction ok So this electron will move very easily in the z direction if it goes, if it follows this sigma bond, because actually the overlap between these different integrals is very good. And so easy motion means essentially a small effective mass. So the contribution of this overlap will give me a light band here in the z direction. Okay? But at the same time, if the same electron, which is in this orbital, wants to move in this direction, we'll have to jump from one orbital to the other using actually a tie bond. So this means that in the x or y direction, I will have actually a heavier mass. And so this orbital, Z, will contribute partly to one band giving a light mass and another band giving heavier mass. If you repeat the same thing with the red orbital, the one in X or Y, you will get an opposite behavior. So for the red one, we will get that in the z direction I have a pi overlap, so we have this, while in the x and y direction I will have a sigma overlap, so in this direction I have sigma, sorry, and this is more like this. And that's what we have, this thing. So in this description, my energy, my balance band is made of, I need to have three different bands. So one is the one along Z, and then we have two along X and Y. they will be fully degenerate in the center of the brain-wound zone. So here I will have three degenerate bands, and then the, say, this degeneracy is partially removed when I move away from gamma in these two branches. Okay? So this is what we get from our, say, various-level picture where we decompose the bonding and anti-bonding state, sorry, the overlap integral in the sigma and pi contribution, and we build a toy model, let's say, where we only have this effect in a cubic lattice. But still you see that this band structure closely resembles what we get from actual calculation where we have here full degeneracy and then this degeneracy is partially removed when I move away. However, these bands are not made only by one orbital, let's say, in one direction they are mainly the contribution of the pz for example and the other the contribution of the other two. So let's see if this is actually consistent with more detailed calculation. So let's focus more. on the conduction band. So what we discussed before was that essentially I have two degenerate states forming one band and one not degenerate state forming the other. But if I look at if I make a deeper, let's say, comparison between our calculation, the one we did with our models, and more refined models, we notice a relevant difference here. So let's focus on the balance band. You see actually the gamma point, the states are not really degenerate, so we have a separation here. So we have two degenerate states at the top of the valent band, and then we have here one state a bit below. So this is missing in our plot. And the second point is that these two states here remain completely degenerate in this region of the brillou and so on, while here we see that we have already partial removal of this degeneracy. So even the top two states are degenerate only at the gamma point, but they are not degenerate in the other direction. So this is the case also, for example, we look at the band structure of germanium and focus on the valence band. We have the same effect. So there is something missing in our picture, and so far actually we never discussed about spin, the role of spin in the band structure, and the role of spin-orbit interaction.