\chapter{Lecture 4}
\fig{1}{sell.pdf}
\fig{2}{sell.pdf}
Let's briefly. Recap what we did in the last lecture before we actually. Enter into seeing how actually. Um, speed, not with interaction in the balance band allows. The generation of spin population in the conduction band of a semiconductor. So, what we discussed in the past lecture was essentially the following since we now know from type binding and the violence band, it's coming from. P state, which means that the. Quantum number for the angular momentum is equal to 1. We need to consider also an additional terminal, which is a. spin orbit interaction which is proportional to some positive constant times the scalar product between the orbital angular momentum and the spin state and we have seen how we can actually write down this Hamiltonian in terms of essentially the total angular momentum the orbital angular momentum and the spin angular momentum. And we have seen that since J can take two values we will have a splitting of valence band state. And this, we calculated this energy, which is a split of energy or also spin orbit energy. So, this Hamiltonian will, so to calculate it, we need to define the total angular momentum of our state. Of course, the total orbital angular momentum and the total spin angular momentum. And we know that this is one half. This is one in our case, and this is essentially the value obtained by replacing 1 and 1-half and 3 and 1-half. There is another operator which commutes with this Hamiltonian, which is JZ. While, for example, L with the M, the projection of the orbital angular momentum, and let's call it SZ, the projection of the spin angular momentum, they are no longer good. quantum numbers okay so i cannot use them to to level the state this can be easily seen for example if we write down again our l dot s in this way And so we see that obviously SZ does not commute with this operator because there is SX and XY, and in the same way LX, LZ, sorry, will not commute with this operator because we have LX and LY which do not commute with LZ. So eventually our state will be labeled by the value they have of the value of J, three and a half, and the possible value that JZ can take. so I can have states with plus minus three and a half and state with plus minus one half in terms of the projection of a total angular momentum. And then the other state, the one, the split of band, will have only two possible projection. Okay? So our final picture we got was that essentially we had This kind of dispersion in the balance band. Well, we associate to this band. Let's maybe do it again here. so let's say this band we call it the heavy hole and it's made up of state with a j quantum number three and a half and j z plus minus three and a half then we have another band which we indicate as a light or band where j is still three and a half but now the projection is plus minus one half and eventually we have this split of bend where the quantum number for the square of the total angular momentum is one half and here we have plus minus one half so and i essentially just gave you the result of this separation and I mean how we can write the eigenvector of this state and these are given by this coefficient that this combination of coefficient which are the clebsch-gordan coefficient and we we pointed out that the aviol have this kind of behavior but they do not have any orbital in the axis I've chosen for quantization while the light hole Feature, I mean, a mixing of a different states and, for example, from the light tool, you can immediately see that these states, they do not have a well defined projection of the spin. Yes, spin up spin down as we mentioned before the projection of the spin. As well as the projection of the orbital angular momentum, they are not a good quantum number, but these states, they had a well defined value. Of J, square and also the projection of J. Then we moved into looking at what happens when I consider this state as initial state for an optical transition. And I take as a final state the conduction band. So, our initial state are essentially is aviol. Or light or that we defined and our final state will be. Uh, the s state, and here I can indicate. Call this up, for example, or I told up where in the case of the violence band. This up or down refer to the value of J, not to the value of the spin. So we've seen that in the case of the light or the projection of a spin is not defined. And here I can have, for example, spin out and spin down because in the conduction, but I don't have any, um, speed on with interaction. So, when we go to to derive the, uh. Selection rule, we see that essentially, uh, we only have 1 term remaining. Which is the 1 containing essentially. The product between the block function of initial state and the block function on the final state. And we have seen essentially that the only relevant matrix element. Are those involving, let's say. one orbital and the derivation of in the same direction of the orbital so the only terms actually which are relevant are this pcv factor and the only derivative is different from zero or the one where i derive at px along x py along y and pz along z so relying on this we have calculated the probability or say the matrix element for X polarized light.
\fig{3}{sell.pdf}
So for example in the case of the aviol to conduction band we obtain this value while in the case of aviol to light or to the valence band we got this value. We did the same with the Y polarization and with the Z polarization. And I was mentioning in the last part of my lecture, I think it was a bit I mean, not so clear. What happens here is that essentially we cannot distinguish one direction from the other from in a cubic crystal. So whichever direction I rotate my crystal, I should get the same absorption. And so, for example, if you consider X polarized light and you add up 1 half plus 1 sixth, you get two-thirds the same happens here and the same happens in this direction so even though i have different absorption from the avion and light all i still preserve the symmetry of my system so whichever polarization i use i get the same strength of this transition coming from different states so and this for example i mean we we can see the different absorption strength of the heavy hole and light hole in this plot so what is this plot showing so it's essentially showing the matrix element that we calculated and we calculated essentially just in some principal direction so for example we can take this direction here as the x direction and this one say as the z direction and here we have it in a polar shape where we can calculate it with any orientation of the electric field and k is essentially our quantization axis and so we see as a as a consequence of the fact that the light hole the heavy hole sorry do not contain any components from the p z orbital that this transition can be excited only by some, say, in-plane radiation. Okay? On the other hand, the light hole, they do contain components from the different, all three different p-orbitals, but, I mean, the strongest absorption is obtained when light is polarized in the z direction. So we see that this is much stronger than what you get, for example, for this polarization. So this is not exactly zero, but it's much smaller than what you would get for Z-polarized light. While in the case of a heavy hole, transition with electric field along the zero direction are completely forbidden. So one thing we were also looking at while deriving this selection rule is what happens to spin.
\fig{4}{sell.pdf}
So, what happens to the spin angular momentum in this transition? And what we have seen is essentially that now I can consider transition between – let's limit ourselves to aviol and lytol for the moment. Okay. If I use a linearly polarized light, I would get the same probability of having, let's say, an electron here. We spin up and an electron here will spin down. So if you look at the selection rules, the transition between aviol up, S up, and aviol down, S down, they are identical. We always get the same matrix element. Okay? And the same is true in the case of a light hole. So if I look at the transition between Lytol up and S up or Lytol down and S down, and even if I look at, I can also look at the mixed case in the case of Lytol because they contain both spin orientation. I don't understand why. Okay. So you can look at this mixed orientation. Sorry here. So, you can even look at this. Mix orientation, but still you would get. That within your polarized light, you get from this transition. The same amount to spin up and of spin down. So basically you are not able to change. the spin population in the conduction band what i would like to show you is that this is not the case when we use a circularly paralyzed light Okay, so if I use circularly polarized light, what I need to do is essentially changing the unit vector describing the polarization of light in this way. So we'll have an . . Ah, scusate. . This will . So what I was saying, what I wanted to show you is that when I consider circularly polarized light, the vector describing the light polarization will take this kind of shape, let's say. So, I will have the X and Y component with a 90-degree dephasing. So, the plus sign, we take it as the polarization for a sigma plus electric field. So, let's say a clockwise polarization. And the minus sign will correspond to signal minus, which is an anti-clock-wise polarization. Okay? So, let's see what happens when I excite, for example, a transition like the one between the aviol up conduction band up with a sigma plus transition with sigma plus polarization so in this case we if we look up how this ideal upper state are obtained from the combination of the different P-axis and PY orbital. Our matrix element in the Fermi golden rules will be written like this. then i will have my dipole operator and i need to take only the derivative with respect to x because sigma plus correspond to ex so to the polarization vector ex plus i ey and here i will have this kind of derivation okay so this imaginary unit will appear in front of the derivation along y because coming from essentially the multiplication with a light polarization unit vector and then here I will have my s state up okay so if I do this calculation I will get from this term I will get minus 1 over root square of 2 PCV. Okay. So here I need essentially my pre-factor. But I will leave it, I will just in secluding this PCV. And the second term will be minus 1 over root square of 2, which multiplies E times E, which is another minus 1, let's say. And then again, I will have a PCV factor coming from the derivative of the Y orbital with respect to the Y direction. And so you see that essentially we have zero. So, this transition between this EVO lap and S-up state with clockwise polarization is forbidden. And actually, we could have got to this conclusion, I mean, before without using this selection rule. Because essentially in this transition here, we are violating the conservation of the total angular momentum. Okay? Because essentially we have an initial state where the projection the angular momentum is plus three and a half h bar and then we have a photon which is clockwise polarized and so this photon will carry another h bar of the angular momentum of the projection of the angular momentum and in the final state i only have one half h bar because the photon has been absorbed and i have one electron in the s state and so you see that these two quantities are indeed different so i'm not conserving angular momentum so this transition between this initial state and this final state will instead be possible with the sigma minus polarization.
\fig{5}{sell.pdf}
So if i use the anti-clockwise polarization my balance will become this one and in this case i'm actually conserving the projection of the angular momentum so So this transition is possible. So this is typically plotted in books and papers using this kind of energy diagram, energy and angular momentum diagram. So here I put my two states of the conduction band. So one will have a projection of the spin plus 1 half, and the other one minus one half. So this will be the S up state and this will be the S down state. And here I will have my state in the valence band. So this energy difference is essentially direct gap energy gap and this will be the state with the projection of the angular momentum equal to plus three and a half so what we call the heavy all up this will be the light hole up this will be let's use it lighter down and this will be the ideal down so here the angular momentum will be minus one half and minus three and a half okay so what we are we were discussing before is that if I use for example sigma minus polarization I can do this transition will be possible and another transition that will be possible is essentially this one because also in this other case if I make the balance of the angular momentum, so this is for the heavy-hole transition and this is for the light-hole transition, we would get 1 half h-bar minus h-bar, the angular momentum carried by the photon, which is equal to minus 1 half h-bar. So these two transitions will be possible and all other transitions will not be possible. Of course, if I switch the polarization, with sigma plus I will have these two other transition which are allowed so now the point is that you see that with this this way we saw let's say let's stick to the Sigma minus case I'm injecting electron with spinning in both Uh, check both, uh, state of the conduction band, spin up and spin down. But let's check about the relative strength. Of these 2 transition, because the initial state of different, so it might be that I have a relative. Intensity in the transition from and the 1 started from the light. So, let's again use our. Decomposition of the balance band state. In the px orbital and calculate the transition between. Up. And spin up, but this time using sigma minus polarization. So, essentially, let's say if I'm smart enough to. Copy this. I'm not. Okay, let's write down the equation. Okay, so the only thing we change now is the sign of a. Of the, due to the polarization of the vector now we have a minus. So now we will have. If you make the calculation minus 1 root square of 2. And minus 1, root square of 2. So now the transition is allowed. And this matrix element as a value of minus 2. By the by the root square to, so the intensity. Of this transition comes from the square of the. Matrix element between the initial and final state. So, if I take the square. Of this value, I get 4 divided by 2. I get to square. Okay now. Let's do the same for the. Transition started from the light hole. So, we say that the conservation of angular momentum allows a transition between light hole up. And the S down, so here we will have a state with a. 3 and a half and 1 half 1, 2 number. And now this matrix element. will be essentially this one so this will be minus one over the root square of six Px plus I Pi spin down minus 2 Pz spin up then I have my combination of momentum operator and direction of electric field and this is now spin down okay so if you do this you would first get minus one over root square of six pcd coming from this times this you see that the spin let's say it's in the right orientation to allow this transition and then the second one will be when when I multiply this by this. And again, because of this minus sign here and of the imaginary constant, I get minus one over the root square root six BCD. Of course, we don't have any transition involving Z state, essentially for two reasons. First, we don't have any light in this direction and also mean the spin is opposite. So this matrix element would be zero. So, now I get minus 2. Good square six six again if I take, let's say the square. Of this transition, I will get essentially. 2, 3rd square. So, now you see that the 2 things are different. And if I make the ratio between the. s up transition and light all up s down transition how we get two pcb square divided by two third pcb square and so the ratio is essentially free so every old transition is three times stronger than the lighter one so what you typically find in in papers is that here you put a number free indicating the intensity of this transition and here you put the number one indicating the intensity of this our transition and of course if you go from left polarization to right polarization you will have identical phenomena phenomenon but with a reverse speed so essentially what happens here is that if i go with a photon which is resting at the energy gap of my semiconductor and the circular polarized let's say it's left circularly polarized, I would get for that I can define an electron spin polarization in the conduction band, which is essentially, let's say that I will send four photons. I will get that three of them generate spin up in the conduction band, and one of them will generate spin down. and the total number of photon will be, of course, 3 plus 1 equal to 4. So if you make this calculation, you get simply 1 half, which means that you will have a 50\% spin polarization in the conduction band of semiconductor at the direct gap of a semiconductor. So a lot of these experiments are done with gallium arsenide and free 5s, but actually it works only also in germanium because the symmetry if i look at the direct gap transition because the symmetry of the states will be exactly the same.
\fig{6}{sell.pdf}
So this is a summary of what we derived so far so essentially you see the selection rule that we have derived in addition what is we have in this plot that we didn't consider is what happens when the photon energy reaches the also the split of band so in this vertical direction we had an energy scale so this is the energy gap and this is essentially what we call Delta zero the split off or spin orbit energy And these are, of course, the speed of band. So we see now that if the photon energy is, let's say, large enough to include also the speed of energy, I will have more transition, which are allowed. And this transition will essentially kill the polarization. because if you have now two electrons, say let's still consider the blue transition, you will have two more electrons populating the spin-down state, and so eventually the total spin polarization will be zero if you excite also the split of band. So let's look at this in a bit more, let's say, complicated and refined way.
\fig{7}{sell.pdf}
\fig{8}{sell.pdf}
So, this is, for example, what you would get. So, what we started so far is this single point here. Okay, so what happens at gamma. If the photon energy is essentially. This 1, and you see that we got, we get this 50\% polarization, which is. This value here if I increase the photon energy. I will excite, I mean, state from, let's say, other regions. And this will bring essentially a smooth decrease of the polarization like this one. And as soon as I – you see here in this electron polarization plot as a function of photon energy where is a kink this kink corresponds essentially to this split of energy so at some point i will get also transition from this region but here i will have a limited region so a limited number of states so let's say the joint density of state for this transition will be smaller than the one that i would get from this one so i don't take actually immediately zero i have a decrease in the spin polarization that goes to zero only when i'm well above the the band gap okay so one thing we can do to actually obtain a larger spin polarization is the following so let's assume that I can remove the degeneracy between heavy oil and light oil. So in the semiconductor, we have seen so far that the gamma point, heavy oil and light oil, they have the same energy, but actually by using a strain, in particular, compressive strain, as we will see later, I can remove this degeneracy. So essentially I will have an energy at which I excite only the heavy hole. And then I will need a larger photon energy. To excite also the light tool. So this means that in this energy window here. My polarization would be 100\% because it will have. Only this transition active and so this is what you see. In this calculation for in the case of a strained layer. So you see that the beginning. You have the 100\% polarization then you have some. Smooth decrease when you start to excite also the light tool and you go back to this. 50\%, as in the case of the constraint case and then. Again, when this energy is a split of energy, you have an another contribution. Coming from the split off and going to this channel. And, uh, giving you, uh, let's say, uh. Reducing this pin polarization. So, let's see, uh, so we, we. How we can experimentally measure. This kind of a phenomenon. You say in the old days in the 80s, it was possible essentially. To do photo mission, and it's still possible, but I mean, this is the 1st way this thing has been observed. Let's say, uh, doing a photo mission. But also resolving not just the energy, but also the, um. Spin orientation of my exam on my electron. So essentially the same in this experience, this is an example of a setup that was in user in our department until a few years ago. And here you have the sample. You have the lighter extracting. Your your electron, so you need a photon energy. Taking an electron from the conduction band with the balance, but sorry to my balance band. The conduction band, and then this and electron can escape. From the surface and to do this with. It is very common to deposit on top of your sample. Which reduce the work function and so allows the. Uh, the electron to escape more easily. From solid so once they are, uh. Generated, they are essentially generated with an with an, uh. The spin orientation is the 1 perpendicular. Through the direction of a beam. And so these electron are extracted. And they are rotated in a condenser essentially. And they are sent to this detector, which is able to, uh. Detect the spin polarization and this detector. We also works essentially with spin orbit interaction. So, it essentially consists in a piece of a cold. Which has a high value. Of atomic number, so it has a strong spin orbit interaction. And so what happens is that when an electron comes in this direction. With a spin in this direction, it will have. More chance to be scattered on 1 side. So more luckily more easily. Go in this direction then on the other, because here, essentially, we have again a spin orbit interaction between the spin of the electron and the orbit in this case is the direction. It's a motion. It's not the orbital motion around the nucleus, but it's a motion given by the acceleration here, which is several tens of kilobytes. So you will count here with your detect with your electron counter more electron in this side. That on this side in the same way, if you reverse the spin. The electron will go more. Can you still see the screen? I think yes. So the electron will go more in this direction, so you will count an asymmetry in your system. And what you typically do to make these measurements more sensitive, you continuously change light polarization. So you go from positive to negative in such a way that you have some point more electrons here in the other case more electron on the other side so with this measurement for example you can compare different samples so this is the case of two germanium layer two germanium samples this one is unstrained let's say and this one has a 96 percent compressive strain. So you see that as we showed before, in the unstrained case, you see a lower polarization as compared to the strained case. You might also notice that this value is not actually 50 percent, it's only 40, and this is not 100 percent, it's only 60. And the reason is essentially that we cannot detect the electron injected right the bottom of the conduction band because we need to give them extra energy to escape the crystal so we only basically can see let's say this part of the electron spin polarization in the case of this unstrained sample and let's say this part of the electron spin polarization in the case of the So, this measurement was actually, I mean, we use it as a physics department a few years ago, but then it was very, I mean, it is extremely complicated because you need to prepare your sample in vacuum and measurements are extremely long.
\fig{9}{sell.pdf}
And so, we moved to a different approach where there is no vacuum involved, but only relatively simple measurements. Sample preparation, so let's say we, if you take a semiconductor. So, this is silicon, but we can use your medium or guy, you must unite and you. Then you place on top of it by means of a. Evaporation a very thin layer of platinum. So platinum is also a material with the highest. So, it has a stronger speed orbit interaction. Then you go with circularly polarized light, you excite your spin. And together with the electron. This being diffuse inside platinum and here, I mean, without any. Real really high voltage this electron they do experience the same. Spin orbit, uh, asymmetric scattering. So let's say. Electrons with the spin in this direction will go more likely. This way, then in the other way, so you will have an accumulation of electrons. And this accumulation of electron will result in a. Voltage this data V that you measure. So if you reverse the polarization. You will measure the opposite thing. This phenomenon is called. Inverses spin all effect. Okay, and this is essentially the. Conversion of a. Spin current into a charge current current, or if you want to have a spin signal into a. Current or voltage signal. Okay. And so here you have a, I mean. Essentially, the detector has been replaced by this thin platinum layer. And we can understand what's going on inside the. Our cynical doctor silicone is example and. To do this, we can exploit the well known. Equation, which regulates. Drift and diffusion in semiconductor so you have seen this probably in electronics so we can define.
\fig{10}{sell.pdf}
An electron current. Which will contain 2 terms 1. It's a drift current due to the electric field and the other 1 is the diffusion current due to the gradient. This current needs to be continuous. So, it's divergence will be essentially equal to a term of generation. Which is this 1, so photo generation and the term of recombination. Which is stands for shock, every dollar combination mechanism and of course. This equation needs to be coupled with the. What's on equation where essentially you have a. The free charges electronic laws and the fixed charges in this case, we were considering a negatively doped semiconductor. So these are. Of course, you can do the same with also. Then what is interesting for our purpose is that you can do the same with spin. So you can define the spin density as a difference of the population of spin up and spin down. Define spin current as the difference between these two currents and then you will have a diffusion equation, a current definition from the spin and also the continuity equation Of the spin where you have essentially. This additional term, so this W is essentially given by this shock. Say, the spin in our semiconductor can disappear because the electron disappears 3 combines with the whole, or it can disappear because the spin itself is lost because of some, uh, spin scattering mechanism. So this is. Take it into account by this number and with this model, we, we, you can analyze it. The photon energy dependence of different semiconductors. So we did it. So this is essentially. Galleons tonight. Geminium and silicon, so each 1 of them has his own. Lifetime, and we see that we can at least. from a qualitative point of view, in some cases also from the quantitative one, reproduce our spin, inverse spin-on effect voltage and so understand, basically calculate or measure, let's say, the spin lifetime.
\fig{11}{sell.pdf}
Another interesting way of doing it, which is, I mean, even I find it beautiful from the visual point of view, Is also essentially. Using this kind of micro structure where we have a. A series of. Metal part, which are this 1. Which are equally spaced and then we have a. A detector so in this case, we had, we were comparing 2 kind of detector. Uh, 1, based on them on a magnetic tunnel junction. So, if you're following lectures of professor, but that you surely know better than me what it is. So, in this case, it was iron with the. Uh, but these are oxide and this 1 is based on our platinum detector. Using the in the spin on effect, and then what you do in a conformical microscope set up. You go and generate spin locally. They are generated essentially. At the edge of this. Plot met, uh, metallic side, because at the edge of this metallic stripe, you essentially have. The right polarization due to the electric field associated with this, uh. Contribution of this may act as small antenna. Let's say you have a light polarization of the speed. So basically you can generate spin on 1 side. and the opposite spin on the other side so if you scan with this light pin this platinum pads you will get a signal in the two detectors the one using inverspinal effect and the one using the magnetic tunnel junction and this is what happens so this is a comparison of the two detectors so this is just the reflectivity of our surface so it helps us understanding where the parts are so the parts are this black dot here in this map and this is essentially the signal of the spin polarization so negative or positive and these lines are just essentially cut through this shape. And so you see that essentially you have this alternating sign of the spin up and down, up and down, depending on which edge you are illuminating. But this oscillations are beautifully contained in the envelope of a decaying exponential, which allows me to, for example, measure the spin diffusion length. Okay, so at some point if I go too far, I won't see any spin. And so this actually experiment was proving that we can detect this signal also without any magnetic material. So it's essentially just to show you that even with semiconductor, you can generate an unbalanced population spin and do spintronics without actually using any ferromagnet in your setup, but using light to inject the angular momentum, which is then essentially transferred through spin. Okay. Okay, so this was 1 example of, uh. The application of the speed orbit interaction. In the conduction band and before, I mean, uh. Concluding this first part of our lecture devoted to bike semiconductor, I would like to show you another method to calculate the band structure, which is known as the K-P method. So, I don't know if you want to make a short break or not. If yes, write it in the chat. Yes, okay. So let's take a 10-minute break. So we meet here at the, in 10 minutes or so. Okay? All right. I'm here. So if you have question, you can ask. Okay, we can restart.
\fig{1}{kdotp.pdf}
And so I will introduce now and then we continue a new. Argument, let's say new method. To to calculate the band structure of a. Semiconductors is called the K dot P method because essentially. We see that a relevant part of the Hamiltonian will be the dot product between K, the crystal momentum. And P, the momentum operator, so let's see. Which are the fundamental points of this approach for the calculation of bond structure. So, we know that, uh. The Schrodinger equation for a crystal. Can be written essentially in this way. So we will have a. Kinetic energy operator plus some potential, which is the crystal potential and we know. That we function. Need to be written in this form. To satisfy, uh. Block theory. Okay. Well, the way function will generally depend. So we are a function of the position in the crystal, but they will also depend on and the band we are looking at and the position in reciprocal space. So here we'll have the end bands forming up my crystal. So, T here, I indicated it as a. Kinetic energy operator we know, but it takes. This form. and we can also for our purpose it will write it down in this way so we consider this minus H bar square nabla square as two times the application of the momentum operator so let's say if you have a given wave function, okay, a genetic wave function, say psi, if you apply once the momentum operator P, you get a vector, then you apply, you make the dot product and you get back a scalar, and this is perfectly equivalent To apply this operator here and so this is why we will rewrite. Our kinetic energy operator simply as. B square divided by. And this is perfectly equivalent to what we would get. In a classical mechanics, so there the momentum is simply. Times the velocity. so if i make this square i essentially get the kinetic energy in the newton mechanics okay and so let's see what happens if i decompose my kinetic energy term as the the successive application of the momentum operator to my wave function. So we start by applying to our block function this momentum operator once. okay and so I will have a first term when I derive the phase part of the block function and this will give me simply h bar k all these are vectors times the wave function itself and then another term where I apply this derivation to the block function And this will give me something like this. Okay. Then I take this equation here, and I apply once more The momentum operator. Okay, so to this expression, so I will not go. So, essentially all the. Different derivation, but you see that you will have a. Different terms and eventually we end up with an expression like this. you Okay. Well, essentially, I have a. Use this parenthesis to point out that the 2 operators here, the momentum and this. Square of a momentum, they do not act on the on the wave function on the traveling part of the wave function. Okay. So, since we have it also. On the other side of the Hamiltonian and V. Is just a multiplication factor it means that we can simply. Simplify here and remove this part and we will get an equation, which contains only. The periodic part of the block function.
\fig{2}{kdotp.pdf}
So if we now. Multiply everything by 1 divided by 2 and 0 to get. The full expression of the kinetic energy. We would get something like this here. I think it's there is a mistake. This should be. Square yes, sorry. Okay, so we would get something like this. Then here, essentially. I can take 1 of these age actually the age here. And with them. In this position. together with a minus sign, and this will be p. So, this term, I can write it down as 2 h bar k dot p u, where p is the operator, divided by 2 m0. And here, I will have the other term, which is this one. Okay? So this can be easily simplified. Then I will have, I can put back my crystal potential. And on the other side, I will have again my band structure multiplied by U. So this will be again function of N and K. And of course, of the position. Okay, and so this is what we call the, the, the kid of. Hamiltonian so far we didn't do any real simplification because we only, I mean. Follow the select the rules for derivation. And because the derivative of an exponential is again another exponential, we can then simplify the phase factor. So we haven't simplified the problem at all. We only have a different way of reshaping the same problem. What the K dot P approximate method does is the following. So let's consider recast our Hamiltonian in this way. Thank you. this is now, but essentially the Hamiltonian is made of two terms, these two terms here and the other two terms are this one and they will essentially depend on K and on K square. So, at this point, what I can do is I can use a perturbation theory. So, I can assume that I know. I'm able to solve the problem. At K equals 0, so the problem at K equals 0 would be identical to my. Uh, the solution so. I know. The solution to this problem. Okay, and so based on the solution. But I've calculated at K equals 0. I will try to get the solution at k different from zero. So from a graphical point of view, the method works as follows. Let's consider reciprocal space, k vector, and the energy. What I'm doing here, I'm assuming I already know the problem, the solution of the energy and the wave function. you you you you you you you Thank you. ok sorry i don't know what happened so now i share Okay. So, let's. So, what we were saying is that we, we can. Take the full of the kid of P method and for the moment set K equal to 0. So, we will be left with this problem that we consider. our amperetope problem so we will assume we can calculate the position of the different energy levels and also of the different the value of this block wave function and so this is typically done so it's not done numerically so you just essentially need the group theory and some fitting parameter to place these different states so once we know the different state at k equals zero we reconstruct the state of k different from zero in this way:
\fig{3}{kdotp.pdf}
So let's assume that we have a band here like this and this will be state one two three four five i will essentially calculate the value of for this band number four as a linear combination of the states at gamma for the other states combination a1 with state 1 a2 state 2 a3 state 3 and so on so i will use this new Combination of state to actually reconstruct. The band dispersion of my system so again. We are starting from the base, which is in case is a. The value of this block with function at gamma equals 0. And linearly combine them to find the solution of K different from 0. The basic difference with, uh, uh, tight binding. The tag binding method is that now these are not orbitals. These are already a set of a complete base set of normal. Function because they are the solution of this. Problem, uh, which, uh. At K equals 0. Okay. So, to show you how this work, let's. Try to, uh. solve we write down the Hamiltonian for a very simple problem so let's assume that I have in my k equals zero two states one is here and it's let's say an s-like state and one is here and it's a p like state and the energy separation will say the energy of this two state from zero will be plus half of the energy gap you you Sorry again. Apparently also the Wi-Fi here is on strike. so let's go to the final let's see if we are able to continue for another 15 minutes okay now everything should be working so what I was saying let's try how this K.P. approach works in this very simple case where we have essentially only two states to build up our wave function. So we already know that these states have a solution of the k equals zero Miltonian. So let's say if I take, if I solve this equation with the s state i will get half of the energy gap and s and if i solve this equation with the p state i will have minus half of the energy gap. Okay? Now, assuming that we know this, we now want to solve the full problem where, in addition to these terms, we also have the k dot p term. okay and our guest solution phi will be a linear combination of our base which in this case is made only of an s and the p state so in a more in a real application you will have several bands not just two so now as compared to tight binding here we had a clear advantage that s and p are now really orthogonal because we have a solution of the schrodinger equation at the k equals zero so So without any ado, I mean, we can write down the matrix, the Hamiltonian in a matrix form. So we will have column labeled with S and P, and rows labeled with S and P, and we can fill this Hamiltonian with the different components this will be our Hamiltonian we are considering. Okay? So let's start by calculating element H11. So this will be S H11 times the cat of s so now if we consider the terms coming out from the multiplication of this part of the hamiltonian with the cat of s this will simply give us the solution of our Amperturb problem. So it's essentially this equation. So I will get E gap divided by 2. The other term will be this one. But now we can notice that we have a h bar m0 k times s p s but since the momentum operator changes the parity of the s state this term will be zero and the final one will be the multiplication with this factor and here we will simply have plus h bar k square 2m0 s times s so essentially since the middle term is 0 our first element of our matrix will simply be the energy gap divided by 2 plus this k-dependent term. Of course, if you make the same for the other element of a diagonal, the H22, you would only get the difference that the energy gap here is minus. So the value of the Ampere-Thor problem is minus energy gap divided by 2. So you would get something like this we don't need to actually calculate it in detail so let's look now at the out of diagonal terms h12 so this will be s we have the unperturbed Hamiltonian the k dot p part and this constant term term.
\fig{4}{kdotp.pdf}
So now what is going to happen so this first term will essentially give me s times minus energy gap divided by 2 times p okay because it's essentially this product is the solution of the amperetone problem so we can let's say rewrite it in this way and this is now clearly zero because i'm projecting two orthogonal states one or the other the same thing is true also for this last term here because i simply have a multiplication factor sp and so this is also zero so the only term left will be this one which will be something like H bar divided by M zero K times S P P. Of course, this is the operator of the momentum operator, and this is the P like state. And so this is essentially a PCV. What we before we call the PCV, uh. Matrix element, so I can write down this element like this. And I will have this kind of behavior, so my matrix. Will be completed by. This kind of. Well, I have, let's say, forgotten intentionally the vector sign because, let's say, we will consider the solution in one direction, or if you prefer, we will consider like a spherical, let's say, dispersion around the gamma point. What will be the next step that we will address in the next lecture? So now we have this Hamiltonian. We can find the eigenvalues, which will be our energy as a function of k. So I will have two bands coming from these two different states. and then once I know the eigenvalues I will be able to find the eigenvector AS and AP which will also be a function of my k vector. A quick run through the slides, so this is just a summary of what we did so far. Essentially, we applying 2 times. The momentum operator, we could rewrite the generic equation for a. Show the equation for a crystal. Making only the new part of the block function, the periodic part appearing and then we can see that I can treat this problem as a perturbation theory. So, this is kind of different from what we've probably seen before, because. Perturbation theory is typically used when you have a physical perturbation. So, for example, you apply an electric field to an atom and by perturbation physics, you get the stark effect. For example, or you apply a magnetic field and you get the Zeman effect here. It's more like a mathematical trick. So, okay. Switching on K and making it different from 0 is not the. Coming from an external perturbation, but from the mathematical point of view. It still works, and so the solution you can also address this problem and this is done in the. Part of the book that I put on, uh, we beep. You can also address this problem using, let's say all the toolbox of, uh, uh, perturbation theory. What we have. uh use today was using a matrix representation using a finite base set so in our Hamiltonian that we derived this was a two band Hamiltonian.
\fig{5}{kdotp.pdf}
Here you can have several bands and so typically in the literature k dot p model are identified by the number of bands you use So typically a six band model, it's used to describe the valence band. So essentially six bands are the two avial state, two light state, and two split of state. When we talk about eight bands. When we talk about. 8 bands we add the. Conduction band as well. So, the S, conduction band, the, the data gap. And, of course, the larger is the number of band you include in your model. the better you can reconstruct your band structure away from gamma so if you stay in a very small region around the gamma point you need fewer bands if you want to move out a lot you need more bands and so the ultimate model let's say it's the so-called 30 band scale of p where you use 40 states in the at the gamma point and this is covers essentially the full so you can calculate all the band structure okay so next monday we will solve this determinant and see how the dispersion goes you can already see it in these slides.