\chapter{Surfaces: Heteronucleation, Strain. Epitaxy: Growth Modes, Misfit, Pseudomorphic Layer}
\fig{18}{3-Surfaces.pdf}
So today we continue the problem of nucleation and we move from homonucleation to ethonucleation in the first part of the lecture. In the second part of the lecture we address a second problem of growth that is epitaxy. I just summarize. I told you that you have two main difference between fields. apart from technical details like thickness and so on. But in particular you can grow films that can be flat, so layered on a layer, or made by island. What does island mean? It means agglomerates of atoms. So this means... This is the substrate. Layer by layer, it means a film with a uniform thickness. made by layer of atoms one on top of the other but according to Knapp-Hills made this way with a given roughness. Which one is better? depends on applications. We see some examples of applications So this is the first classification and this is the classification which will address the problem of alteroges nucleation. The second classification is ordered film or amorphous film. Ordered, each atom is in a proper place, amorphous is completely random. For example, crystals or classes. This will be addressed by the second part of the course, that is epinexy. Just to summarize, what is nucleation? Remember that nucleation means the condensation, essentially, of a cluster of atoms, a nucleus, started from what? From a vapor super-subterranean. We started homogenucleation, we started homonucleation that is starting from a vapor of a given element, you create a solid nuclei of the same element. So in homonucleation that was involved just one chemical element and two phases that are vapor and solid phase. We studied, we understood energetically, we found the critical radius, we found the energy barrier. Today we see the same, exactly the same formation, but on a eternucleation. What is the eternucleation? Eternucleation means that in this case you have two different materials. The materials you want to grow and the material where you want to go. Essentially it will be not so difficult, we will just extend the theory of homogeneous nucleation of a single chemical species to this problem. In this case we have two chemical species, the substrate and the deposit material. For example, you want to iron or germanium, iron is the first element, germanium is the second one. And two phases, because the depositing material is both in solid and vapor phase. Because you want to deposit iron, you will have iron in vapor phase, and then iron deposited on germanium. So, you have three actors. The depositing material in vapor phase, the depositing material in solid phase, and the subsurface. We have to understand which are the relation in these three actors. We take essentially the same approach we use in home manipulation. that is we wrote the free energy change per univolume, sorry, the free energy change not univolume, the free energy change that corresponds to the formation of a given nucleus. Before, which kind of, which is the shape of the nucleus? The shape of the nucleus was spherical. Why spherical? Because it was a nucleus that was grown in the vapor phase. so in general think about a drop of water drop of water is not spherical because there is gravity but otherwise it would be spherical in this case because you are drawing your material on the substrate you do not take a spherical because you have to wet the substrate so in this case the shape minimizing the exposed surface is this, is a section of a sphere. It mainly is a drop of water. Imagine a water on a surface. It is solid, but the same. So in this case, the volume is a little bit different. This is the free energy change. We have a lot of terms. Why? The first term is the term associated to the volume. Before was the volume of a sphere. in this case is the volume of this part this region this region can be seen as a section of a sphere of radius r and it's possible to demonstrate that the volume of this part is a coefficient depending on the shape multiplied by r cube the volume element then what you have if you remember in homogeneous nucleation you have just one of these three energies these energies are associated to the creation of a surface why? because before there was just one surface, that was the surface of the sphere in this case you have how many surfaces? you have two surfaces because one is the surface between the substrate and the solid. The second is the surface between the bubble and the solid. As before, you have to introduce the three interfacial energy per unit area. Remember, gamma was before we called about the interface surface between the energy. It's the same. Typically, a surface is anyway an interface between two elements. So you can call it surface or interface. But before you had just one interface that was between solid and vapor. So there was just one of these terms. In this case you have two interfaces. Because you have three interfaces. Interface. F is film. Film is essentially the solid phase. interface film vapor the green one interface film surface substrate the violet one and interface substrate vacuum and we will see later why there is once more the violet one with the minus sign anyway you have three terms due to the fact that you have three possible interfaces between these phases. These three terms are related to surfaces, so it will be proportional to the square of the radius.
\fig{19}{3-Surfaces.pdf}
Okay, this is simply if you take this by geometry it is possible to calculate which are the volume, this area and this area and this is just a formula you can find in books of geometry they are not so relevant okay now we have to understand why this sign in particular what is this energy this is the energy associated to the formation of your nucleus okay so the fact that you have this term the fact that some vapor goes in the solid state and as you remember this is a negative term because condensation from a super saturated vapor is a spontaneous phenomenon then you have these two terms what they correspond to the fact that if you create these nucleus you will create two surfaces the first is green one and the second is the violet one so you have a plus term plus sign but what happens imagine that you start with the substrate uncorrugated by any film. Then you create a film. This means that the surface of the substrate with respect to before is reduced by the varioed area. So you have this contribution due to this. You are creating an interface between film and substrate but at the same time you are removing the same interface between substrate and the bubble. So you have to consider both these terms. These two values of the interfacial free energies can be different. This is the reason for which they do not simply cancel. So in this way, if you know all these terms, you can understand which is your energy. The energy is gaining or not. There is one more consideration that we will use later if you remember when I explained you what is the interfacial free energy sorry the interfacial yes energy I told you that the interfacial energy is the same of the interfacial tension because if you think the fluid these two quantities are the same Interfacial energy is energy need for the creation of a surface per unit surface. Interfacial tension was the force need for creating a surface for the unit length. If you remember, the units are the same. But also, we found by a simple picture that there are the same parameters. So we can take this also as force per unit length. Now look at this point. This point is the border of these nucleus. This point is the interface between substrate, film and vapor. You can apply here the equilibrium. In the red forces you can apply the equilibrium. If the drop in the film is stable, this point must satisfy an equilibrium of the forces. and in particular in which direction? in the horizontal direction so in the horizontal direction you can write the equilibrium of these forces and it is this one and it is related to what? which is the parameter that is relevant? this angle because this tension will be tangent to the angle so you have to take the cosine of this angle What does it mean? That if you ask which is this angle of the film nucleus? It's not as you want. It's strictly related, it's called the wetting angle, to the three interfacial tensions or interfacial energies. So if you know these three values, you can understand which is this angle. from zero means essentially the film is flat or more than zero the film is not flat but we'll come back on this later now we stay in the energy field.
\fig{20}{3-Surfaces.pdf}
If you look energies this is the same graph I made with homogenous nucleation that is how this free energy of formation depends on the radius. And even in this case you can find the point at which the first derivative is zero, this is the maximum, and you can find the corresponding radius and the corresponding value. It's just more complicated than before, because we have different terms, but essentially it's the same idea. ok same as homogenous nucleation remember that with this theory we don't explain why a nucleus is created we just explain how this nucleus will change after creation so imagine you have some atoms that by random motion we start to form a nucleus. In this case a nucleus not everywhere but on the surface. Then what happens to this nucleus? It depends on the critical radius. If your nucleus forms in this region, it will decrease the radius in order to decrease the energy, so it will be destroyed. If it starts above this radius, by increasing the radius, so growing will reduce energy. So it's a formal process. So as for the region of nucleation, you have a nucleon with a minimal critical size in order to grow. Clearly, here in this course we are interacting with which is growing. So we are interacting with this region. This means that you find this radius or, if possible, reduce this radius as much as possible the truth make the growing more fabricate.
\fig{21}{3-Surfaces.pdf}
This is just the calculation of I just take this equation okay where I substitute the value of the coefficient I've seen in you see in the other slide and I used also this relation the relation of equilibrium between forces and I obtained this final expression this final expression is the energy barrier for this kind of surface, this kind of shape. And it's possible to see that this is exactly equal to this term, which was the same of homogeneous operation. Well, this is the gamma, the interface energy between film and vapor, essentially solid and vapor phase of the same element. Why this term is a simple geometrical term? That depends on the wetting angle, angle theta. And if you plot just this term, it's called the wetting factor, you can observe this trend. What happens? When theta is zero, what does it mean? essentially you have that this nucleus is very flat which is the limit of a nucleus very flat is a flat film that is the situation that typically not every time but typically we want to achieve if we want to achieve a grow layer by layer so typically we want to achieve this situation In this case, note that this term is zero. What does it mean? That the nucleation for the films, if possible, and later on we'll see if it is possible depending on this, but if it's possible to obtain nucleation for the films, the energy barrier is essentially zero. So it's a process, spontaneous process. So if this condition is satisfied, then nucleation of film takes place spontaneously. The opposite equation is this one. If theta is 180 degrees, the wetting angle is essentially 180 degrees means this. So your section of the sphere is a sphere. And in this case this coefficient becomes one and you have essentially the only duplication. So in this case the contact with the subset is just one point, but all the surface of the sphere is exposed to the bubble. So it's a reasonable result. If you consider this system, you can forget about subsets, you just disturb this one, you just end this term. This is consistent. The result of this line is, if the film can grow as a film, the grow as a film, film means a flat film, the grow as a film is spontaneous, because there is no energy barrier. essentially every nucleus will go.
\fig{22}{3-Surfaces.pdf}
Now we come back to the discussion of this formula let's call it the Young's equation that allied remember by imposing the equilibrium of the forces in this point and this gives us a relation between the cosine of the angle and the difference in the interface tension this interface tension are numbers that can be found if you have a couple of materials you can find on internet, on books, these values okay and you know that this wetting angle determines the shape of this nucleus means flat film or not flat film depends only on this value nothing more first situation what happens is theta is larger than zero if theta is larger than zero you have not a flat film you have some section of spheres so this situation imagine that from one nucleus we start a system like this so you have a lot of elements they are called island island means that theta is larger than zero so the cosine is smaller than one and this gives you this relation between the different interface tensions numerically this term can be neglected in the sense that this is smaller than the other ones, this is just a speckle. So at the end you can make this relation a relation between this and this, that is, I I think it's better to think about them not tension but energy. This is the energy associated with the formation of the substrate to vapor. This is the energy associated with the formation of film vapor. What does it mean this without this? That the free energy associated with the formation of the film larger than the energy associated with the formation of the substrate-vapor interface. So, in principle, it is better for the energy-bio system to leave some part of the substrate uncovered, because energetically it is more favorable to not having the substrate covered. So this is an energetic interpretation of this formula essentially. So in this case means you try to reduce the wetness of the substrate to reduce energy. For example, just an example, the growth of metals on semiconductors. If you look at the different energies, in general, the position of metals on semiconductors is made by island. for cluster not for flat fields due to this theory then you think about application and you need to grow flat fields or semiconductor so how to do if this seems to prevent it because you can have other tricks this is just what happens when you send atom of for For example, gold on gamma-senite will grow by island. Then you can do some more operations, we will maybe discuss in more detail later, when we speak about deposition, like heating the substrate. If you heat the substrate, the solid island starts to melt, forming a flat beam. But it is something more. It is just to tell you, if you just try to grow, what will you obtain? instead if you have this is equal to zero is that what is called the layer grow it is a flat film and in this case the cosine is equal to one so this is this is equality you can also put the sign of larger than larger than is not mathematically logical but if it happens clearly you stay in this situation in this case is the opposite if you forget this this case is more favorable energetically favorable for the system to wet the substrate instead that they do not wet because the energy associated with the interface between film substrate is smaller than the energy associated with the between film and vapor is more the range associated to the sub-server these are the two main difference between the growth in terms of morphology. Morphology means blood fills all items.
\fig{23}{3-Surfaces.pdf}
This is just a summary, just to comment. This is interesting when you start to grow. Why? Because imagine you have a substrate of Then you start to grow a layer of your film. Then you want to add material. Because maybe here I grow, I don't know, one micron. If now you add material, your new substrate is the film below. So essentially, your grow, if you think, is between different materials just at the first stage. but later on when you wet all your substrate by your film a new kind of evaporation will be something more related to homonucleation because the material will be the same so this theory with substrate different from the film we just describe the first stage when you are wetting the substrate with something that is not the same material as the substrate. Later on, essentially you have this kind of deposition, which is called homoepitaxy or autoepitaxy, autonucleation. Essentially, film and substrate are the same material, so these two coefficients are the same and this is zero. In this case, you are sure to stay in this case, so you have layer by layer. So every step after the first one is energetically favorable. So if you are able to add the substrate once, you can know the film layer by layer. this is the situation, you add layer by layer but the substrate is another layer of the same system itself and so this condition is automatically satisfied because this is equal to this perfectly equal the problem maybe is when you have what is called a multilayer multilayer made by alternating stats For example, imagine you have two different materials, green and red, AB, AB, AB. This is more complicated. Why? Because it's clear that if we imagine that from A to B, this is satisfied or this is satisfied, If you reverse the role of film and substrate, you will be satisfied. So if the first layer is layer-grow, the second one will be iron-grow. So essentially this should be possible to produce layer by layer, because of this region. what you can do the first the first thing is maybe this material is very similar for example when you grow some kind of photo detectors or laser made by semiconductor you simply have different level of semiconductor that are very similar so these coefficients are essentially the same is not outer epitaxy, but is similar to outer epitaxy. This is the situation. If they are very different, in this case, layer by layer is not guaranteed. Maybe you can help it, for example, by heating, by removing the diffusion, but in principle it is not guaranteed. So in general it is better to go layer by layer with similar surface energy.
\fig{24}{3-Surfaces.pdf}
This was when the only energetic terms involved in the equation were the volume energy, which is the energy reduction of the system due to the creation of a volume that is due to the condensation from the vapor to the solid base and the energy terms due to the creation of a surface but you can have also additional terms because you can have other interactions in your system for example a typical interaction is an interaction related to strain what does it mean strain look at this picture this picture, we see in the beta-seq what does it mean a crystalline structure. But essentially, crystalline structure, each atom is placed in a proper position. But not all the materials have the same crystal structure. So maybe you have the substrate with atoms are the edges of a cube with a given dimension and the film with a different dimension. This kind of rope we see that is not energetically favorable. It is energetically favorable when one atom is on top of the other. So what happens in this case is that you have to strain the film in order to accommodate with the same lattice parameter in the plane of the substance clearly this is a deformation think about like a material that is stretched there is an elastic energy associated with this operation this elastic energy is positive because you have to give energy to the film in order to move the atoms in order to make this situation like a spring if you want to increase the length of a spring you have to give to the spring energy. So what happens? Simply by this picture, the thermodynamic picture, you can simply add your equation to this term. The disproportion of the volume and the first term and to this energy. This is the energy strain free energy change per unit volume. It is the free energy change due to the strain per unit volume. What can happen? You can recalculate these two terms and in particular this includes here these additional terms that is added to this one. was before is now you can have two situations there is an error no okay yes remember that this term is negative because it's the free energy associated the formation of the volume that is negative because your spontaneous process the condensation from the vapor to the volume if the second term is positive as in this case that is you need energy in order to create your deformation this means that this denominator this term decreases with respect to this one because you have something a negative and a positive term so this increases with respect to this one what does it mean it increases? it makes more difficult the nucleation apart from the formula I think it's quite intuitive you need to give more energy to the system to the formator is more difficult than before. By the contrary, if your chance and this term is negative, you will add negative with negative, so this with the prism. So in this situation in which the strain induced by the growth is energetically favorable, that is negative, in this case it's things of favors denucleation, which is the situation for example because imagine that it's not the film that will strain but the substrate maybe you have a substrate that was initially strained for some reasons due to the growth the strain of the substrate is relaxed in this case it's a gain from the substrate because it was before there was in the substrate an elastic energy energy. Now it's elastic energy removed. So this means that it's energetically favorable. So in the second case, nucleation is easier. Typically what happens in the first case, typically what happens is that the substrate does not change. What changes is the film and in this case this situation is more energetic and demanding than this. This is a typical But we can make on this in epitaxy in the next lecture. Any question on this? The message is this, depending on the three energies of the system, you can distinguish between layer by layer and island row. There are two kinds of morphologies. Two kinds of morphologies that can be determined in device. device made for example a tunneling junction or any device like CMOS you need layer by layer but maybe if you need some device related for example to the catalysis of a material you need as large surface exposed as you want as you can so it's better to go by island or maybe you want to grow a material that is just a capping layer, just to protect it doesn't matter if it's layer by layer or by island, it's enough that at least there is some wetting of your system. Ok? Okay, but the last information is here you can add as many terms as you want. Maybe you can have other additional terms due to the presence of impurities. This is a very general picture in which you can add all the terms you want. OK. If it's .. Sorry. But is the equation one is both for the ? Yes, because it's not distinguished between one or the other. We will see that in crystal in films. You can easily, slightly change them. But not the equation, but the free energies. But the equation is the same. Okay, so we move to the second part that is related to the detoxing. We start this part today and we finish today or tomorrow. But anyway, tomorrow we start with the main part of the course, that is the position. From tomorrow we start to address which is the position. How to deposit, which is the technique, the theory and so on. So the theory part will be finished today or just a few minutes tomorrow.

\chapter{Epitaxy}
\fig{3}{4-Epitaxy.pdf}
Okay, we move to epitaxy. Epitaxy is the second fundamental aspect of films. What is epitaxy? Epitaxy comes from the Greek and means the extended single crystal film formation on top of a substrate. you start from a sub-circle you add atoms but not in a random way after the row the atom will be distributed in an ordered situation this picture was a picture from the last from the surface lecture which I show you which is a single crystal you remember that single crystal is something that is periodic across the full volume you can take one cell and repeat it by translation you obtain the full crystal Epitaxial is fundamental in many applications like semiconductor, thin-thin device technology but not all applications so maybe in your work we will not need epitaxial because there are some applications, for example For example, coatings, coatings means protective layers. Recording many magnetic materials are crystalline, but many others are not crystalline. It is better than not crystalline, because if they are crystalline, they are fixed direction, which must apply magnetic field. If they are not crystalline, there is not this problem. Anyone of you is following this course? Robert Harko? Ah no, because it's the third year. If you will follow this course in your career, you will understand exactly why material and crystalline are more complicated than crystalline. But in general, recording is related to recording by magnetism. think about the display of the monitor, it's not crystalline. Which is crystalline? For example, some kinds of tunnel magnetoresistance devices. Tunnel magnetoresistance is tunnel junction. some electronic devices may be there based by stacked AB AB AB structures mainly microelectronics or essentially every device based on quantum effects because quantum effects are typically based on we study later what are called the Bloch's equations equations and the blocks equation are equations that are able to predict quite everything from the specific heat, from the color to the electrical conductivity, but they require the periodicity. So the periodicity means you need a system with crystalline. So quantum effects, in this case MgO based on magnetic resistance is the quantum effects, superconductor quantum effects so in all this case you need epitaxial crystal is a synonymous for epitaxial one epitaxial layer is a crystal layer there are two synonyms as i remember at the end of this slide there are some additional slides you can take a look which for example this is a spain detail just to have an idea.
\fig{4}{4-Epitaxy.pdf}
Manifestation of epitaxy which kind of epitaxy can have? the kind of epitaxy depends on the structural analogy between film and substrate think for example about the picture of before in which there were a substrate with a given square lattice and on top a film with a different square lattice you can have low analogy or high analogy high analogy means the lattice parameter of substrate and film are the same maybe there will be some strain as in this case you can see that in this case clearly the film is the film is not by square but by rectangles because it will be straight but anyway there is analogy at this interface. This case is called high level of literacy. What happens if instead the stratum between film and substrate is not so large? That if you have not the possibility to strain the cell of the film in order to match, because the strain must be too large, what you can do is this situation, create effects. That is, not all the atoms here have bonding with the upper and the lower. It's called a defective film. This is a low level of epitaxy. This is a high level of epitaxy. This is just an introduction. There are three families of epitaxy. There are three families of epitaxy. Hormoepitaxy or heteroepitaxy and teterlayer epitaxy. see the tail the second one because it's the most used and just a mention the first the third.
\fig{5}{4-Epitaxy.pdf}
Before do this i have to introduce you which is what does it mean the crystal structure and how you can describe a crystal structure this is a part of the theory that people that we will attend the other registrar and we will do detail but in this case I prefer for some of you that you don't know what is just to introduce just a very short introduction imagine if you have a simple cubic lattice what does it mean the atoms are on the edge of oxygen. This is a very very the most simple crystal structure it's not a realistic structure, typically material does not crystallize with this structure but it's the easiest way to explain, so I use this ok then think about one cell, because all the crystals can be obtained by repeating this cell. These equations are just equations that satisfy this repetition process. This is your cell. This is a cube. Each side of the cell is left gaping. How to label the lattice size? You think one of the atoms are the vectors of the cube. 0, 0, 0 point. All the other ones are labeled by their x, y, z combinates. E means A. So, for example, this one will be A00, because it's along the x-axis, so A00. But you divide by A, so it's 100. It's the way in order to call the position of the atoms. for example A00 is 100 your notation A0 that is the violet one is 110 and so on now we are interested in which is the distance from the origin to our point in just the Pythagoras theory ok this is for addressing the points.
\fig{6}{4-Epitaxy.pdf}
Then we are also interested in addressing not the points but the directions and in particular this direction the direction parallel or coincident with the side of the cube along the x-axis is from 0, 0, 0 point to 1, 0, 0 call it this way with square brackets indicating the final point if I write 1 0 0 between square brackets it corresponds to this direction from the origin to 1 0 0 this is the direction for what concerns one plane a grid plane you can call it indicating the perpendicular direction to this plane for example the green plane is perpendicular to 1, 0, 0 so you call it 1, 0, 0 but with rounded brackets so without brackets means the point, square brackets means the direction from the origin of the point, rounded brackets means the direction perpendicular to the plane ok, so there are the ways used in crystallographic notation to address the crystal direction this is important why because you have to define your system how are you this was just an introduction of the solid state physics.
\fig{7}{4-Epitaxy.pdf}
Then call to homo epithelial epithelium home epithelial epitaxy I told you that is the most important part of the epitaxy because this part we use home epitaxy the substance of the film is made of the same material like homonucleation but here we are looking not on the morphology but on the crystal structure in epitaxy the substance of the film are made by different materials homeopathy clearly, every time you have to have agarotic film, because after the first step there will be atheropathy, then it will add layers of one material on layers of the same material, so it's homeopathy sometimes you can use also intentionally, for example imagine you have a silicone substrate you want to go to the silicone sample and some cmos maybe before to add the material you want for example the oxide you can add additional silicone okay so you have your substrate your substrate is cleaned as you told you possible you put in your machine in the position machine the first step is you add more silicone why not for question of chemistry because silicon is silicon because it can have less defects because it can be pure the welfare cell substrate our first abstract when you buy the purity is something that you pay sense as much as pure in terms of contamination as much as larger the cost so maybe you can accept to have a subject is not superior but after this you add other silicon for example, you reduce the effect or you can doping depending on it. You know what is a p-doped and n-doped semiconductors. Maybe if you want to grow p-doped silicon on an intrinsic silicon, you can do this. Why this homopetaxy not eletropetaxy? Because typically doping of a semiconductor is some part for millions. so essentially is the same so you call homo epitaxy even if maybe the silicon is not pure but doped okay atheropitaxy in order to obtain epitaxial structure epitaxial means when all the structures for device structures for devices for example tunneling junctions and in this case this is a 10 10 is a transmission microscope we see in the last lecture it is a way in order to make a picture of the cross section of your device looking at the atoms because it's sensitive to atoms these are called atoms this for example is a device for the transmission that is made by multilayer AB AB AB of first-cell conductors gallium arsenide and aluminum arsenide these are very similar in the face tension so they can grow layer by layer clearly they are grown not this way but this way but if in this case you need epitaxy so they are just example of hetero epitaxy.
\fig{8}{4-Epitaxy.pdf}
Okay, now the point is, if you have a homeopitaxy, look at the first column, it's quite easy to have a good crystal structure. Why? Because if the substrate has a given crystal structure, the film will have the same crystal structure, the material is the same. So every time you have homeopitaxy, you are sure that if the substrate is crystalline, the film will be crystalline. and then there will be no difference between film and substrate it will be the same it's called matched structure matched means that the lattice parameter of the substrate on the film are the same sometimes in aether epitaxy can happen the same the two materials can have the same lattice parameter so even in this case you can have aether epitaxy with a matched structure What happens if these are different? So, you see that this square is different from this square. Or you can also see here the side of the subsurface is different from the side of the field. In this case you can have a raw strain. Strain means that the film will strain, will deform, in order to accommodate at the interface the same lattice. Because clearly what is important is the accommodation at the interface. So if you have x, y, z, what is important is that in the x, y plane, there is accommodation of the lattice, not in the z direction. So typically, a strain, for example, a compression in the plane corresponds to elongation in the outer plane. But to the cost of elastic energy. because it is energy, to elastic deformities. The third possibility is this situation. It's co-reliance, what does it mean? That for some mechanism, for example, creation of defects, you can reduce your mismatch, mismatch is the opposite of matching, only to this region, the interface region. But on the top, the effects are this structure. So essentially, instead of deformating all the film, you just create defects at the interface. But in this way, you cannot accommodate on top, on this defective interface, the film with its proper structure. Even in this case, you need energy, energy from water. water for example for breathing bondings so in both cases you need some energy but there are two possibilities these are the three phases matched strained and relaxed clearly this structure is no more crystalline in this structure you have a translational symmetry in all the three directions. In this situation, you have translational symmetry only on x-y direction, not z. In this case, you completely lose translational symmetry. You can apply translational symmetry to film or to substrate. But for example, if you have a current flowing this way, this current will see some of the interseminary.
\fig{9}{4-Epitaxy.pdf}
I just propose you two examples because it's true that you can in order to accommodate cube on cube with different dimensions you can stretch but you can also decide for example to rotate there is no reason for which you can put one cube to the other cube but the one on top cannot be rotated. I just propose two real examples. The first is the growth of barium titanate or strontium titanate. These are two oxides. We will enter in the test in the oxides we will study. And I will present also what is barium titanate when we study the position. Okay, you grow experimentally. Imagine you are growing this on this. What does it mean? We switch on the position system you grow. Then after the growth, you can look which is the orientation, I will present to you which is the ways in order to measure the crystal orientation, like serine fraction. And you see experimentally that in this case the growth is cube on cube. The biocontamination of the crystal direction, 1, 0, 0 direction of Stanford is parallel to 1, 0, 0 direction, or you can grow iron on MgO. If you go iron on a geode, you find a different situation. The cube of iron is rotated with respect to the cube of MgO of 45 degrees. This is a bit of a mistake. Anyway, because the film on top will be crystalline, but with a different structure. Why? Which is the element you can take in order to distinguish in general from epitaxial growth and not epitaxial growth. There is just one parameter that you can, or two parameters, that you can take in order to understand this. It's not a general rule, it's just an indication. the two parameters are the crystal lattice the crystal sites a a of the film a of the substance and in particular what is called the lattice misfit lattice misfit is the difference between the lattice parameter of the substrate and of the film normalize the film clearly if the two are the same as in homo epitaxy this will be zero or if they are very similar this is very small in general, often but not always the epitaxial relationship that is for example the not rotation on the rotation can be predicted on the basis of the latest fittings argument the orientation of your film is the one that gives you the best lattice fit or the smaller lattice piece now we'll analyze the tail these two examples we see what does it mean remember that if you have this parameter this parameter must be minimized minimized means if you can sometimes you cannot in this case you cannot have a p-text.
\fig{10}{4-Epitaxy.pdf}
First example if you look at the two lattice parameter that is the two sides of the two cubes they are essentially the same so this misfit is practically zero zero means two percent two percent one percent are zero this means that you can grow cube of cube because this element is very small. Maybe all matter is not in zero so maybe you have a change of the film in order to accommodate the substrate. Sorry, yes. And in particular because the film in principle is larger than the substrate, the film will decrease its lattice parameter in order to fit with the substrate. but this is very small effect in this case epitaxy is cube on cube in the case of iron if we imagine to grow cube on cube the two lattice parameters are so different that this is too large for allowing epitaxy you can see these are the two squares corresponding to the projection there what you can do if instead of considering the lattice parameter of iron you consider lattice parameter of iron multiplied by the square root of two that is what happens if you rotate 45 degrees this number is very similar to this one and this corresponds to a mismatch very small so in this case epitax is allowed provided you have the rotation of 45 degrees okay even in this case there is a difference so even in this case it will be in this case the film will be strained with respect to the substrate but not so much so this is prohibited because the mismatch is too large this is a loader because this mismatch is small okay these are just indications maybe you have not the first not the second, maybe you cannot so in this case you cannot have a middle so the rotation can I put one time or one so in some cases you can have also other rotations maybe if you have some situations like this typically it's not but maybe you can also have if for example this situation typically will not happen because in this case it's better to have a deformation it's all cube by cube but maybe in intermediate situation you can you can also have more complicated situation it's just an idea okay iron mgo grow by crystalline for example but with the In terms of the block equation, this creates a problem not provided the fact to consider in the equation this change of orientation. So the key point is the film must be crystalline, the substance must be crystalline, maybe they have the same crystal structure, maybe not. but anyway, epitaxial means film is crystal this is just two examples taken from papers this is baryutideinette, below strontium tideinette it's difficult to see from there but these are atoms, perfectly aligned atoms this is not a picture by barypoin, this is a picture by transmission electron microscope and this is the same ironized geomix the part where that part of the geomix in this case you have atom by atom.
\fig{11}{4-Epitaxy.pdf}
I told you that we would mention in very short also the other two kinds of epitaxies graph and tilt layer epitaxies because they are not so used but it's just worth to mention because it's a course of graph epitaxy is a way to deposit crystal films or on amorphous substance so while in homo and hetero epitaxy you need a crystal substance to deposit a crystal film this is the only way to deposit a crystal film on not-crystal substance. Typically, if you deposit a not-crystal film on a crystal, you try to deposit a crystal of anamorphose, you can obtain a situation like this. This means the growth of maybe polycrystal, that is, crystal films, but with random orientation. So imagine that your films start to grow with crystal structure. but not extending on the full substrate so you will have some crystals made this way some this way and so on how to force them in order to grow with all the directions parallel or 1, 0, 0 direction of the crystal parallel if you don't have a crystal substrate you can use nanofabrication nanofabrication means you pattern your substrate with some structure that forces your polycrystal to be oriented parallel to the structure so it's a sort of using lithography in order to make a crystal film from an amorphous film I think it's the only way in order to start an amorphous film to obtain but clearly this requires lithography so it's more complicated because you need before to make lithography second is tilt, lift and layer the point of the pitaxii in this case the difference the first case the difference is that the substrate is not flat but is patterned the second case the substrate is not plain but is diagonal. So in the second case the substrate is not this but this. I exaggerated a bit. This angle is very small. In this case it is possible to demonstrate that thanks to this accommodation you can deposit film not parallel to the plate but impunated, and this can accommodate some part of the mismatch. That is, you can grow tilts of a given material on a substrate of another material with a mismatch between the two lattice cells, but without creating any defect, any strain, because of this tilting. I do not enter more details. If you are interested, there is appendix before the calculation. is a way to partially relieve the strain. This is not so used. Why? Because these are expensive to buy. So typically, growth is of a minute repellency, especially because the substrate are the substrate you can buy. In this case they are expensive. In this case you need eutrographs. but there are two possibilities that can be used in case or maybe you can use in a master thesis or so okay apart from this came back for a therapy.
\fig{12}{4-Epitaxy.pdf}
Just a quick summary we have epitaxial artificial microstructure substrate can be amorphous so the advantage is substrate can can be amorphous. The cost is you have to first make lithography before. Theta-layer epitaxy, you use, the cost is you use discrete substrate, the advantage is you can have the relaxation of the layers. So growing themes with different lattice parameters with respect to the substrate, without strain, without effects. homo is the irriterum epithelium epithelium the substrate is the same on the film in the epithelium you have substrate different from the film and this is a typical case because almost doesn't make any sense because in this case you simply go this way okay but not used Graphic Epoxy is a way for example to obtain self-assembled structure. In the first lecture I told you the difference between the bottom up and top down approach. Top down is you remove material, bottom up is you increase material. But in the bottom up approach, how you can decide how to grow your structure, this is one way. way. You have artificial microstructures that force not only the crystal structure but also the geometrical structure of pure material that will go.
\fig{13}{4-Epitaxy.pdf}
Okay, just come back to this picture. This is a picture of before, just to answer to the question of the question about the fact that these can be applied to production of the motor sphere the question is the same even if the theory would be free we see also the years before people made this theory then they made this theory if you look at a paper as well complicated because each one one interpreted this in terms of other phenomena. But anyway, the equation is this one. Which is the difference with respect to amorphous? In the case of amorphous, it doesn't matter how much material you have. The material is the same, independent of the fact that it is thin or thick film. In this case, you can also have a dependence of the number of layers deposited. and is the number of layers and it will enter in these two parameters related to film these two interface energies could depend not only on materials but also on how thick is your layer because having one monolayer, two, three or more can affect this parameter not this parameter because this parameter is not related to films only these two and in particular why you can have a possible change in lattice constant and structure of the film surface for example if you have strain in this case this term film at the film surface not strain, film vapor maybe the interface will be different For example, if you remember the effects of work function. The interface energy caused by the difference between lattice constant and crystallography structure over layers abstract. Over layers is a synonymous with film. What did I do before? If you have a strained film, the interface energy will be different if I respect non-strained. The volume strain accumulates in a pseudomorphic over layer. The last part of this lecture is what about a personal model. The point is, in this case, you must pay more attention to these equations because maybe the independence can change the evolution from one to the other. So, for example, you start by layer by layer growth because the first equation is satisfied. Then these parameters change and you move to island growth. This can happen for reproduction fields. Typically, typically, but not always, it does not happen for a modus film. Remember this is a general theory, it does not apply to single cases, it applies to general cases.
\fig{14}{4-Epitaxy.pdf}
What is a pseudomorphic overlayer? A pseudomorphic overlayer corresponds to the first few monolayers of the film, ml for this monolayers close to the interface remember that you have your substrate your film maybe if the substrate and the film are different you can have at the interface a strain or the formation of defects but later on the film will grow as forgotten the substrate so the pseudomorphic overlayer is a layer between the substrate and the film in this equilibrium configuration so no strained simply imagine you have your iron with its proper lattice parameter your MGO with its proper lattice parameter and at the interface a layer that will accommodate one and the other this is called pseudomorphic over layer. This is just an example and you can define a parameter with the lattice mismatch strain, as we have seen before, define this way. We see that it will give you an indication of how thick is this over layer. If the misfit is zero, f is zero, there will be no over layer because there is no mismatch of the data. This layer will contain both effects of strain layer will be associated to an energy. So this energy is an increase of volume of strain energy that increases with the field thickness. Because if you have strain energy, strain energy is proportional to volume. If you increase the pressure of the layer, the total elastic energy will increase. So what happens at the second point? Can you imagine if you arrive at even critical thickness, after which this energy is too too large and something will happen.
\fig{15}{4-Epitaxy.pdf}
You can have two mechanisms. First, option one. At a given thickness, you have a transition from 2D to 3D go-mode, from layer by layer to island. Essentially, you have this equation, film, film, film, this is a pseudo-mode for the layer, and then island we start to grow. It's called Strachan crust and grow mode. This situation. Here, atoms are alive and wet all the layers below. You can see here all atoms will cover completely this layer. Here this atom will cover completely this layer. After this, atom will arrive and proceed to form an island instead of covering all the bottom layer. Clearly, if you want to make an epitaxial film, this is not the right situation to try to avoid. Why this? Because because this layer will be strained. You can see here, these are circles, these are ellipses, because they will deformate in order to accommodate this lattice patch. So you increase the energy at a given point. Why you move from this situation to this one? Because these ones are graxed. this is the crystal structure of this material so no more strain so they prefer to create island instead of increasing the pseudomorphic layer growth simply for energetic reason and this is the summary this is layer by layer this is the island growth and this is the layer plus island growth equation and this critical thickness means the thickness for which the pseudomorphic layer becomes unstable and creates island. Now imagine you have to grow a device. You start from your substrate that you grew your film. Imagine that is a tunnel in charge. So you need this thickness as much precise as possible. So you don't want island. If the thickness of your device is smaller than the critical thickness, you are sure to stop your grow before island can start. On top of these you will grow other materials, for example, this object of iron, MgO is the first film, iron is the second one. It doesn't matter even if there is this method of relaxation, because you don't have alcohol to relax, because the thickness of your device is smaller than the critical thickness. instead if the larger the critical thickness will have a situation like this is completely wrong for this kind of devices so the point is the critical thickness should be as large as possible because as large as possible means that if H star is as large as possible we we stay in this situation every time we'll have ULA strained but crystalline and for example this is the case these are the typical structure of magnetic tunneling junction I think that I reported the arbenics of this iron MgO iron this iron MgO is lightly strained but not so much and we make calculation by transport code and so on can predict exactly this particular result this critical thickness increases as the mid-speed decreases generally if the mid-speed is smaller the critical thickness is larger it is the problem of layer can be thick the limit is if the match the mismatch is zero you have the thermotic layer with the same lattice parameter of the field of the substrate and so it can grow with it okay so once more you have to try to find the materials in which the mismatch is as small as possible because you are growing on a string film so we respect the form of before you have to remember that the you remember you had the energy of the volume that was a negative energy because it was favorable the condensation from the vapor to the film. But in this case you have an additional term that is the strain energy of the volume that is positive because it is a cost energy and this is proportional to the volume. So with respect to the theory of before of the nucleation, you have to add this extra term. So the reason is this is not relaxed this is strained. As you increase, you increase the strain energy. strain energy. OK this is the first way.
\fig{16}{4-Epitaxy.pdf}
There is a second way that is the misfit is accommodated between the two lattices typically how? First has very small thickness and modality there is mismatch The film strains elastically and adopts the lattice constant, the substrate, essentially as before. The first layers are strained. Then, what happens? Before, what happens is that, after the end of the Pseudomorph layer, islands start to grow. In this case, we'll have defects. the defects or dislocations means essentially that some atomic bonds are broken the symmetry dissection is lost and but in this way you can accommodate a film with its proper lattice parameter on the sub-circle with a different lattice parameter there are two different situations even in this case the point is I start with a substrate then I have my pseudomorphic layer at a given point all the islands start all the misfits start which is the critical thickness of the pseudomorphic layer after which the effects will start to appear this location starts from here, it's the same as before. The mismatch is smaller, this thickness is larger. So as before, you stay in this situation. Which one will take place? It depends on the materials. For example, when we are growing semiconductor, typically semiconductor on semiconductor, like germanium or silicon, this is the other materials you have to go I don't know anyway the point is this happens because the energy contained in the strain layer becomes too much this energy is proportional to the deformation imagine to elastic energy of a spring if the formation is smaller because the Misfit is smaller then you can increase the stickers okay clearly these create some problems in some devices for example if you have a device if a current moving in this direction we see a larger resistance if i respect to a current moving in this direction moving in this system because the presence of this location the effects are on, it creates trumps for electrons, which are fatal, and it creates resistance. So this is a problem not common, not one of the old, but all electronic devices. better to try to avoid to have effects that can be a problem many for example imagine you want to go need for your device a germanium layer and you go to germanium layer on a silicon substrate for silicon are cheaper and understand What you can do, if you go germanium on silicon directly, will leave a lot of dislocations. So in this case, for example, this, I don't remember exactly the lattice parameter, but there is a difference of 4 or 5\% between germanium and silicon. So the xenomorphic overlay is very small and dislocations start to appear immediately on germanium. What you can do is to use intermediate strains, for example, of JxC1-manx with x varying in order to accommodate position by position the strain. the way in order to accommodate this strain there is a group in Como that is expert on this for example, there is a way in order to accommodate, try to avoid this kind of problems clear? any question? If you have any question even later on you can write me or the next lecture.
\fig{17}{4-Epitaxy.pdf}
All the other part of this lecture are appendix in which this case for example present KMR based materials to see the what I just mentioned.
\fig{18}{4-Epitaxy.pdf}
\fig{19}{4-Epitaxy.pdf}
\fig{20}{4-Epitaxy.pdf}
\fig{21}{4-Epitaxy.pdf}
\fig{22}{4-Epitaxy.pdf}
\fig{23}{4-Epitaxy.pdf}
\fig{24}{4-Epitaxy.pdf}
\fig{25}{4-Epitaxy.pdf}
\fig{26}{4-Epitaxy.pdf}
\fig{27}{4-Epitaxy.pdf}
\fig{28}{4-Epitaxy.pdf}
\fig{29}{4-Epitaxy.pdf}
\fig{30}{4-Epitaxy.pdf}
\fig{31}{4-Epitaxy.pdf}
\fig{32}{4-Epitaxy.pdf}