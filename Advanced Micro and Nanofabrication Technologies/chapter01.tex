\chapter{Introduction to Advanced Micro- and Nanofabrication}
\fig{3}{1.introduction_fin.pdf}
Okay, so today we will make a brief introduction of advanced micro- and nano-fabrication. You will see what does it mean and which are the driving force and which is the evolution in the years starting from the 15th century to nano-fabrication and where we are here now. So this is the outlook of the lessons. So this is the introduction, we made some history. And this is the, let's say, this part of the greatest silk in the Smurfs law is the driving force that brings everything, think is that advance everything in the real of micro nanofabrication. So the need of creating integrated theory. We will see something about this and where we are now. So the idea is that we are moving and they need micro and nano real. So if we start with something that is made by the we will start with the, let's say, 1 mm range in which we have the tissue, then we move to the human hair in which we have a diameter between 60 and 200 microns, a human cell that can be in the range of 10-20 microns, and then the bacterial cells that are, let's say, So, this is the micro range and in this case we relate to the micro fabrication. And from the point of view of what men are making in this range of dimensions, we will see that above the millimeter we have the precision engineer between yes above 100 micro let's say in this case the tools that are made are made with some physical mechanical tools like the drill the cutter and so on and it's something that we buy in different shops for example so brico center to say something something for the other land because we have to i don't want to advertise and anything and then we have need to if we need to move below 100 microns the fabrication process becomes more difficult and require a higher degree of technology in the At the real Mach 100 microns, we have the MEMS sensor. MEMS is the acronym for Micro Electrical Mechanical System. We will see something about this. We use application in many, this is a sensor, so it's mainly in the sensing area. For instance, all the accelerometer and the gyroscope we have for instance in our coal, but also in the cars and so on, are made by 9 sensors. Then if we move in the range between 1 and 10 microns, we start to have the first transistors that were developed in the 90s. In this sense, it's a minimum feature size, so the minimum achieve in this kind of advice was in the order of 5 microns. So for this technology, we use what is called conventional lithography and what is called silicon layer technology. What does it mean? Silicon is the main material that is used both for man's and from pistol, we can see it and so all the technologies that are related to the fabrication of something from silicon is called silicon technology and the idea is that with this dimension we can inspect our device with optical microscope because the dimensions are way longer than the wavelength even if this picture I report here as scanning electron microscope picture, but in any case we can check them on the optical microscope. Here there is reported this LIGA process. LIGA process is a process that is made mainly from the MAMS sensor. You will see something in one of the lectures and LIGA is a German acronym and means that it's a process in which we have optical lithography we will see it in a few minutes and these two processes this is galvanic a formal uh garbano form and form that means once we made this optical lithography we will see we will create a mold with some electro deposition is a deposition that deposit metals this metal create this mold that is a stamp and then the last process it is up formal means from the stamp I create the object. So it's a way to make something that is big but it's a very fast way to make a lot of serious stuff all together. And okay then we move to the nanoworld and the nanoworld in let's say in the world of nature is represented for instance from the virus that is in the order of 100 nanometers, head DNA that can be of the order of 3-4 nanometers and then we move to outdoors in the order of Armstrong. What is what happens with the man-made stuff? Here we have something that is from the technological point of view more advanced. In this case, I report one of the many examples we can have. This is a single electron transistor. It's a transistor in which we are able to control not the current, but only one electron at a time. To do this, we need a very small dimension. This is even bigger because it's 100 nanometer diameter. but for sure in this case we need some other lithographic process that are involved for instance also conventional transistor nowadays have some minimal dimensions that are in the range of the nanometer size so now the state of the art we will see is three nanometers and to do this in In a laboratory environment we use for instance electron-bublitography that uses electrons to make the process or some other technique that is scanning problitography that uses an anometric tip to make the fabrication process. In industry we will see something about this. They use very expensive machinery that are related to extreme UV litography that allows to get this resolution that is well below the wavelength of light of the visible light of course there are also some other examples that are taken from let's say more from the chemical point of view that is the molecules of medicine that are synthesized for instance this is an example of this or this is a quantum dot that can be created by mechanical approaches and that is used in our screen to emit light and this is for instance was the object of the last year Nobel Prize in Nobel Prize. So let's see now, oh, just and lastly, in this case of course we are not able to see what we are doing with optical microscope, so we use again electron beam microscope that use electrons or some other technique, some scanning probe that use the interaction of the nanometric tip to let's say visualize some properties of our object for instance the morphology of the object the electronic electric properties productive properties and creative properties and so on.
\fig{4}{1.introduction_fin.pdf}
Let's make a look from the historical point of view now of micro nano fabrication so i reported this algorithm that is from richard fayne payment that is we will see one of the inventor of nanotechnology he said if an apple is magnified to the size of the earth then the atoms in the air are approximately the size of the original So if we think that the network can be bigger than the earth, an atom is bigger than the network. And the idea of micro-nanofabrication or nanofabrication is to push the limit of fabrication towards the atom. And Feynman, that is a very famous physicist, he won the Nobel Prize and his reputation reputation is not so good because he is a little bit misogynist, but okay, he was really a very good scientist. He is considered the father of nanotechnology. Here I report a lecture he gave on the American Physical Society conference in 1959 that is called a plank of movement at the bottom, an invitation to enter a new field of physics. So his idea is first, let's say, a question that was of course trying to stimulate the discussion is why cannot we write the entire 24 volumes of the Encyclopedia Britannica on the head of a pin?
\fig{5}{1.introduction_fin.pdf}
So the idea is that he make a reasoning that I report here in which he will, he is demonstrate that is possible to do this of the theoretical point of view because we are able to, let's say, to use, to go down to the dimensions that are a little bit bigger than atoms. So in principle we can do this because we can use the atom to store the information somehow. So the point was, it was feasible from the theoretical point of view, the point was to have the technology to do this. So he stimulates a lot the scientists, especially the experimentalists because he was more a theoretician one. and he also made a bet. He promised to give some money to the person who will be made to do this. And after some years, the inventor of electro-bibliography was indeed able to write, not only the encyclopedia, but, let's say, a good part of it. And so from there we started really with the nanotechnology era. This is from the scientific point of view. But of course there is also a more, let's say, point of view that is more related to the market world and we will see. So now we will focus a little bit more about what happened on the industrial point of view and then during the course we will come back to the physical and research side. So the idea is that we have a lot of applications of micro and nano fabrication, not only our laptops or phones and so on, also everything needs micro nano fabrication, every device. So of course the main application is related to electronics. Here I have a few examples. This is the processor of Apple A11, a little bit old, but of course here we will see more in detail something about this stuff. We have a complete and complicated architecture with CPU, memories and so on and all of these are created with nano-fabrication tools. The idea is that we will see to do this kind of stuff, it takes two, three weeks, a lot of steps and it's very complicated. We have not only processors, as I said before, we can have manned devices, accelerometers, and of course here we reported some displays, for instance, that can be used. This is a bendable display and also in this case we use nanosupplication tools that incorporate also chemical approaches. Then we have photonics. In this case, the idea is to make this microstructure or nanostructure in order to manipulate the light. This is a metasurface, it's also called a photonic crystal, and this structure is, let's say, a lattice, it's repeated, and what happens is that the light that imping on this structure is not seeing the single object but is seeing a continuous structure with new properties that are arising from the fact that we have this nanostructure so it's not a let's say a single relationship with a single object but is the array of objects and in the surface having different properties from a flat one there is also another application that is really let's say now already caught at this quantum photonic so the idea is to create some circuit that use light to mimic quantum processes to make, for instance, calculations. Then we can also, when we speak about tests, we can speak about sensors. Now there is a lot of study, research, but also, let's say, quite already objects in real life that are used as sensors and that are, let's say, wearable. For instance, our watch, many of us have, let's say, ways to measure the pulses, the electrocardiogram, and so on. And also, for instance, drug delivery that is implanted under the skin. For instance, you can think about devices for insulin that are already commercial and used, and they are done with micro-contamination. Food, also in this case we speak about sensors. One of the ideas is to have sensors directly on the production chain of food in order to understand if there are contaminations. I think about the salmonella, listeria and so on, but also biodegradable sensors. So sensors that we can eat and then we check what happens inside our stomach, for instance, and so on. So energy. Energy maybe is one of the most important challenges of the next years. We have, let's say, conversion, so for instance here I put some photovoltaic cells, but also storage that is important, for instance, from electric vehicles, and this is a very, let's say, the production of batteries is something that is very important, and there is a lot of research on that. Again, mechanics, so we have a mechanic precision, here I report this is the nozzle of a printer that can eject ink, for instance, and this is made with some micro-publications, the process is used for instance to do this kind of stuff and some precision mechanism inside some for instance, analog watch, there are stuff that are made with microcomputation. And last but not least, biology because now there is the idea to go down in science also with the laboratory because one of the main problems is when they have to do some experiment or some study is to collect the biological sample that is normally in a very small quantity. So we need a tiny lab to handle a tiny amount of biological samples. And so there is a team of a lab and chip that are miniaturized labs that are made with micro and nanofabrication and that are also useful for let's say make some screening on film and then there is the point of for instance tissue engineering in order to recreate artificial organs and so on so this This is of course a list that is not exhaustive at all, but for the moment I say we will focus more on electronics because traditionally it was electronics that start everything.
\fig{6}{1.introduction_fin.pdf}
So when we speak about electronics, we speak about the semiconductor manufacturing process that is report here so the idea is as i said they speak about semiconductor manufacturing because the base is always for the transistor at the moment is still the silicon silicon is cheap there is a lot of silicon in the earth so it's very convenient to use it so they start with the fabrication of wafer that is made in some reactor so it's a chemical procedure and it's the same time they have to design the circuit that is made by people for electric and electronics and information technology background once we have the design of the circuit they have to design the mask So what does mean mask? Mask are all the steps that are required to do the device. We will see at the end of the course all the steps that are required to do a transistor that is used in the processor that are commercially available. so we can add 20 and 30 steps. These steps are called the mask because we will see normally they use a mask to say make the main process that is the optical lithography process. We will see the next slide what that is used. So once they design all the processes, they start with the process that is the wafer manufacturing. So they start with the bare wafer of silicone and they make a lot of deposition, lithography, action ion implantation, thermal treatment, we will see quite every of these aspects. aspects so also this step are made several and several times so the focus let's say of the course will be from this side so the wafer manufacturing then they will finish everything doing some contacts some canalization this is important to mount and achieve then on a wafer that is normally 300 millimeter of diameters many chips are taken so they check the quality of the chip and then they cut them and they mount them on the the chip carrier so in order to put them in our devices So this first part is called front-end processes and the last part is called back-end processes. The idea is that the factories or the factories that are the industries that make this kind of process takes, as I said, two or three weeks to do that. So there are only few industries that are able to do state of the art processes that are in our phone and in our laptop. One is Intel, one is Samsung and one is TMCC factory. So the technology is so difficult and this process requires so many years of optimization and it is a very high technological degree. The machines themselves that are used to do this process are very complex that only three producers in the world can do this in the state of the art, let's say, product. So we will see why this is so complicated during the course and now.
\fig{7}{1.introduction_fin.pdf}
Okay, so let's see what is an example of a simple example of manufacturing. Just understand why I speak about mask and I speak about different steps to do something. Let's start with the idea of I want to put these two metal nanostructures on my substrate, that is the silicon substrate. So these nanostructures are two simple squares of gold, for instance. How can I do it at the micro- or nanoscale? We have two approaches. One is called direct-edge process. So what I do is this, I take my substrate and I put on top of some gold. The gold covers all the substrate. Okay, so we have all the substrate that is covered. For the deposition of gold, we can use different techniques and you will see the first part of the course. Then we put something on top. This stuff is called resist and is a polymer format. This polymer normally reacts with the light or with the electrons or with other energy forms and react in the sense that it changes its properties and in particular its solubility properties. If you maybe have been familiar with the camera, an analogic camera, the analogic camera has a film. The film is impressed by the light and then you can print from the film the photograph. So the idea is the same, the resist is something that can be impressed by the light. And how can I create this green structure on our resist? I use a mask normally. A mask is something that, for instance, stops the light here on the top of the green, let's say, rectangle and keeps the light flowing in the next part. So then I do a development and during this development the part of the resistors that are exposed goes away and the other stays there. So using a mask and light we can create this pattern by exploiting the response of the resistors. This pattern is a mask, it's called again a mask because it protects the underlying layers from the other forces. In this case we do etching, so we mill away, we remove the material and we remove the material in all the parts in which the material is exposed apart from the part here in which we have the release. So the idea is that in this case the resist protects the laminate layer, we remove the resist, normally made by acetone is very simple, and we end up with our metallic nanostructure. So there is another approach that we can use, it's the lift-off process. So in this case we start with the gold on top and we remove part of the gold here. We start with the substrate and then we add the gold subsequently. So again we do the lithographic part, so we create the mask with the resin using light and then we deposit our film of gold. In this case, the film covers every part of the surface, but a certain part is on top of the resist, while another part of the wafer is in contact directly with the wafer. Once we remove the resist, we put the wafer in acetone, for instance, the green part dissolves, and with it also the gold goes away. And so at the end we finish with the gold only in the part that were exposed during the process, so that were not covered by the rings. So during the course we will see, in the first part we will see these two, this link, the deposition, while during the second part of the course we will see the other part, so the lithography, etching, and the resistive movements. And so, of course, if you have any questions, do it.
\fig{8}{1.introduction_fin.pdf}
Everything here is done in a clear-roam environment. This is the clear-roam of polypharm. These are two pools that are used to make some chemical process. For instance, the development of the resist and so on. This is a machine that is called mask aligner that is used to impress the resist with UV light. We will see it after a month. There is, if you want, you can turn off the clear room. Why we use the clear room? Because of course we are dealing with something in the nanometer, micrometer range. So if some particle, dust, is covering the surface, the device is already damaged. Because of course the dimension of the particle can be even of the dust can be even higher than the direction of the device you want to use. And there is also another, let's say, important point that is the environmental condition. All this stuff requires, for instance, some chemical process, some heating, some chemical reactions reaction that can that have to be done in the control condition, control means control humidity, temperature in order to have a very reproducible process otherwise it is really complicated to have a receive to do something if every time the environment condition changes so we have to let's say to put this suit in order to stay there, gloves, mask and so on. And you will see if you want to enter in the laboratory and see what happens. You will see something more about this on the lesson of November 19.
\fig{9}{1.introduction_fin.pdf}
So now we start with the history of lithography that is going a lot of centuries ago because it starts with the process developed by Duret and Palmiciamino, the artists. The idea is that to create this kind of structure in metals, or in normal metals, so they start to develop this process. So they take this metal slab, they put some wood on top, and with the season they make the pattern. Once they made the pattern, they use nitric acid in order to etching the metal, so to remove the metal, and then they create this engraving in the metal that are like bas-solid. So this is the start of this kind of technique. This is not really really because we need another step that was invented a few centuries after by a that is the step of time. So we need to print on, for instance, paper. So the idea is always the same. They use this Cavalier limestone, limestone is in Italian felt art. and they put on top some text with these crayons and they made always with this scissors, they made some pattern, some drawing. Then they treat in this case the limestone with a small quantity of nitric acid and Arabic gum. So the gum coming from the tree, the Arabic gum. In this way what happens is this, the part of the stone that is covered by crayon is is able to accommodate the water. so it's hydrophilic, sorry, so once the labriscope is sprayed with nitric acid and Arabic gum, it becomes hydrophilic, while the other part, the crayon part, is in fact, so it's hydrophobic. So the idea is that they put some water on top, and the water goes only in the hole here, And then they put ink. So the ink is fat, so it's rejected by water. And what happens is the ink is deposited only on top of the fat. And so at the end we have this stamp, a mold, that they can use to print the paper. So the idea is this. And they started with this. And this is the start of, let's say, all the printing process and so on. And this is the first idea of lithography.
\fig{10}{1.introduction_fin.pdf}
Okay, then we make a little bit of a change ahead because we start with the idea of the electronic industry. So, in reality, the first, let's say, electronic devices that were used, and they were used already during World War II, were the vacuum tubes. Do you know what are vacuum tubes? They are like this tube in which we have a very small concentration of air, so we have So we had vacuum. In this way, from a filament, like the, let's say, old lens, we were able to emit electrons. The electrons were traveling in the vacuum and they were catched by, let's say, an anode that was accelerating the electrons, and the electrons arrived to the anode, creating, let's say, a sort of circuit. The idea is that with vacuum tubes you can implement some logic because you can switch on the filament and you have to apply a voltage to the anode. Once the voltage is high enough, the electron can travel towards the anode in the vacuum. If the voltage is low, the electrons have not enough energy to arrive to the anode and so we have a knock and door situation. and this can be 0 or 1 logic input. So it was really convenient indeed now they are also trying to do vacuum tools with nanotubes of carbon but anyway it was working very well but the point was that they were very big and bulky and they consume a lot of energy So the idea is that, for instance, they use it on the war plane and Boeing was mounting this kind of vacuum tubes and they used 300 up to 1000 vacuum tubes and they were bulky and big so there was no space on their plane at the end because they have a lot of vacuum So they tried to find another technology that is more compact and they started to develop something that was integrated in some material and not using tools. And in 1950, Shocklane created what we call a thermal junction. So it's again a three-terminal device. One is used to control a current that is flowing between two terminals. So the idea is that a current is controlled by a current. I don't enter in details because then we move to other technologies. So with this bipolar junction, they start to make the first integrated circuit. And in 1958, let's say, two different industries, that is called Per Child, made the first integrated circuit with this bipolar junction. That is this one, this very big. Then in 1958 arrived Texas Instrument and Texas Instrument made this bipolar junction in a way that start, say that was able to start that is called complementary wave of putting the device and was the first starting of what is called cmos technology we will see a little bit afterward and so in 1958 we had let's say the first processor let's say a large circuit that was made with this four transistors and five resistors that were able to do simple operations, adding, subtracting and so on. So, one chip like this and one very simple processor can cost up to thousands of US dollars at that time. So the real, let's say, changing of view was with the Apollo mission. On Apollo mission they put a lot of this integrated circuit in 1969. So this was the real change of everything because people started to understand that there was a possibility to make computers, processors and so on. so they started to there was a request in the market so the price dropped from 1000 to 20 30 euro dollar story so the idea is that after a follow mission they start to produce everything and so the state of the art let's see so we start with something that is on the order of some millimeter and we arrived here today today this is the a17 processor and here i report some of let's say something that effort for advertize put on the website so this is the processor and three nanometers is let's say we will see the technological also is the minimum feature size that is on that processor so the minimum dimension of a part of the device is three nanometers we have let's say some transistor and this is the clock frequency keep in mind these numbers because we will see that we have some problems with this so this is the schematic of course this processor is very complicated now they have this high performance core that are state-of-the-art transistor very fast and also power consuming then they decide to put some let's say part of cpus central processing Here that is, let's say, less, lower performance but has a lower power consumption. Power consumption is the biggest problem now for this industry. And then they have some part that is the neural network in which they have some device based always on the technology of silicon that try to mimic how the brain is working. There is a lot of research on this stuff, they want to compute it. So this is the neural engine. Then they have the GPU, so the graphic process unit that helps to control the image and so on. So this is all the stuff. Going to Samsung, Samsung sells stuff, this is the last processor, 3nm and also 3.4 GHz of maximum speed. This is Intel, Intel they say that they have 7nm of new future cyber technological load see. This number now has only a commercial meaning more than a real street technological one. So maybe seven nanometers for Intel can be more similar to three nanometers for let's say Apple and Samsung. So the idea is this. Of course this is a processor made from a laptop so let's say the frequency, the clock speed, so the speed of the processing is much higher because they have more efficient way of disperse the heating because heating is one of the main problems of this technology now.
\fig{11}{1.introduction_fin.pdf}
Okay, so I don't want to make a course of electronics, you already follow it, but it's understand what is the technology we are speaking about. So the technology on which everything is based nowadays in the industrial world is MOSFET, so field effect transducers. We have a source and a grain in which the current is flowing and a gate that is a contact here that is able to control how much current is flowing between these two terminals so it's a three terminal device what happens is that in this case we can implement always zero and one state so bit of information because we can have that the gate is able to switch on or switch off power device because it's able to create here a channel for the current to flow from source and drain. So MOS is metal oxide semiconductor because the gate here has a metal contact to apply voltage. Here we have an oxide and here in yellow we have the semiconductor. The idea is that this creates a capacitor that is able to move charge. There are two types of MOSFETs depending on which kind of charge are moving. If the charges are moving from source to drain, we have our poles, we have a P-MOSFET. If the carriers are electrons, we have a K-MOSFET. Normally, in this kind of transistor we have two let's say of these devices that are put one on the side of the other and this called complementary uh cmos complementary um architecture it it is we have one p mosfet and on this the other side the n mosfet in this way they stop the device is more stable i don't have very detailed just to let you know there is there are some let's say figure of merit and this is how let's say this kind of transistor are working so So, Vgs is the voltage we apply to the gate. And we have different regimes. The first regime is this one, here in red. If we make some voltage between the source of the drain, in principle the current should run from the source of the drain. but if the gauge voltage is below a threshold no current is able to go because here in yellow the material is not a good carrier, a good conductor, it's a semiconductor, it's not able to make an electric pole jumping from here to here. So to do this we need to increase the voltage of the gate and in this case we create here, below the gate, a conductive channel. How we create it? We move some charges from here, from the, let's say, bulk of our semiconductor here. In this case, a channel is created and the current is switched on. Here in green there are these deflation regions, that is the region in which we have, let's say, no possibility of insulating the region. So in this case what happens is now that we are able to move, let's say, charges here, we have a current and if we, let's say, increase the voltage between the drain and source, we are able to increase also the current at this point. But we have a back, we can reach a saturation. At a certain point, in a certain condition, with a certain, let's say, voltage between D and S, that is a number that is related to the gate condition, we arrive at the saturation. We are not able anymore to increase the current. The point is that we want to have two levels in our device, one off and one on level. So ideally we want that in the off status we have zero current even if we apply a high voltage here between source and drain. And in the on status we have a very high current because we need a good signal to have the bit 1. So we want to have an efficient way of carrying, let's say, the car.
\fig{12}{1.introduction_fin.pdf}
Okay, this is only a recap of what happens because we are speaking about this object. And now I will introduce this man that is Moore. He was a scientist. he was the one that founded a factory that was called NM Electronics and now is called Intel. So he was a very important scientist and entrepreneur. So he formulated in a very famous paper this law that is called Moore's law. This law is a law that predicts how much the semiconductor industry will increase in the area. So he said that the driving force of moving along with the technology is an economical driving force as always in the industry. idea is that he said that with the unit cost falling as the number of components per circuit rattles by 1975, so in 10 years from his prediction, economics may dictate squeezing as many as 65,000 components on a single silicon shield. The idea is this. He predicted that the number of transistors in an area of an integrated circuit doubles every two years. So every two years we have a scaling down, the dimension of the transistor we are scaling down in order to accommodate in the same area the double number of devices. This was the main motivation was the price drop for each component. So this is the graph that represents more or less this. Here we see that we have the number of components, so the cost of the process for each device, and here the number of devices. We start here and he said if we increase the number of devices per hour area the costs are going down. We reach a minimum, a minimum because at a certain point the process of packing all the devices in the area becomes very difficult and costly. So, the costs start to increase. But when we are here, we jump to another technological level. So, they made some technological advance in the production of such scientists and so they are able to start again and to go down in the products cost and this and this and this so he predicted that his law was let's say at least was not valid for 10 years but we are still there and the law is still valid. So I don't know if this is a real prediction or is it a self-realizing prediction. So this is the driving force that forced all the industry to follow Moore's law. So it's a law, but it's also something that was pushing the industry to do this. and this is the prediction of how many components inside the federated seal which were able to last.
\fig{13}{1.introduction_fin.pdf}
So moore's law was arriving let's say up today now in every let's say proposal i write to obtain some grants from europe i say moore's law is coming to an end we need to go to another technology to magnetism, to spintronics and so on. I write about every time this, but this is not true because still we are dealing with CMOS and still we are scaling down, following more or less the same. This paper is 2016, but we have more advanced other data, but I will show you. of let's say this is following. So this is the transistor of chip that is increasing and this is the clock speed. So as I said before we have a limitation. So even the state of the art transistor are not going up to 6 MHz of clock speed and in that case the computers use water to be refreshed because the heating is a big problem. So they're squeezing down the devices but they are not able to dissipate heat and this limits how much current and how much fast operation are going. So we have seen some problems, it's not only an idea.
\fig{14}{1.introduction_fin.pdf}
So scaling down is convenient from the economical point of view, but also from a performance point of view, because the device becomes faster, okay the electrons or course has to make a less road to go from source to drain so they are faster less power consumption because we did this time is the contact and so on so we are able to apply higher electric field given lower voltage and each cheaper as the most predicted because you use the the same area, so the same material to fabricate more devices. There are a lot of problems. Apart from the dissipation, the main problems are, of course, the cut fabrication. Using UV light, UV light has a wavelength of 365 nanometers to impress exists and we want to design something that is 300 meters so of course we have a technological problem then we have some degradation of the device characteristics we will see we have some fundamental let's say problems we will see and also some problems in the material because of course if we are dealing with smaller and smaller pieces of material we can have some defects and we have also a degradation of the mobility that is the ability of the carriers to move fast inside the material so a lot of problems but also some problems so how to scale down where you will see this, this is easy.
\fig{15}{1.introduction_fin.pdf}
We will, this is our, the top view of our transistor, the MOSFET, so source, gate and drain, these are the top contact, this is a width of, let's say, that is this one, and the length that is normally the length of the channel, that is the most meaningful, let's say dimension because in the distance the current has to travel from source to drain so it determines a lot of properties. So if you think to let's say let's divide w and l by 3 we end up with 9 devices. So the idea is that if we scale by a factor of 3 we scale the F factor of L to the square root, 3 to the square root 9. So, Moore's law means double every year the number of transistors. So, every two years, we have to scale a factor, L, that is equal to 2 to the square root, under the square root. So, in order to have two devices. Okay, so they try to follow this. Let's divide every time W and L by 1.4 x 14 and so on.
\fig{16}{1.introduction_fin.pdf}
So, every time they do a scaling, they say they reach a new technological node. The technological node is represented by a number. As I said before in the Samsung Apple Processing, we have 3 nanometers technological node. At the beginning it was the length of the gate. So the idea is that in this case the technological node was 130 nanometers, that is the length of the gate that the current has to travel from source to drain. This is a static electron microscope view, side view of the CMOS. Just to remind you, this is the contact and this is the tiny layer of silicon dioxide of insulator to the capacitor. And these are only 2.1 mm that are 4 lo of silicon atom in silicon dioxide. it's a very very thin layer and we need to put for atom on top of this so it's not so easy but then in 2011 to answer to a lot of problems of stability of our transistor in Berkeley they invented this new three-dimensional design of the mosfet that is called fin fat. The fin means in Italian, finna, fin of the fish. Okay, and the idea is that since once we squeeze down the dimension we have a lot of problems for controlling the channel to the gate. They use this three-dimensional structure in order to control the channel from the gate from three sides. Here we control the channel from one side here on the top, while here we control that also on the two sides. So it's a three-dimensional structure. So in this sense, the gate length loses a little bit the meaning of minimum of minimum feature size because what is the gate length now is this distance or the tree let's say side in which the gate control let's say the channel so it changed a little bit this is a side view of this this is the infector, it's cut here, so here we have a cut and we see here the D-P. And the idea is that afterwards they changed the concept of node, it's not anymore the the gate length but is a very complicated formula that takes into account the distance between the contacts. So with this formula that changed from founders to foundries, the minimum feature size is more or less related to the minimum distance between the two contacts, for instance the gate and the source contact just to have an idea. Now it has always also a commercial meaning so it's less rigorous let's say it's not so far along this formula.
\fig{17}{1.introduction_fin.pdf}
So to start now we know what is a technological node so we see what happened starting from the start of the moore's law up to here 2019 we have some updated afterwards so they start following more or less moore's law every three years and then they start to make a very aggressive Here they follow Moore's law every, let's say, two years, as he predicted, but now in 2000, Intel is starting to be more aggressive. Let's say the 90 nanometer technology node has a channel length that was way lower. And they go like this, so this is only a matter of definition, up to, let's say, 32 nanometer technological motor. At this point there was no able to do any device with this architecture because they had a lot of instability, and they start to introduce in this period the thin-fab technology. So everything changed and we have another scale. Of course it's less aggressive, but also it changed a little bit everything, the introduction of the infect. We will see something about what happens from the present case and the future. Okay, we will see this.
\fig{18}{1.introduction_fin.pdf}
What is the main problem due to the scaling? So the main problem is this one. It's quite easy. We apply a voltage here. If we apply a voltage here and we have the source and the drain here that are these two, which we apply another voltage and we have an electric field that is this one. The electric field here that is moving the carriers from here to here is normal, has a straight line, the current can go low from here to here quite simply. But if we squeeze down the dimension, now the electric field is like this. it does not anymore the line fields that are parallel and we have a lot of side effects and so on. So the point was that in this configuration we were not able to reach saturation, we were not able to control the device and even if this graph are misleading this current here was way lower than this one. so it is a problem so what they try to do is that first okay we have this peculiar electric field that is made by this we change the gate because also the gate at prior voltage that change everything we change the gate we make the capacitor again larger and to do it larger we need to have a very very thin oxide a thin dielectric in between the two album of our capacitor and so we will establish this time so once we scale down the dimension we have to scale down the oxide after reaching that four row of atoms of silicon we saw on the picture of the transducer report.
\fig{19}{1.introduction_fin.pdf}
There are other effects that are reported here and the charcoal channel effect where the channel of conductance is very very short the first one is the mobility of the carrier so if we put a voltage and we apply a voltage to a small dimension the electric field is higher okay because it scales down with the inverse of the dimension so the electric field becomes higher and it can affect the mobility of the carriers. It can damage the silicone itself that is behind the wheel. So this is one problem and so at the end we end up with a very small current. So the on situation is much similar to the off situation and we are not able to to discriminate to the two levels. Second problem. If we have that the source and the terrain are able, are very close together, the electrons or poles are able to move from one side to the other even if we have not gate voltage. It's an effect of Karmelit, someone of you already showed this concept in a lot of courses. So what happens is that we have a on and current even if the device is off with the game caught under the threshold. So also in this case we are not able to discriminate very well between on and off situations. And at the end this is the more complicated stuff but it's always even by the fact that we have high electric fields. At a certain point the high electric fields that are between source and drain are able to move again the carriers that are the charge that are here that was creating the conductive channel And so here you see that the sky, the green, light blue region that is insulating is going here under this kind of chamber. This is because here near the drain all the charges goes away. And so we have no more conductive chamber. This is called pinch of the chamber. and then we have a look at it. Okay, this is only to show you, we have still 10 minutes to show you which are the problems and now I will show you the solution.
\fig{20}{1.introduction_fin.pdf}
So the advance of nanotubble irrigation is also related to overcome this physical irrigation and so they introduce some new tricks. For instance, they tried with the 19 nanometer technological mode to introduce strain to increase the mobility. Then they changed the oxide that was created the capacitance because we cannot go below four atoms of silicon as a layer because we are not able to control that process, so they they moved to the high clay, you can see, and then they cannot overcome this short channel effect and so they moved to the free-effect architecture. This paper was 216, this is wrong, and so they predicted that from here, 5 nanometers, they weren't able to use CMOS anymore but this is not true they are the projection is going to up to 230 we will see with the cmox configuration up to 12 ohmstrom 1.2 nanometers i don't know because it is an atom one, we will do it in 2030, we will see the graph.
\fig{21}{1.introduction_fin.pdf}
So the idea is for instance, new technological advance. This is our semiconductor, the idea is this one, you will see that they put this there is silicon here and in this P MOSFET they put this silicon germanium contact, this let's say contact under the source and drain in order to compress here. This increase the mobility. This is due to the band structure of silicon. I don't enter in detail, there is a back up slide of this if you are curious about this. For the M MOSFET they put some some other material here, the silicon nitride, in order to get some tension here and restore the mobility. So they engineered the material in order to put strain, compression on tension, in order to restrain the everything. And in this case, they were able to increase the mobility, so E-ON and so increased E-ON or E-OFF ratio.
\fig{22}{1.introduction_fin.pdf}
The other aspect was the oxide of the capacitor. So we saw before here, okay here, that to have a good stability, the capacitor here need a very thin dielectric. But what we can do, as the capacitance depends on the dielectric constant of the material, we can not only decrease the thickness of the capacitor, the rule of thumb is if we scale if we have the channel length L the oxide thickness has to be 45 times smaller so with this silicon oxide if we have 100 nanometer lengths we need a oxide that has a thickness of 100 over 45 nanometer length so in this case with a gate length of 35 nanometers we have a thickness of 0.8 nanometer for atoms and this is not feasible anymore so the point is that since the capacitor depends on t but also on the dielectric constant we can increase the to compensate the fact that we are not scaling down the thickness of the oxide. So instead of using the K of silicon dioxide, we use this high K dielectric that has a very high dielectric constant, that is 25 times silicon, and this is okay because it has to compensate the fact that we are not able to go down go down with the thickness of oxide of the board since the capacity tanks is directly proportional to k while is inverse proportional to t this the two effect compensate and this was a big big improvement and then they moved to the the FinFET mode.
\fig{23}{1.introduction_fin.pdf}
FinFET architecture, I already said this, is a three-dimensional architecture. So the chamber is not more planar, like this, but it's three-dimensional. In this way, they surround this kind of structure and it's more able to control better the the channel itself because the electric field that is applied is more uniform. They also put a lot of field inside the same gate, so they parallelized the channel in order to have more current that was flowing but was very well controlled because the gate was surrounding all the channels. But also this is an image of the field, we have different seminal fields, the gate and the source and the drain. There are other contacts that are put there to stabilize the sea moss.
\fig{24}{1.introduction_fin.pdf}
And now what happens, this is one of the last steps, what happens now, so we move from the planar transistor with high-speed electrics to the thin-fet, and now this is, let's say, the prevision. So they will move to another technology that is gate all around effect, we will see, or complementary effect. And this is expected up to 2030. This here is the wave length that are the technology they are using to do lithography. These are excitement later, we will see during the lectures what are. In mention is we put the nexus, the objective of our machine to do the result of the input. And this is a completely new technology that was developed in Poland by ASML, that is one of the most important industry of europe in the technological world because it's the only one able to do this machine that you use this very long way.
\fig{25}{1.introduction_fin.pdf}
This is the Gate-All around. So, gate all around is named by Intel, reborn is called by Apple, nanosheet by Samsung, everyone has this name. So the idea is that we move from the infant in which we have the gate that was surrounding the channel by only three sides to this geometry in which the channel, the green one is embedded inside the gate, and the gate controls it from all the sides. These are two very suggestive advertising movies of Ronin Seren and his Samsung, that show how this kind of technology is built.
\fig{26}{1.introduction_fin.pdf}
And it would go take two of these uh the intolerant effect and we put one on top of the other we make a complementary mass technology with two gas to the other then now the idea is to move completely vertical devices here the current is flowing in plane and the others in this device they is flowing out of the air. So it is a completely new approach that is more prone to the weather.
\fig{27}{1.introduction_fin.pdf}

\fig{28}{1.introduction_fin.pdf}

\fig{29}{1.introduction_fin.pdf}

\fig{30}{1.introduction_fin.pdf}

\fig{31}{1.introduction_fin.pdf}

\fig{32}{1.introduction_fin.pdf}

\fig{33}{1.introduction_fin.pdf}

\fig{34}{1.introduction_fin.pdf}

\fig{35}{1.introduction_fin.pdf}

\fig{36}{1.introduction_fin.pdf}

\fig{37}{1.introduction_fin.pdf}

\fig{38}{1.introduction_fin.pdf}

\fig{39}{1.introduction_fin.pdf}

\fig{40}{1.introduction_fin.pdf}

\fig{41}{1.introduction_fin.pdf}

\fig{42}{1.introduction_fin.pdf}

\fig{43}{1.introduction_fin.pdf}

\fig{44}{1.introduction_fin.pdf}

\fig{45}{1.introduction_fin.pdf}

\fig{46}{1.introduction_fin.pdf}

\fig{47}{1.introduction_fin.pdf}

\fig{48}{1.introduction_fin.pdf}

\fig{49}{1.introduction_fin.pdf}

\fig{50}{1.introduction_fin.pdf}