\chapter{Surfaces: Properties, Coverage, Surface Energy, Homonucleation}
\fig{3}{3-Surfaces.pdf}
We have to start surfaces and the growth in some surfaces because in order to understand what is the position is the growth you first have to understand with your surface and our film can grow on a surface so this part is a physical part okay and we take to left okay the question is why surfaces are important in principle the world animation but why surface are important that is important because it is a place where you grow and depending on properties of the surface and the testing or properties on the relationship between the surface and the material to grow you can grow different kind of things what does it mean yesterday we made the example of uh an injunction telling you that are system made by a layer distributing layer and included between two conducting layers because the tannin improbability by coming with transmission depending exponentially on the thickness of the insulating layer it's clear that if you have an insulating layer there's a thickness it is not continuous that is not uniform we'll have something completely wrong because if you have for example thickness with some ripples it means that the transmission will be completely different from one point to the other so for example in this case you need very flat steam so if you manage to deposit on top of the metallic layer your insulating barrier you must be sure that it is completely flat by the By the contrary, you can have other applications if you want to support this stream, not flat but rough. Rough means, in this way, instant coagulation. For example, if you are interested in material or chemical catalogue, in this case, largest this is a surface for example for gas okay so uh one surface is important the surface is important because it determines the growth that is takes place on the surface in general how growth takes place these the wagon holds the animation anyway starts from here the atom will not wet completely the surface because there is not enough. So we start with the concentration of atomic classes of group of atoms on the surface. Then you add more atoms and you will produce a continuous film that is not completely flat because you have group of groups of atoms. the others were under the field, and so we have a complete coverage of the substrate, but not the film. Then you can continue, you can thicken the film, and so increase the thickness of the film. What happens to the topmost layer? This situation is similar to this one, the correlation, but you can also have situations with a flat surface. This, by the point of view of what is called the morphology, is a plus-versus-thrust. But there is enough information about the crystal structure. Now, crystal structure is something that is very fundamental in this kind of, in many kinds of applications. You didn't study solid state physics, right? So, the theory of physics is based on the fact that main systems are crystalline. Crystalline means the atoms are placed in fixed positions. And typically there is a symmetry by translation. So you take a single cell made by two atoms, and you can replicate it many, many times in each direction. This is a crystal. single cell can be can be cubical can be hexagonal it depends on the different uh crystal you have or more complicated but in general imagine you have a single unit and to replicate and then to the ether you can apply some theorem quantum mechanics and block theorem that allows you to just using this symmetry. There are three kinds of crystal truss, not crystals, but degree of crystal truss. The first is what is called a single crystal. Single crystal is Okay, single crystal means All the atoms are ordered. This is a view of topmost view. But in general it is a three-dimensional view. All the atoms are ordered. So if you can design like a grid, with in each corner there is one atom. This for example is a single crystal. For example, diamond. Diamond is a single crystal of carbon. Because all carbon atoms are perfectly positioned in according to a lot of the but anyway then also ever another equation in which locally you have this crystal structure but not global disease that called for crystal if you take group of atoms they ever this crystal structure but not everywhere okay many magnetic materials like thermonolides as you can see or you can add an amorphous solid. Amorphous solid means atomized disorders like glass. This order determines many properties of the system for example which is the difference graphite and diamonds not the material because it's carbon anyway but graphite as an order is not three-dimensional but only two-dimensional there are layers that can move one of the others the path graphite is black diamond is transparent and this is because Because you to the crystal structure, we have some additional properties with respect to graphite or with respect to single atoms carbon, then gives you the fact that diamond is robust, is transparent and so on. So, you can have films that are crystalline or not, flat or not. How growth takes place? Typically you start from this equation, cluster of atoms, then the cluster central aggregator, completely wet in the surface, aligns to what is called a film. A film is a block of atoms on top of a surface. The film can be flat or flat, crystal or not crystal, but film is this one. We speak about film where the thickness is from new layers to or micro more or less. How we can study this? We can study this by two, essentially two approaches that we will see. The first is a microscopic approach in which we will see what happens looking by the atomic point of view. is atom by atom what happened if you have a beam of atom going to the surface what happens looking at the behavior of the single atom and this is the first picture that we show you today the second part we start today finish next week will be otherwise a microscopic view what does you mean? You don't look atom by atom, but you look, you feel like a continuous entity. Essentially, you look a group of atoms like a drop of water. And in this case you create thermodynamics. You see that this approach is more general, but it's less detailed. do these approaches you can understand a lot of things about these skills and in particular how to understand for example if you have a film of iron on a subterranean medium which kind of growth do you expect so with these electron surfaces and the next lecture on the system you can continue to accept crystalline versus not crystalline flat versus the flat point so we move before starting we discussed about the domestic point of view we have to discuss to moghejati surface that is: what is a surface?
\fig{4}{3-Surfaces.pdf}
When you have a surface you have a surface when you add a soliter it is from when you start when you will study or you study the qualitative you consider a solid made by atoms that are infinite in principle when you study about by using quantum mechanics, consider infinite solid in all directions. We will not use it physically, because it is impossible. Essentially, we consider that periodic boundary condition, that is, the solid, that's dromia, that's diatomia, that we have passed in this point is equal to this one. Anyway, you consider something that is into the quantum field. What does it mean to have a surface? you have to cut this make the solid so introduce a breaking of the symmetry this will create some new effects that are not present in the bulk so surface of additional effects will affect the balance this is a key point because when you have to work with a surface You cannot treat the surface as the bulk. You cannot tell the surface the same properties of the bulk. You have to consider it as a different entity. And this is a key point. Why? Because if you have the surface is on the surface, it will have all the deposition effects. So we have to study the interface between what happens when one atom goes on the surface. And in particular we speak about adsorption and desorption energy. Adsorption means the energy when one atom from the source will be adsorbed on the surface. Desorption is also possible that one atom on the surface will be essentially evaporated. so they have to process adsorption this option if adsorption not absorption because absorption when the atom goes inside the surface adsorption means goes on the surface of the is attached to the surface but stays on the surface not inside Okay, what happened on surface? Surface has different properties from the bulk, and in particular, the tectonic processes have different properties.
\fig{5}{3-Surfaces.pdf}
What about the transfer properties? Imagine that this is your crystal. Imagine this crystal is locked, repeated infinite times in all directions. Direction, direction and also in all directions. Now, if you cut this system, simply you remove all the atoms on top of this line, you just have this continued defect. This is possible, typically not. Why? Because when you break a symmetry, the top layer will see a different atomic environment with respect to the bottom layer. Why? Because if you look, for example, at the third layer, the third layer sees atoms on top and on bottom. If you look at the top layer, you will see atoms on the bottom, but not on the top. This means that the energy of this atom will be different. essentially because if one atom expects to have as you know, as you said, six neck holes the top one atom will have only four of them so it's different, you have two bondings that are not saturated with other atoms so something will be different in particular the top layer will have a different energy with respect to the bottom one So what happens? Three stars tends to go in the minimum energy state. So it's possible to calculate. This is just a description. When you will study policy physics, you will understand in more detail. What happens is typically that these topmost layers will change in order to minimize energy. And there are two possibilities. The first, it will be detached from the bottom. The distance from the top layer, from the first to the second, will be different the distance from the second, third, fourth, and so on. Or, you can rearrange not only the top layer, but the, say for example, the two top layers, in order to minimize that. processes that minimize that. It's clear that this situation is different from this one. Because, for example, the author is different, we are different. So the properties of this surface can be different from the properties of this surface. Or of the bulk, because this is a bulk situation. It is different. So this means that due to the formation of the surface, you can have different surface properties with respect to the bulk. It is a point that you must understand because if you have to grow something on silicon, you must not consider the bulk silicon, you must consider a surface silicon.
\fig{6}{3-Surfaces.pdf}
A similar discussion can be done on electronic properties. This is more related to quantum mechanics, so I try to avoid it to enter into the detail. Here on metal, what happens is that the churning equation predicts that due to this transition, in addition to the valence band state, you can predict that it is on some certain state, it is state, the state above the Fermi energy. It's pretty good so the same as funny and can move Typically never metal all the lessons a little energy or it can be It's not exactly true because one electron then can be removed from the bank or not by a poplar can also arrive here and stay trapped here. So not all the electrons emitted by a metal, so removed from the valence band of a metal, can go pot-emitted. Some of these electrons can be trapped in the surface. This is called work function. So this means that if you want an electron to be pot-emitted, you must give this electron an energy larger than work function. the photonetic effect of Einstein, even in the photonetic effect of Einstein, there is this parameter, the wall function. The wall function is the energy of this strapping state. So this means if you are dealing with a surface, you must not consider the electronic structure of a balance with this one, but also the position of this state. In the absence of this state, any photon can produce potamation. Any photon can produce an energy smaller than the semiconductor. But because of this, the non-substantial potamation will need an energy at least larger than the nuclear one. A similar situation happens in semiconductor. In semiconductor, what is the problem? We conduct, for example, silicon atoms. Each silicon atom has four bondings to the surrounding atom. But if you cut the surface, you have some bondings that are not sealed by an electron. They are free. This will create, by showing the equation, a similar surface state that propagates inside your system. and in fact what happens is that if you remember from second-class technology, electronics, you have typically valence conduction bands and semi-energy loads. What happens in this case is that you use this type of state to make a banding of the band. The band banding. The shock barrier is not used in electronics. It's a problem in electronics. The problem is that something that can be used in electronics is a problem. And this band are not flexed but tend to have a curvature close to the surface. So this process must be taken into account. Apart from the same point is take care with surfaces. Don't consider surface as a bulk. Okay, this was related to surface. So you have the surface of the substrate with some particular properties.
\fig{7}{3-Surfaces.pdf}
Now what happens when you send atoms on the substrate? So you try to deposit. The green is your substrate, the red are the deposit. deposit atom can do first of all they can be adsorbed adsorbed means they enter in a transition region in the region close the surface where they stay attached what happens how you can deposit you must have a solid state solid state is the surface and a vapor phase you study this kind of process as you have a vapor phase of evaporating atoms that won't deposit and a solid state of the surface so the atoms essentially from the vapor phase when they attach the surface go in a solid state there's a change of phase so imagine that you have this atom in a vapor phase that interacts in a transition region This is an exothermic process. They will send the particles, but the atom will attach, which means that the energy of the system will decrease. So it will release energy. Exothermic reaction means that it is a spontaneous reaction. As a matter of fact, if you send a beam of particles on a surface, they will attach. otherwise they will not start. There are two types of this absorption reaction. Physical absorption. The particle is attached but retains its identity. And the forces are typically van der Waals forces. Or chemisorption, in this case the particle changes its identity to ionic carbon and bonding with surface atoms. if you send iron on... I think it's not iron, but... send iron on germanium, we'll have iron on the room. But if you send... oxygen on silicon, we'll not have oxygen on silicon, we'll have silicon oxide because it will send a reaction there are two different kinds of effects which one depends on your material here there is a graph that show you essentially which is the distance on the surface related to chemical molecules and physical molecules and which is the potential energy of the two. You can see the chemisorption decreases the energy, and makes the molecule closer to the surface. Chemisorption is less strong and less close to the surface. But this is not a graph related to a given material, this is a general graph, because some material can be dissolved some other can be kept sold typical oxygen reacts with everything a noble metal ash gold does not react with anything so gold typically is probably okay in the fall we are not interested in the detail of this because it depends on material but we are interested in defining an average energy that is a composite energy of absorption so we consider all the possible absorption reaction absorption states and we take this energy that is negative because the reaction is exothermic so is energy released by one atom when it is absorbed But there is a chance also to do the same in the opposite direction, that is the desorption. One atom that is absorbed can desorb. And you can define in a way a composite energy of desorption, that in this case will be larger than zero. Because if one absorbent atom is in a minimum state energy, if you want to remove it, you have to give this energy.
\fig{8}{3-Surfaces.pdf}
The question is, how is to quantity? There are quantities atom-related. And surface-related, because it does not depend only on what you want to remove. what you want to deposit but also on where to deposit influence the field because a key point of the vacuum deposition technology is how long does it take to deposit a field so how these energies determines these essentially this raised okay assume to have a vapor stage contained at atoms atoms then that will deposit at a partial pressure that will condensate on a surface many times so this reaction is a condensation because it's from a vapor phase on the solid phase of the conversation. Just a name, only substrate is the place where you deposit. So if you have a germanium substrate, it's a germanium substrate. But maybe you have a germanium substrate or a silicon substrate. Many times on top of the silicon substrate you grow a silicon dioxide layer because silicon is inconductive. Maybe you want a silicon insulating film, silicon dioxide. Then on the silicon dioxide you grow other materials. For me substrate is the layer on which you grow. So and the first step will be silicon and the adatum will be silicon dioxide. And the second step my substrate will be silicon dioxide and the adatum will be here. of all. Okay, we can write a simil-pistar equation that indicates how the substrate coverage varies with time. What is the substrate coverage? It's essentially a percentage of coverage of one layer. Imagine you have a layer. Theta equal to zero means no coverage. The substance completely free. Theta equals one means that all the layer is covered by one monolayer of atoms. Okay, so this model consider only one more layer coverage. Cannot explain the roughness. And just explain how long does it take in order to obtain a full coverage one monolayer. First question, how long does it take? Second, is it possible to have full coverage? Because remember, you have absorption reaction, but also the absorption. So maybe if the absorption is stronger, the equilibrium situation cannot be a full coverage. So this equation is quite easy to understand. The time derivative of the subscoverage is made by two terms. related to the position that is related to evaporation per term proportionate to this k absorption we see later with this for the pressure because you know that larger the pressure larger the number of molecules large larger the flux of molecules so larger will be is the demolition rate from the lecture on vacuum technology, remember, plus the proportion to pressure. Multiplied to what? What is 1 minus theta? Theta is the coverage. 1 minus theta is the percentage of the substrate that is free. Okay? So, this, the subscoring increases proportionally to the pressure, so how many atoms arrive, proportional to how many types are free for an atom. Because if you have a monolayer completely covered, you cannot have no more growth, because remember that I speak about one monolayer. So if you have in a given position one atom, you cannot add another one in this model. So this is the first term. The second term is related to desorption. The desorption is proportional to water, to how many atoms are attached. So adsorption is proportional to how many sides are free, while desorption is proportional to how many sides are occupied, so they can desorb. Okay. These two constants are the constants related to the, essentially the classical description of thermodynamics. The exponential of minus the energy divided by kb, kbc. The temperatures are dependent. And note that the absorption energy is negative because it is a exothermic. So it will reduce the energy of the system. So this means that this coefficient increases when the absorption absorption energy in absolute value increases. What does it mean? Larger is the reduction in energy due to the absorption, larger is the coefficient. Means this process is more probable. For what concerns the desorption, in this case the desorption energy is positive. So when the sorption energy increases, so this means, the sorption energy is how much energy is needed to the atom to be dissolved. If this increases, this will be decreased, because easier is to dissolve the atom, larger will be the rate of the sorption. If the atom is very difficult to be removed, because the energy needed is very large, the probability of the source will be small. Okay? So, we have this equation. It is not a complicated equation. material that you have on the subsoil of which is made a subsoil off because it depends on the kind of reaction you have because in some cases once you attach there is very difficult to remove for example if oxygen reacts with silicone it will create a silicon dioxide is a molecule in other case for example example, if you have a Nobel medal, Nobel medal of reactive anything, it's easier to remove. It depends on both. it is possible typically I know that you have just one but it's possible maybe it can depends on the activation temperature maybe there are some reaction that can be activated by the temperature for example if you spend the iron on silicone you can have iron bound on silicone by 100 or you can have an alloy iron silicone it depends on the temperature if the temperature is very large you can have this alloy what determines this temperature the fact is you are heating the software or maybe the position techniques because there are the position technique more energetic like sputtering and less energetic in some deposition techniques where the atoms of the uh the other atom arrive with very large kinetic energy they will heat locally and then we create so in principle it is created it's right wow depends what you want to do because if you want to deposit iron on silicon you don't want to obtain an alloy okay or maybe if you want to deposit you're going to obtain for example silicon dioxide you can decide to start with silicon send oxygen and then exploit the fact that you have a reaction It depends on what you want. We see that there are deposition techniques, for example, silicon dioxide, in which you evaporate directly silicon dioxide molecules. Other techniques in which you evaporate silicon in a atmosphere of oxygen, and so you obtain this. There's another way for doping, for example, you know that semiconductor, like silicon, germanium, diamers and iron, can be doped in order to obtain a pure n-type. In this case what you do is you send beams of dopant and you explore the fact that it will react and so they will substitute some fluid and so on and so you use this. Everything depends on what you want. Now these are differential equations not too difficult, you can solve them. And find the solution. Quite easy. The solution is one. You have this term multiplied by one minus the solution.
\fig{9}{3-Surfaces.pdf}
If you look at this picture, you have in black, look at this equation. It's a constant term multiplied by 1 minus exponential minus t divided by tau. So you have to keep running this. The first is this one, and the second is the time constant. First, look what happened for the last time, t10 to t2, so above the last time constant. It means we should be the equilibrium of the final coordinates. That is given by this fraction, cp divided by 1 plus cp, where k is this ratio between the absorption divided by the distortion. What does it mean? That, in principle, is not guaranteed that you can obtain a true coordinate, because you have a compensation between absorption and distortion. At a given point, you can have an equilibrium between absorption and desorption, so the full coverage can be smaller than 1. Why is it equal to 1? Why, in the limit, Kp is 10 to 1? Because Kp is much larger than 1, this is 10 to 1. So in order to obtain a full coverage, you need to obtain the parameter 10 to 1. What does it mean? This parameter is made by the product of Kp. P is easier. Larger is the pressure of the evaporating gas, larger is the class, larger is the number of atoms that will go to the surface, and so larger is the coverage you will obtain at the end. So one way not to increase the coverage, but increase the pressure. Okay? Or, increase this coefficient. This coefficient is K absorption divided by K distortion. Larger is the K absorption, which is larger than the gain in energy when you absorb. Smaller in K distortion, so larger is the distortion energy needed to dissolve. and so large is the term. But, till absorption, till absorber, you cannot apply with them. That depends on the materials. Till pressure, we can apply with this. We can decide which is the pressure. Till coefficient also have an influence on the time constant. Because the same of yesterday. If you have a process, you want this process will end in minutes not days so which is tau even tau is proportional to water in this hypothesis is the one that disappears and so then the tau is proportional to k absorption of the pressure to minus one so larger the pressure smaller the content cotton so acting with a large pressure means that you have a full coverage in a very short time okay this is another indication is the first indication of which is one parameter that can determine the growth of the pressure so if you are able to change the pressure you are able to do it okay.
\fig{10}{3-Surfaces.pdf}
Pressure enters into this equation into the coverage at each time and into time constant. But what does it mean pressure? Because yesterday we told for example that having a small pressure the chamber, the low pressure, is good for evite contaminants. Now I'm telling the posi, having a large pressure means a larger deposit rate. You have to distinguish, there are two different pressures. This is the pressure of the evaporant material, the material you want to evaporate. This is the partial pressure of this material. This partial pressure is typically a local pressure, because if you have an evaporator and a source and a material you have a beam of particles that they're not diffused in all the chambers they will be concentrated on the material so it will be a local pressure so this pressure that acts in this equation is the pressure of the material you want the positive the idea of you have a gas uniformly distributed with the pressure this is a macroscopic quantity equal in all the chamber is not true for this process okay by the contrary the pressure of which we spoke yesterday was the pressure for the digital gas that is a global pressure everywhere in the chamber typically this is much much much larger than this this can be 10 to minus three millibars but locally these are 10 to minus 8 times 10 take the attention is different we are speaking of this pressure and this is the pressure on which we can act we will see when we speak about the function techniques how we can act on this pressure and determine this pressure that determines okay no because this is a local pressure so it will not affect the pressure of everything You can tell. In principle, you can have the total pressure as an average, but it's not exactly 1 plus the second divided by 2. It's not the same, because this is everywhere, just by local. Local means 1 centimeter squared. Yes, much, much lighter. No, it must be higher, otherwise it cannot be positive. Yes, only to speak when you are working with the equation process. Initially you have your evaporator, and you measure maybe 10 to minus 10. Then you switch on evaporator. From calculation, we see that it's possible to calculate the local pressure. That is something that you can calculate, but not measure, because there is not a local measurement. You can measure just the global pressure. If you switch on the evaporator, and the evaporator has a local pressure of 10 to minus 3, by sure you will see in your measuring system that in 10 to minus 10, it will become 10 to minus 9. So you will see the factor. But anyway, it's not just an average A plus B divided by 2. Okay. As a matter of fact, one problem in this case is like the contamination of the chamber. For example, if you are evaporating germanium, you are sure that after many evaporations will have the chamber covered by germanium because it is true that you will concentrate evaporation on the cell on the tablet but we will see that the evaporation is concentrated but not concentrated so maybe you will have a contamination everywhere essentially the residual gas pressure will be partially made by the evaporant pressure okay it's important this is different.
\fig{11}{3-Surfaces.pdf}
This was the microscopic view means we started the what happened locally atom by atom we see that the faster to cut the surface you think electronical structure difference you see that torsion and the in the morphology, not in the morphology of the universality, but in the transopsy in particular, of atoms of the system. We spoke about a microscopic view. Why? Because we spoke about a monoreacrobe. Now we see, defined by another quantum view, that is a thermodynamic spectrum of information, what we will do is to make a theory that is very similar to the theory of water condensation so there is the water condensation the condensation from the vapor phase imagine the vapor water on a surface okay imagine in a storm when in the morning we have the the windows covered by water because of the temperature between the outside and the inside. This is called the duclation. Duclation would mean that you have a vapor, from this vapor you create a solid. So duclation means the creation of a new phase from the vapor to the solid phase. Duplication is one approximation, as before. Before we approximate the path, we are drawing just a formula layer. In this case, it's an approximation. Why? Because, in duplication, typically, think about a drop of water. A drop of water is not a film, it's a drop of water. so it's just an idea I don't understand how the go takes place and typically is what happens the first stage of the go before the wedding session nucleation means you have some elements nucleating on the other plane for the bonus up then is that there were water with wet everything the water is the nucleation not from water but from water twice because you're sorry say you can distinguish two kind of information very study for one homogeneous heterogeneous homogeneous means you have a vapor phaser and you can create a solid of the same material so you have two phases but one material vapor water you will create ice heterogeneous means you have for example vapor water you create ice but where on this table so we have two materials and two phases water, solid water, solid water, is heterogeneous. Heterogeneous is something more related to the law of the universe, because you are depositing something on a surface. Homogeneous is not, homogeneous is simply you change the phase. But in order to understand heterogeneous, you must start from homogeneous law. Okay? Homogeneous is not just a case study, it can because in magic you want to grow silicon substrate not on the silicon substrate but on silicon substrate how you can grow a silicon substrate starting from nothing so you will have a silicon vapor that will condensate starting to grow silicon so this is essentially the early stage of formation because in this case you are not depositing on a substrate you are depositing the substrate and there are some questions that we try to ask under which conditions the nuclei nuclei means the elements that form are thermodynamically stable because maybe nucleation is not thermodynamically advantageous so maybe it's better to stay in the water phase what is the determinant the energy as in thermodynamics. As in chemistry. I have a question. Which is the role of the surface energy? The surface energy is the energy to create the energy to create a surface. How large the nuclei are? They can grow or they stay very small. It's different technologically because you want to grow a silicon substrate you need that more atom added to the atom already present. The process will not stop. What is the energy barrier? Until the growth mode, flat or nothing? And, ok, the strain is another point. So, with this part of nucleation, we will in particular arrive at the result to be possible to distinguish space due to a property of the materials, which is the grow mode. That is, when one film grows on a sample flat, that is one monolayer over the other monolayer, or rough. Rough means that atoms, instead of adding one on top of the other, try to form island this lecture will give you the answer the other lesson is for the one on the beta efficiency will give you the answer on when atoms are crystalline or when atoms are developed okay after this we will move directly technology the position of technology.
\fig{12}{3-Surfaces.pdf}
We start with thermodynamics thermodynamics is just some what we will arrive today to homogeneous population and in particular this theory is also called it's a very old theory because it's a copy of the theory of capillarity of the old theory stored in place from liquid or vapor fluid particularly for vapor fluid.
\fig{13}{3-Surfaces.pdf}
First we have to define what are surface energies or surface forces. Remember that you have a solid that is nucleating from a liquid or a vapor phase. Consider vapor phase. I wrote liquid because this theory was used for liquid. By this case, you have a vapor. The position, anyway, starts from a vapor phase. A solid nucleate, a vaporite ion is a colloquial vapor phase, but it starts between A and E phase. What does it mean? In order to have a nucleation, it's necessary that your vapor phase is not stable, because if the vapor phase is stable, there is no reason for nucleation. If the temperature in this room is above the zero Celsius there is no reason for the water to condensate. Okay. So you need to have an instable vapor phase. Because in the nucleation that we create part of the molecule we produce a nucleus made of many atoms but we will not discuss how many atoms the community is. And this means that you create both a volume of solids with a given surface. Or the volume is made by atoms. The surface, the border, and in particular the surface of the interface. The surface is the interface between the solid and the vapor phase. We start to speak about interfaces. Through this interface, through the creation of this interface, you associate interfactional or surface free energy. They are synonymous. Interfaces from vapor to solid. If you stay inside the nucleus, you see the surface with respect to the vacuum. Okay. The surfaces are more energetically favorable with respect to bulk or less. That is, the system prefers to have a larger surface or a smaller surface for the same bulk, for the same volume. In other words, it's better to have this shape or this shape if the volume is the same. One. You might think about the water, the water, the water is in the top. It's also in the elongated form because of the gravity. So, the formation of surface is energetically unfolgable. And we define two quantities. The first is surface energy. Surface energy is defined as the excess energy at the surface of a material compared to the bottom. In order to create a surface, we need to give an energy, additional energy. And the surface energy is the relative work done on a material for increasing its surface area by delta E. So, with an energy you can give, you know, the increase, the surface energy. Surface energy is the surface energy is implicit for unit area, it's not one energy, identical to the other. This is the model you typically imagine in blue square. you will increase this blue area of this quantity, gamma, this variation between the walls needed in order to increase this, divided by the variation of the area. Or, if you consider that this system, how you increase the surface? By applying a force. So the wall is the force from this increase, divided by the area, the area is 2L delta x, it means 2 times the area of the first angle, 1 times because we imagine to have like this, if you increase you have the area on top and on the bottom, so increasing the area of this blue element means 2 times the area on the bottom and so your gamma is made by this area So, gamma is the surface energy per unit area needed to increase the surface of a given area. We can define another force, this is the surface tension. The surface tension is defined as the force required to move this side per unit length. So you have this force divided by 2L. Why? Because in this case, this L can be taken two times because you are increasing this part, and so this part is the force mass structure on the top and on the bottom. Generally, this parameter tells two different aspects, one is a force to influence, the second is an energy to influence area, but they are the same. If they apply to a liquid, they are exactly the same. If they apply to a solid, it's not true, because in this case you have a work to displace molecules inside the solid, it's more complicated. But in general, this theory you take every time surface energy equal to surface tension. Remember that it's wrong to tell surface energy equal to surface tension. Dimension is wrong. Simplifico. Surface energy is the area, surface tension is the area. So we use in this theory, sometimes we use the surface energy, sometimes surface tension. Sometimes surface tension, but remember that we are the same. So we will, at a given point, we will make a balance of the forces. So we see the interpretations of surface tension. When we make a balance of the energy, we use the surface. These values are dependent on the material. Altamuletics. So for example, these are the surface energies of the material.
\fig{14}{3-Surfaces.pdf}
Okay, start with homogeneous nucleation. Today we just do the homogeneous nucleation, then we leave the heterogeneous nucleation for next week. Homogeneous nucleation of which? Of what from what? Homogeneous nucleation of a spherical solid phase, why spherical? Because a sphere is the volume is the geometrical body smaller area so if the nucleation of a spherical solid phase of radius r is only parameter from a super super super saturated vapor Okay, two important. This is not the process. The process was the policy something on a sub. In this case, there is no substrate. Then inflation of a particular particle. Whether. It's interesting when you want to go, for example, a deep layer. We should play something on itself. And is a model we see that if you understand this model understanding of the heterogeneous nucleation that is more realistic to be practically immediately. What does it mean super saturation? Super saturation means that the partial pressure of the vapor of the compound be the positive is larger than its vapor pressure. Each material and its proper vapor pressure. The vapor pressure is the pressure at which vapor and solid phase are in equilibrium at all. Okay? For each temperature, there is a vapor pressure. For example, the chill the vapor pressure of vapor and water at 100 Celsius is one atmosphere. This is the vapor pressure because in this case the liquid and the vapor phase of water coexist. Here we are more interested in vapor-solid phase. ok so you stay in a situation with super saturation the vapour pressure will stay larger than the vapour pressure means condensation can take place ok you have a vapour to solidify the condensation what you are creating If a vapor will condensate, it will create a drop. This means that it can account for an energy related to the creation of this volume. So before there was no volume, now there is a volume. Delta Gb is the chemical free energy change per unit volume during the vapor phase transition. So if you create this nucleus, this will involve this chemical free energy change. While this is the volume, because chemical free energy change per unit volume multiplied by the volume, you obtain total free energy change of pure vapor due to the creation of this sphere. And this depends on the radius. For the condensation reaction from vapor to solid, this free energy is given by this expression. It depends on the temperature, on the atomic volume, and on the ratio between the vapor pressure and the partial pressure of the vapor, Td, and the vapor pressure of the compound, Px. okay is better model but because we stay in a super saturated situation is a ratio is positive for deposit larger than one so the logarithm is larger than zero and so this is negative because there is a mean minus sign so this means that the creation of a nucleus in a super saturated condition is a a systemic process it's a spontaneous process because it's super saturated so we tend to condensate okay if it's not super saturated if the same pressure that is smaller than the pressure will not condensate it will be positive and so the process will not happen so the super saturation is a condition in in order to obtain nucleation. Okay? So, when you create a nucleus, your system decreases the total energy. So it's a spontaneous process. So why there is not nucleation everywhere? Because there is something more. You are also creating a surface, not only a volume.
\fig{15}{3-Surfaces.pdf}
And surface means that you are creating, if you are creating a surface, there is an associated surface per unit of the system. It is given by gamma, remember, was the energy due to the creation of a surface per unit surface, multiplied by the surface of the system. So the total energy is in one. Then you need some of these things. The first is negative. So this means that this process is convenient, favorable. The second is positive, because you remember that you need a work in order to increase a surface. So if you want to have a sphere, negative starts from a sphere, this sphere. And you move this sphere, gain energy, in the fact that you increase the body. But you lose energy, in the fact that you increase the circle. So you have a balance. You have to understand when this balance is directed towards growing, or why this balance is directed towards decreasing of the sterol. So the question is, in which condition the process is spontaneous? When this is negative, when the fact that you reduce the energy by creating the bulk, it's more relevant that you are decreasing the energy, sorry, that you are potentially, this energy negative is larger enough to value than this one. Or when there is equilibrium. equilibrium means at a certain point we stay in a radius in which this is zero stable.
\fig{16}{3-Surfaces.pdf}
How to do we just find the third derivative equal to zero find first we find the equilibrium position The equilibrium position means which is the condition in which the energy is a maximum or a minimum. And you obtain just by doing these two values. So you can plot this cube this way. And you find that this point is a maximum, not an end. point corresponds to a radius r star and to a free energy that is g star. The g star is different from this point to this point. r equal to zero means non-nuclear, there are non-nuclear. There is non-nuclear. r larger than zero means there is a nuclear, which but even larger, this is the radius of the value. Okay, note that both these are positive. Radius is positive because there is a minus, but this is negative. That gv is negative because this three-emits variation is only the volume. While this is positive because it is a square. What is that history that represents the energy barrier in the nucleation process to be able to compose nucleation to start with the limiter? Imagine to have a nucleus that accidentally falls. Accidentally because you have more than one atom that impacts and that is untouched by some force. So imagine you start with a nucleus with a given radius, maybe a radius here smaller than L star. What happens? That the system will tend to go in the direction minimizing its energy. The direction minimizing its energy, this is one, it will start to destroy, because it will minimize its energy by reducing the volume. because in this case the term used the surface is more important than the body so if a nucleus tags with a radius smaller than f-thugger it will not be able to inflate okay you will kill a solid but the contrary if the nucleus has a radius larger than a star in this case the trend the direction will be to increase the radius because increase the radius is enabled to decrease the free energy so the point is everything depends on which is the initial volume the initial radius of the nucleus clearly how you can understand the initial radio frequency in the form when many atoms will interlock randomly if this value is very large and start very large it is unlikely that such enough large atom will nucleate and enough large cluster will nucleate so this process in this case, the nucleation will not happen. You are not able to make the nucleation. By the contrary, if this value XR is very small, every nucleus that will form randomly will be able to grow. So the point is how to decrease this XR. That's the idea.
\fig{17}{3-Surfaces.pdf}
But it also depends on these three parameters. Gamma and delta Gb. Look in particular to delta Gb. If you have a delta Gb, why look in particular to delta Gb? Because gamma is fixed. Once you have a material, gamma is fixed. You cannot play with gamma. Because gamma is an energy greater than material. material but you can play with delta gb because delta gb depends on the ratio between the pressure now the vapor pressure at a given temperature is six years but you can act on the pressure of the vapor that decides okay because remember that the uh v the pressure the buffer, the pressure, you can decide by adding gas to the chamber or not. If you increase PV for a 50S, you increase the absolute value of this. If you increase this in absolute value, you reduce the S-price. So if you increase pressure in your chamber once more, it is an advantage, because in scale you reduce the critical radius and so smaller nuclei start if you start with smaller nuclei produced by randomly they will start to grow and will remain stable for growth Moreover, also this is smaller. So, in general, work at larger pressure is an advantage both if you look by the microscopic point of view for the coverage, also if you look by homogenous migration. So, increase the pressure as much as possible is an advantage. It's possible to calculate, you can find on the book, I suggest you to want to look, that also you can calculate the formation rate of nuclei and the formation rate increases. In other terms, if you want the rate means how many nucleus are created per second, okay, this depends on the pressure. If you increase the pressure, you process typically 30 degrees. How to increase the pressure? One way is if you have imagine you have this chamber full of water, full of bubbles, when you increase the pressure, you increase the pressure in the chamber, so you put more water in the same volume. Or we see that you have the position technique, in this case the local pressure can be increased typically by increasing the temperature of the soil.
