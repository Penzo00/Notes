\chapter{Thermal Evaporation: Layout, Vapor Pressure, Evaporation Rate, Point Source}
\fig{3}{5-PVD.pdf}
Okay, so from today we move to the most interesting part in the course, which is the positional. We speak about position techniques and how you can deposit different kinds of tools. So, with respect to the other parts, in this case we make examples, we see application more directly. First of all, what is a deposition technique? Deposition technique means to comfortably transfer atoms from a source to a target. Source is the origin of the atoms. Target is where you want to put the atoms, the target. You want to transfer atoms from the source to the target. What does it mean? Contrarily, it wants to decide how many atoms, for example, per unit time, per second, or which is the angular expansion of the atom. All the atoms must arrive perpendicularly or doesn't matter. It's this idea. In this sentence, there is all the deposition. The question is how to deliver. So you understood what you have to do. You have understood how you can do. And you have two ways. The one is physical wave, and it's called physical vapor deposition. Why vapor? Because the way to transfer atoms from the source to the target is anyway putting them in a vapor phase. The reason for which I explain you what is ethanucleation and homonucleation. It is anyway the condensation of a vapor phase. So you start from a source that is solid or liquid. And then you arrive to deposit a film that is solid, of course. From here to here, you have vapor. So the source material is physically transferred from a condensed space evaporator to the target circle. What you have to do is to find a way in order to make atoms to be removed from this source, solid or liquid, liquid, put in the vapor phase, moved to all the substrates, and then deposited. It is a physical way because you physically move atoms, or physically you impart kinetic energy to atoms. The second way is the chemical vapor deposition. In this case, what you use is a volatile compound of the material to be deposited, chemically to produce a non-volatile solid, just graphically. In this case, you want to deposit, I don't know, silicon, so you start from silicon, you sell silicon, and you have silicon. In this case, you start with some compounds of a silicon that stays in the vapor phase, that's volatile. This will react with some other compound, for example, I don't know, in one deposit, for example, a gas of silicon, a vapor with a silicon, and another vapor, second vapor, when the two react, they will produce silicon solid and another gas. So, the gas that is produced will be waste, and the silicon will be positive. different in vector here the source is solid or liquid and you have to create a bubble starting from the source in this case the source is the vapor itself they are completely different we start with this and then we move to this here chemical reaction are involved here is just physical transfer there are also to some technique that has like a mix between the two but in general start with the two techniques separately these and these.
\fig{4}{5-PVD.pdf}
Choose the objective of the deposition technique this is a slide maybe i already showed the first lecture you have two objectives obviously the first is a teleprompter deposition because you have to start to use an interstitial Then we have the structure that is not part of the simple made by sex. Which is the task in this case? If you get high purity of the material, you want the quality of very pure material. Because maybe you have to exploit the transport property, the material must be pure. thickness i already explained this if you want to build a tunnel barrier or in general a layer you need a very precise signal uniformity means the layer must be uniform both by the point of view of the chemistry if it's silicon dioxide must be silicon dioxide everywhere not silicon oxide or dioxide and also uniformly means also by the point of view of the thickness Multi-layer doesn't mean that typically devices are not based by just one layer iron on MGO, but yes, a multi-layer, many layers. For example, if you want to build a magnetic tunnel junction, that is a system made by simply a magnetic material, an insulator with a magnetic material, and you want to build these on silicon, you need 12 layers. Why? Because you have, for example, to match the piece of steel, so you need many layers to match the piece of steel. Or you have other requirements. So in general you need multi layers. So you have to deposit many materials, one after the other, with a good uniformity, a controlled thickness that can range from few monolayers to microns, and high-publishing. This is the first task. The second task is during the water potations. Because I told you that you can have during the potations what they call additive processing. With the mask you deposit the resists on some plate, then you add material on everything, Then you remove the resist by chemical acting. This way you leave only certain regions of the material covered by water removal. For example, insulating layers. In order to insulate, imagine that this is insulating layer. It serves to insulate this part from this part, for example. Or to make electrodes. Electrodes are placed where you have to weld a wire. so you need the electrodes that are very thick but in this case you have different problems because in this case you have you need large area and no thickness well once you have a good instrument you need not more layers but microns all the same for later the purity thickness controller is not crucial why because it's enough the gold is conductive or silicon dioxide is insulating. It doesn't matter if you have some impurities if the electrical properties are correct. So you are not supposed to stay with these properties. Typically it's not a single layer. A single layer can be silicon dioxide, it's a typical insulator. If you want to make it better, you can use a layer of silicon dioxide and aluminum oxide. The same for the later with the latest made by gold or chromium gold. So the color completely different. In this case, you need maybe many evaporators in order to use many layers. In this case, just one or two are enough. In this case, you have to be sensitive to the single layers. In this case, it doesn't matter. But typically in this case, you are maybe depositing on a single device. Here you need a larger area. Why larger area? Because in this case, the test is ready, you can deposit the larger area, but due to lithography you can choose exactly where the material will attach, the other part will be removed from the area. But this part will be taken over by Daniele later. So these are the two main kinds of deposition, the main application of deposition. We will see now in detail, with our technique, which technique we can use for the first, which we can use for the second, or both.
\fig{5}{5-PVD.pdf}
Okay, now we focus from the first lecture on the physical rapid deposition technique. The question is, which is the physical mechanism for transferring material from the source the target through a vapor phase so essentially which is the technique that allows you to transform the solid or liquid source into a vapor you have two classes the first is evaporation evaporation is simply you heat the source you will start evaporating or sublimate sublimate from the solid state evaporate from the liquid So evaporation caused by absorption of thermal energy from liquid or solid sources. The second is sputtering by bombarding solid sources with energetic elements. We will enter in the case, but in this moment I just want to make you the classification. What does it mean? In the first case, you have your source, you heat the source like heating water, and we start to evaporate. So you just need electrical, sorry, you just need a way to heat. In the second case what you have, you have a gas of inert atoms like argon. They are accelerated, they bombard your source and by scattering they make the atom of the source to be ejected. In this case heating is not needed. You need just to have another gas bombard. So here you have to keep your sample, use the source to give a temperature. In this case, the sample can be used for temperature, but you need a gas to bombard your surface, you know, your projection. We will see about each kind of technique, which has advantages, disadvantages, why you can use the first or second. So at the first you have other classifications, the teaser. Which is the way to provide thermal energy to those atoms? Because remember that everything is done in vacuum. They have electrical heating, like just a Joule test. Electron beam, you can turn an electron beam to keep locally. You can use a laser that can be focused on the other thermal. So this is just the classification. We will start today from this part, and in particular we will start to study in general what does it mean evaporation, which is the many parameters that determine evaporation. Then we will move next week to which is the mechanism, we will see the different deposition techniques by term of evaporation, then sputtering. In all cases, the emitted atoms traverse high vacuum regions before the deposit on the target. So, the reason for this is explained in the first lecture. Why you need vacuum? To reduce contamination, first. Second, to allow the particles that are emitted from near to arrive there in a straight line. avoid this pattern. So this is the region for which we need a pattern. Here I indicated the different lectures in which, for example, we will learn.
\fig{6}{5-PVD.pdf}
So, let's start with terminal operations. I start with terminal operations because it's the easiest to understand. It's also the easiest to make a model. We see that it's quite easy to make a model of terminal operations. so easy with sparkling chemical deposition each technique each different way of making that chemical deposition has its properties this is the by basis system used for the position process by terminal operation the material is your source the substrate is your target what you have You have a vacuum system. This is a vacuum chamber. It's a vacuum, so you need a pumping system. This is not reported here, but it's written. Vacuum chambers or somewhere they will pump. There's a pressure monitor. A monitor which is the pressure inside. So what you have to do? You heat your material. How you heat material, we will see in detail how you can heat your material, but just think about heating by dual steps. Here we have voltage generator, we create a current by dual steps, we heat at a given temperature, we start evaporating, like electrical heating for cooking. Then we start evaporating and we move to all the tasks. Now the point is how you can measure which is the position rate. a key parameter why because it wants the positive but know how long it will take to have these sources withdrawn we will see today how this positive calculator is the producer based by physical but in general what we calculate is just an indication not precise if you need some precision you need a measurement direct measurement so you switch on the source you need some instrument with measure which is the position of a typically it measures how which is the thickness deposited on a given surface in a unit size this is the crystal molecule we see the crystal is a system that is placed here and measure how much material is the point okay so in this way you can measure in real time or before to switch on or after switch off which is the position okay this is a base system of operation all the system operations have this feature the source the target and the crystal molecule. This is the evaporation cell, the thermal evaporator. The material is inside this tube. Here, through the current connector, moving here, we start evaporating. And we exit from this hole. If you look at the tail, here, this hole is covered by this material. This is called a shutter. Why? Because when you switch on the source, it takes some time to go at a distance, to go to the transceiver. But if you have your software exposed to the transceiver, you will know what is the positive in the transceiver. So all the sources are being shuttered, it is just covered. You switch on the source, the source starts to evaporate, but the shutter will block the evaporation that will not arrive on the shutter. When the source is at regime, you can open the shutter and then the material starts to evaporate. In the system more advanced, there is mechanics. You just measure on a monitor the thickness rate, when you see the thickness rate is a constant, just switch on the shutter mechanically and maybe you test the computer how long you have to take the shutter on so the source will evaporate then we close okay it's a better way to precise instead of reaching on and on the evaporator because we take time to switch on into the shop. Okay, so you can see here the tube, the charger, here there is a connector and there are other connectors for the water because Joule effect means heating. Heating means everything will be in a temperature, but you want to have a temperature only the source. water in order to cool down the system apart from the water okay this is a general description of a system then can have difference on the way you heat this maybe it's not true effect maybe it's laser maybe it's a little beam but anyway this is a generic system general system The material is a liquid or solid source. We will see when it is liquid and when it is solid.
\fig{7}{5-PVD.pdf}
This is a hypothesis that we have to look at in this lecture. That is the thing we start today and we finish on Tuesday. First we will say how to understand which is the evaporation rate for the source. How many atoms per unit time per unit surface are evaporating from the surface? And this is related to a key parameter that is evaporation. Second, we see this positional geometry because evaporation is not deposition. If you evaporate, I don't know, 10 to 20 atoms per second, will not deposit 10 to 20 atoms per second. Why? Because such atoms will be lost during the movement from the source to the target. Not because of scattering, because of the same vacuum, but simply because of the geometry, because this atom beam will not pop to the target, but will go to wherever. So, the position geometry will determine the position rate of the target, And also we see the field thickness repository. So this is a key parameter in order to understand how uniform the EU film, uniform means how the thickness of the film is constant. And we choose the deposition rate that depends on the evaporation rate and the deposition geometry. Together, the defense defines the deposition rate. That is what you need, because the information you need. Third, we'll take a look on the field theory. There are the three parameters that the fundamental will consider. And there are three parameters in general for any evaporation technique. So we just discussed them with reference to the thermal evaporation, but there are equal for any evaporation technique. Okay, start with a little bit of theory.
\fig{8}{5-PVD.pdf}
Daniela told me that if you use it, it sends you the value. It's a planche. Plunge time. Okay. Evaporation rate. This is a theory of 1882. So it's a very old theory. The test is that if a liquid has a specific ability to evaporate at a given temperature , think for example water. Water evaporates at 100 degrees. The evaporation rate is proportional to the net pressure above the liquid. It's something we already found. Which is the area, remember that when we talk about the equation, by the microscopic view of the surface flow, we discuss the difference between the ambient pressure and the pressure of the vapor material. Ambient pressure is the pressure of the contaminants inside the chamber. The vapor pressure, the pressure of the vapor material is the pressure just close to the source. This is the net pressure above the liquid, this pressure close to the source, that is strictly related to the evaporation of the material. This net pressure is the difference between the equilibrium pressure, the vapor pressure of the evaporant, and the transplating pressure acting on it. The transplating pressure is the ambient pressure, the pressure due to contact. The vapor pressure of a material in general, in general definition, is the pressure at which the vapor and liquid phases are in equilibrium and coexist. For water at 100 degrees Celsius, the vapor pressure is one atmosphere, because at one atmosphere the water, the water and vapor are in equilibrium at 100 degrees Celsius. As you know, the pressure in the mountain, the water boils at a temperature smaller than 100 degrees, because the pressure in the mountain is smaller. So there is a correlation between the vapor pressure and the temperature. Every material has this vapor pressure, depending on temperature. This is the formula we found in the first lesson. The flux, the evaporation flux for molecules in the square root system, that essentially is the same of the gas-in-fever flux equation that we found in the second lesson. It is the same. This evaporation that was not evaporation but absorption. But it's the same equation. Which is the difference with respect to what we calculated before? The difference was that here, what is important is the difference between the vapor pressure and the hydrostatic pressure. While here, there was just one pressure. And the second is, we have this coefficient before. What does it mean, this coefficient? This coefficient is called the coefficient of evaporation. What does it mean? Not all atoms that can evaporate will evaporate, because maybe one atom that will start to evaporate will be retracted by your source. So, this is a coefficient from 0 to 1 that tells the fraction of particles that effectively evaporate with respect to those that potentially can leave.
\fig{9}{5-PVD.pdf}
So the maximum evaporation rate is obtained when? when this coefficient is 1 and the pressure is 0 if you think if you look at this difference the evaporation pressure minus the contrast pressure pressure is something about 10 to minus 8, 10 to minus 10. In i-vacuum, ultra-i-vacuum. Vapour pressure, we see that it's something around 10 to minus 3, 10 to minus 5. This difference is practically the vapor pressure. So, this difference doesn't matter. in fact if you say neutral gas, if you have to make, I don't know, 10 to minus 6 minus 10 to minus 10, it is 10 to minus 6. So in general, in the formula before, this approximation is not a fixed approximation, it is what happens in practice. So you can find what? You can find two parameters. So, first one, the first is the flux, molecules or particles for cm2s. And the second is grams for cm2s. There are the evaporation rates. Remember that we are speaking about evaporation from the source. For the moment there is no target. It doesn't matter if there is a target or not. It's just evaporation. I stress this point in order to avoid confusion. Okay, the evaporation rate depends on which parameters. The mass, but the mass is constant, it depends on what you have to, you want to deposit, you evaporate, sorry, to evaporate. It depends on the temperature and depends on the pressure. So you have two parameters, pressure and temperature. But in practice it is one parameter, because pressure and temperature are related. Think for example water. Water, the vapor pressure relates to the temperature. So at the end it is one parameter.
\fig{10}{5-PVD.pdf}
Pay attention, the dependence on the temperature is not so strong, because it stays in the denominator with square root 2. The strong pressure is on the pressure. song story is strong influence is on the pressure because the flux is a particular this is a typical class of capa on equation of thermodynamics that allows to understand essentially which is the pressure is a partial term okay this is obtained at thermodynamics i don't demonstrate this i don't remember the demonstration honestly this is the equation for solid vapor and liquid bubbles it is this p is the vapor pressure and p is the corresponding and delta h is the difference between the vapor and condensed space in terms of entities, while V is the difference between the volume of the vapor and the condensed space. This equation works for solid and integer. We speak about condensed, but here it is the same. The two phases are in equilibrium, coexisting. Along these lines, it can be calculated with these equations, using these values for solid vapor liquid vapor or liquid solid you can find this along this user there is an equilibrium for example here there is a condition between liquid and water if you increase the pressure the temperature increases is what i told you for water if you decrease the pressure because you go in the mountain the equilibrium temperature decreases the water boils at a lower point We can make some approximations. The first is, in general, the volume of the vapor is much larger than the volume of the condenser space. Vapor is everywhere, condenser space is just related to a film or a pot of water. So in general this difference is equivalent to the volume of the vapor. You can take the vapor as a perfect gas. It's an approximation, but you can take it. And use it as a perfect gas flow. This variation of entity in general depends on temperature. But we can simplify and say that this variation of entity is just the molar heat or evaporation. It does not depend on temperature, it's just a constant. With this approximation in mind, we can solve this equation, because that equation becomes this equation, where instead of... here we have p divided by rp, because of the ideal gas flow on the volume, and this is a constant. This is just a differential equation, you can solve it, and you can obtain this result. So this is the pressure in logarithmic scale as a partial temperature. It is different from depth, because depth is linear scale instead of logarithmic scale. So typically we use logarithmic scale. So we can see that the pressure depends on the temperature, how? drastically because if you increase the temperature of 100 Celsius, you have an increase of the pressure of about 100 human degrees. So, 100 Celsius means 10 in terms of pressure. So the effect of the temperature of the pressure is very very relevant. It's not linear, it's exponential. It's a tiponical.
\fig{11}{5-PVD.pdf}
This equation is not so correct in general. It's possible to correct, but the trend is the same. anyway what we will use in our our water is not this ideal flow but it's a real law someone measured this cubes for the different materials each one of these lines corresponds to one element one chemical element or one molecule that evaporates as a molecule Here there is a temperature, the very first scale, in Kelvin. Here is in Celsius. Here the vapor pressure in Thor, the very first scale. This is the vapor pressure in atmosphere. You can see the shape of this fluid. But it works. There are experimentals. So in general you have to take these. This can be found on internet where you want, you can just check the pressure of the elements and you can find this. There are two of these, because you cannot record all the materials here. There are two of these, one with some material, the other with some other material. In the examination, we will provide you with the paper view. Ok, how to use this? this is the experimental vapor pressure that is along this line you have the equilibrium between the vapor and the pressure solid the powerful and the condensed phase of which material condenser phase can be liquid or can be solid these dots along the line this is the number of elements This small dot, a point in size, has the, called the melting point. Means that the material along this line, for example this line is chloride. Above this point, the salt is liquid. Below this point, the salt is solid. like water. If it is water, here the height is removed. So, if you say, for example, if you are evaporating from this point, you are evaporating from a solid source. If you are evaporating from this point, you are evaporating from a liquid source. This horizontal line here, dashed line, Correspond to atmospheric pressure. This is the boiling point at atmospheric pressure, the boiling temperature. If there is water, this temperature would be 100 Celsius. This is exactly the shape of the cube before. Before my cube was low pressure, temperature linear. Here I put the same cube, low pressure, low temperature, and you can see that the shape is the same. It's not a straight line, but it's just a monochrome.
\fig{12}{5-PVD.pdf}
How you can use them? First, here you can find the melting point of different elements. is the temperature at which, above which the material becomes liquid. For example, imagine to want to deposit iron, no, sorry, deposit nothing. Imagine to have iron. In the iron, the melting point is here, below this red dot, dot there was a dot of the making point one hundred eight one thousand eight hundred fifty kelvin at the pressure of three minus two dollars why the morning point is one if you see for example gold the gold meeting point is 500 celsius smaller easier to evaporate gold than iron. But the boiling point is not so important, but in general the point is the melting not the boiling point, because you are not evaporating from the second pressure.
\fig{13}{5-PVD.pdf}
The question is, which is the general criteria for discerning between evaporation from a solid or a liquid? Why this is important? important because you know that all materials apart from mercury are in solid state there is no material that naturally stay in liquid state so if you buy any material gold it is solid state what to do then you put this material inside what they call the crucible like that to put your material in a solid state i don't know in some form of powder or so on then you eat for example by electrical heating then you need that evaporator when directly from the solid state of first medium to melt become liquid and then evaporate. Technologically different because in one case you need a vessel containing a solid, in other case a vessel containing a liquid. If it's a vessel containing a solid, maybe you don't use a vessel, you just use a rotiscosa cylinder, a rod of material, with a system to take it in place and then use directly evaporate. If you expect to melt, you need something to contain it. It's a technological problem, not physical, because physically it doesn't matter if it's liquid or solid, but technologically you have to know exactly what will happen. What is the general criterion for reserving between evaporation from a solid and a liquid? It's just a criterion. The question is, which is the vapor pressure typically needed to evaporate with a reasonable What does that mean a reasonable rate? A rate that allows you to make a micron in one hour, not a micron in one second or a micron in one year. It's just an indication. The answer is 10 to minus 3 to the power. It's a typical pressure. The rule of thumb is a melting is required if the pressure at the melting point is low point is lower than about 10 to my control this is just a first indication so you have a material you just look the melting point where is that below 10 to minus 3 tor in the green region or above in the red region this gives an indication of probably which will be the melting state For example, iron, iron emitting points, yes. So why? Because the temperature of metal is larger. So probably not evaporate. We don't need to arrive to such large pressure. Gold is different. Gold is here. It's very probable that gold will evaporate. you don't need to arrive to such large pressures most at one school gold is different gold is here it's very probable that gold will evaporate from the liquid state so we'll hit first it will melt then it will evaporate okay this is just a general rule but case by case we will see what happens typically all the metals like uh metal like iron chrome titanium or israel nickel silicon we start from solid station you see the gold germanium sodium cesium starts to the metal state so for example the silicon germanium are two semiconductors very similar but one starts from solid the other from metal.
\fig{14}{5-PVD.pdf}
I show you three examples if maybe you took a look about the examination you can see that some of the exercises proposed some of the exercises proposed in the examination are problem of this kind is calculated, the operation rate calculated, time needed, etc when there are studies exercise clearly I give you on the back of the paper so the okay who is there is a delay from what I do example iron what you have to do that is not the question but the question there is even an evaporation rate find the not given a depression find a pollution rate anyway which is the pursuit procedure when i ask you to study in general at evaporation process you have to take this out you have looked at the elements I asked for example this case iron this is the cube of iron I just highlighted this is the experimental vapor pressure of the iron these are the flux on atom centimeter square seconds or gram centimeter square seconds of the iron I simply put instead of the mass M the formula is n okay so these are the fluxes depending on water on the temperature on the vapor pressure okay note that the influence of the temperature is mainly via the pressure and not the temperature because changing samples of 100 cells here over a no from 1200 1300 not a large effect in this burner but then we have a larger spectrum so the temperature dependence is essentially restricted the pressure so these are question which are the operation rate corresponding to a vapor pressure of 10 to minus 3 torus let's take the line corresponding to 10 to minus 3 find the interceptors and you find the pressure Oh sorry, the pressure, the temperature is 100 1600. okay note that this chart seems here not so precise because not so the scan was not so perfect but it exactly what we find in internet so in general it's impossible to calculate this value to fall out of the figure you can just calculate the value of the precision of some tens of degrees so in general when the examination maybe all of you have the same result clearly you cannot answer me 1400 but if you ask me 1600 1650 that is famous okay because this is not precise at okay so we have a question a component to this upper question I do just take the population line with these two numbers in mind to calculate it plus atoms in your square second there are the population I don't know what happened here.
\fig{15}{5-PVD.pdf}
Imagine that there was just before. Okay. Okay, now this was about evaporation, not deposition. But imagine that all evaporated atoms will deposit on the substrate or on a film, which is deposition rate in tercobonolipid type. To assume that there is no problem about geometry, each evaporated atom will deposit, one to one. can calculate which is the rate in terms of monolayer per second of the position of these atoms. The same formula we found when we speak about bars. You have divided flux from the density of atoms of centimeters squared for monolayer. Remember that this 10 to 15 was an average value typical of every surface. If you want to be more precise you don't think this, you don't consider this, but you consider the real value for the iron atom, not for this one. In the same condition, which is the time needed for one minor cover coverage, if you reverse this, it is how many monolayers are deposited per second. If you reverse, how many seconds are needed for depositing one monolayer. This element is fundamental because if you know the series of one monolayer, you can understand which is the time needed for the positive and negative signal. Example, which is the position rate and time for one monolayer at this pressure that are exactly the value of the previous values. 8.5 milliseconds at this temperature. So, you can imagine that, imagine you want to deposit a layer of iron made by thermal layers. It is not a good temperature, because we need 85 milliseconds. So you need a new shutter of your source to open and close in 85 milliseconds. But if you want the body, for example, I don't know, one micron, one micron is more or less 6,000 monolayers. It can be right in this case. Six times monolayer is many seconds. This is just a calculation, I told you by mind. If you know the interlayer spacing of the two layers separated by this distance, you can calculate the rate in terms of nanometers of the four layers. This is nanometer per second, this is the time for one nanometer. But you can also change in this case. Imagine that I want to go slowly. To go slowly, I decrease the temperature. If I just decrease the temperature of one other magnitude, sorry, not of magnitude, I decrease the temperature of 100 Celsius, okay, I will reduce the pressure of one other magnitude. So the rate will decrease of one order of magnitude and so the time will decrease of one order of magnitude. So the strong influence on the temperature is not here, but in the pressure. There is some influence here, but the passage is not exactly, there is not a rate of same here, but it is a small variation.
\fig{16}{5-PVD.pdf}
The second example was not exactly the same, but taken from one of the examinations of the last year. That gave a lot of problems to students. Why? Because in this case, I asked the source temperature, and which is the state of the source, needed to obtain an evaporation rate of which are given values. In this case, I do not provide the pressure, I provide the evaporation rate, so I don't provide the parameter, but there is something. The formula you have to use is the second one. So you know this value, you have to find temperance and pressure. You know that the temperance and pressure are related by the degree of energy. That is the term for the temperature. I define them. What I want, how I can find them if I don't know the relation, analytic relation between them. Remember what I told you just before, that the influence of the temperature is mainly through the vapor pressure and not the temperature. So what can do? fever is here is the blue cube the start for the night I 850 Kelvin and stops here is a more less power so they did a so we can do it let's assume one temperature for example 1000 100 and calculated for reporting the operation pressure which will be the parameter will determine the evaporation rate as i told you the temperature in this server is not so relevant because taking 1100 100 1200 1300 here there is not a lot of difference between the years so you start with a temperature just a test temperature chosen somewhere here not maybe not here not here but somewhere then you calculate the potential responding pressure and you evaluate the rate The rate is too large, because you could have 10 to minus 5, 10 to minus 4. So what you do with the rate, in order to increase, reduce the rate, you can, for example, first you start with 1,100, it's too small. So you increase the temperature. You found the impression, you calculate the new rate. In this case it's too large. You can go to the middle point, 1200. And there you've seen a result that is closer to what you wanted. Not difficult, but it was a nice thing for many of your colleagues. So just start with one temperature, start with your calculator, by iterating the problem. This is not correct because it's not 2.2. But in order to be more precise, you should need a chart that is more precise. So its approximation, you can have, I can accept, 10\% is right approximation. So this is the result. okay not to leave we are just with a razor who won't the iteration no more than you so a sort it is at 100 to 1000 for the cabin they require the operation later the source will be a solid state because these points I don't know trust me this point is here the line here is below the point of metering this point is not in a solid state this point would be met so you have closed the mounting body.
\fig{17}{5-PVD.pdf}
Here I propose you some exercises. The solutions are in the last slide. There are typical questions of declinations. but I think Daniela uploaded all the information from the field on the robot.
\fig{18}{5-PVD.pdf}
Third example, in this case I want to move this bar to close it and, ok in this case you have a different system because before we started the position of single element I'm even here in this exercise there is copper magnesium and germanium and molybdenum There are single elements. Single elements. In this case it's different. In this case we have an alloy made by barium, titanium and oxygen. Barium and titanium are metal. Oxygen is in a natural state but gas. So, the idea is we want to deposit these materials with a corresponding geometry. We want to deposit barium T of 3 not barium T of 2. material is a fabric you know what is a fabric just to tell you that these are not simply just example without a position but he is a certain size and typical size is that if you move the parameters, the material stays humanized. You can change the humanization direction by ... ... ... ... very fast no okay this is the one after great is a material in which you have polarization versus external signal if I need to read Then you can remove the external filter and the material stays for a while. Now we try to apply to the computer. This computer is new. It's a new computer. Maybe we can avoid animation. I can show you the PDF version so that maybe PDF can work the room is a nightmare my computer is like mature okay so I'll try to And we start at the same time explaining what is a serrated just for you to know. Just a minute, I will start the computer. Okay, what is the ferroelectric? For what is the use of the ferroelectric? Look at ferromagnet. Ferromagnet is a material that can be magnetized permanently. So you can magnetize by applying the ferrocele. This magnetizer, where? Imagine to apply the ferrocele remover. If the field is positive, it will come back in this position. If the field is negative, it will come back in this position. So, for example, this is not cold, this is not cold. so you can monetize the magnetic layer for example this with north south or the opposite what is this is a bit of information because depending the part that you have on the top for example a north or south is an information a binary information many many memories are based on this technology. It's a magnetic memory. The memorization of the bit is memorized in the magnetic state of this MRAM. There are there are many applications. Which is the problem of MRAMs? How you can write a magnetic state? Or you can use an external magnet, but it's not good for a memory that is inside here or you can use the biosavart effect is a wire if it occurs it produces magnetic field depending on the direction of the current will change the direction but a current is consuming energy about the other side so a magnetic memory in order to be is a continuous memory, power consuming memory. And, and that is the energy drive. And one problem of our technology is that many energy that is produced is used for the formative technology. I think in 10 years 30\%, 30\% of the energy produced by power plants will be used for information technology, not for heating room, not for for information technology, because information knowledge is extremely demanding. If you make a question to such GPP, you consume the same energy that you use to drive a car. Okay, so the idea is try to use some other kind of memories that consume less power. For example, in the case of pergoletic memory. Why pergoletic memory consumes less power in variety? Because in order to write, we just need an energy field. But the energy field can be applied without current, with just a static voltage. We apply a static voltage across a valentine's layer, plus, minus, minus, plus, and we reverse the correlation of the energy, without consuming current because it's an insulating insulating will flow currently so theoretical memories are one opportunity in order to not solve but try to reduce the problem okay okay with this in mind we try once more but now it connects Thank you. Thank you. point right, we start just one bug in time, right? then what is the the wrong file How to be able to finish this work? Okay so, what is bargain tightening? You know how to deposit them. Which is the problem? The problem is that you have deposited half an oil, made by different oil. Each one with its proper mapo pressure. But not only. Oxygen is a gas. So we will be depositing in a different way than other materials. So what you can imagine for that parameter oxygen, imagine to deposit, to evaporate a source made by an alloy of barium titanium, in order to replicate barium titanium in your target. Is it possible physically? because in order to obtain a given rate you can see that you need a very different uh bubble pressure okay corresponding to a very different temperature what does it mean sorry the different pressure is not so different but the temperature is very What does it mean? Imagine you have your source made by your alloy. You heat your source at the same temperature. You cannot heat at two different temperatures. But if you heat at the same temperature, you will obtain very different rays. So in your source, barium and titanium will evaporate with different rays, and so it's impossible to obtain the same spectrometry. typically barium evaporates at a temperature much smaller than this. It's easier to evaporate. If you evaporate barium titanium, practically you obtain a film of barium, not titanium. So this means that it's impossible to use, in this case, titanium evaporation from a single alloy contained to replicate the security. What you can do? You can do two different sources. Instead of using a single source of barium titanium, you use a source of barium, a source of titanium. You eat each one at the proper temperature needed to stay the rate you want. That must be equal. So you can replicate the geometry. okay it's more complicated because you need not a single source but two sources this might be used in parallel with the power per template you know stay in order to stay the same rate what concern oxygen there is an additional problem oxygen is a gas not evaporated by a source so you what you can do is insert gas with a valve inside your chamber so you will evaporate volume plus titanium to process in atmosphere of oxygen atmosphere does not mean one atmosphere means in a partial pressure of oxygen so inside your chamber we have the partial pressure of volume titanium but also the partial pressure of oxygen that is released by a bottle of oxygen so the evaporation this case will not be a vacuum because we have a given pressure of oxygen the evaporation alloy is more complicated.
\fig{19}{5-PVD.pdf}
This is explained in a more theoretical point of view because compounds can have different evapopressure and you can also have dissociation reactions during evaporation in some cases you need multiple sources in order to evaporate alloy For example, evaporation with dissociation, what does it mean? It tries to evaporate, I don't know, silicon dioxide from a source of single dioxide, it will dissociate into silicon oxide plus oxygen. So if you start evaporating from a source of silicon dioxide, you will obtain silicon but silicon oxide is not a good insulator. So it's not useful. What you can do? You can do the position, for example, of silicon in a partial pressure of oxygen. This way you can. Remember that these are thermal operations. Pattern will be somewhat different. Or you can have different composition. The composition, for example, you have gallium arsenide, which is a most used semiconductor, in which I think that arsenic the arsenic starts to evaporate. If you start from a gallium arsenide source, you will deposit gallium to participate in the process. In other cases, for example, silicon oxide, not dioxide, silicon oxide, or manganese oxide, in this case you are lucky, because the bonds between the atoms are so large that you can evaporate as a wall. If you start from a source of magnesium oxide, it can evaporate magnesium oxide, because there is no nitroside, but it's just some bits. In general, you need some bits in order to evaporate. Let me show you this again. Okay. Okay. This was for what concerns the evaporation. I told you that, I told you that it has to be up to the point.
\fig{20}{5-PVD.pdf}
Evaporation. How many atoms are evaporated from the surface? position geometry which is the geometry from the source to the target and last what about the future now we start just two words about the geometry this is the second step i told you that you don't calculate the position rate how many atoms are deposited you need to know the evaporation rate and the position this is the second element the push geometry answers to the question what is the mass of unit area that can be deposited on a surface film by a given source by a given source means by a source with a given evaporation The film uniformity, related to this, there is also the film uniformity. That is, we see that geometry will determine not only the rate, but also the uniformity. And we see that it's made in an opposite way. If you increase the rate, that is uniformity. To increase the uniformity, that is the rate. So we have to decide what we need. Here are two examples of sources. The first is a classical example made in all, everywhere we have a point source of an energy field of 93 degrees. This is a point source, imagine a point source that evaporates in all directions with isotropy, and a green circle, the euclid. this sample is not perpendicular to the radius but is the infinitive of an angle theta with respect to the radius okay? theta equal to zero means the sample is perpendicular to the radius the second source is a Moriarty source, it is a surface source, you can imagine that this is the source the pagina given area not point but area this is the sample that can be placed just on top parallel or on top not parallel or off from the perpendicular so in this case the time triangle the one is angle is the same of the other the second is angle between the normal surface source and the line connecting the surface source and the unit. Then we start with point source. The surface source will be a tension with the point source integrated over the medium.
\fig{21}{5-PVD.pdf}
We start with point source. Point source means that the evaporation particle originates from an infinitesimally small region of the point source. The source is a pointer of the point. You can assume that the source is not just a pointer, but a small sphere. A small sphere is anyway not isotropic, but smaller with respect to this region. So you can define the infinitesimal ball region on source and uniform mass evaporation rate from the region. That is, evaporation rate from this region is mole to square second. We define, we do not consider a point but a sphere in order to use this parameter, that is, square second. This aspect is a distance r from the from the polysorbate. The total evaporating mass is the integral of the mass of the evaporator weight, gram per centimeter square second, integrated over the source and over the event agent, from zero to zero. It is the total mass evaporating mass. This mass is the policy on the surface, on the substrate, where if imagine that this is my point source, this is my point source, this is my substrate, the beam will arrive on the substrate this way. So the beam will see this area of my smartphone. But if I make the smart phone this way, the area seen by the beam would be the projection of this plane. As a matter of fact, if I take the smart phone this way, I will deposit nothing, because the beam will arrive this way, on this space will be nothing. So this means that the affected area seen by the beam is the projection of the area of the surface on this plane perpendicular to the surface. So this area dAp that is equal to dAs, that is the infinitesimal area of the substrate, for the continuous angle takes into account the potential. This following proportion is applied. Assuming that all evaporating masses in the full deposit All the atoms evaporating that come to the surface are deposited. This is the ratio between the essential areas. The mass per unit area deposited is equal to this term. The total mass divided by the surface is here. mean the mass emitted from here is disposed everywhere here. So here I calculate which is the mass that is deposited on my infinitesimal area of water, not of the surface, but is projected here. Essentially, I calculate the infinitesimal mass deposited on this line, P a c. Then, I multiply by the cosine and obtain at the end the deposited mass per unit area on the structure. Then consider the distance and then consider the cosine vector, which is the orientation. This is the final formula for the point source.
\fig{22}{5-PVD.pdf}
Which is the dependence. First, the distance. Far is the task of the source, smaller is the weight. Why? Because the mass, imagine the mass evaporated by the source is the same. If it is dispersed on a larger sphere, this is the source, this is the third substrate, the mass will be dispersed on this sphere. If the radius of the sphere increases, the quantity of material going to the substrate is the same. So if you want to increase the rate, you have to decrease the distance. This distance is the physical distance if we use the machine. No more than 1 meter, not less than 2 meters. And the cosine, clearly, geometric orientation is a substrate. It's better to have the substrate perpendicular to the radius, not this. Why, in some cases, we can have the substrate tilted? For example, if you have to use a multi element operation source, like the source of barium and potassium. In this case, imagine you have, this is the source of barium, this is the source of titanium, this is, for example, this angle cannot be produced. Maybe you can decide to rotate the cell in such a way that it can be extracted as a new world. And maybe not because if it's one continuous liquid, the liquid will go out. So the reason for this is angular is that sometimes you are forced to have an angle Okay, it's just to tell you, which is the drawback of this, that to reduce the distance, you increase the rate, but you decrease the uniformity. So it's good for example to want to grow a layer where the uniformity is not so relevant or an electrode the uniformity is not so relevant but the layer is thicker. So you need a larger thickness without so large uniformity. So you can grow very quick, sorry, very close to the field that you need to the body a field material but not so thick for example so you can say clearly which is the problem of same father you are the positive gold or platinum of the story for the uk to waste gold they waste platinum neutral, because there is enormous strength. This is another problem. Sometimes industrial machines, in order to not avoid this problem, but to reduce this problem, they deposit many subjects at a time. So, for example, maybe you have your source here, along this sphere, you put many factors perpendicular so in this way you can deposit many subs at a time so you use the quantity of material situation remember that this is a technological point so when you build when you design a deposition machine you have to consider all the aspects not all the physical ones but also the of the dimensions right and also the problem of safety but for evaporation machine the product safety are smaller than other machines for chemical deposition is a nightmare to get got it is not physical so safety must be very very relevant uh i don't know if you would not enter the polypharm now no when you enter in polygons there is a traffic light red yellow blue because in polygons there is a chemical vapor deposition if the transglide is red don't enter because there is some gas everywhere it did not happen because they are safety but the case is red is yellow means pay attention that some gas lines are not working properly maybe they are just in stock so if you see that yellow light don't panic maybe it seems it's switched off a line in this case it is level line this means everyone everything works in one larger chemical reposition there are other problems but not problems but avengers but okay on tuesday we move to surface sources and i hope the computer will work any questions So, for example, for the end-user... Impossible. What you can expect is a level of disinformity. We will calculate next time. Is disinformity. You can maybe accept 1\% of disinformity. It depends also on how large is the structure of the system. So, if you want to move the sample, you have to move the sample? Yes, of course. Here, I don't tell you how to move the sample, because it's some... But typically, you can move the sample, rotate the sample, for example. Of course, it is something you can do. Typically, for example, in semiconductor, this is done. Sample is maybe not moved, but rotating, because rotating is easier. Maybe you are depositing not on the center, but far from the center, but you are rotating it. By calculation, you can understand that this allows you to be simple. Or you can use a different technique. You won't need a very uniform sample. For example, you can use chemical vapor deposition. Each technique has its proper drawbacks and advantages. That is right. Any more questions?
\fig{23}{5-PVD.pdf}

\fig{24}{5-PVD.pdf}

\fig{25}{5-PVD.pdf}

\fig{26}{5-PVD.pdf}

\fig{27}{5-PVD.pdf}

\fig{28}{5-PVD.pdf}

\fig{29}{5-PVD.pdf}

\fig{30}{5-PVD.pdf}

\fig{31}{5-PVD.pdf}