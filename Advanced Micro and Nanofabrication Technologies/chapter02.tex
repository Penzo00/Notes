\chapter{Process Flow: Deposition. Vacuum Physics: Pressure, Flow}
\fig{2}{1-Introduction.pdf}
\fig{3}{1-Introduction.pdf}
Part of Daniela is up from the big part out obtain the device, but before you obtain the big part. So the idea I want to try to explain you these processes in general, in order to point out where we have to act. Here we say just a simple example, I took an example of my experience. Why? Because of this example, if you're curious or interested, I can give you all the data because it's not covered by copyright. If I speak about CMOS, there are some hidden procedures and so on, because it's something industrial. It's all related to science. And essentially, to my experience, it deals with the fabrication of a photodiode, a very simple system.
\fig{4}{1-Introduction.pdf}
What is the idea? What is a device? Very basically, is something that must use electricity because in order to obtain a signal or to manage a signal you need electrical current so it's a something that must be electrically accessed anyway in this case for example we speak about the photodiode what is a photodiode photodiode is a device that must collect the light the light must be converted in electrical signal the electrical signal must be measured this means that apart from the structure of the system the central system is due to water to what you want to do for example you can use different semiconductors depending on the wavelength of the light world measured but at the end of the story you need a window for accessing the light and at least two electrical contacts in order to measure. How to do this? This is just the basic physics. Maybe I don't know the entire data case because maybe you did not study, or you did not study the semiconductor physics. But it is that you exploit the property of semiconductor in which if you have a photon of a given wavelength, this means photon of a given energy, You can promote an electron from valence to conduction band. This electric conduction band is a free electron. Then can produce an electrical current. So the valence band electrons are fixed here, can be moved by an electric field, you obtain a current. So the idea is, you have the light, yes. You promote this electron, yes. This electron can be moved. This is a current. This is the physics. The point is how to measure this card. This involves a lot of fields, physical semiconductors, electroelectronics, material science.
\fig{5}{1-Introduction.pdf}
But in this course, we are interested that. Not how to build a structure, a structure means many layers of different materials, not how to be how to design this heterostructure that is which material to choose for example in this case this was not a conventional photodiode it was able to measure not only the light intensity but also the light polarization is the reason for which there is for example iron magnetic material but in this moment doesn't matter about which is the reason for the choice of material the point is how we can from this heterostructure that light on the paper can work, how we can use. So we need an active area for the protection. An area where light can arrive. Where to the semiconductor. But if you look at the device made in this way, light must cross some many layers before coming with germanium. Must cross the gold, the iron, MGO. So there is a technological problem. If you want the light to arrive to the germanium without being attenuated, you need the structure on top as small as possible. This is one of the problems we have to face with. So practically this structure, technologically, you need to have this as small as possible. Then how to measure? You need two electrodes. So, electrons means places where you can put a wire connected to a multimeter, for example. These electrons must be made of which materials? Typical metal, gold. Gold typically is made by gold or chromium gold. But pay attention because these electrons must not be in an electrical coctail between them. insulated one with respect to the other. So what we have to put an insulating layer. We will discuss a lot of how to grow an insulating layer. And insulating layer means you have to grow something that electrically insulate the two electrodes. Okay. These are all technological problems because if I just look by structure to my point of view of physicists, not engineers, physicist for me it's enough to have a structure gold iron geogermanium all this problem pertain to technology this for example the structure is a three-dimensional version of the structure in which you have first electrode second electrode and this violator is the insulator that allows to insulate the two electrodes and on the bottom the same conductor And this is a picture from the top. Here you can imagine you have two wires that are connected to a multimeter. If there is a light illuminating this active area, we will measure by a multimeter a current flowing between these two electrodes.
\fig{6}{1-Introduction.pdf}
How can I build this? First I need the idea of which material I have to deposit. Then, from what we start? Typically in this case you start from the semiconductor, as in CMOS. Semiconductor is, in this case, is germanium, because we were interested in a particular wavelength measured by germanium. But it can be silicon, it can be non-ganon arsenide. The semiconductor typically is a substrate with some dimension. Dimension is measured in inches. inches. One inch is 2.54 centimeters. In industry, as large as the wafer is better. Why? Because from the same wafer you can obtain more and more devices. The point is, if you have a big wafer you can cut in pieces and obtain many different chips um but which is the person that produced this wafer not typically industries like st electrical electronics intel they buy weight from other providers in my case i bought the wafer for another provider but the first problem was if you take the wafer the wave wafer may be encapsulated in plastics but when you're exposed to error you will immediately covered by contamination contamination what is carbon in the air in milan or the pound carbon oxygen. Oxygen will react typically with semiconductor, germanium or also silicon, making oxides. Water, the water in the air. So this is a problem because if you want to deposit something on this substrate, you have first to clean, to remove all. This is a point we will discuss today in more detail later. So we have first to do cleaning. Before to think about depositing about depositing anything, you have to do cleaning. How you can clean? Clean chemically in order to remove the contaminants or directly just before the deposition. And this will be, this is just a summary. We'll be in more detail of everything later. What's next? Then And this is the part of the course. Grow means adding on the germanium substrate different layers of materials. Which characteristics? We see this is just an example. In this case, we need to have a very ordered system, very epitaxial system. Means that all the atoms were in the right position in column. This is typical of all quantum devices. All devices based on quantum properties, think about the Bloch's equation of solid state physics, they need to be symmetrical about repetition. So in this case, it's a kind of device that is a reduction. So you grow, but what you intend to grow typically is not what you grow. Because we will see that there are many problems, many drawbacks of the growth. So we have then to characterize. Characterize means look what you have grown.
\fig{7}{1-Introduction.pdf}
Then imagine that you have your structure, your heterostructure. It's just a block. It's just here, Polaro square, depends on the substrate. Then now you have to obtain the device. So what does it mean? Move from this to this. The layers are, see if you look at the colors, are the same, but with different shape. And moreover, there are additional layers that are needed in order to use the device. Is he just a slide that we level from the standard at the end of the course. The point is, how you can make it all up here I told you what ink optical lithography, you know, to define all later or later on a little bit. You don't define the area that you want to remove or to add. Then you can remove some material by by physical action. etching means remove material only where you want. So by optical lithography you define where you want to remove material. Then by physical etching you remove material from that area. Or you can add material. You can move, remove or you can add. For example the electrons typically are deposited at the end of the process. So they will add later. There There are many techniques that can be used. Here I just listed the technique I used in that exactly in that device. If you are more interested, in one of these slides, there is a reference of the paper.
\fig{8}{1-Introduction.pdf}
So we arrived to this device. This is a picture by optical microscope. These dimensions are some things that the emitter was 100 micrometers to 5 micrometers. We have different device, but now we have to electrically access them because clearly you cannot simply buy the probes of a multimeter here because it's a system with dimension of micrometers. How to do? You use a sort of a bonder. It's called a workbonder. essentially is a system that was a microcopter that puts wires made of steel, aluminum or gold on the electrodes. And that is wires are from the electrodes to a typical typical board, external board where you can use a conventional thin soldering. So the idea is that with this system, you move from the device to a macroscopic board. There on the macroscopic board you can go by multimeter or whatever. So this is the result. In this case you can see that it's not maybe easy to see from there but in the center there is a lot of device because typically you don't build a single device but many of them and all these wires are used for connecting this device. So there is the device inside. For each device there are two wires. The wires arrive here on these pads. On these pads you can connect by probes or by classical tin solder.
\fig{9}{1-Introduction.pdf}
That's the end. Yes, if you stay in laboratory. If you don't stay in laboratory, no, because you cannot provide this kind of objects to someone that won't use it. it you need to make packaging packaging means you insert in a device like this is a commercial prototype not based on our system it's just a commercial driver in which this saline this is the active area and this connector is a electrical connector with inside there is the first contact and the shield outside the second contact so at the end you can imagine that all you work arrive to here there's a photo is a system you here you connect with a cable it's a standard cable connect a multimeter then you switch on the light and you measure a current okay this was just to explain to you which is the full process then where we are right I think it was a slide of Daniela I think.
\fig{10}{1-Introduction.pdf}
First part of the course where we will focus on thin films go the position of the other structures but not only because I told you that during the procedure of photography you need to remove material etching but also you need to add material so we also have to deposit sometimes not sometimes every time some materials during the rhetoric pattern so my part of the course will focus on this part that is the just as first step but also this part the position during the lithography the technology technological problem is completely different because here you are depositing on a substrate or a continuous material here you're depositing where maybe on material that was already patterned or called by resist and it will be a difference because in some cases for example there are some techniques that are not suitable deposit on this kind of material or technique that can damage the resistor so if i deposit i damage the photography so we see in the tail which are different between the two techniques which technique we can use here and which technique we can use there this was just introduction.
\fig{11}{1-Introduction.pdf}
Just a summary on which problem the position can address we will see many of them we have 22 hours we cannot see anything but many of them. First of all, materials can have metals that are conductors. You can have insulators. You can have semiconductors. You can have oxides that are typical insulators. Depending on the material, you can have different techniques. Or inside the same technique, you can have different parameters. Stoichiometry, maybe you want to grow a single material, gold, a contact, or an alloy, for example, magnesium oxide, or some magnetic alloys, a permalloy that is nickel iron. The surface, maybe you want a flat surface. Flat surface means flat in terms of atomic distance. So flat means, for example, three layers of atoms, no more than this. Or rough. Rough means it doesn't matter. If I want, for example, to deposit a material to protect a layer from the external exposure, for me it's important that it's a thickness larger than a given value, but it doesn't matter how much larger. The order can be crystalline, practically ordered and in picture, or amorphous. Amorphous means like glass. The molecules can be distributed randomly. The thickness. You can have thick or ultra-thin. Ultra-thin means some atomic layers. Today there is a lot of research about what they call 2D materials, like graphene. to deem is a few atomic layers thickness or maybe you want to deposit insulating layer that can be micrometer of thickness. It is easy to imagine that there is a lot difference, a big difference between before if you have a technique in which you want to deposit ohms of material maybe in one minute, you cannot use the same technique to deposit micron because it will took ohms. So there will be different techniques depending on the rate of the position you want. Uniformity, maybe you want a material that is the same thickness everywhere or doesn't matter if the thickness is different. The substrate is not only important which is the material you deposit, but also where you deposit. The substance can be conductive or insulating. We see that insulating is very difficult to manage. Flatten or rough. Process step, we've already told you before. You can deposit on unpatterned system, just the initial stage. You deposit on the substance. Or patterned. So you're depositing maybe on regions covered by resist. And last, which is application. Because depending on application, have different techniques. In this case, for example, you can deposit thermagnias, thermolatrix, semiconductor, superconductors, and so on. So just to tell you that the position is quite complex, because there is a lot of this problem to be, and we will consider many of them. This is a fundamental rule. There is not a unique solution. Different techniques can be used for the same scope or different scopes of the same technique. So in general, when the question of examination, the question is what you can use for the positive T layer of silicon oxide. There are as well, more suitables, less suitables and as we're completely wrong, but there is not typically only one as well.
\fig{12}{1-Introduction.pdf}
Okay, and this is the sum of the questions we'll try to answer during this part of the course. First, where we will grow, in which ambient vacuum, and that is the part that we start today. On what we grow on surface, we have to study, not so much, but a few hours of surface physics. because you need to understand how to grow before we have to start to understand on what you grow, because the physical surface is completely different from the physics of, for example, bulk materials. And we leave that part to the theory part. Then, which kind of films we will grow? We can grow epitaxial or amorphous, but because epitaxial is very relevant, I will make a lecture on epitaxial. Then finally we will arrive to the deposition technique. And we see different kind of deposition techniques based on physical mechanisms. You give kinetic energy to a material. This material will move to the substance and will deposit. All chemical techniques use chemistry. And finally, as a last argument, You understood where to grow, on what to grow, how to grow, but you have understood what you have grown. That is the characterization technique. Because at the end, you have to look if you obtain what you want. So for example, check the spectrometry. The spectrometry is right. If you deposit silicon dioxide CO2 or silicon oxide CO, it's completely different. So we had to check that the position was right. You have to check, for example, the surface. The surface is flat or not flat? You can use microscopy. So we just show the lecture on this is 22 hours. This is the reference book. But I think all I report in this slide comes from the reference book. There are some copies, I think, in the library, Polytechnico. So you can take a look. question. Okay, the second part we're going to look at. Just a minute. Just a moment of silence. Just a moment. If you agree, I don't make the break and I finish at 14. I don't know Daniela, but I will finish.
\fig{2}{2-Vacuum.pdf}
\fig{3}{2-Vacuum.pdf}
Okay, second part, we have to deal with the question, where we go, where we say which ambient. You cannot grow here in this room. And the reference is chapter two of this book. What is vacuum? Vacuum is a definition that means not absence of any matter because it's impossible. Even in space there is matter. But an ambient in which the pressure is smaller than one atmosphere every time the pressure is smaller than one atmosphere means there is vacuum there are different levels of vacuum that are not clearly defined if you look at different books you can look at different ranges different names of the ranges in general different ranges are these ones from not that the order of magnitudes is from one bar atmosphere to 10 to minus 12 millibars there is a range of 15 10 to 15 from ambient atmosphere it is the typical pressure to the minimum atmosphere you can achieve for us are particularly important eye vacuum and ultra eye vacuum and medium vacuum, the central ones. Note that when you speak about vacuum, what is relevant is the exponent. So there is not also a large difference between, I don't know, two per 10 to minus nine on three, three, uh, for 10 to minus nine millibar. What is important is the minus 9. Okay, why is it in the vacuum? Which is the objective of the Grotto techniques, is to deposit layers of material, in many cases, atoms by atoms or molecule by molecule, on a solid surface. So the idea is you have a flat surface and you send atom by atom. Okay. So why do you need vacuum? Here there is a list of motivations for using vacuum. That is, remember, vacuum is synonymous for pressure smaller than ambient pressure. We will see many of them in detail during the deposition technique. First, to reduce particle density of undesirable atoms and molecules that are called contaminants. If I want to deposit iron on germanium, I don't want to deposit at the same time carbon of the air of Milano, or water, coming from my brief, or oxygen. So I need vacuum. Why? In order to reduce the quantity of the material that I don't want to deposit. This is the basic idea. If I stay in vacuum, I have a large partial pressure of contaminants. So I am able to deposit what I want. This is the basic idea. We will make a quantization of this. Second, imagine you have, I want to deposit an iron atom. The source is here, the substrate is here. The deposition will arrive in this way. But if I have some other atom inside, maybe there will be a scattering. So I will not be able to transfer my material from the source, the target, to the substrate. Source is from where I take my atoms, substrate where I send my atoms. Clearly, if I reduce the particle density inside my ambient, I will reduce this risk. So in this way, I can be sure that if I produce an atom of the source, it will arrive on the substrate without any scattering. The third will be involved in a particular technique in which we use plasma. So we'll leave for them that technique. And also this is related to sputtering technique. This is to chemical vapor deposition technique. And also this one. The last one is very, okay you are young but typically use led lights but you know there are also lights with the filament the light bulbs are in vacuum inside the glass there is vacuum there is some material used to obtain vacuum why because otherwise the filament would burn simply by coated by oxygen So there are some techniques in which in order to deposit use heated filaments and so you need vacuum to preserve all filaments from burning simply. So there are some main reasons in order to obtain vacuum. Which vacuum? We will see for example for the first two motivations, which kind of vacuum you want, the better vacuum you can. When I speak about the better vacuum means the low pressure so increasing the vacuum means reducing the pressure in pretty much first case if you stay in ultra vacuum is perfect.
\fig{4}{2-Vacuum.pdf}
Basically how is did a vacuum system vacuum system is a system made by a serious chambers connected where everything stays in vacuum. Inside this system, you cannot see anything that is not in vacuum. So it's completely closed. For example, this system is a system made by two green chambers dedicated to deposition and a red chamber dedicated to characterization. So you can deposit by two different techniques. you can move your substrate from one point to the other, clearly without touching. It's everything mechanical and then you can characterize. Everything there is in vacuum. Now the question is, how could you obtain vacuum? By pumping. There is a pumping system that allows to obtain vacuum. It seems easy, but it's not so easy. Why? Because you want, there are different kinds of vacuum you can obtain. And depending on the kind of vacuum you can obtain, there are different pumping systems. Two main problems. Maybe it's enough for you to use low vacuum, so a few millivolts. Or you need ultra-high vacuum, 10 to minus 10 millivolts. And this is the first problem. Second problem, you can work in different regime. What does it mean? If you have a system, you can take the time you need in order to put in a vacuum, but then it will stay in a vacuum forever. Or maybe you have a system that is in air, by your hand you insert a sample, a substrate, you close it, you want to pump the system to high vacuum, not taking one year, taking few minutes. So it could be different. In the first case, you have to preserve the vacuum. In the second case, you have to obtain vacuum in a very short time. So depending on these, there are different possible solutions. And because this is a technological course, we will have to look at many of these solutions. For example, we want to deposit an epitaxial film. I started to use some names by molecular BMP. That's it. You need to go to a vacuum. To avoid incorporation of contaminants. So you need a pump. Down to the ultra level, 10 to minus 10 millibar. But it's not a dream. You do not increase the pressure of the crease. The pressure stays at 10 to minus 10 forever. So it's difficult to maintain, but once obtained, or you have a system in which you have to insert backhand substrate, you close the door, then you start pumping. You want to go from one bar that is 10 to 3 millibars to 10 to minus 7 millibars in a few minutes. Means there is a factor 10 to 10 of difference of ratio between the two pressures. And we see.
\fig{5}{2-Vacuum.pdf}
Now we have to come to a more physical, mathematical physical part in order to better understand what, which are the key parameters important for development technology. that also are the key parameters that we will use to decide which kind of vacuum you need. Because technologically we see that there are different pumping systems, but also different deposition systems, and the difference means easy or not to manage, means how much they cost, how much is the dimension. So the choice is not easy. Okay, we start with just recalling kinetic theory of gases. Everyone is conscious of the kinetic theory of gases studied in the past courses. Fundamental physics. Yes, you remember, I can skip these slides or not. Maybe not. Okay, just very short because I use these lights in order to arrive to a key point. I will just check in the world of registration. Okay, kinetic gas is imagine you have a gas in a closed system with uniform pressure, temperature, density, classical gas. For the moment, we do not discuss about pumping. We just discuss about static pressure. Imagine you have a system, at a given pressure, then you close, and the pressure stays constant. No net gas flow, just equilibrium conditions. There is a random motion of particles due to temperature. There is elastic collision between particles and force, and there are no forces between particles. Particles are independent. It's the idea of ideal gas. In the case, you can define the Maxwell-Boltzmann formula of the distribution of the particle as a function of the speed. This function is the number of particle per unit speed divided by the total number of particles. Okay, it's the classical Maxwell-Boltzmann distribution. And as you can see from this picture, but also understand it depends on what temperature mass, because if the temperature increases, the average speed will increase. See for example, hydrogen at two different temperatures. If temperature increases, the particles tend to be faster. But also the mass, heavier particles like aluminum at the same temperature of hydrogen is a smaller speed because heavier means more inertial smaller speed and you can define by calculation the average speed velocity and the mean square velocity and they are all related to the temperature divided by the mass you can also calculate this is a typical result of the kinetic material gases the average kinetic energy that this is the kinetic energy of a particle with mass m this is the mean square velocity and this is related to the temperature so essentially the final conclusion is the temperature is a measure of the kinetic energy of the gas but now we are interested in pressure.
\fig{6}{2-Vacuum.pdf}
Pressure is the fundamental parameter in in vacuum technology is the key parameter. And what does it represent? It represents the momentum transfer from the gas particle to the container walls. Is the scattering with particle the walls? Because particle cannot scatter each other because they are independent. There are no forces between particles, but they can scatter with respect to walls. By using the ideal gas law in the kinetic theory of gases, you can obtain this result that tells you which is the pressure as a function of the average speed, the average quadratic speed. A consideration on what is the pressure, because we use pressure a lot during this course. pressure you know that is a force per unit area that is the unit in the international system is pascal anyway so pascal is not a good unit to be used for example if you think about atmosphere atmosphere 10 to 5 pascal is better to use water or the atmosphere that is the pressure at the sea level at a given temperature and so on. Or the bar. Why? Because the bar is very close to the atmosphere and is not an international system unit, but is just a multiple of the international system unit. So one bar is 10 to 5 Pascal and is more or less one atmosphere. Or you can use the torre, that is the historical unit, and one torre is, you know that this is the atmospheric pressure one dog is more or less one millibar more or less because i told you that when you speak about pressure what is relevant is not the decimal digits but the real order of magnitudes as a very very rough approximation you can tell that one is one millimeter very very rough approximation why it is because if you work with gauges measuring the pressure. Unfortunately, in some, I think, UK, they don't use Pascal, Bar, et cetera. They use TOR. So maybe you have your system measuring in TOR. And so there is a nightmare. How to convert as a zero approximation is one billiard.
\fig{7}{2-Vacuum.pdf}
The hypothesis of kinetic autologousness is that the collisions are only between particles and walls but this hypothesis is realistic what does it mean Imagine that you have a large quantity of particles inside your gas. There is also a probability that one particle during this path will cross another particle. As a matter of fact, the kinetic degree of gas is based on ideal gas, that means our gas at low pressure. So from this theory emerges this fundamental parameter, the mean free path, that is the mean distance traveled by a particle between collision for example you start from the source your atom your atom will travel for one mean free path if imagine that this is the so the source is a substrate i want the atom is able to move this way without scattering If the Mifry path is 10 meters, it is likely that the trajectory will be straight. If the Mifry path is 1 centimeter, no chance. You will start to scatter, so the probability that at least one particle will arrive there will be not zero, but reduced. So the Mifry path determines how many scatterings you have during your path. how you can calculate them imagine to have two particles with dc is the collision diameter of the particle means if the two particles imagine the particles as spheres spheres means physical spheres or also the fact that one particle can talk with another particle by forces if If the two spheres are separated, they will not collide. If the spheres get in touch or try to intersect, in this case there will be a collision. So the idea is two particles collide each time their centers are spaced by a distance smaller than dc. I can try to change the game and then imagine that two particles are not two equivalent particles, but one is a point particle. The particle two is a point particle. Why the particle one is a sphere with a double diameter. By the point of view of the condition for on the condition for the collision, nothing changes because the condition the collision between one and two happens only D is more than to DC. Sorry, is more than the DC is the same. Okay. If you look at that, that one of these one is the same two enters in contact with one If this D is smaller than this radius. So you can make these kind of. Particle one as a target area that is the area. Relative collision. That is the area of a imaginary sphere. Imagine to look by this side is the projection of plane. is this one. Why is that the sphere, imagine you have a sphere, if you look by one side, you don't see a sphere, you see a circle. So if a particle will arrive, it doesn't matter the three dimensional was important just the two dimensional. So particle one has target area, this one. Imagine that this particle one travels exactly a mean free path. This means it will sweep a volume, cylindrical volume, where this is the target area and this is the mean free path. Okay? If during this travel we cross another particle, there will be a scatter. So which is the condition to having at least one collision, at least one scattering is that this volume multiplied by the particle density and number of particles per unit volume, which be one. that is inside is another way to tell inside this volume apart from particle two the particle one that is at least one particle two this condition means somewhere here there is at least one particle so there is at least one collision from this condition you cannot take the mean free path as a function of water the gas density and the effective dimension of the particle, the collision particle. The gas density goes to the denominator. It's clear that larger is the gas density, larger will be the probability of having any collision. So this is the first indication. The mid-free path increases if the gas density decreases. But gas density related also to pressure. So if you pressure the crisis, the military power will increase.
\fig{8}{2-Vacuum.pdf}
Okay now you can use ideal gas flow and in particular you can define the particle number and as a function of the constant, the gas flow and the Avogadro number cost. And you can rewrite this equation of the mean free path in that way, making explicit the dependence of temperature and pressure. Why? Because at the end, you want to have a way to calculate the parameter of the gas, but using parameter that you can measure. So in this case, the temperature, the pressure, something you can measure. So you're interested not on the dependence of atomic elements, but the dependence of what you can address. For example, think about a mean free path at room temperature and ambient pressure in this room. Taking a diameter, a diameter of five ohms. is essentially the dimension of the range of the atomic force. You obtain 52 nanometers. This nanometer is 10 to minus 9 meters. Essentially means that every particle in this room will collide. If this is the source, in order to have a position on my substrate, I need to stay at a distance that is much smaller than, or smaller than, 50 nanometers it's impossible is the first reason for which you cannot work for the position in ambient pressure because the main free path is too small you have to do what to reduce the pressure because if you reduce the pressure of a given for example to a factor 10 to 5 means that the mean free path will increase of a factor 10 to 5. okay so this just indicates why you need vacuum Then you can complicate the discussion, telling, okay, but maybe there will be any way, even if there is a collision, collision, maybe after more than one collision, the particle can anyway arrive on the substrate. So maybe I start from here. My particle is not straight, but it goes this way, many scattering and then they arrive. It's possible to make this calculation by using Monte Carlo codes. The result is not so different. And at the end of these calculations, you obtain this empirical formula that relates at room temperature. Many vacuum systems are at room temperature. So use room temperature because we see that if you stay in a vacuum system, apart from some cases there is no need to heat or cool down so you start with that so in this case the mean free path can be estimated by this formula is inversely proportional to the pressure here in tor is in millibar okay here you can see the same in this graph here you have the pressure in torr on top of millibar here there is the mid free path in centimeters and here in nanometers it's a log log cube and if you take this in a log log cube you obtain just a straight line i indicated by different color different regimes low vacuum that is is from ambient pressure to one millibar. That is a 10 to minus three atmosphere, more or less. Then medium vacuum to 10 to minus three, more or less millibar. High vacuum, ultra high vacuum. All the deposition techniques work in this range, apart from cases, for example sputtering, but stay in this range. Which is the corresponding mean free path? you can see that here, for example, take this point that is the crossing point between the high vacuum and medium vacuum. It will correspond to 10 to 1 centimeter, 10 centimeter. So it means that in this zone, the mid free path starts to be interesting. If you go in ultra vacuum, you have a power of 10 to eight centimeters. So this means that when you stay in high vacuum, you can put your source, your substrate where you want, but the trajectory will be straight. Okay. In general, for pressures smaller than 10 to minus 3 torr, we obtain a MF3 path larger than 5 centimeters. Essentially, from this point, maybe 10 to minus 4 torr, you can deposit without problem. Even because if you consider a chamber for the position, the chamber of the position is not big as this room. It's something of one meter dimension. Okay, so up to this point, what we did? We tried to extract a fundamental parameter that is the mean free path using what? Two parameters that we can measure, pressure and temperature. Temperature by thermometer. Pressure by using a gauge like a barometer.
\fig{9}{2-Vacuum.pdf}
Okay, up to now we discussed what happens in a closed system, okay, where no particle can enter, can exit. So it's closed. Clearly, this is an ideal situation, but it's not a practical situation because we need a system that is not closed. We need to make a particle to move for the position. We have to make a substrate to move to enter the chamber. So this is just an ideal situation. In order to come into reality, we need to consider what is a gas flux. That is a flux, a net flux of particles, a net movement of particles. You can define this quantity that is a typical flux. Flux is the speed multiplied for a density. It represents the number of molecules or particles, I, in the slides, typically use molecules or particles as synonyms, it's not different. Striking a surface perpendicular to x per unit time in unit area. So it's a particle per unit area per unit time, area per unit time, particle on centimeter square second. You can use Maxwell-Boltzmann distribution as well, but it is different from the other I used because in the other one it was a distribution of the intensity of the speed. Here is the motion of the x component of the speed. Anyway, you can calculate from this the flux because you can take the expression from here you can obtain the nx you make on this calculation and you can obtain the total flux for a gas with density n molecular mass m and temperature t okay and you obtain this result at n so the flux of a gas at temperature T with mass m and density n is this one. I repeat, what is the flux? It's the number, imagine you have a net beam of particles and you have a surface. It's the number of particles that arrive perpendicular to the surface per unit time, per unit area. Imagine that you want to deposit a material, there is the source, there is the substrate, this will be, for example, the number of particles per unit time unit area that we deposit on the substrate. Or the number of particles per unit time unit area that are emitted by source. So it's important, it's fundamental in order to understand which is the rate. What is the rate? Rate means how many particles you are depositing per unit time per unit area. Okay, you can use either gas law, you can remove the particle density that maybe is not a point that is useful to measure, and you can re-obtain the flux as a function, as before, as the pressure and temperature. Pressure and temperature are the two key values. enters everything. With pressure you can calculate everything. So by summarizing, this is the gas impingement flux in terms of particle centimeter square. This is instead the gas impingement flux in terms of moles centimeter square. I simply divided by the Avogadro number. Imagine you have a given gas because if you went to deposit iron, you have iron. So for a given gas fixed the mass, the flux will depend on the pressure and temperature only. Because typically the temperature is fixed, the room temperature, the flux will depend on the pressure. Note that the flux is proportional to the pressure. So as I told before, there is not a unique description or unique rule. Depends this formula means that it's better to stay at large pressure or small pressure. It depends on what you want. If you want to deposit a material, it's better to stay at large pressure because the flux will increase. But if you imagine that this is not a material that you want to deposit, it's the flux of contaminants in your chamber. So in this case, it's better to stay at the smallest pressure as you can to reduce the flux of contaminants. Later, when we discuss about the position, we see that when we speak about the position, this is not the pressure in the chamber. This is the partial pressure of the gas that you want to deposit. This is completely different. But in general, they use this one. Numerically, if we take all the constant, Avogadro number and so on, this is the use.
\fig{10}{2-Vacuum.pdf}
Application, we see two application on this formula for the gas plants. First application, imagine to have a vessel, a box, a closed box, apart from one hole of a given area. Imagine that inside the gas you have a given pressure. Outside the gas the pressure is zero. Okay? The question is, which is the flow of gas escaping from the box? Why it will escape? Because, as you know, if you have a system with two different pressures, molecules will move from the zone with larger pressure to one with smaller pressure. I want to measure this flux in maybe a strange unit, but we will see why it is important. volume flow per second, that is, how many centimeters cubed of gas per second will leave the vessel. We can use the formula we know. The rate at which particles leave... No, the English is wrong. The rate at which particles leave the vessel is... This is just a refusal. is the flux multiplied by area. Because remember, the flux was particle centimeter squared second multiplied by area is particle per second. From here, you can arrive to the volume flow per second multiplied, dividing everything by n. Okay? Because particle per second divided by particle per centimeter cubed, you obtain centimeter cubed per second. And this is the volume flow per second. Now you substitute your flux and the pressure used by the gas law, you obtain the result. The result is that if you measure the flux not in particle per cm2s, but in cm3s, the result does not depend on them, on the pressure. Depends only on the temperature and the mass. So we found a parameter that seems strange, but we'll see how it is used later. This parameter has a strong advantage. It does not depend on the pressure inside your chamber. Okay. So if you imagine to work in a system with a fixed temperature, as it happens typically, with a fixed mass because it's just a gas. This is a constant. This volume flow per second is a constant. It does not depend on the pressure. It doesn't matter which the pressure is high. Numerically, this is the solution. Clearly, it depends on the area. This is a geometrical consideration. For example, which is the volume flow for the air at room temperature through a hole with area of 1 cm2, very small area, like a coin or a piece, is about 12 liters per second. It seems a very large quantity. 12 liters per second will flow across this hole. Okay. I don't know if today or tomorrow we will see the application of this.
\fig{11}{2-Vacuum.pdf}
Second. Second application. Forget about the first one. We will reconsider today, later or tomorrow. Think about the image that you want to quote the surface with a layer of atoms. What does it mean? You have a germanium substrate and you want to cover with a layer of iron. Layer of iron means in each point you must have one and not more than one, not less than one atom of iron. This means to cover the surface. Consider a surface exposed to a flux phi of gas with a molecular mass m and temperature T. How long does it take for the surface to be covered with just one more layer of gas particles? This is an example deposition process. First point is how many atoms I have to deposit. So it's clear that the monolayer is a thickness of one atom, but how many atoms I have per centimeter squared. Typically clearly depends on the kind of atoms I want to deposit. But a typical value is that imagine that the atom deposit with a square lattice, each atom is an accordion of the square with distance between atoms of three ohms. He's just a number that is used as a reference, but is not far from a. It's just an average on all the possible configurations. In this case, you have a surface density of around 10 to 15 atoms per centimeter squared. Okay. If you calculate the ratio between the flux and this atomic density, you can obtain the number of monolayers deposited per second, because the flux is a particle from centimeter squared second. the density is particle for centimeter square in one monolayer, you obtain the monolayer per second, just dimensionally. So how many monolayer are deposited per second in this system? Assuming all impinging atoms sticks, what does it mean? That maybe not all the atoms that arrive on the surface stay attached, maybe they are scattered. assume that everything arriving will stay attached. So we obtain the number of monolayer per second. If you reverse, you obtain the number of seconds per monolayer. That is the time needed for depositing just one monolayer of material. So the question is, if I want to deposit a monolayer of iron on germanium, how long does it take? And if you calculate this time, you obtain what? You obtain this expression. Inversive proportional to the pressure, proportional to mass for temperature. If you have a fixed gas, M is fixed, temperature to room temperature, so it's inversive proportional to the pressure. Larger the pressure, larger the density, smaller is the time. because you have a larger number of atoms available to the position. Example. We have air. Imagine to stay here in this room. Air, this is the mass of the air considering the average between nitrogen, oxygen and other elements. At room temperature and ambient pressure. 3.5 nanoseconds. In 3.5 nanoseconds, every surface is covered by whatever. Or in ultra vacuum, if we stay in ultra vacuum, the time is 7 hours 21 minutes. You note that there is a factor of 10 to 13 that is related to a factor 10 to 14, which is a different pressure, because one atmosphere is 10 to 3 millivar why this is 10 to minus 10 millivar okay it's clear that it's completely different it's impossible no impossible depends on what i will tell you later.
\fig{12}{2-Vacuum.pdf}
This is was just the formula i showed Impossible to do what? I stay in this room. Imagine that I have a germanium substrate. I told you that if I have a germanium substrate, before to deposit, I have to clean. Imagine you are able to clean. You go in a chemical bath, you clean by fluoridic acid, you remove all the contaminants. Then you take out of the bath, and what happens? In 300 seconds, the contamination will come back. Essentially, it will continue to be contaminated. This is a spectrum of mass in which the pressure is a function of mass number of the content in the air. You can see essentially hydrogen, carbon, water. These are carbon, oxygen, nitrogen. This is carbon oxide, carbon dioxide, and so on. These are all contaminants. So if you stay in air, it's impossible to have a clean substrate. It's impossible. So which is the point? If I want to work on a clean surface, I need to reduce my pressure, which is the rule. The rule is, imagine you have a system in which you have to deposit some material. And in order to deposit the material, you need a time, t star, maybe in order to deposit a layer of, a proteic layer, you need half an hour. which is the rule. In order to be sure that your surface stays clean during all the deposition time, you need that T star is smaller than this contamination time. I will add much smaller, it's better, but anyway smaller. Does not mean that any atom of contamination will arrive on the substrate, because this number is for full coverage, but essentially will reduce the number of contaminants. So, for example, in this case, in ultra-high vacuum, 7 hours 21 minutes. If you take 10 minutes for depositing, the surface will stay clean. If you stay in 10 to minus 8 dog, there is not a so long vacuum, the time will be 4 minutes. So it will be more complicated. We will see there are some tricks that we can use in order to help avoid contamination. But this is a general rule. If you stay in ultra-high vacuum, it's better. So the question could be, why do not stay in the vacuum every time? The answer is because the difficulty to obtain vacuum is proportional to the level of vacuum. That is, obtaining low vacuum is not difficult. Obtaining ultra-high vacuum is not a nightmare, but almost a nightmare. Okay. So this is the reason for which sometimes there will be some compromise between contamination and vacuum you can achieve. Okay. So UHV, anyway, is required when managing ultrafine fields. Another point, it depends on what you want to deposit. For example, imagine two extreme cases. One is, think about glasses. Glasses sometimes contain anti-reflective coatings, for example the blue light. It will be thick, I don't know, micrometrics I think. In this case, it's important that during the deposition of this ultra layer, there will be no contaminants. but not so relevant. So you can accept a given quantity of contaminants. The same if you want to deposit, I don't know, a protective layer or an electrode. If you want to deposit an electrode, a metallic electrode for making contacts, it must be metallic. But even if it contains some contaminants, it will stay metallic anyway. So with this, less relevant, So you can decide to work at a larger pressure because it doesn't matter. But imagine you want to deposit a tunnel junction. Do you know what is a tunnel junction? Some of you, yes, because we speak at the UN. Do you know what is a tunnel junction? Okay. Imagine you have a metal, an insulator and a metal. And imagine that is a device with components. Classically a carbon can flow. Quantistically, yes, because there is a tunnel effect that tells you there is a given probability for a current to flow from this point to this point. This probability depends on what? On the thickness of the insulating layer. Because if it's, it depends exponentially on this thickness, because if this layer is too thick, the probability of tunneling is definitely smaller. What does it mean? Few nanometers, no more than these, two, three, four nanometers, no more than these. Two, three, four nanometers means one nanometer is more or less four atomic layers. So we are speaking about 10 atomic layers, 10, 15, 20. In this case, the contamination is more relevant. Why? Because if you have contaminants here, maybe these contaminants can break the insulating. And so instead of insulating and so tunneling will become conductive. First. Second, because quantum mechanics works here because everything is ordered as in picture I showed you before. But contaminants does not, are not ordered. They go somewhere. can break the symmetry and so they can remove the effect so for example for tanning junction because of the low thickness because of the symmetry that is needed because you have to preserve the insulator effect you need that to work in ultra vacuum to reduce the quantity of contaminants it's just an example but there are also other examples typically if you want to grow So insulating material, any contamination can produce a breaking of the insulating. Breaking means no more insulating. Think about the dielectric. The breaking of the dielectric is the same. Okay. Just last, yes, for today is, if you look at that formula of the contamination time it depends inversely on the pressure this means that the product between the pressure and the time is a constant for a fixed max temperature you stay at room temperature everywhere you have a fixed gas this product is constant so a typical unit that is used is the surface exposure measured A language defined as one language correspond to an exposure of 10 to minus 10, 6 torr for one second. There is torr, not millibar, because a unit of UK. And also why I speak every time of flux in terms of centimeter square, not meter square, because it typically units used, unit used in this kind of theory. theory also books so i prefer to maintain the same notation used in the book so it is a unit telling that i can deposit one lag mirror corresponding to expose the system to 10 to minus 6 torque for one second minus 7 for 10 seconds and so on and this corresponds for example uh to try vacuum to less than one moment layer but it's not so relevant is important is that if you find in a book i exposed one material to one language means that i exposed for a time and a pressure so that the product is good.
\fig{13}{2-Vacuum.pdf}
This is just a summary of what we told today but we'll restart tomorrow from this summer question Even curiosity, but maybe not today, but in the future. Curiosity about the position and so on. Today was theory, unfortunately, but... One on gas blocks. So when we were calculating the EGOR 2 of the segment, we said that it does not depend on gravity, and only on temperature, but then the temperature does depend on the volume of the ideal gas law, right? So we can always transpose it. In the ideal gas law, you have two, three parameters. There are density, volume, temperature, pressure. Two of them are independent. So pressure, temperature are independent in the sense that they determine the density that enters in this equation. Even though the box is of a fixed volume, which we know the value of. But gas can move. So it means it is fixed volume, but the density can change because the gas can move. In this case, I understand what to do. But fixed volume means fixed number of moles. In this case, instead of using if you think about the volume is fixed but the number of moles will change because we change density that is the number of particles per unit volume so what will change is the number of moles in the gas law so pressure and temperature are two independent variables that determine the other.\\
Today we continue with the background technology and we finish with technology so for tomorrow we can start with the surface science just a general comment i will present some of this slide not some i will present these slides at the end of the slides that we present there are some happenings that are more data than I present in principle examination what I ask is what I explain so the rest remaining part for example demonstration theories and so on it can be interesting looking at them but it's not part of the examination I think on the website there are some examples of examinations so for example you can find last year there was a two or three times a question about technology so in principle to the question of technology you can already ask that now so this is more this way you can take a look of an examination and also on the level of examination right because in the second year that examination is written before it was just organ but we decided to change okay uh this is just a summary of yesterday yesterday what we told was we discussed the vacuum vacuum is pressure smaller than ambient pressure and we told that the key parameter in vacuum technology is the pressure okay because typically the temperature is typically room temperature the mass depends on the gas you're using if you're using air it is made by different components and you can take an average mass but the parameter which can play is the pressure because like we see today by parking machine you can change the pressure the key parameter where the mean free path it is the average path between that a party can travel without having any scattering. So in principle, Mifry path means this is the source at which distance you have the first scattering with some other particle is the Mifry path. Then we calculate the flux. Flux is something related to a beam of particles moving in one direction. We calculate the flux, this measured perpendicular to the surface, in number of particles per centimeter square second. And you can see that all these parameters depend on the pressure. The flux is proportional to the pressure. Larger the pressure, larger the flux. Because larger the pressure, larger the number of particles. Mifree-Pav is the opposite. Larger the number of particles, easier will be the scattering, so the Mifree-Pav will decrease. Then we calculated this volume flow, of cm3 volume per unit time of gas flowing and this will be a point that will take a day. And last the deposition time, the coating time, means the time needed in order to deposit one monolayer of atoms taking an average density of 10 to 15 cm2. also we point the fact that in the fields for research development the pressure can change of 13 orders of magnitude because from one atmosphere that is around one bar to ultra vacuum that is 10 to minus 10 millibar there is this factor so we have to we we will study the pumping system then I am able to range between these enormous differences.
\fig{14}{2-Vacuum.pdf}
This is just a summary of this result. This is the molecular density, this is the molecular instance rate flux, pressure, mean free path and the monolayer formation time because they or related with the pressure, you can plot in the same graph as a partial differential. And in particular here are indicated the different ranges in which these techniques that we will study are inserted. And we will discuss when we will study techniques why a given technique needs a given type of one group. OK, this was just a summary, then you can go and study the gas flow. Gas flow means that yesterday I told you that you stay in a closed box. In this closed box there is no pumping, there is essentially a gas inside this box. You can have a movement of this gas, but it's not a net gas flow, so maybe you can have some atoms that will go to the surface, reduce the pressure, but it is not a net gas flow as for example produced by a pump. How to have a net gas flow? You need a pressure gradient. So pumps will produce a pressure gradient that will produce a net gas flow. You can consider three different flow regimes, depending on the local geometry of the system, pressure, temperature, type of gas. This means that exactly as when you study gas dynamics, you know that you can have a laminar flow or a chaotic flow, okay, the formation of vortex. Laminal, vortex. It's completely different. Even in this case, and you can use an adimensional number, the Knudsen number, in order to estimate which is the kind of flow you have. In particular, these adimensional numbers is the ratio between the mean free path and the typical dimension of the system. Dimension of the system means, which is the typical distance, for example, for the source and the subset. anyway it ranges from 10 cm to 1 m, no more than this. So this number is not very variable. Why? The mean free path can be very variable because it depends on the pressure. If you take instead of this value, this parameter, the corresponding expression we found yesterday, we can obtain this relation. And here you can see, as a function of the system dimension, as a function of the pressure, you can see different regimes. The regime to which we are interested is the first one, that is the molecular flow, the laminar flow. Because in this case, the Mifry path is larger than the system I mentioned, this means that the trajectory are straight, so you can use kinetic gas theory. And it's this one, it is the typical one of evaporators, electron spectroscopy, electron macroscopy techniques. So in general we consider this regime. The other regime, for example the regime of viscous flow, is present when you have a high pressure, you have a high pressure for example in some chemical vapor deposition technique because in this case the fact you have a gas is fundamental in order to activate reactions so we see the difference between the two but in general when you speak about physical vapor deposition technique it's molecular gas flow.

\chapter{Vacuum Technology: Pumping, Pumps, Vacuum Systems}
\fig{15}{2-Vacuum.pdf}
Now we come to pumping what is pumping pumping means removing particles from one side moving to another side. So for example you have a system here, atmospheric pressure, you take particles from here, so you use the pressure, and you send the particles into the atmosphere. For example, I don't know if you have at your home this small machine in order to make vacuum for the food. You have plastic bags and you can put the food in vacuum, because in this case it can be preserved for longer time. It is exactly the same. You will remove particles and put in the atmosphere. A gas flow in general is driven by a pressure difference. You need a pressure difference. So you can consider two boxes connected to one hole with different pressure. The molecules will move from the region with larger pressure to the region with smaller pressure. We define this strange parameter that is Q. Q is called the gas per output, that is a strange word, in Italian it's portata. Gas per output, that is expressed in pressure per volume divided by time. Pressure per volume per unit time. Later we'll see which is the physical meaning. For the moment it's just a definition. And it's the product of a pressure drop and this C. This is called the conductors of this water. If you have two regions with different pressures, each one with the pressure, with the gas in equilibrium itself, at the boundary you have this hole. So the conductors is referred to this hole. So we can define this gas drop output as the product of this conductance multiplied by pressure drop. This gas drop would present only if, if and only if, there is a pressure drop, otherwise it cannot. Which is the net gas flow in the hole. It will be the product of the area multiplied by this difference flux because in principle to that macroscopically you have more particles from one to two then vice versa but it's possible both so in principle you can define the net gas flow as this difference okay and it corresponds to water this is particle per second If you divide by the gas density, it will be cm3 per second. If you remember, cm3 per second was exactly the same unit of the volume flow we introduced before. So essentially, I called it B dot. B dot has exactly the same dimension of the conductance. And this is possible to demonstrate that it is the same, the same quantity. So essentially, this volume flow we studied yesterday, looking at what happens if you have this part of pressure P and outside the pressure 0 is the same if you calculate pressure 1 and pressure 2. essentially what i studied yesterday i just called p1 equal to p p2 equal to zero but it was exactly the same which is the point the point is if you remember this quantity that is the volume flow volume per unit time that yesterday seemed a strange quantity did not depend on the pressure it's essentially a constant once you feel the temperature the mass and area this is a constant so in this equation c does not depend on the pressure okay which is the physical meaning of this gas throughput this is just a trace in order to indicate in order to indicate physically what is If you use, imagine the situation of yesterday, in which the P1 is P, and P2 is 0. Q is conduction for pressure, but the conduction is the derivative of volume with respect to time, because it was volume per unit time. Because the pressure is constant, you can obtain this result. Then, by using the ideal gas law, you are clear. just to tell what is gas drop-out is a quantity that is proportional to the number of molecules of particles crossing the hole per unit time. So it's a quantity Q proportional to how many molecules are moving to the hole per unit time. It's proportional because we multiply by kVt, but this is a constant, a temperature constant.
\fig{16}{2-Vacuum.pdf}
Okay, maybe it can be easier to understand if you make a parallel between gas flow and the classical electrical current flow. Q, the gas throughput, particle per unit time, takes the role of the current, the electrical current. Electrical current is electrons per unit time. So particle per unit time, electrons per unit time. The pressure drop, that is the force producing the gas flow, is equivalent to the voltage drop, that is the force, electric field, producing the electron flow. And the conductance here is exactly equal to the conductance here, the reverse of the resistance. So they take the same role. and as you remember this quantity does not depend on the voltage exactly as this quantity does not depend on the pressure so it's just a parallel between electric world and gas flow world the equations are the same the fact that they are the same formally it means that you can use the same rule for example a combination of different systems later we will see just an example but in principle if you have more holes for example in series or in parallel you can take exactly the same rule with which you sum the conductors electrical conductance okay we see the advantage Here is just some elements that we can find in typical vacuum systems, for example holes, for example tubes connecting to chambers, for example annular links, and to each one is connected to conductors. The conductor depends on what? Depends essentially on the, only, not essentially, only on the geometric dimension. case just on the area in this case on the diameter and the length and so on so it's like the electrical resistance the resistance is something defined by the material by the length and so on it's exactly the same okay and you can use them in this equation in order estimate which is the gas throughput for a given pressure drop.
\fig{35}{2-Vacuum.pdf}
If you go here, you can see for example an example of application. I do not enter into details, you can look at it. But for example, imagine you have a vacuum system and a pump. We'll see later how pumps works, but typically pumps are systems that cannot connect directly to the chamber because of physical dimensions. So between vacuum system, the pump, there is a nipple. You have to understand which is the conductor on this nipple, because we'll see later that it is fundamental in order to determine which is my ability in pumping the gas. And in order to do this, you can use this formula. For each of these elements, you have a tabulated value of the conductance. For example, in this case, what you have? First, the vacuum system speaks with a hole that will... What is a conductance? It's an obstacle to the gas, essentially. Think for example the electrical resistance. The electrical resistance is an obstacle to the current flow. As a matter of fact, you need a voltage drop in order to have a current flow inside the resistance. In this case, for example, we have the gas everywhere. In order to force the gas to move only in these regions, it means that some particles, for example, will scatter here and here. So this restriction will be a problem and will need a pressure drop to be overcome. Then there is this cube. And so these are two obstacles and you can calculate the total conductors using Sinti's rule. there are two conductors in series, we use the sum rule of the series, remember this is not the resistance, the sum rule of the inverse of the resistance.
\fig{36}{2-Vacuum.pdf}
And that's all for example this is a typical system used to connect two machines, we see a lot of this system in the laboratory. This is a real system. And okay.
\fig{37}{2-Vacuum.pdf}
\fig{38}{2-Vacuum.pdf}
This is even a more complicated situation. We do not have only this restriction and this tube, but also some obstacle inside the tube that is used for other applications. In this case is more complicated. So the conductors is reduced. The point is, more element you have, smaller is the conductance. If you think about resistivity, larger the resistivity is, larger is the voltage drop you need for moving a current. This is the inverse of resistivity, so smaller it is, larger will be the pressure drop needed to make the gas flowing from here to here. So the problem is when this becomes too low, too small. And came back here.
\fig{17}{2-Vacuum.pdf}
You understood which is the role of the conductor. The conductor is the same role of the inverse of the resistance in an electric field. Something that in order to have a current inside electrical circuit you need a voltage drop in order to have a gas flow inside a conductance you need a pressure drop now the point is who can create this pressure drop it's a pump later we will discuss in detail some pumps and we'll show in detail how pumps works technologically Before we understand how a pump can determine the pressure in the chamber, because the point is what parameter is interesting for me is the pressure inside, because the chamber is the place where the position characterization will take place. So my final objective to obtain in this chamber the pressure I need for my operation. Which are my, what I can use, a pump. Which is my obstacle, the pipe for example, or the conductor, the elements, the producer conductors. Okay, so the pumping speed is defined and the ratio between the throughput divided by the pressure. Okay, where consider a parallel with current. If you have a circuit without nodes, the current is concerned. Current does not disappear. Imagine you have a circuit for voltage drop across all the elements you have the same current. The throughput has the current. So the throughput here and here is the same. You define the pumping speed. The pumping speed is a quantity that is related to the pump. So forget about the chamber, just look at the pump. For now. Maybe it is too complicated this way. Consider the final pumping speed. Doesn't matter where. The pumping speed is the ratio between the throughput and the pressure at the pump inlet. that is in the place from where particles are pumped away. The throughput is Tor per liter divided by second, pressure per volume divided by time, the pressure is Tor, if you make the ratio you have liter per second, volume per unit time. So the pumping speed is how many liters per second you are able to pump, is to remove from the place where you want to move. For example, forget about the part of this moment. Imagine you have just this part, the chamber. You don't know to which is attached the chamber. You don't know what is below section 1. You can anyway calculate the pumping speed, because the pumping speed here is the ratio between the throughput and the pressure in the chamber and this is how many liters per second are extracted by your chamber up to now it doesn't matter how, it's just a number ok note that conductance and pumping speed have the same dimension but different meanings because the pumping speed is something related to a single chamber with a single pressure while the conductance is related to an element with a pressure drop, but they are the same as I mentioned. Problem. You cannot have a chamber directly connected to a pump because any time you have some element to connect. When you go in the lab, you will go in polyfabric laboratory. And you can ask to Marco Azz, he's the person that will... I don't know, you already meet Marco Azz? Yes or no? Okay. Marco Azz is a technologist of polyfabric, that is also a physics engineer, because he was my old PhD student. You can ask him to show directly how systems are made. chamber, the nipple and the pump. You can ask what you want from Marko, since Marko knows everything about this because he started in engineering and he worked on that machine for many years. ok problem consider a pipe with a conductance C connecting a chamber at pressure P to a pump with pressure P to P pressure to pump the turn output is constant along the chamber like the carnet in a suit pressure is P is the pressure measured in the chamber that is at this section because you suppose it to be uniform in this chamber S is the pumping speed related to this chamber what does it mean? is numerically the ratio between the throughput and the pressure how many liters per second are extracted from here? Pp is the pressure of the pump measuring this section, because these two pressures are not equal. If they were equal, there would be no throughput here. So you need two different pressures. And Sp is the intrinsic pumping speed measuring this section. They would be different from this, why? Because it's different, this number. So the point is, I am interested in calculating the pumping speed in this chamber. But what I have, what I can calculate? I can calculate the conductor of the pipe, because geometrically you can calculate the conductor and the energy. You can also find on books, there is a table of conductors. You know these values because when you buy a pump, the seller gives you this value. This pump is able to have a given pumping speed and the base pressure that is this value is a given value. It depends on the pump. So the only parameter to be determined is the effective pumping speed. with the capacity of your system made by the pipe plus the pump to extract molecules of the chamber.
\fig{18}{2-Vacuum.pdf}
And I do not follow the formula, it's very easy to see, to calculate, you can obtain at the end this relation. The effective pumping speed, that is how many particles per unit time are you able to extract from the chamber, is the intrinsic pumping speed of the pump, something that depends on the pump, that is on the manual of the pump, divided by this one plus this fraction, that depends on the conductance ideally if the conductance would be infinite what does it mean infinite conductance with the electric analogy it would mean zero resistance the effective pumping speed would be equal to the intrinsic pumping speed if the conductance is not infinite but finite means the resistance is larger than zero, this will be smaller. This means that because of the conductance, you lose pumping speed. This is the problem of conductance. It's the reason for which I told you that conductance is an obstacle to pumping. Why? Because it reduces the effective pumping speed. So the ability of your pump to remove particles from your chamber. As smaller as the conductance, as smaller is the effective pumping speed or in other words if you need a given pumping speed here and you have a large conductance you need a pump that is over as I mentioned this is just what I told you for example later in the lecture we study we will see what is a turbopump but in general turbopump is a kind of pump able to put the system in a high vacuum condition or even to maintain ultra high vacuum condition so it's a typical pump for system that must stay in 10 to minus 7, 10 to minus 8 millibar If this pump is a pumping speed of 2000 liters per second, and imagine here a pipe, a cylindrical pipe, with this diameter and this length. This is a typical vacuum element that is used to connect a big chamber with a pump that anyway is not smooth. will reduce the pumping speed to one third of the intrinsic value. So if you want to obtain a pumping speed of, for example, 670 here, you need to buy a pump, a turbo pump, with a pumping speed of 2000 per second. These are real numbers. I try to use in this course as much as possible real numbers. numbers. These are typical real numbers. This is a typical real number of a turbo pump. It is a typical real number of an impulse. Note that this impulse is not so small, because the diameter is 5 inches. But anyway, it's an obstacle for pumping. Okay.
\fig{19}{2-Vacuum.pdf}
You can stop me if you don't understand the funny question. For the people that stay at home, I am not able to see, look at the chat. So if you have any questions, just switch on the microphone and... Ah, see? It's new. Okay, I can't see the chat. But anyway, you can even switch on the microphone and ask me so up to now what we discussed we discussed what you obtain remember that is not a physical course a technological course we have to give you some indication in order to be able to design or manage a deposition or lithography system. This is related to a vacuum system. So, for example, if you want to buy, build a laboratory, and you know that you need a given pumping speed, go on the catalog, and you see, I know that I have to buy this pipe, which is the conductance, then, which is the pumping speed of the pump I have to buy, or could I reduce the pumping speed of the pump, larger pumping speed means larger cost. By changing the pipe in order to increase the conductance, so all engineering problems. Now we move to a second point, that is this related pumping speed. But why I'm interested in pumping speed? Which is my final objective to obtain a given pressure. So my final objective to obtain a pressure of 10 to minus 8 millibar. Remember that when I speak about pressure I use every time the unit millibar or tor. So when I speak about 10 to minus 7, if I forget to tell you the unit, it's every time millibar. So 10 to minus 5 millibar is 10 to minus 8 bar that is my heat atmosphere. The point is my final objective is to obtain a given pressure in my chamber which is the way using the pump. So the pumping speed is a parameter for which reason I calculate the pumping speed. I calculate the pumping speed because this will determine the final pressure I can obtain my chamber. This is the final formula indicating the pressure inside your chamber as a function of time. Because you can imagine that, imagine you have a chamber in air. Then you switch on your pump, the pressure is reduced. How long it takes? Think about the vacuum machine for the food. how long it will take in order to go in vacuum? this is another technological problem, you don't want to wait one year, maybe one hour can write, but maybe in the industry one hour is not right, one minute. this is the formula, i will not demonstrate here this formula, but you can find in this appendix the derivation of this equation is a typical exponential decay equation. The derivation is very easy, but just treat it or prefer to avoid it. This is the pressure of the function of time. Bi is the initial pressure times zero. Imagine that the initial pressure is ambient pressure. P0 is the ultimate pressure of the pump, means the final pressure that can be achieved by the pump if closed on itself. Imagine to have a pump completely closed, that is the pump does not pump anything apart from itself. This is the pressure that can achieve. It's not zero for the reason we'll see later but is anyway a value very very small you can imagine. So the point is, which is the pressure P inside my chamber as a function of time given an initial pressure in my chamber and the final pressure that can be achieved by the pump is given by this equation. that tells water that in principle if you wait for an infinite time you can achieve in your chamber exactly the same final pressure of your pump at the end there will be an equilibrium but how long it will take? it depends on water the effective pumping speed is the reason for which we calculated the effective pumping speed and the volume of the shape. The time constant of this exponential is volume divided by the speed, that is this equation. So you can see in this equation, there is the conductance. If the conductance increases, tau will decrease. It will be faster. If the conductance will decrease, tau will increase. So the conductance will determine the effective puppy speed and through this, the final pressure you can achieve or better, not the final pressure you can achieve but the time needed for obtaining a given final pressure. Just understand, if this tau is one year. It is clear that I cannot wait one year. So my point is, if this is one year, which is the pressure that can obtain in one hour, for example? I make two examples of two pumps I will describe later. The first is a pump that is called a rotary pump. A rotary pump is a typical pump used to evacuate a chamber for ambient pressure to a lower pressure, typically 10 to minus 2 torr. Later we will see why. So the pressure drop is 10 to 5, a factor of 10 to 5. time for ambient pressure I inserted same values typical note that this is the effective pumping speed, it's not so large because these pumps have not so large pumping speed I inserted an element of the given conductance no, sorry this is the cylindrical chamber so the volume of the chamber that ends in the equation simplicity I forget any pipe so there is any conductance. Imagine the conductance is infinity. I want to evacuate the chamber from particles, reduce the pressure. I want to reduce the pressure of my chamber from ambient pressure to 10 to minus 2 torr. Note that this is not the ultimate pressure of the pump but I don't need to achieve this in order to achieve this I would need infinite time for me it's enough to arrive to 10 to minus 2 how to do? if you take this equation and you extract time you take the time needed to achieve an even pressure if you know the initial pressure, the final pressure the ultimate pressure, the volume and the pumping speed so the point is, if I insert this number here, which is the time needed to achieve this 10 to minus 2, is 3 seconds not so long why is not so long? because the time constant is 20 seconds ok, so is very fast. What does it mean? Imagine you have a chamber in ambient pressure, you want to put it at 10 to minus 2, just switch on the pump, wait 3 seconds, half, and it's 3D. Okay. Clear how to obtain, you start from this, if you want to calculate the time in order to obtain a given pressure, this is the equation. Second example, imagine you have a turbo pump. Turbo pump is a pump that is used in order to obtain a low pressure, high vacuum pressure, or eventually ultra high vacuum. A turbopump has an ultimate base pressure of 10 to minus 9 torr. But a turbopump for the reason we'll see later cannot be started at ambient pressure. A turbopump must be started at 10 to minus 2 millibar torr. So in this case the initial pressure is not ambient pressure but 10 to minus 2 torr. and the ultimate pressure that the pump can achieve is 10 to minus 9 torr the question is how long does it take in order to achieve this pressure infinite but you can tell what does it mean to achieve this pressure means to achieve this pressure apart from 1\% of the difference. So to have 10 to minus 9 torr or 10.01 to 10 to minus 9 torr is the same. So the question is which is the time needed to achieve this value of the pressure, say 6 seconds, not so long. The time constant of this system is 0.34 seconds, very fast. These two are typically used in cascade. The first is used to produce a 10 to minus 2 torque pressure at the inlet of this one that we use to obtain pressure. Many times some pumps are using cascade. The first is called a forked pump to an ambient from 10 to minus 2. The second 10 to minus 2, 10 to minus 9. The couple rotary ok ok the number I used are realistic numbers so if you look this number you can tell me there is no problem to obtain vacuum just need a few seconds to obtain vacuum I use realistic values so why I told you yesterday vacuum is a nightmare for experimentalists because this is not the true history but it's not all the history, there is something more.
\fig{20}{2-Vacuum.pdf}
As a matter of fact, the pumping is needed for different reasons. The first is for what is called volume pumping, that is, inside your chamber you have a gas, you want to remove gas from the chamber. So physically, to take molecules from inside, particles from inside, and move to outside. And this is the equation, and we see that it's not dramatic, because you need seconds without conductance. If you have a conductance, you need maybe not seconds, but minutes, but it's not a problem. But that's something more, because a chamber is not made only by a volume, but also by its surface. A chamber is made by steel, stainless steel. That is a material that is pure, but there are some problems anyway. Internal surface means that, imagine your chamber, apart from what is inside, evaporator and so on, like a box, typically cylindrical, not square, a cylindrical box with internal walls of the cylinder. From these walls you can outgas some chemical components. Why? First is permeation from seals and metal walls penetrated by all gas molecules. These walls are 1 cm thick. But anyway, there is a chance that any molecule can enter from air, can interact with the wall, and penetrate through the wall inside the chamber. Remember that we are dealing with very small pressure, so it's clear that this is not a macroscopic effect. But just very few particles are enough to increase the pressure, for example, to 10 to minus 9. So it will be a small effect, but it is present. Permission from seals and metal walls. Metal walls are the walls of the chamber. What is seals? In order to connect the different elements of a chamber, you have some object connected in some ways. Some ways means, I will show, but you can also ask Marcoasa, are made by the connection made by copper links, typically. that are quite smaller. Through this copper rings you can have a permeation of molecules. So this is a source of contamination that cannot be removed by volume pumping because it's continuous. Volume pumping, once you remove it, you remove it. In this case, this process is continuous because the chamber is exposed to ambient. You can also have diffusion for plastic and seals with gas dissolved. Typically for vacuum you use metal surface, because you can produce metal surfaces with larger level of purity. But sometimes you also need plastic or seals. Think for example to wires with the coverage of plastics. In this case, in a plastic you can have dissolved some gas molecules. When you put in vacuum, these gas molecules are dissolved. So putting a gas molecule inside your system, sorry, a plastic element inside your system is a problem. But you need, because in some system you need to have plastic. okay even in this case is not a this option that at certain point stops is a sort of continuous the short distortion. There is both. This is due to the fact that inside your plastic there are some molecules that when the plastic is put in vacuum are disordered. But there is another problem. Here your chamber is in air because initially every chamber is in air. Then, when it's in air, it happens what I explained to you yesterday. If you expose a surface, a clean surface to the air, in few nanoseconds it will be covered by contaminants that are carbon, oxygen, water. Nitrogen is difficult to attach, so nitrogen is not a problem. because nitrogen tends not to react and to be attached to the surface. So in the moment that you start pumping, you will have your surface of the wall, that is made by stainless steel, ultra pure, but anyway is covered by contaminants. And so there are contaminants that you have to remove. While you work with vacuum, every time you need to use gloves. Why? Because your skin is covered by oil, by organic elements. Organic means carbon, oxygen. It's devastating. Once there was one of my colleagues that had this attitude of... And he had a problem with a drop of blood entering the chamber. Impossible to obtain ultra-high vacuum. Until we found the problem. Because it continued to dissolve. If you touch the vacuum with your hand without the gloves, you will see the traces of your hand and you cannot make in vacuum. In vacuum means in ultra-high vacuum. So this effect cannot be bumped as this one, they stay and this is the final equation.
\fig{21}{2-Vacuum.pdf}
This is not only this depending on time, but you have also these elements that do not depend on time. may be reviewed but not completely reviewed and this graph indicating this contribution you can see that depends on time but the volume can be packed very fast but it's not a problem surface absorption can be removed diffusion can be removed permeation cannot permeation will continue anyway this is the reason for which you need to pump every time without this in principle you can pump up to here and stop pumping the system will stay pumping but not so you have to continue to pump so mass pumping time to achieve ultra high vacuum is spent on moving gas from surfaces and the surface are the true problem Now the point is, imagine a wall to obtain ultra high vacuum, means 10 to minus 10 millibar. That is a condition needed for some depositional spectroscopic techniques. Imagine that pressure in torr due to permeation with a larger stronger pump can be reduced to 10 to minus 12 to make it a little bit easier but 10 to minus 10 ok. So the point is if I want to obtain a dry vacuum I have to find the time at which I obtain the pressure I want. typical pumps will take one month, more than one month. So, or I have time to switch on the system and wait for one month, with the pumps continuing to work, or I can use a trick. That is, in general, elevate temperatures promote degassing. What does it mean? If you heat a surface, surface you force the surface to dissolve the contact. So if you hit this surface you dissolve this gas. Take for example water. If you have water on a surface and you hit the surface the water will evaporate, for example. So a solution is what is called bakeout. Bakeout means you heat your chamber walls So you hold your chamber to 200 degrees Celsius for one or two days while the system is continually pumped. How to do? You use essentially resistance coils. You put this resistance across the chamber. You use the aluminum foil like cooking in order to uniform the temperature. The chamber is made by metal so the thermal conductivity is very large. In this way you can promote degassing. And in this way you can obtain ultra-high vacuum. Why? Because you are removing the most relevant part of your contamination. So typically, you have your pump, you start your pump, you arrive, I don't know, here. then we turn back out, you can decrease the temperature because you promote the gas.
\fig{22}{2-Vacuum.pdf}
Now just a quick discussion about pumps. I do not get into the details of pumps because I prefer to integrate into the details of how the system works. So the information you need in order to understand. What is a vacuum system? A vacuum system is a system made by different elements. What you have? Connection between different chambers or pumps. valves to separate chambers and pumps because in some cases you need to separate chambers for example you want to put a chamber in error you don't want to put all the system in error so you have valves to close gauges to measure pressure gas flow temperature glass window in order to look at right fast entry doors means like doors to be opened to insert the sample and then close sample holders because inside the sample cannot be touched is in vacuum so you have some system then can host the sample sample and substrate are the same can hold the sample and move inside your system for example move to a depository chamber to another and then all the tools you need deposition characterization cleaning and clear you have the pumping the pumping serve to achieve vacuum yesterday in the initial part of the lecture i told you that you have kind of regime starting dynamic dynamic means i stay in pressure I want to go in vacuum static means I stay in vacuum I want to preserve the vacuum what does it mean preserve continuously remove the degassing from surface because even after back out anyway there is degassing.
\fig{23}{2-Vacuum.pdf}
Just very quickly vacuum pumps you have these categories gas transfer and entrapment. Entrapment punch means you entrap the gas particles. You have gas particles in the chamber or permeated from the walls, they are entrapped by elements that can take them there in these elements and so essentially they are removed from the chamber. or gas transfer. Gas transfer means in this case you entrap so molecules are more free to move. In this case you take molecules from the chamber and put the molecules outside, there. Two different kinds. Pump does not remove molecules by accepting forces, clearly you cannot exert a force on the molecule because a molecule is a neutral molecule, it cannot exert a later concern. You alter the natural molecule motion by a trapping or making them to move somewhere. Okay this is just what I told you. Between gas transfer pumps we see difference between positive displacement and kinetic pumps. Now I present a list of the pumps with some characteristics.
\fig{24}{2-Vacuum.pdf}
The first is the rotary pump. We have already seen the rotary pump and it was the pump and example from ambient pressure to 10 to minus 2. If you look on YouTube, there are very good simulations of how the pumps work. There are simulations released by producers of the pumps. In this case, you have essentially here you have the chamber, here the ambient. We have the system rotating that is able to take air from here to here. So removing air from here means removing air from the chamber. It's a positive displaced pump, in the sense that it displaces the gas mechanically. because air enters here, this rotates, this moves physically the molecules in this direction. Not each molecule, but moves the whole block of molecules. This is a pump used for water. Used to arrive to 10 to the minus 2 torr or 10 to the minus 4 torr. Not smaller than this. the pumping speed is not so large so why it is used? it is used because other pumps we see cannot be started at ramean pressure but need a lower pressure so use this kind of pump as a pre-stage to other pumps so this is for what is used advantage and disadvantage we have a rotating system Rotating system that sees the ambient pressure. Means you can have friction. Friction between rotating part, friction with air, because at the end they see air. You need oil. Oil to... is a lubricant, to avoid heating, like your car. But oil, if it's heated because of the movement by friction, well, it can evaporate. Oil is made by carbon. Carbon is the worst nightmare of vacuum. Because while water is enough to heat the sorb, carbon is not. So what you have, you have what's called traps. Traps means between this pump and the turbopump, there is an oil trap that is made by crystal that adsorbs oil. So avoid the fact that the oil will go into the chamber.
\fig{25}{2-Vacuum.pdf}
Other kind of rotary mechanical pumps. The only difference is the way in which the displacement is made.
\fig{26}{2-Vacuum.pdf}
Now we move to gas transfer kinetic pumps. In this case you impart kinetic energy to the gas. This is a turbopump. Other pump I show an example. Turbo pump is a pump that can arrive 10 to minus 10 torr with a pumping speed that is 100 times a rotary pump or less. It works because you have a set of rotating blades, like a coil fan, a set of rotating blades that impact, that by impact, give a kinetic energy to the molecules that are moved from one blade to the second blade to the third blade, and this way, by impacting the energy to the molecules you take the molecules outside the chamber. The idea is the molecule arrives here, here you see the first blade that gives kinetic energy to the second blade and so on up to the exit of the chamber. This is I think the most used pump, it's a mechanical pump. Oil is not needed because in this case the pressure is smaller so the friction is small. But in order to be sure that the molecule can go from one blade to a second blade, you need the mean free path of this molecule to be larger than the distance between the two blades. So this means that you must stay in vacuum, because if you stay in air, the mean free path is few nanometers, the distance between blades will be few millimeters, it's impossible. So this pump must work in this pressure range. you need a rotary pump before in order to make this pump to work.
\fig{27}{2-Vacuum.pdf}
Diffusion pump diffusion pump is a pump that is based not on a mechanical part moving, mechanical part moving some can be broken so it's not so good for a certain reason it's made by oil you have oil here, oil compatible with vacuum, you can make oil boiling, boiling oil is vaporized, it goes to this tube, then it exits here, this drop of oil entraps essentially the gas molecules. by gravity they come to the bottom and so in this way they remove the particles they are called diffusion pump because they are based on the diffusion of oil which is advantage no moving part that can be damaged avoid moving part is better but is not so pure because anyway you have oil or the oil will stay in the pump, chamber is outside, but anyway there is some risk in past times they were used to put in vacuum foods, for example in industry they were used to put in vacuum foods as turbo pumps you cannot switch on at room temperature, because at room temperature, no sorry, at ambient pressure, oil will burn You must stay at smaller pressure in order to the fact the oil will evaporate without burning. I tried to switch on a diffusion pump because one of your teachers, that you know but I don't tell the name, told me, not a problem, switch on the pump even you stay at 10 to minus one toro, no problem, oil completely burn. smell it was one of your pictures but it's a secret okay so advantage don't be part disadvantage your oil so for example not suitable for ultra vacuum applications.
\fig{28}{2-Vacuum.pdf}
Gas entrapment pumps these pumps based on the fact that you can entrap molecules somewhere How? By condensation or by chemical bounding. In this case the particles are not moved outside, are not removed, but are taken inside the chamber, inside the pump. So at a given point the pump will be saturated. You have to refresh, in the sense to clean and restart. a cryopump is one of the most used pumps with turbopumps and in this case what you have essentially you force the condensation of few molecules like imagine the condensation of water in the morning on the glass of my windows there is water why because outside is cold inside is hot so there is condensation it's exactly the same you make condensation of the gases on the walls of this pump, it is made by a lot of cylindrical cylinders in order to increase the surface, that are at a temperature below 120 K, because below So 120 K you can condense water. Water in vacuum conditions. You can also use a temperature of helium, that is a few Kelvin. In this case it would be very powerful. As a matter of fact with this pump you can achieve ultra vacuum. Temperature smaller than 20 Kelvin, 10 to minus 10 T. It's closed pump. you do not have any opening towards the ambient. It's an advantage because if you have a power problem, as with storms in the summer, it's a nightmare because everything stops. If everything stops, if you have any connection between the chamber and the ambient, if the pump does not move particles from the chamber to the ambient, the opposite will happen simply because the pressure drop. With this pump, no, because they are closed. If they switch off, they stop to pump, but they do not speak with the ambient. Even in this case, you need initial torque pressure that is smaller, 10 to minus 3 torr, because otherwise you will produce ice. If you produce ice inside it will stop to work. Ok. So no moving part, no oil. But you need a cryostat in order to have the low pressure you need, each pump has its advantages and disadvantages.
\fig{29}{2-Vacuum.pdf}
Ion pump, another gas and carbon pump, in this case is based on the fact that you have a very large potential that is able to ionize the molecules these ionized molecules can be driven by magnetic field towards the wall High voltage means few kilovolt and high field means few kilogauss. But once it's switched on, it starts to pump. In this case, the pressure must be 10 to minus 5, 10 to minus 6, very lower. otherwise this effect will not reproduce such a plasma cause we see when we speak about Spartan but a group of charged particles can produce a plasma and this plasma can destroy the pump so you must stay as low pressure so that you can direct particles without producing a plasma. A plasma is like a storm a lightning is a plasma you don't want to create any lightning for this you need to stay at small pressure so in this case in order to use this maybe to evacuate your chamber with a turbo pump a rotary pump is not enough but then you can switch on this it's a closed pump.
\fig{30}{2-Vacuum.pdf}
These are the two kinds of pumps made based more or less in the same effect inflammation pumps titanium means you heat a titanium when you heat something with evaporator titanium has the capacity of chemically bind with contaminants so you evaporate titanium it will chemically bind the evaporants, by gravity will come back and then it will come back with contaminants. Getter pumps are similar, in this case you have elements called getter that are essential absorbers of other contaminants. For example, the old light bulbs inside they had getter materials in order to maintain a vacuum because if you had a light bulb in order to avoid the filament to burn you need a vacuum but vacuum must be maintained but you have no pumps you have this greater material that are able to absorb some gases okay this was just a short list of the many different kinds of pump you can use the information can take is if i tell you i want to a cheap ultra vacuum which pump could you use dot answear rotary possibly dot answear diffusion if i ask i have a i want to evacuate a chamber that is in air which pump can i use dot answear to me turbo etc as with me rotary.
\fig{31}{2-Vacuum.pdf}
Last part last 10 minutes Just a few words on what is made of a vacuum system, vacuum system in general. All the vacuum systems are more or less made in this way. Maybe there are more chambers, but this is the basic idea. We have two chambers. One is a central chamber and one is a sample preparation chamber. Here you prepare the deposit, characterize and use samples. to introduce the sample. Which is the point? In order to obtain high vacuum, we have seen that you need times. So if you are able to obtain high vacuum or ultra-high vacuum by breakout, please do not destroy this vacuum. So if you have a chamber that stays in vacuum, it's better that the chamber must stay in vacuum. But if you want to insert a sample on the extractor, so you have to open the chamber, use a different one connected by a valve. So you have two chambers, one that can be pumped from air to a vacuum in short time. A second that is in a vacuum to a vacuum every time. this is static pumping, this is dynamic pumping. This chamber typically will contain just one element, one holder to take the sample. Then you have some system to move the sample, open the valve, move the sample from here to here. All this system of pump we see now step by step step by step how to insert a sample, how to extract a sample. So do vacuum chambers. Two pumping systems to vacuum. Two rotary pumps, you see why, one venting valve and inside what you want, the position, characterization, what you need. ok, imagine you want to insert a sample in a chamber in order to make a deposition, just for example first, red means all these valves, ok, here I forget to tell you that you have also valves these are valves, there are four valves, open or closed Red means closed, green means open.
\fig{32}{2-Vacuum.pdf}
So let's start with all valves closed. You open this fast entry lock. Fast entry lock is just a window that can mechanically open. A window or a door. Why I brought nitrogen? because typically when you have a system that is in vacuum and you want to put it in air why? because in order to open the door the pressure difference between the chamber and the ambient must be zero otherwise you cannot open something vacuum it's better that in order to make the chamber that was in vacuum to put the chamber in ambient pressure is better than the pressure is nitrogen because nitrogen is very easy to pump because nitrogen does not react with the walls otherwise oxygen can react it's true that there is an essential nitrogen but not only nitrogen there is also oxygen and there is also water if you use pure nitrogen essentially what you do is you cover your walls with nitrogen. So the walls are protected against other contaminants by the nitrogen you have inserted. It's called venting in nitrogen. You just use a nitrogen line, nitrogen is not so expensive. This way what you do, you can help your system to protect of contaminants and so you can make faster the pumping. So you open your door, here there is nitrogen at ambient pressure, you insert with your hand with a globe your sample, then you close the door. and you switch on this valve and you pump this system by a rotary pump to 10 to minus 2 torr at the same time this imagine that at the same time all the pumps were on you don't switch on off and pump the pumps are always on why because switching on off a pump can be damaged, can be not good for a pump. So a pump is better to stay always on. So all the pumps are on, but they are closed, they pump on itself. Then you open this valve, means that this pump can make this chamber to 10 to minus 2 torr. Then you close, you open this. This turbo pump sees 10 to minus 2 Tog, it's able to pump, so in a few minutes this chamber will arrive in high vacuum. Then you can open this valve and move by a holder, mechanical holder, the system, the sample from here to here. is pumped by a cryo pump there will be a vacuum it doesn't matter if there will be a different pressure from here to here not a problem if the fresh pressure difference is 10 to 100 because when we close it will make the pressure back to the initial value then you do your procedure you i don't know deposition or characterization here there is a cryo pump just an example you can have a diffusion ion turbo pump okay you finished your operation you move your example back here then close all valves just open this valve that is a valve for nitrogen for venting when the pressure here is equal to the pressure outside you open this is the typical operation of every system maybe you have more than one deposition chamber if you For example, you have different techniques. So you can, for example, have our system, you have a chamber for molecular B epitaxy, a chamber for fast laser deposition. You can add how many chamber you want, but typically this is every time present. Okay, and this is how you can play with valves and pumps in order to make this operation faster. in the system I used, that is not the top system, this operation from year to year needed half an hour. There are also other systems in industry that are all automatized. just open put not one substrate many substrates close go home come back this morning and we'll here we all use up the deposit because everything will be mechanical in a laboratory typically it's mechanics you have to have but it's not necessary.
\fig{33}{2-Vacuum.pdf}
\fig{34}{2-Vacuum.pdf}
It's just an example i think that's all this is an example of my system in which we have okay this is just an example of our made a vacuum system but i think it's better you ask marco to show you what you use is a copper gasket it is something used to maintain vacuum but ask marco for you this and that's all this is just an example of a system like xpu this is the grab this is the chamber called fast entry lock for inserting the sample that is connected to one to three chambers in ultra-E vacuum, this is high vacuum, and by this transfer arm we can move the sample from here to everywhere. These chambers have different levels of vacuum, this is ultra-E vacuum, ultra-E vacuum, because this is for electron spectroscopy. This is for molecular bioequitaxy, this is for passing laser deposition. That's just an example. Many of these systems are in polypharm. OK, this was just an overview on the vacuum technology. So if you enter in a laboratory, you would be able to understand what is happening. Maybe you're not able to understand which pump is. But you're able to understand from a system where is the fast electric lock and so on. typically a fast entry lock is a place where there is a door that can be opened manually and a typical window in order to see what is inside remember that inside the chamber you have vacuum vacuum means 10 to minus x outside there is ambient pressure if you stay for example in ultra vacuum between the inside outside there is 10 to 13 difference so all the doors all the windows that are made by glass must sustain this pressure drop so it's not so easy because there are for example thick windows because they must sustain these pressure drops that is not small.