\chapter{Magnetostatic Energy and Shape Anisotropy}
Okay, today I want to see with you something that was written in some of the slides of yesterday's lecture, which is the concept of magnetostatic energy. So yesterday we have seen the magnetostatic framework, which means, say, essentially in case of absence of currents, you can use now a scalar potential, write a Poisson or Laplace equation, solve it for the potential, and then find also the demagnetizing field, which is the field produced by the magnet. And we applied this concept, this machinery, this theory, to three cases. The cylinder uniformly magnetized along its axis, and the same cylinder magnetized in the right direction in the sphere. And we have seen that was the last concept we asked the electric that in case of surfaces of the second order, an ellipsoid, a sphere, a cylinder, all these kind of surfaces containing a thermomagnetic body, you have the peculiar condition in which the demagnetizing field produced by the magnetization is uniform inside the body and is connected to the magnetization by the demagnetizing tensor. Okay, this was the last concept. Okay, Okay, what I would like to do today is to see in more detail the energy cost for building up a magnetic configuration. The cylinder magnetized along its axis, or the cylinder magnetized along its right-hand direction. Do they have the same energy? Do they have the same energy cost for building up these two configurations? The body is the same, but just the fact of magnetizing with the wrong one, acting on the other one, does it create a different cost? And which is the energy cost that we have to pay? That's the basic question. And now to answer this question, we have to see now which kind of energy contribution should come into play. So that's a question for you. Now let's imagine that, and now you're seeing something there. So I should, let's do this. Let's start from scratch. You have a cylinder or a sphere. So let's imagine that we start from our cylinder. And this was our reference ray, XYZ, which is an infinite cylinder, and the magnetization can be pointed like this. Which kind of energy could be involved in the calculation of the total energy of this system? coming from the cause of negative. So which kind of interaction could come into play and give here a contribution to the total energy of the system? Please? Obviously, dipole-dipolar interaction between the magnets. Of course, we can't hear where the first line is. Dipolar interaction between... What's your name? But this shouldn't be the first answer to my question after the course of Jack The first answer should be, if you have a ferromagnet, which is a basic exchange interaction, once you say okay let's calculate the exchange interaction here okay but just to clarify the story when you write down the exchange interaction we will see it I'm not sure today probably tomorrow when you say exchange interaction so So essentially, the Hamiltonian that you have is written like this, minus j s1 dot s2. Well these two are two operators in principle, and the spin space. So everything depends on the angle between the spin. Well let's assume that you are treating now the spin as sort of a quadratic one often. So what this reminds us now, it's interesting, the last one again, is the angle between the speeds. So that you have a difference of field ratio for the same world space in the particle space, okay? So it depends on the angle, power or how many times it makes a difference. And in general, exchange energy comes from an angle between the speeds, which is not equal to zero. So the ground state for the magneto all the things alive. And for any kind of perturbation, you need an angle between negative and positive, you have an exchange energy column. That's the same. And now, with this in mind, we should say that there is no difference in exchange energy between this configuration and the other one. So let's take now the same system, okay, in which same reference frame. Now the magnetization is right there. That's not fantastic. If Ham is pointing like this, there's no difference in exchange, interaction, energy. What's the reason why there's no difference? Because this Hamiltonian is totally isotropic. So what makes a difference, just the angle between the spin? If the spin is at the point of this, or right there, there's no difference at all. See, it's totally antithrombic, the exchange interaction. What makes a difference, just the angle between the spin, not the absolute direction of the spin. So, as soon as you start with an item, a node, another term, a magnet, and you shape it into a cylinder, direction of the speed, followed by the axis of radio, doesn't make any difference in terms of exchange energy. So in terms of exchange, it's not different. And now the real difference that comes into play here is when you consider the bipolar interaction, which is exactly that leading to magnetostatic energy. When you say magnetostatic energy, you are saying dipolar interaction energy.
\fig{3}{Lecture_2 Thermodynamics of magnetic bodies.pdf}
And so the first task of today's lecture is this one. Try to calculate, find a way, the expression for calculating dipolar interaction energy. When you imagine that you have a body like this one, the shape is whatever you want, in which you have now the spin arranged in a way that is arbitrary. They can be pointing in any direction in the space, depending on what? Depending on the way you magnetize the system, depending on the fact that you have a domain wall inside, straight configuration. But in the end of the day, as soon as you have a configuration of the spin of the magnetic moment on each lattice point of your crystal, you have now a configurational energy. It's exactly like electrostatic configurational energy. So it's just the fact that you have an arrangement of spin pointing in an arbitrary direction in the space. That creates now a dipolar energy. What does it mean? It means that if you have this dipolar moment, this magnetic dipole here, it fills the field produced by all the other dipole plates in all the other lattice sites. And the interaction between the field produced by all the other lattice sites and the first one creates an energy that depends on what? On the configuration of the dipoles in your body. So it's something connected to the configuration. so I'm not asking myself which is now the energy cost to create that configuration I'm just saying whatever is the way you create that configuration when you have this configuration there is an energy associated to the interaction between that dipole and the field produced by all the other dipoles in the body this is magnetostatic energy now the question is how can you calculate When you have an extended body with a strange configuration of the dipoles and so on, how can it really tackle this stuff? And this is the way. So first of all, you say, OK. You have a lattice with magnetic moments mu i. Okay, different I is the index with which allows you to distinguish the different dipoles in the lattice. It could be a cubic lattice, could be an hexagonal lattice, depends. Iron is a BCC. When you have nickel, it's an FTC. When you have cobalt, it's an ACP. So different structures, so you have a dipole in the lattice. And then you say that okay, I have the field, you have this dipole, this mu i, and all the other dipoles produce here a field which is the h i, okay? It's produced by all the other dipoles. And now you have the interaction between this mu i and this h i. And this is a sort of Zeeman interaction, okay? It's not an external field, it's the field produced by all the other diaphrases. How do you write down the interaction energy? It's a Zeeman interaction, which is minus mu dot bi, which means minus mu i dot mu zero h i, which is exactly what you write down here. So the interaction energy will be minus mu zero mu i dot h i. Okay? But okay, now you have to pay attention not to count the couple twice, because of course when you do this and you make a summation over the i sides, so you are counting interaction twice, because when you jump from one to two, okay, on one I calculate this. When I jump to two, I calculate the same story, but I'm counting it rationally to one and two twice. So in order to get rid of this, or say, on my smallest mistake, you place in front this factor one out, okay? So this is the interaction energy between that field and this one. But if I want to count now the energy corresponding to all the couples of dipoles in my body, I have to put inside, so outside this factor 1i. And you take the summation over the i, all the sides in your lattice. so this is in principle what you have to do and now the problem is how can you calculate H high okay is the summation now here on all words that's that's interesting okay interesting oh yeah I J and J I remember the meaning of this one. You know, one two two one. You don't have to count the carbon one. Now how can you make it? The basic idea is that represented here. You have a body like this, which is an extended body. Now you say, okay, let me sit on the dipole new pile, this one. Okay, I sit on this, and now I make separation of two regions, a sphere which is concentric to the dipole, nu i, and the region which is outside. The region outside is that dashed here in this board. And I treat the problem in a different way. For the region which is the dashed region, I use the continuum approximation. I don't want to see the crystal structure here. I'm just considering not the peculiar crystal structure, the SCC, DCC, ACP. I'm just considering a magnetic body and I want to use now just a concept of the demagnetizing field coming from the theory of the magnetostatics that we have seen in So I want to use a sort of demagnetizing test of that kind of total magnetic charges and so on. But inside this field, I want to go into the detail of the structure, the crystal structure, and inside the sphere, I really consider that having DCC, FCC of ACP lattice. The question could be, why are you doing that? Because in the end, it's clear that the field, H-I, this one, will reflect also the arrangement of the atoms around the single dipole. But clearly, the fine structure will be very relevant for a distance from the dipole, which is not too large. When the distance become too large all the diapos are more or less in the same position So you understand that the problem of the real crystal lattice is really something which is relevant when you are quite close to the single iron D1 At the distance which is comparable with or to the lattice constant Okay, so when you exceed when you are 10 times the lattice constant far away So the Euclidian structure over there is no more relevant. It's a sort of continuum of . And so what you can say, OK, inside this sphere, in a radius which is larger than the lattice parameter, of course, because the Euclidian is no longer than the lattice parameter, but it's just the signal you want. It must be larger in order to include different lattice sizes, but we will see not too large. And I anticipate something just to say that half half should be what? R is larger than A, the lattice constant, and lower than what is called the exchange length. I will define this parameter in a formal way later on in this course, but this is the typical length, typical distance, over which you see a tilt of the sphere. Let me anticipate something that you already know. Exchange interaction is an interaction that tends to keep all the spin following. The ground states correspond to all the spin following. Any perturbation of this is an excited state, which has an energy quality. So it's just the tendency due to exchange interaction. This is not the unique, the unique, I'll say, energy to be considered for the equilibrium configuration of the magnetization. As a matter of fact, in a body, in an extended body, sorry, I didn't see something. In an extended body, you have domain walls. So the main wall is a surface which separates two parts with a magnetization which is opposite of 90 degrees. So for some reason that we will see, in a real body there is a field between the spaces. There are some regional niches which jump from one domain to another domain, and in that region there is a rotation of the space. Now, even in a domain wall, the rotation cannot take place in just a single light-year structure. The reason is quite simple. The energy cost would be too high. A jump by 180 degrees over just a single light-year structure will imply a cost in terms of the exchange energy that is unaffordable for the system. So the rotation of the spin in the domain takes place over a characteristic distance, which is the domain wall width. So in any case there is a distance over which the spin can rotate. And this distance is defined as the exchange length. We will find a more formal definition of this form later on. But clearly, and that's important at this stage, You chose now a sphere, which is a concentric sphere around the type of mu i, whose radius is lower than Lx. And what is the T-impact of this? Why do you choose that? If R is much lower than the T-impact, you can safely assume that inside the sphere all the T and all the lattice points are parallel. Okay? So what you're saying is that if this is true, then you can say that in your sphere Here you have your dipole which is the new eye But let's imagine that you have a crystal lattice, which is this one which is a cubic one All the spin are parallel And this simplifies a lot the calculation because now the choice is done in a proper way. You choose a surface with the full symmetry sphere and inside the sphere you have all the dipoles are parallel. This is the easiest way for calculating the effect of the peculiar crystal lattice structure that you have in your model. So coming back to our calculation, now our target is calculated h high, which is the field produced by all the other diagrams. And now we have divided the body into two regions, a sphere of radius R, capital R, inside which we are considering the detailed lattice structure. So we are treating the problem exactly like this, with a summation. outside we're using the continuum of approximation. And now how can you proceed? So the way you proceed is this way. You say okay, let me do this. So this is my body, okay. Now I say that the field produced by this can be divided into three contributions. So I take now the field produced by the region which is outside and for doing this I say okay this is equal to this story. I take the same volume here, plus what? A sphere. Okay? Here, let's assume that the magnetization is pointing like this. So it's equal to a body with a magnetization point like this, treated in the continuum, plus what? A sphere whose radius is r with the magnetization point it down. In this way you create an oval, okay? A body with a cavity inside, a straight cavity inside. And then you add, it is MS, if you want to know what I'm using, using just hand. And then I take my sphere of radius r, which is treated in the continuum, not in the continuum, much better. In the case of the real lattice, it could be a cubic lattice. This is the superposition of the three elements, creating a difficult division what sum is exactly in the field that's supposed to have. So essentially I was imagining to have this right here, this is the dipole, this is near high, and now this is the decomposition that I'm using. Okay, now in terms of field, if h high is equal to volt, if I take this body, the demagnetizing field at the point i is what I'm calling here HM. Capital HM is the demagnetized field at the point of the i-dipole produced by the whole body treated in the continuum approximation using the concept of the demagnetizing tensor, magnetic charges and so on. And then I have to, let me do like this, It's more beautiful. And then I have to consider the contribution of this one. What is this? It's a sphere. And what's now the demagnetizing field of the sphere? Minus one-third m. But okay, m, now we should say, in principle this is minus m if you want to be consistent with the same thing. This is m, this is minus m, okay? So you have this minus m, so this means that the demagnetizing field produced by this one is something which is pointing upwards, okay? So I have to write down what ms divided by 3. This is exactly, or ms, sorry. Sorry for the confusion. In the end, it's a saturation magnetization of that body, but I was using just M without this substitute. M divided by 3. Yeah, because, okay, you say it depends. In this case, do you understand this? No, I don't. No, you don't understand. Okay, fine. If you have a sphere, the sphere with the magnetization pointing like this, Okay, the demagnetizing field will be pointing like that. Okay, it's a demagnetizing field. So at the end of the day, it's aligned to M. And the demagnetizing field is one-third, minus one-third M in terms of vectors. But as this one is pointing down, the demagnetizing field is pointing up. So if you want, you put here the M, which is a vector, and you find something which makes sense. Because M is pointing upwards. The first one, in the first element? Yes, exactly. We are considering the demagnetizing of the blood? Yeah, this is the demagnetizing of the whole body. And you should have different signs then, since one is magnetizing one way and the other No, okay. Yeah, you are right. But this is HM, it would be pointing like this. Okay. HM will be pointing down, and the other one will be pointing up. Not exactly down, because it depends on the shape, but it would be opposite to the M. But in case of the sphere, it's really opposite, with the coefficient which is one-third, minus one-third. And HM, which is minus one-third M. In our case, it's the other way around. m minus one-third m, which is pointing up again. Okay? Like m. And this is the original two minuses you have here. If you want this minus one-third m, where m is minus m. This means plus m divided by three. Are you okay? Good. Plus what? plus the contribution at this side of the dipole, which has sit on the different sides here. So plus H I prime, which is the field produced by all the dipoles, considering them as individual dipoles. And see, really, the magnetic field H produced by all these dipoles are arranged according to the lattice that you're considering. Okay, now this is something that you can calculate. This is some, okay. The 9\% is the ring in the left-hand side of the element. Are all the parallels? Yes, definitely. Yes, they are all parallel. We will see this in the calculation in a while. Okay? Good. Let's move on. on. Now essentially this is done, so we in principle depending on the shape we can calculate it using the demagnetizing tensor for instance. This is something which is easy because it's a sphere. Now the problem is how to calculate this contribution here. So the big advantage of this choice is that you have a sphere with all the dipole fibers.
\fig{4}{Lecture_2 Thermodynamics of magnetic bodies.pdf}
So now we have to really go into the detail of the field produced by each dipole. And so you find that, okay, you are inside the sphere, and you have this geometry, the dipole mu i and the dipole mu j. As you see, they are parallel, even in this sketch, okay? They are parallel because of that assumption over there. And now you have to apply the formula for the dipolar field produced by one dipole at the the position of the first one. The formula of this one is a formula that we will use a few times in this course. So it's the dipolar field produced by a dipole at a quite large distance. So it's one over pi minus mu j, mu j. So now the the substitute j is now moving in such a way to account for all the other sides around the mu-tribe. So it's a sum over j for rij, which is lower than capital R, which means inside the sphere, of minus mu j divided by rij to the third. You remember that the dipolar field decays as one over R to the cube. And then you have this much, much complex formula, which is mu j dot r i j by r i j, divided by r i j to the 3. So that in the end, even here, you have a dependency like 1 over r to the cube, which is exactly what you expect for a dipole. So this is exactly the summation that you have to carry out in order to find the value for h i prime. And now probably it's a good time for a small exercise. Let's try to make this calculation, to do this calculation for a simple cubic lattice. Just understand how you can find a real result. Well, we start from the vectorial expression, and now we consider a cubic lattice with the axes which are x, y, and z. And now you say, okay, I start from this formula here, and I take the projection over x, for instance. I consider just one component, which is hx. So it could be hi prime x. So I take this formula here, which means it's a summation. I take 1 over 4 pi in front, everything, and then I get what? Okay, I have a summation of the different terms. Okay, but I don't need this one. The first one is minus mu j, and I take minus mu j x. Divided by r i j to the third. Then I have plus three times. Now I have this dot product, which involves mu j dot r i j. Let's now write this dot product using the Cartesian representation of the vector, which means so I have mu j so it will be mu j x y x i j plus mu j now y yij plus mu jz zij multiplied by now I have to make the projection of rij which is xij divided by rij to the fifth. But now we can simplify the story because we have to remember that inside our sphere, all dipoles are parallel, all dipoles are parallel, which means that I can forget this index j, so it's always the same. So one could say, okay, this is equal to one over four pi summation of this minus mu x as is reported there, divided by Rij to the third plus three times, and then you have mu x xij plus mu y yij plus mu z zij multiplied by xij divided by aij to the cube. And now don't forget that you have to make this summation which is for all the j which are inside the radius. Yes summation over the j. okay and now let's make let's carry out this kind of summation let's consider now the second term here if you now multiply this you will find some terms like this so you have three times sorry and then you have this contribution mu x xij second plus mu y yij xij plus mu z zij xij. Okay? Now, the contribution of this term here and the other term here will be zero when you take the solution. And the reason is very simple. These have all terms both in x and y. So now you have a rep, so in your lattice, you have one on the right, one on the left, okay? And they give you a contribution to the opposite side. So all these terms are odd terms, which means that they carry out the summation inside the sphere. That's the reason for choosing the sphere, is the surface is the body with the maximum symmetry. really capitalize this choice right now. All are parallel and we are using the sphere. So this means that they will give a net zero contribution in the summation and the unique to surviving in this one, because this is the even term, Xij. So you're left with a calculation of the summation of this term here. But now you are in a cubic lattice. as you are in the cubic lattice, x minus z are two limits of change, so they are totally equal. So that I can say that this summation for j inside the sphere of xij the second divided by r i j to the fifth can be written like this one so i have this let me also write three times three times the summation of the j I don't need it, the summation of j, or, yeah, I need it, okay? So let's imagine that you have this one. You can write this one as one-third r ij to the second. Do you see it? r ij, it's not too much space here, but okay. Rij to the second will be xij to the second plus yij to the second plus zij to the second. Okay? If I take the summation of this one, as all these three terms are equivalent, the summation of this one will be exactly one-third the summation of Rij to the second. Okay? So I can find that this one, this summation here, corresponds to this. And so this and this make it out, and finally you are left with the summation over j of 1 over rij to the third. multiplied by what? Multiplied by mu x. And so, with this in mind, as you have this one, so all this term is provided with plus the same term which is here. And the net result is zero. You see? All this term here is giving you what? In the end of the day, you're left with just this term here, with the same term here with the minus. End of story.
\fig{5}{Lecture_2 Thermodynamics of magnetic bodies.pdf}
What is the main result here? The result is that in case of the cubic lattice, the contribution of all the dipoles inside the sphere is zero. The magic of the dot block is zero. You can check the iteration in the final, it's correct, it's zero. interesting what does it mean? It means that in this case, in principle, you can forget about these two points. You just have the first two ones. But this is a general result of... Is it peculiar to the cubic lattice? Is it just peculiar to the cubic lattice? It depends on the theory itself. We have used... We have deeply used this term there, the cubic lattice, where this symmetry is actually the center of the sphere. If you take another crystal lattice, those will not be zero. But again, if you understand now the way you carry out the calculation, those will be in any case connected to the direction of the sphere inside the sphere. And in general, you can easily figure out that this term can be written like this. Minus is a tensor which multiplies the magnetization. Just tell me. Why am I saying that this makes sense? Because inside the summation, you always have the mu. And the mu is what defines the space, is pointing to what, and the single dipole is connected to the magnetization. The magnetization is the magnetic dipole divided by the volume. So in principle, the magnetization is defined like this. is the sum for j, okay, if you want r ij lower than r of what? mu j divided by four pi r to the q. This is the magnetization. The magnetization is proportional to mu, okay, and as you have mu inside all this this summation in the end you will say that depending on the shape of the body sorry depending on the machine depending on the crystal structure you will have that the h high prime will be sorry this is the business living not saying that is equal to this h high prime is equal to that expression here. That's a general expression. And so you are left with this general expression for the magnetostatic energy, the configuration of energy depending on the Euler interaction, the u-ham of these lines, minus mu 0 divided by 2, the summation over i of ui dot h-hi, where h-hi is the sum of h-ham plus one third m minus the stencil lambda capital lambda multiplied by have. Okay? Are you okay with this? The question is for yourself. I have a question. Once again, not now. Why did we lost the weeks we were going from the... What we were going after... From there, I didn't lose it. I was just considering that, okay, If you want, I can write also by mu x here. Okay. So it should be mu x, mu x, and mu x. I was just considering everything without mu x, but I need it because without this one, I will not find the cancellation with the first one. I was just considering everything but this pre-factor. Okay, Mattia? Thank you. Good. Now, in the calculation of the magnetic energy, we are here. Now we know how to carry out the summation, but as you know, as it happens, so the summation is easy if you go to the continuum approximations. Instead of carrying out the summation, you move down to the integral, you say, okay, I associate now to each mui a small cube in a lattice, in a square lattice, in a cubic lattice, and then I move now to the integral in the volume. So instead of mui, now I see the appearance of what? Of M, the magnetization. Because now you can say that if you want, that's another way, the new I can be written by this is the magnetization by A to the cube, okay? And A to the cube is exactly d tau, the unit volume, okay, or d cube half. base association you transform now the summation into integral which is exactly this one. There is nothing particular to that. And now you distribute this scalar product to the different terms appearing here within brackets and you get these different terms alpha, beta and gamma, three distinct integral and the integral is is over the volume of the volume. It's over this volume here, okay? This is the integral. Three different terms, alpha, beta, and gamma. Let's have a look to these different terms and comment on the entity, meaning, and so on. Let's start from beta. Beta is minus mu zero six over six, m squared over in beta. It depends on what? Essentially not in the integration of m, okay? Not integration of m. Depends on the magnitude of m. So it's maximized when m is maximum. When m takes the maximum value, case of full magnetization when exchange interaction is so strong so that all the magnetic moments are parallel and so on. So in this sense is something which is maximized when M is maximized, when the temperature is very low so that you are not activating magnets so that the magnetization is maximum that you can have and so on. And in this sense this is a term which is very similar to the exchange interaction energy. Wow! Can you comment my statement? Now my statement is this is a term which is similar in its form, not in its origin, similar to exchange energy. This seems to be somehow connected to the same thing. Do you remember the Weiss theory of third magnetism? So, if I'm considering M to the third as a means of the analysis of the magnetic energy? Yeah, so in the Weiss theory of third magnetism, which is, of course, was not a theory, was just to say, a phenomenological way for accounting for such a very large value of the cooling temperature. So the idea is that if for iron you have to heat up up to above 1000 Kelvin in order to destroy the magnetic motor, it looks like you have a strong internal field which keeps all the magnetic moments aligned. And in the Weiss theory of permanganate, so you say that there is a sort of internal field, which is responsible for the alignment of the spins, which in the end represent the exchange interaction, but at the time of Weiss theory was the internal field. And so you say that there is an internal field H, how do you call it? I don't know. Vice. The vice field, which was written like a lambda coefficient multiplying M. So that the energy, the exchange energy, the exchange in this theory was written like, Okay, it's the interaction between this field and m, which means minus mu zero, lambda m squared, and d tau. But what about the value of lambda? The value of thousand, okay? And so you immediately see that this term is really similar to this one. But there is a big difference that the lambda factor is missing here. So this means that this contribution is similar to the contribution of exchange energy in the continuum of exhibition, apart from the project is 1000 times low, which means, okay, you can neglect it, forget about it is is there, but it's not very relevant, because as compared to exchange interaction, that will be an energy contribution that we will definitely take into account in the micromagnetism theory. So that contribution is negligible. Good. Let's move to the third contribution, which is gamma. In this contribution, so we have seen, for instance, that in case of a cubic lattice, this contribution will be, the gamma contribution for a cubic lattice will be equal to zero. In other lattices it will be different. And so it is strongly connected to the symmetry of the crystal lattice. So it's an energy contribution that cannot be neglected at all, but it is strongly related to the crystal structure. And it contains the anisotropy of the crystal lattice. In a cubic lattice, if you move along the x direction or along the diagonal of the cube, our situation is different. So if the magnetization points along the direction of the size of the cube or the diagonal, you expect to find the difference, which is coming from the peculiar shape of the capital lambda a tensor that describes how you connect the H-high prime to the magnetization. So this is an anisotropic term that we will find again when we will treat magnetic energy, and especially the so-called two-Ion crystal anisotropy, magnetic crystalline anisotropy. And then we are left with this one. In the end of the day, even though the calculation of the configuration of energy leads to the summation of these three terms, this one can be neglected. This one is conventionally considered into the anisotropy, magnetocrystalline anisotropy energy, and you are left with just the first two, which is the so-called magnetostatic self energy. I will use during this course, not just me, but the international community is using as also the formulation for the magnetostatic energy or the polar energy. Which takes this form here.
\fig{6}{Lecture_2 Thermodynamics of magnetic bodies.pdf}
So far, by solving that problem, we found that expression for the magnetostatic energy, which is minus mu zero divided by two, the integral over the volume of HM dot M md tau. Is the integral over the volume possible? more. Now, here there is a demonstration of something, okay, you can go through that demonstration, it's quite easy, I can just outline it, there's no physics inside, but the result is important in some sense, because this is an integral over the volume, and you can transform now into an integral over the space, the world space, of something else. Let me first comment on the sign of this magnetostatic energy. One could say, ah, my God, there's a minus sign, which means that the energy is negative. No, it's positive, and the reason is quite simple. As we have seen, the demagnetizing field tends to be antiparallel to the magnetization. So this means that this dot product is essentially negative, okay? It's providing you with a negative contribution, and so for this reason, that minus by minus gives you plus. This is a positive contribution. It's a well-defined positive energy, the magnetostatic energy. OK? Now, you can transform this integral in a different way. So I just outlined things. You say, OK, this is an integral over the volume. Yes, but the magnetization is zero, is non-null, just inside the volume. So you can just rewrite the same integral as an integral over the whole space, because outside m is zero, so nothing changes. So this is the transformation integral to the minus sign. And then just you remember that, okay, this is the connection between v, h, and m. v is equal to mu zero h m plus m. So you rewrite instead of m, from this equation, v divided mu zero minus h. Okay? the whole space. And now you say okay, I just distribute this product here, I get two integrals. The first integral is the integral of the whole space of hm dot b in the tau plus mu zero divided by two integral on the whole space of the demagnetizing field of the stray field squared. Strictly, in the field where you find the magnet, the straight field or the free diagonal, demagnetizing field is strictly, they are exactly the same thing. Good. Now what you can see is that this first integral is zero. It's a new integral. And the reason is that you have the dot problem, two fields, and the divergence of one is zero, is the divergence of B is zero, and the curl of H is zero. This is something that we have seen. In magnetostatics, the curl of H is zero, there are no current amounts. And in any case, the divergence of B is zero. Now, with these two ingredients, you just rewind it a bit, because you can say, okay, the divergence of B is zero, you can write B as the curl of A, of this sort of vector potential here. And then you write H dot the curve of A, you use this vector identity, you find here the curve of H is zero because there are no currents, you can't set this contribution, and you're left with the integral of the whole space of the divergence of this. You use the theorem of divergence, which means the integral of the whole volume with divergence of A cross H is the integral of the whole surface, which is at infinity, the whole space of A cross H. But as for finite source, finite magnet, both A and D, they go down at least as 1 over R squared. This integral is equal to 0. And that's the general situation. When we have the integral of the whole space of two vectors, which are characterized by the fact that the divergence of the first one is 0, the curl of the other is zero, and you add that at infinity they go down at least as one over r squared. Okay, this integral is zero. So you get rid of this and you're left with this expression, which means that in the end you can write down the same expression for magnetic static energy in a different way. We have the first possibility, which is this one. The first one. The second one is plus The integral of the volume of hm squared in the tau. Okay? Sorry. Over the whole space. Over the volume minus the zero m dot hm. Over the whole space hm squared. Clearly in the second case, the fact that we have a quantity which is 1, it is well evident. But also in the first case, because HM and M, they are antiparallel. Okay, now we are ready to go back to the result of yesterday. I want to apply this to the three cases that we have seen, okay? And especially in the case of the cylinder. I remind you that we have seen two cases. The first one was a cylinder which was magnetized along the z direction. Okay, the second case was a cylinder which was magnetized along the radial direction. our calculation, so let's say m was pointing like this. The result was that here the demagnetizing field was equal to zero. And the reason is that magnetic charges are pushed, a plus infinity or minus infinity, there is no net contribution. Here the demagnetizing field was equal to minus m divided by 2 by ux, as m was pointing along x, or if you want minus m divided by 2 vector. Okay So probably it's even better to write it like this. Minus m divided by 2. Let's calculate now the magnet's static energy. Let's see, let's take the first expression. Minus mu zero divided by 2, hm dot m, and the tau. But HM is zero HD, sorry, HM is equal to HD, okay, same story. But HM is zero. It's zero. There is no demagnetizing field, so there is no interaction essentially of the magnetization with the field produced by the magnetization itself, because that field is zero. If you move to the other situation, it's not the same. It's not the same at all. Because you find that the magnetic static energy in the second case would be minus U0, the integral over the volume of HM, which is minus M squared divided by 2 in V tau. And so, of course, if it was an infinite cylinder, what we can do is we can calculate the magnetic energy per unit length. Let's assume that we take now a length, which is L. Matteo, any problem? No, no, no. Okay, good. So this means minus mu zero divided by L, minus by minus is plus, m squared, and now we have pi r squared, this was the radius, pi r squared multiplied by L, which is exactly the volume. And this is different from zero. Now, this is the energy cost. Ladies and gentlemen, which is the preferred configuration for the magnetization, 1 or 2? 1. The energy cost is 0. The energy cost in terms of the polar energy. For this same term, it is the same. This was the first statement of my lecture today. but on the other energy, the first configuration is much more convenient, when in the second case you have this energy cost. And what is the outcome to the implication of this? Just because you have a specific shape, there is a clear preference of the system to allow the manifestation in one way or another. This is not written in the exchange system, it is not connected to the drug. This is the reward of dipolar energy. And so, we are seeing what is called shape anisotropy. Chez-Panaisotopy is a consequence of dipolar energy. The term of dipolar energy, of magnetostatic energy, depends now on the direction of the magnetization with respect to a specific axis defining, I don't know, the rotation axis of your system, or I don't know, the cube, the side of the cube of your shape. So, the way your hand handles with respect to the shape of your body makes a difference, a big difference. And this is inherently contained in the form of the demagnetizing tensor. If you assume that you have now the demagnetizing tensor, what you know is that you can always write, if the demagnetizing tensor is well defined, if the surface containing the magnetic body is of second degree, like in case of a cylinder, for instance, like in case of a sphere and so on, like in case of an ellipsoid, in this case you can write that HM is equal to minus N multiplied by N. But if this is true, you can rewrite the magnetic static energy Em like minus mu zero divided by two, the integral over the volume of hm dot m. This is essentially want M with the minus which comes there plus M, M applied to M and beta. What happens when you do this? As I told you, you can always write your demagnetizing tensor in a diagonal form. you can properly choose now your reference frame, which means that the n is always something like n and y and z . All the other components are zero, okay? It assumes this diagonal form. And with this in mind, you can easily prove that We can easily prove that the magnetostatic energy in its general form can be written like this. U0 divided by 2, the integral of the volume of water. Okay, you carry out multiplication, you get Nxx by Mx, then you carry out the dot product you will find Nxx Mx squared plus NyY My squared plus Nzz Mz squared. Okay? Now if this is a general expression of the magnetized austatic energy, you You understand that the anisotropy comes from the peculiar form of the magnetizing tensor. For a sphere, no anisotropy at all, because you have that Nxx is equal to Nyy is equal to Nzz is equal to one third. So in the end you get what? the magnetic static energy will be just mu zero divided by two, and so you get what? Integral of the volume of one-third mx squared plus mz squared, or d tau, which is, for example, mu zero divided by two, m squared divided by three, by four, yeah over the model okay no anisotropy at all the three components are totally equivalent okay but if you take now a cylinder that's not exactly the same infinite cylinder situation is different the tensor is 1 half 1 half and 0 ok this is the tensor this is F this is this expression so when you write in this case the magnetostatic energy it will be minus u0 divided by 2 minus minus is plus and so on and then you get 1 out nx sorry mx square plus 1 out my square plus plus nothing 0 mz square So you immediately realize that. Now if you put your magnetization along Z, you find zero. If you put the magnetization along X, you find something which is different from zero. So using the concept of the demagnetizing tensor, So you immediately find that the form now of the magnetostatic energy contains the shape anisotropy. It's inside. And we can see even a more general situation, which is that of spheroids. Sorry, by the way, I'm jumping back. I'm sorry. If you want to find this, for some reason with the chalk on my fingers it doesn't feel which is not good at all. These things you can find these things more or less here yeah so it is in the first part of this life okay.
\fig{9}{lecture_1 basic magnetostatics.pdf}
So, let's imagine that you have this situation. First case, oblate steroids. They correspond to this shape here. It's like a rugby ball, okay? x, y, and z with the axis of r, b, and c. Okay? The alpha axis, a, b, and c. So this is definitely a second-order surface you can write for defining this. And one now can say, okay, let's write down the magnetostatic energy for this body here. Of course, what you know immediately is that this body has a cylindrical symmetry around the z-axis. So you expect to find that nx will be equal to ny. Sorry, nxx equal to nyy. They will be equal. And then there will be an nzz. And now my question for you is this one. which is the relationship between Nxx and Nzz, which is the bigger. This may be like a mean considering their axes and the inverse proportionality relation. like I imagine that... So for you, an X is larger or smaller? Is larger. Why? Because I have, let's say, less space inside the material to magnetize, so, people are saying that something should be equal, I imagine that the magnetic response points along one axis that is shorter than the other should be written. Interesting. I'm very interested in this. Because, okay, the way you perceive things, the feeling is important in physics, very important. These days, in our lab, there is something which is extremely strange. There is a result that I don't understand. Something that could be if it's true it could be a sort of revolution so i i believe that it's more now don't tell the guy okay my first my experience has been there is something more okay this is true i have to wait some other experiment to see if it's true or not but what makes the difference it's very often the feeling of feeling. I don't really understand what is feeling of love, but with some experience, we start perceiving the feeling that it should be like that. Which means that if someone tells me, how can you say that I would say something like Eduardo is saying, saying something like this. Eduardo is not saying exactly the core of the thing, but I understand the problem in his mind, the idea is a good one. And the feeling is important here in science. Many times the feeling is guiding your choice. After why, after many stories, after some few you understand why. But the feeling is fundamental and you should have a good feeling in science. You know the good feeling is difficult because you can't have the proof. At first sight, you enter a room, you see a spectrum, you say, oh my god, this could be related to that. That's intuition, that's a field, and the first approach is fundamental. Now, let's put what the others are saying in a more quantitative and understandable program for everybody to explain. It's true, it must be bigger, but what's the reason? So, let's try to go back to the general meaning of the component of the demagnetizing tensor. Okay. And we can use also the example of the of the cylinder which is very very representative of the story. Anyway, when you say that HM is equal to minus NXXNYYNZZ MXMYNZ. Okay? This is the way you find things. which means that you can write it down as minus Nxx, Nx, Nyy, Nyz, Nz. So, Nxx gives you the idea in a quantitative way of how the body responds to the application, to the presence of a magnetization along the X. If nx is big, this means that if you have mx, you will have an x component of the magnitude of infinity which is b. If it is zero, this means that the body does not respond with a component of h to mx. So in this respect, now you can understand what happened, for instance, in the case of the cylinder. If the cylinder is magnetizing that direction, okay, along Z, you find that the magnetic charges are so far away, there is no demagnetizing field. So this means that n ZZ is equal to zero, okay? And so you have a general rule, the longer the, that I'll say the size of the body in one direction, the lower the response to a magnetization point in the same direction. In this sense, even though it's not an inverse proportionality, but here roughly say that the body is elongated in one direction, along that direction the components of the diminutive tensor will be small. It's not exactly an inverse proportionality, even though some texts report this, but this is wrong. It's not an inverse proportionality in the end. It's not like one over the side, but it goes like this in terms of inequality, in a qualitative term. Okay? And, in the other case, you have mx, for instance, you have a finite hx, hmx, minus m divided by 2. And this is why. Because magnetic charges are very close each other. So the demagnetizing field, because it is M, the demagnetizing field is not negligible at all. So the smaller the size in that direction, the larger the components of the demagnetizing terms. You see my point? Fabrizio, you okay? Yeah. No, you don't, you're not okay. Now m is constant or over the space? Yeah, of course. m is uniform. It's not, yeah, like also for the Yeah, if m is not uniform, you cannot apply the concept of the demagnetizing tensor. Okay, that's a general assumption. Only in case of uniform m, you use the demagnetizing tensor for calculating the demagnetizing field. That's a basic assumption. If it is not uniform, you must refer to simulation in some way. But, OK, this analytical framework is very powerful because it gives you the way for making some calculations. So in the case of the problem that we have right now, you can run some simulation. And students are doing that extensively. But at some point, I need to write down some equation in which I want to make some modeling of the story to understand the trend, the physical. And for this, you have to make some assumption. In some case, even though M is not perfectly uniform, say, OK, let's assume that I can use demagnetizing tensor just to write a formula that allows you to understand how things are going physically. But the basic assumption is here, M is uniform. Good. So now it should be clear, the story. large size, the body in one direction, small value of the demyelinated ion, so we'll call it along that direction. So coming back here, yeah, I'll cancel this and place it on that space. So in this case, this relation holds true. Nxx is larger than Nzz. So writing down the expression for the magnetostatic energy, we discovered something which is interesting, also called future. So the magnetostatic energy will be near zero divided by two. And so I will have NXXMX squared plus, okay, I have to write NXX, but okay, let me say they are equal. So I will just replace again, MY squared. plus nzz mz squared. Okay, this is the magnetostatic energy in our case. But now I want to say, okay, it's an integral, but if we assume that everything is uniform, it's not really an integral, is a multiplication by the volume of the body. Do you see this? It's really the question by Fabrizio saying about, okay, m is uniform, so in principle all this integral is just an integral of the constant, okay? Nothing changes here. So you can write down the magnetic static energy per unit volume, that will be near zero divided by two, and then you're left with this expression here. And now, just with a very simple mathematical tree, we can find an expression which is more understandable than this one. But the trick is that I will add here nxx by mz squared. Because if I make this addition and I subtract it, here I will find nxx by m squared. So by doing this, I will find out nx as mx squared plus my squared plus mz squared, in the end is m squared, plus, now as a hundredth term I have to subtract the same, nzz minus and x, x, m, z squared. Okay? And now, the interesting part of the story is that, hey, this is just a constant, okay? What does it mean as a constant? So, as it contains m squared, independently on the orientation of m, and it is a very different value of mx and imz, this is a constant. And a constant for an energy term that may cancel. So, energy is defined without peculiar considerations, it's not really interesting. a constant plus the zero divided by two and that's that minus and xx and z squared okay so it seems that the unique dependency that you have is a the dependency on the Z component of the magnetization. So if M for instance is pointing like this, and theta is the polar angle, it depends on what? M squared, why? Because N squared on the angle is left. Okay? And now, what about the sign of this term here within brackets? Is it positive or negative? It's negative. Okay, this is the expression for a uniaxial anisotropic term. What do I mean by uniaxial and rhizotropic? What is clear is that here everything depends on what? On the angle that the magnetization forms with the z-axis. And if this coefficient is negative, so now you can make a plot of the magnetostatic energy as a function of the angle theta from 0, pi and 2 pi, and you will find that, okay, this is negative, okay, it's a negative value, so this means that it was not a very good choice, this one, because of the problem of the blackboard. Okay, as this coefficient is negative, for theta equal to zero, cosine of theta is of course one, okay, so you have what? You have a negative contribution, which is, it's a minimum for this function. It's minus the cosine square of the angle theta. starting from this point here. When you have that, sorry, you have phi over 2, you have the cosine of pi over 2, which is 0, so you are here, and for pi, as you have the cosine square of theta, we're going back here. So this is the the profile of this shape I resolved. You have two minima, one is here, or zero, and the other four pi. And you have a maximum which is at 90 degrees. What does it mean? It means that this configuration is a minimum, but also the other configuration is a minimum. And the maximum is for m in the equatorial plane of this prolate ellipsoid. Any questions? No. This is the uniarcal anisotopy that you can have. And the better case of oblate steroid is a rugby ball. If you move now to the other situation, is that of proletal and so on. spheroid or ellipsoid. In a prolative spheroid, you have a sort of disk. So it's lumped something in there. X, Y, and Z. Again, A, B, and C. So A is equal to B, and is larger than C. Please. Sorry? Yeah, it's exactly the same, sorry, sorry. Yeah, sorry. This is oblate. Yeah. And collate. I can tell you how to, in my mind, when I have to remember this distinction, I use a strange story. You know, the blade is an older of monks here in Italy. And usually they are not very very... they love food. So that's the reason why there was something more of a blade than the prurates or what. But that's my way. With all due respect for these monks. Nevertheless, I made a mistake. Anyway, okay. In this case, you have a different relationship between A, B, and C. And, of course, this implies that when you write down the same expression, Nxx would be lower than Nzz. The mathematics is the same. Now, if I have to write down the magnetostatic energy per unit volume, I will find exactly the same expression, because the symmetry is the same. Now, if I rewrite this, this will be mu zero divided by two, nzz minus nxx by the cosine square of the angle theta by m square. but the difference is that now this two pi will be positive and the implication is that in this case zero pi two pi is the same here, so the magnetostatic energy, the unit volume will be a cosine square of the angle theta. So in this case you have this. So this corresponds to this. Yeah, so this is like this, like that. Then what is now the message? You have a disk here, and now there's a preferential orientation magnetization in the equatorial plane. This is called, this is Yoni Aichalagy-Gyatopi. This is called easy plane and this is a big one. Why? Because, okay, now the easy axis here is the z-axis. What does it mean easy axis? It's the axis along which it's easier to find the magnetization. It's the easy axis. The axis along which the magnetization loves to stay. Here you have an easy plane, so the magnetization allows to stay in the plane.
\fig{10}{lecture_1 basic magnetostatics.pdf}
In the slides, you will find that in case of the steroid, there is a calculation, this quite famous paper by Holtzburg, which calculates the XF4 of the components of the demagnetizing tensor, assuming that A, B and C have a well-known value. This is just a detail if you want. Apart from that, I would like to point out that this second example is very relevant for practical and technological applications. The reason is that all the technology of the semiconductor industry is based on planar technology. Everything starts in the wicker and proceeds with a particular deposition of layers, thin film. Now the question is, which is the shape and the It's not a miracle. It's a miracle. But it behaves like a prolate or a blate. It's very similar. No blates here. So what we expect to see is what we can't tell. Something very similar to the case of the oblate spheroid. And this is the last message of this lecture. But it's very relevant because we will deal with Fin-Fins. So Fin-Fin means something like that. This is a cross-section and if you want, it extends like this, okay? Now the question is, where do you expect to find the magnetization? How it could be oriented, just considering magnetostatic energy? Which is the easy direction? Out of plane or in plane? Ah, 50-50. As today is the election day, so who will vote for out of plane? of plane. Raise your hand. Nobody. For in plane, a lot of preferences for in plane. It's in plane because the thin plane is very similar to an oblate. But also physically you can understand why. Because let's imagine now you have two options. In the out of plane option, if M is pointing like this, we have a lot of magnetic charges appearing. We have positive charges appearing on, so this is a cross-section. So positive and negative. And this distribution of charges creates a lot of free field, a lot of field, hm, filling the space so that the magnetostatic angle would be large. Okay? The other situation is this one, in which the magnetization is starting with this, and you have negative charges here and there, but they are very far away. Thin film means that you have an 8-inch wicker, 200 mm diameter, with a thickness of 10 mm. Nothing has converted to infinite distance to the plate. And what happens is that the magnetic charges appear and the edge is not the grateful. But here, so it's an infinite distance, so it brings me off the magnetizing field. So here, essentially, you have HM, which is almost zero, okay? And this means that the magnetic energy tends to be minimized in this way. That's the reason why by shape and authority, we most likely behave in a way that's consistent with the preferential orientation of the magnetization in the thin-thin-thin. That's a fundamental achievement of today's science. And if you have to write down the equivalent demagnetized tensor, some people would say, no way, it's not the second-order surface, this one. Do you agree with me? So, basically, you cannot apply this to the theory of demagnetized tensor. Briefly speaking, yes, but let's assume that we want to write down something which is in the framework of the Mario tangential. And so the n should have brought us which expression? Zero, one. Zero, one. Zero, one.
\fig{11}{lecture_1 basic magnetostatics.pdf}
\fig{12}{lecture_1 basic magnetostatics.pdf}
So good morning everybody and let me summarize what we did during first lectures. Yeah, probably this is not the good. Where is it? Here. Okay. So what we are looking at, a fantastic blackboard. So magnetostatics. no dynamics here we will see afterwards what happens when you start exciting spin waves or magnetic rays or something like that but for the time being it's a static behavior so we are what we are mainly interested in is to understand what happens when you do this experiment you place a magnetic body inside a coil you apply a current so you apply a field and you want to understand how the magnetic body reacts to the application of the external field which kind of configuration can have inside which kind of behavior because this opened the way to different things that you can do with magnetic material but the basic problem is this one but is magnetostatics and it's static which means that has we as As we have seen that the curl of HM is equal to zero. HM stands for the demagnetizing field. It is the field produced by the magnetization itself. There are no currents around. So the currents are elsewhere producing the field that you are using now for stimulating your magnetic body. But there are no currents included in this calculation here. the curl of H is equal to zero, so that we can introduce now the concept of scalar potential, and in the end of the day, we wrote this equation, nablus value of U is equal to the divergence of M. Divergence of M, which behaves like a volume density of magnetic charges, and also seeing that some surface magnetic charges are appearing at surfaces and so on. And solving some magnetostatic problems, we found out that, okay, there is what we call now shape anisotopy, which means that for a cylinder, for instance, okay, when you consider the possible direction of the magnetization, you immediately discover that not all the direction are equivalent because there is a shape which is not submitted. It's not a sphere. So essentially that configuration for M is different from this one. And what is the difference? The difference is that, okay, the stray field that you have is different. Here there is no demagnetizing field. HM is equal to zero. Here HM is different from zero. And the reason is that magnetic charges here that are very far away, here they are quite close. But in terms of energy, we discovered that when you consider now magnetostatic energy, which is dipolar energy that we calculated last time, we discovered that here magnetostatic energy is zero. E ham is equal to zero, and here E ham is larger than zero, is a quantity which is definitely positive here. And so you say, okay, this is not the favor state for the magnetization. Magnetization tends to be aligned along the long axis of your body. Okay? Shape anisotropy. And when you say magnetostatic energy, so essentially you can write down different way, minus mu zero divided by two integral of the volume hm dot m. in detail or plus you do divided by 2 the integral over the volume of H M square in detail okay the shape and is not the piece something which is fundamental the last achievement of last lecture was the consideration of what happens in a thin film thin film is fundamental because for planner processes which are the basis of all the technology that you are we have right Right now, so far, the future, I'm not sure it will be like this, but so far it's right there. So you always start with the position of thin films. And the main result, which was the final result of Tuesday, last Tuesday, is that in a thin film, the magnetization loves to stay where? In the plane or out of plane? In the plane, due to shape and result. because if the magnetization is in the plane you minimize the stray field and so you minimize according after you this is over the whole space this is in the volume is on the whole space you minimize the magnetostatic energy okay it's a good new this one my feedback is working today you know I brought that and during the discussion I say okay well there is something wrong there without seeing which is good okay it is a personal check it's a personal check for for the teacher you have always to check if your brain is working nothing to nothing to dealing with magnets but is dealing with my my safe condition safety condition good okay this which was my last result now today we are discussing the thermodynamics of magnetic material why I'm discussing thermodynamics you know move to thermodynamics when the system is too is too complex to be tackled as I'll say an assembly of individual particles or individual dipoles as in this case. So our real problem is this one. Let's imagine that you are now in our lab that you will visit at the end of the course and now you have your magnetometer which is made of a big coil producing up to two Tesla and then you have a magnetic body which is your sample that you put it inside the pool in between the poles one Jonah they need the seat take a seat and see you but after this please close the door and do not interrupt the lecture and along bye bye okay so what I'm saying is that so So now you put your magnetic body inside the poles of this magnet. And so inside your body, how many dipoles do you have? If you have iron, so iron is a BCC material. If you take iron, it's like, yeah, good. And there is another one which is placed here. And this is 2.87 ohmstrom. now if you have the typical situation which is a material we have a piece of material which is 10 millimeter you can't take millimeter sorry and the thickness could be I didn't know hundred nanometers something like that which is how many out of the oven side how many dipoles so that's a problem that you cannot really tackle you cannot really manage without resorting to the thermodynamic approach which is a way for considering a system in its complexity but as a sort of black box we're not considering the coordinates of each dipole we are really moving to the consideration of some thermodynamic potential that allows you to understand what is going on which has the tendency and so on so that's our problem so So that's our situation. There are too many, I'll say, dipoles, magnetic dipoles over there, so that we have now to use the thermodynamic approach. And the thermodynamic approach is based on the fact that you must be able to write some thermodynamic potential, and typically the internal energy, the Helmholtz free energy, the Gibbs free energy, and so on. So for doing this, you must really know how the system exchanges energy. That's the basic concept. So let's assume that this is your system, thermodynamic system, which is placed inside now the poles of the magnets. So imagine that you have, these are just the poles of your electromagnets producing a field here. You have an applied field, HA, which is produced by these two poles of the electromagnets. So there's a current here flowing somewhere. And now you have your sample which is placed here. And this is your system. Now the question is how this system exchanges energy with the environment. This is something that you need. So essentially there are two ways. Either it exchanges energy by a heat transfer, that is something that in principle you know. But now the question is how the system exchange energy by a sort of work, which is not in this case a mechanical work. We are not applying a force. We are applying a field. But the field drives now what? Drives the behavior of the magnetization. Magnetization will tend to align to the field. And now the question is, which is the magnetic work? So what happens when you apply a field, which kind of magnetic work you do on your system? This is something to be taken into account when you write down the first law of thermodynamics. You write down that the variation of the internal energy will be equal to what? dL plus dQ. Okay? so the internal energy increases because okay you have any net income of energy arising from heat and work but now which kind of work is the magnetic wall so now our problem is this one how can we evaluate the magnetic work that's the main problem to be solved during the first part of this lecture because when we have these then we can move now to the thermodynamics but we we first need an expression for the magnetic world without this we are blocked we cannot continue.

\chapter{Thermodynamics of Magnetic Bodies}
\fig{2}{Lecture_2 Thermodynamics of magnetic bodies.pdf}
This is the introduction now let's move now to the to the way you you perform the calculation for the magnetic work. Just to keep a trace of something that will be visible in the slide. You say okay, essentially the problem is this one. Calculate the total work needed in order to create a given distribution of magnetic field H in a region by a magnetic material. It's exactly this problem. You send the current, you want to create a field H, but in that region, you have a magnetic material. Now, what is this work? It's written as a sum of three terms. So delta L is the sum of the first term, which is delta L prime, plus delta u m plus delta l star. So let me comment on the meaning of these different terms. Delta l prime. The work done by the magnetic power supply to set a given H field, H high. So what is HA here? Look at this definition. The field that you want to set in the end is the result of two actions. The first one is the action of the carbon flow in the cores, which is producing now HA. It's the field produced by external carbon. It's that connected to the Maxwell equation that will say that the curl of HA is equal to J, the conduction current. Ampere low, essentially. You have a current, and you calculate the H field produced by. But there is also a field which is produced by what? by the magnetic charges appearing on the magnetic body as a result of the fact that it assumes a magnetization, a net magnetization, which is not new. So you have these two contributions. The contribution to the magnetic field H coming from the currents in the course of the electromagnet and the contribution coming from the magnetic charges, the magnetostatic field, the stray field that's obeying the equation reporting over there. At the very beginning, I was commenting about that with your colleague, what's your name, Bassett, with Bassett. He was asking me, but okay, in your calculation, you are not considering how you reach this M. Yes, in my previous lecture, I started from this assumption. Let's assume that our body is magnetized with a uniform magnetization. And then we calculate the straight field produced by magnetization. I was not considering the reason or the path leading to this magnetization, okay? Today, I'm doing something different. Okay, now we really look at the real world. In the real world, you have a piece of material. You put it inside an electromagnet. You apply a field sending a current. And now let's see what happens, okay? Now situation is the real one. But the magnetostatic field obeys those equations. Okay? The field produced by the magnetic charges obeys those equations. So we have two fields, this one and the other one. And the equations are different. Okay? For this one, the equations are those reported in the left top corner. And for the other one, the equation will be this one. Okay? this is the field now you can say where is the work which is needed just to create H8 it will be the work needed to create the applied field in absence of the magnetic body okay which means let's assume that you have your electromagnet to create H the applied field you you need some energy some work because we have to create this and if you start from a region which is hem which is a empty in the sense that there is no field now if you apply a field you know that there is an accumulation of electromagnetic energy which is the integral of the volume of H dot M okay from from Maxwell equation so there is a work to create to accumulate this energy just because you are creating a region which is a static field. And this is L, delta L prime. On top of that, there are two other contributions. Delta UM is something that you know perfectly. Now it's UM, also written like EM, is the configurational energy due to dipolar energy. It's the magnetostatic energy. Because when you apply a field and you create a configuration of your dipoles inside your body, you have now dipolar energy. And this dipolar energy is an energy cost to be taken into account. So the coverage that you are sending in the electromagnet, the energy provided by the power supply, goes into what? It goes also in some dipolar energy, which is stored in your system just because of the configuration assumed by the different dipoles. And this term is something that we have already calculated. We know the expression principle. But it's a part of the story. And there is a last term to be considered, which is not the configuration of energy. Because okay, also when we did the calculation for the magnetostatic energy, our assumption was, let's assume that the configuration is a peculiar configuration. Let's now calculate magnetostatic energy. But the question is, the way you reach that configuration will strongly depend on the nature of the material. If you have a ferromagnet, a paramagnet, or a diamagnetic material, an antiferromagnet, or a ferrimagnet, so the energy needed in order to create a given configuration will be different, because the exchange interaction will be different in the different cases. the kind of interaction will be different. So you understand that, okay, there are three contribution and the last one is that considering the nature of each material. This is a configurational energy independent on the nature of the material. A ferromagnet, a diamagnet or a in under Tesla, they will all have the spin all around. and the magnetostatic energy would be exactly the same, but the energy needed to create that configuration would be extremely different for a ferromagnet or an antiferromagnet or a paramagnet. For a ferromagnet, it would be essentially zero. If you have hydrogen, in some cases, the acoustic field is one ocean. You just apply a tiny field and boom, they all switch parallel. Finished. End of story. For an antiferro-magnet, you have to move too many teslas. So you need really a lot of energy to create this spin flip, the total spin flip transition, which will really reverse one of the sublattices so that you get a parallel alignment of the two sublattices. So you understand that the configuration energy, the magnetic static energy could be the same, but the work down to achieve that configuration could be extremely different. And that's the reason for this subdivision in three terms of the total work that you need. There's a portion of the work needed in order to create the field that you will have in absence. So you want, for instance, to put your body in a field of, I don't know, 10 milli Tesla, okay? You need a wall to create this 10 milli Tesla. But then when you put inside the magnetic body, okay, you have to now consider which is the alignment of the spin that you reach, giving rise now to magnetostatic energy, and to reach this configuration, you have to spend some work which depends on the nature of the material, ferromagnet and ferromagnet zone. Is it clear? This is fundamental for the next part. Good. Now, we have to carry out this calculation. and my static energy something that you already know so I skip this part wonderful no okay let's see here and we move to this.
\fig{7}{Lecture_2 Thermodynamics of magnetic bodies.pdf}
This is our problem from the point view of the electric circuit which is used in order to now to to send the current in the electromagnet which is represented by this coil inside the coil you have your magnetic body in green here and there is a power supply. Power supply is this one. It could be a battery, it could be a power supply but in any case do not forget that this guy is able to send the current because it produces what? An accumulation of charges here which produces a field which is a field driving the current here. And inside, you have the electromotive force. There is what? It's not the electrostatic field. It is the electromotive field that's closing the circuit in the battery, which is something which is connected to electrochemical reaction. because in the end you have a positive pole but here and the positive pole will send the current from here to here but inside it's the other way around and this is possible just because you have an electromotive force which corresponds to not an electrostatic field it's an electric field which represents now another kind of interaction typically in a battery is really due to the chemical potential that you have, the afrochemical potential that you have inside. Okay, is E prime, okay, is the field which is responsible for the motion of charges inside the power supply. Okay, and now, as announced before, we have two fields contributing to the total field. So, we have the applied field HA in absence of the magnetic field, which obeys the standard Maxwell equation in presence of Carn. Yeah, because the curl of HA is JA, the applied current flowing in the electromagnet. Okay? And you have now the magnetic material. For the magnetic material, clearly when you have the magnetic material, you move from BA, which is 0 HA, to a more complex situation in which you have that H, is the sum of H applied plus HM. The demagnetized field gives the magnetic charges appearing because you are magnetizing now your body. And as a consequence, you see that also B will be mu 0 by HA plus HM plus HM. And HM and M, they are connected by the equation of magnetostatics, the equation on the left top side. of my blackboard. This is the divergence of HM, which is equal to minus the divergence of M, which in the end leads to the number of square root of U equal to the divergence of M, and the curl of HM is equal to zero. Okay. And now, as we're dealing with energy in Maxwell equation, which is the theorem, which is handling the story of energy, is pointing theorem. Okay. If you want to investigate now some energetic aspect with electromagnetic fields is the point in theorem, which is describing this kind of energy balance. So you have to say, okay, now I have my system. Let's assume that my system is the whole system described here. I take now a surface, which is really containing, in its interior part, the whole circuit, the electromagnet, the magnetic body, the power generator, and so power supply and so on, okay? And this is the surface S containing a volume omega. And now you write the pointing theorem applied to that surface, okay? And now you could say, okay, now I have this system. What happened? I can now have an accumulation of energy inside this volume and the dissipation of energy inside this volume depending on what? Depending on this first two integral. The first one is the time derivative of the electromagnetic energy stored inside that volume. And it's, I thought, no, it's not written here. Yeah, but okay. It's, well. You know that the electromagnetic energy how to call them e h is equal to what sorry is the integral of h dot b and d tau over the volume with one half in front this is the standard for the little static let me feel these the energy density is one of the dot he for netting in these of H okay parallelism is complete so you have this energy here now if you imagine that you integrate over the volume and now you take what the time derivative you will find something like this okay the one out cancel out with the fact that you you have to have this product here H dot B So we take the derivative of the product, you keep the first constant, you multiply by the derivative of the second, and so on. So in the end, you find that these two will head up, and they will not cancel out. This is the time derivative of the energy, electromagnetic energy contained in the volume. And then you have another term. So this could be positive or negative, depending on the fact that you're increasing B or you're decreasing B. So if you are magnetizing, if you're ramping up the field from zero to, I don't know, 10 milli Tesla, so this energy will, if you're ramping down, if you're sweeping down your field, of course, you will see a decrease of this. And there is another term here which must be taken into account. Do not forget that when there is a time variation in general of the field, there are some electromotive forces, and there could be some eddy currents. What does it mean? So if you typically have a conductor, and iron is a ferromagnet, but it's also a conductor, it turns out that, okay, you ramp your field, there is a time derivative of the field, you create some electromotive forces inside, and you start seeing some eddy currents. That's the reason why, for instance, in a transformer, you never have a block of iron, but you have many lamellas with some insulating layer in between. reasons that you want to kill eddy currents otherwise your nucleus or your transformer will heat up heat up and so on so okay in principle you can have eddy current in your process of magnetization in terms of the energy that could be some dissipation of energy into thermal energy which is coming from eddy current and you see the typical structure is J squared divided by Sigma which is sorry which is a power because here of course we are talking about the time derivative of energy contribution so this is essentially a power dissipating over the volume and this should be taken into account so which means you have the electromagnetic energy that could increase or decrease depending now the fact that you have now an increase of the field or decrease of the field but at the same time you can dissipate some energy into some heat good and now you say but now which are the contribution so which is the energy which is now which is the source for all for the increase or decrease of the energy point in theorem states that when you have a surface essentially you have this contribution here so the point in vector is e cross H when you integrate over the surface you obtain when the s is oriented outwards so you find the total flux which is going out from the system as s now you need something which is introducing some energy in your system you're putting in front a minus sign here minus this integral okay and then you have something else which is connected to volt so now what I'm doing essentially is to consider that I'm considering here essentially the magnetic energy okay because interesting in the magnetic work not in other kind of works so essentially what I'm saying is that okay but here inside there is a source of energy which is a chemical source of energy for yourself okay which is not automatically taken into account in the in the point in theory. So I have to consider that there could be some incoming flux of energy coming from what? From the electromotive force, from the power supply. In other words, when you have... So our electromagnet is not based on batteries. It's based on real power supply. But the rector, Donatello-Sciutto, pays every month the bill of NL A2A and so provider of electricity okay so in order to say to have this field to activate so the students are running the electromagnets they don't know exactly how much it costs but okay but it costs because you have to pay A2A in order to have an hour at the end so we have to consider this energy contribution so if you are increasing the the field between the poles of the electromagnet, this is because you are spending some energy coming from the power supply. And how can you calculate this? So if you refer to this simple case where you have the electromotive field here, you have J A dot A prime, which is exactly the density of power provided by the power supply. So you integrate over the volume here, and you find essentially dl over dt, which represents now the work done by the power supply. Good. And now you say, okay, but if I assume that this is an isolated system, which means the surface can go to infinity if you want. So there is nothing outside. So if there is nothing outside and this goes to infinity, essentially there is no net contribution from the flux of the pointing vector here. So you can forget about this term here. And another assumption in this calculation is that, let's assume that magnet, magnetic body, which is insulating. So it could be, for instance, heat, heat from iron garnet, so that you can neglect eddy current. So you neglect this term, you neglect that term, and you find something which is quite easy. I state in that, essentially, the rate of accumulation of electromagnetic energy depends on what? Depends on the rate of pumping of energy from the power supply. You see the point? So if you accumulate electromagnetic energy, this is because you are using a power supply. the energy provided by the power supply goes into some electromagnetic energy. So this is the formal way to derive this, which is good under this assumption. Okay, so in the end you say, okay, I have this work, delta L provided by the power supply, and this work is responsible for the accumulation of energy. So essentially you are here, so you say that, okay, delta L which means you just multiply this by delta T which means that you are getting read essentially of this time derivative of B with respect to the time. You find that the delta L is this one, this expression. And let me write down this because this is very relevant. which means that the delta L, which is exactly this guy here, it's exactly the same, what I'm really interested in, so the magnetic work that I have to do, is what in the end can be expressed as this integral here, the integral over the volume omega of H dot delta B. in detail or d cube R same story good now from now on there is a little bit of mathematics but I'll try to give you essentially the physical meaning of the mathematics then you can also go into the details by yourself the slides out there so you have this this expression here and one should say okay but if I want to understand what happens at the level of my body not my body of the magnetic body this piece of the sample which is placed inside the pulse of the magnet so I'm not really interested in the energy needed in order to create the applied field H a I'll say in principle this is something which is not really entering into the thermodynamics of the magnetic sample that I'm considering. You see my point? So the energy needed to create the field HA, this one, that I will have without the presence of the magnet. It's not something entering in the first principle of the thermodynamics for the magnetic body corresponding to my sample. So I have to create this field, HA, but what I'm interested in is the effect of HA on my body. So this means that if I want to calculate now the magnetic work, the DLM, I have to disentangle here the delta L prime. So I'm not really interested in this term. So this means that with reference, to our problem, which is the thermodynamic problem we have to solve, pay attention that this is exactly the delta Lm I'm looking for. So the first thing I have to do is to take out delta L prime. This is not something entering in the energy balance of the first principle of thermodynamics applied to a system which is the sample, the magnetic body. Is it clear? Not really. Are you convinced? Yeah, more or less. It could be. There are these kind of thermodynamic considerations that are always a little bit tricky. But you know that in thermodynamics, it's always critical how you define the system. So in this case, my system, for the need, for my need. What is my need? In the end, I want to find out the machinery, the methodology, which is capable of predicting which kind of magnetic configuration will be found in my sample when I apply a given field HV. That's my problem. Because if I know this, I can predict what happens and then I can engineer the body, I'll say material, the shape, and so on, in order to obtain a given configuration, which is that needed for a magnetic field sensor, for, I don't know, a memory, for whatever you want, whatever application you need, for a conduit for spin wave propagation, but okay, that's a problem. I want to know how my sample reacts to the application of the instant field, A-train. In this sense, my thermodynamic system is this one. Okay? This delta L is the work by definition needed, if you go back to the previous slide, is the work need, oh sorry. I know, okay. needed to create a given configuration of field, total field, H, which is the sum of HA plus HM. Now, so this is the general problem. I cannot say that there is not this. There is also the need for creating HA, but for my specific need, for my specific purpose, purpose the system the interesting system will be just the magnetic body the magnetic sample so that in the end my the magnetic wall to be placed in the first principle thermodynamics for this one is not including Delta L prime so I have to take what Delta L minus Delta L prime is it more clear now I don't I don't remember your name, sorry, Matteo. Matteo, is it more clear now? Good. So that's the reason why the lecture in presence makes more sense because without seeing the eyes of Matteo, I couldn't be conditioned to better explain something that was not clear. That's probably the unique reason for being here today morning.
\fig{8}{Lecture_2 Thermodynamics of magnetic bodies.pdf}
So this means that now I have to carry out this calculation delta L minus delta L prime. Okay, what I'm interesting in is delta L, yeah, let's see, okay. Delta L M which is delta L minus delta L prime. And what does it mean? This means that it's really the difference between H dot delta B, where H is the total H, HA plus HM, and B is the total B, mu zero H plus HA plus HM, minus just mu zero HA dot delta HA. okay the two situation without magnetic field magnetic body and with magnetic body and now there's a little bit of mathematics you go into it but in the end of the day so look at this line this line here okay you will find out that at some point there is this expression for the Delta LM the real magnetic world we are interested in. The first integral is an integral over the whole space of delta HA dot HM. Now, here we can apply a property that we have seen yesterday, last lecture, which is, okay, HA is a field whose divergence is zero. Why? Because HA is the H field in vacuum. so in vacuum B is proportional to H as the divergence of B is zero also the divergence of H a is zero we are in vacuum okay in vacuum also the divergence of H is equal to zero that just proportionally coefficient is new zero the vacuum permeability now for H M on the other hand you know that as we we are in magnetostatics, the curl of H, M is equal to zero. And now you have the situation in which you have the integral of the whole space of the dot product between two vectors, two fields, vector fields. For one, the divergence is zero. For the other, the curl is zero. In this case, it's zero for a property that we have seen on Tuesday. So, okay, this is zero. Forget about this contribution. and then you find here an interesting integral which is mu0 hm dot delta hm in the tau my god what is this? look at this expression for the magnetostatic energy that we derived on Tuesday so clearly if you take now the time derivative of this one you find exactly this expression here. Time derivative multiplied by delta T. So because now we are no more talking about powers. It's our energy, delta L. The delta E is a variation of energy right now. But you recognize that this guy here is exactly what is the change, is the delta of the magnetostatic energy. Which makes sense, of course, so the magnetic world should contain a part of the configuration energy that you are storing in your system just because you are changing the relative arrangement of the direction of the spins. Okay? Good. And then we have another term, which is this one, which is nu zero d integral over the whole space of H dot delta M. where H is the total H, it's not the applied field, it's H dot delta M. Okay, so this is a relevant result that I have to report here. Delta E M plus mu zero, the integral of H dot delta M and d tau over d O space. okay and we are here it's exactly this expression which is now inside this red box okay now there are some mathematical manipulation that you can carry out so you can say okay but this integral is over the whole space yeah okay but over the whole space M is zero so it's zero so clearly you can replace this whole space with with the Omega okay the volume of your body okay this is nothing but just to mention and now there is another interesting story that you can write down Delta he am not like this but in a way which is slightly slightly different so instead of using so this expression here is connected to this expression for the magnetostatic energy which is this one okay but you know that you can also express calculated magnetostatic energy according to the first formula here with the minus minus mu zero divided by two the integral of of H dot M, H M dot M. When you do this, this means that instead of that expression for delta here, you write this one, minus mu Z divided by two, and then you have what? H M dot delta M in detail. And now. Yeah, I'll say the proof of this story is essentially this one. So starting from this expression here, delta of minus mu zero divided by two, the integral of hm dot m in the tau. If you now apply the chart that the delta is the delta of the dot product, you will find that you have the sum of these two integrals here, in which we have HM dot delta M or M dot delta HM. And now for a reciprocity theorem, but okay, you can easily understand this. Let's assume, for instance, that M and H are proportional, which is the case of what? The paramagnet or the diamagnets and so on. So if you are just proportional, it's immediate to see that these two integrals are exactly the same because now you can replace M with chi by H. And so you will find H dot delta H. And the same will be found here. For a ferromagnet, it's slightly more complex. But OK, in the end, this is the same story. But the main message is that due to this reciprocity theory, in the end of the day, the delta of the magnetostatic energy can be written like this. two times mu zero divided by two hm dot delta m it's a different way for writing the same expression but with this in mind you can say that okay now it's not fantastic but okay it will work delta em so the variation of the magnetostatic energy can be written like this minus mu zero the integral over the volume of hm dot delta m in the tau plus mu 0 h dot delta m in the tau okay but now you can write a single integral which will become mu 0 divided by 2 the integral of what H minus H M and what is H minus H M you remember you don't forget that I can write it here H is equal to H a plus H M the field applied in absence of the magnetic field plus the magnetostatic field so as here I will have what H minus HM Delta M in detail what is that is H a the field apply okay in absence of the magnetic field which is nice because when I use my DSM vibrating sample magnetometer that you will see in in the laboratory, I have a calibration. I know that when I send one amp, I have, I don't know, 100 millitesla, okay? When I put my sample inside, I don't know exactly how many, so millitesla, of course, is mu zero H, A, okay? When I put my sample, okay, things will change, but what I control in the end is the current, and the current is directly connected to HA, So it's good to have HA as a control variable for my experiment. Because HA is connected to JA, and what I control is JA. I don't control H. H depends on the response of the material. My control variable is HA. Do you see my point, Alessandro? It's the current, okay? I control the current, then the system responds and so on. But is the current something that I control? So it's good to find, good to know that in the end, this is HA. Good. And this means that, sorry, for the delta Lm in the end, I can also write down this expression here, which is this one. I'm just rewriting it. Mu zero divided by two. Now, is there, no, there's no divided by two. Divided by two is just in my hand. I was looking why I was writing divided by two. That doesn't make any sense here. Okay. Okay. Is mu zero the vacuum permeability which multiply h i dot delta m and d tau over the volume. Good. End of story. This is the expression for the magnetic work. And now we take a break because now it's a good time. Before this, it was a little bit tricky to stop, but now we can have a coffee. If there is a coffee machine around. Okay. So essentially, we can say that from the first part of this lecture, we know now how to calculate the magnetic wall, and the expression is written over there. Just a few comments and then we move to thermodynamics. Okay, more or less it's something that we already, yeah.
\fig{9}{Lecture_2 Thermodynamics of magnetic bodies.pdf}
Okay, so essentially it's exactly what I was saying. That's an expression which is relevant because HA is the control variable. So the external variable that you can control and the magnetization is in the internal state of variable of your system Okay, you like in case it is like in case of in hydrostatic pressure the cylinder which you have a gas What you do is that you apply now a pressure and you modify the volume. Okay same story and We will review in a while this story of extensive and intensive variable conjugate variable. Anyway, the delta L star, just a comment about there. Delta L star, you remember that in our definition, delta L star is the work which is associated to the specific nature of the material, paramagnet, diamagnet, ferromagnet, and so on. So one Some could say, but okay, let's have a look to this work. So by definition is delta lm minus delta em, okay? Which means that essentially, have a look here, is this expression here, this one. Where is it? Okay, good. this is the expression for the Delta L star okay Delta LM which is the sum of everything the sum of Delta you M or Delta M and this one now it's Delta L star now which is the form of this Delta L star that's quite interesting so essentially you have to calculate this okay well H is integral H dot Delta M and now the problem is H you don't know very well what H is because H is H a plus HM which depends now on the response of the piece of material which developed the magnetic charges and so on. So a practical way for considering this story is to say okay now let's use a peculiar shape of your magnetic body. Let's take a cylinder, an elongated cylinder. Why? Because for an elongated cylinder as we have seen before, so we know that in this configuration the magnetostatic field is essentially zero. It's very very small. Okay. So what I mean is that in this case when you have an elongated body you apply field along the long axis of this, HM is essentially zero which means that H is equal to HA. The field that you have inside, the magnetic field, is equal to the field that you would get without the magnetic body itself so this means that in this situation experimentally for long cylinder H is equal to H a so that Delta L star is equal to Delta LM so for a long cylinder what we have seen is that the magnet of static energy is equal to zero so clearly this term doesn't exist which means that the magnetic work the total magnetic work is equal to Delta L star in the sense that there is no configurational energy there is no dipolar energy okay so when you take elongated cylinder the magnetic work is just that associated to the peculiar nature of the material so there is no contribution for dipolar energy that's a main mess so now let's try to see what happens in practice for the different classes of material that you have in mind and let's start from the easiest one which is a paramagnet in a paramagnet if you apply a field in this direction ha it will develop a magnetization that is power and I'll say you know that for a paramagnet you have a large van like curve with some saturation some point that okay it's an an S-shaped curve, but okay, in the central region of the field, you have an almost linear behavior, which could be this one. So when you have an almost linear behavior, and you calculate that integral, that's the integral corresponding to this area, the shaded area here, because it's H dot delta M, okay? So the area is this one, H dot delta M. So the variable here, which is varying, is delta M. Forget about this integral here. So the idea is that now you're making a sweep of magnetic fields, so you're changing delta M. Then you have to integrate over the volume, but you can also assume that everything is uniform in the volume. So in the end, this integral over gamma, it's just a multiplication by the volume of the body. Like in case of a second-order surface where the concept of the demagnetizing tensor can be used in a safe way. so this is exactly the magnetic work which is needed in order to do door what physically what's your name sorry Nicole you change the position so now I have to know your name now there is also another name missing which is that of the lady on the other side so because I know Alessandro I know Matteo I know Nicola and now I have to know so your name Alessia okay now it's okay so Nicola physically speaking so what is now the physical meaning of this wall you have a paramagnet you apply a field you set the magnetization you are spending some energy to do what physically just to orient the dipole to re-enter dipoles of the paramagnet a zero field all the diapoles are randomly oriented you apply a field and we will see how probably you don't know exactly how but I will explain you how at some point you will assist to some orientation of the magnets the iPod they're statistic statistically assume now and non-zero and non-null average magnetization okay but you you need some wall you need some energy in order to orient okay or we end the dipole good now let's move to the ferromagnet which is now characterized by an easterly this loop now if you imagine to calculate now this H dot Delta M when you carry out a sweep you will quite easily understand that in the end if you make a loop from large negative field large positive and you go back in the end the net contribution will be this dashed area do you see it Mattia Matteo do you see it okay and what is the meaning now of that shaded area physical meaning yeah you have to to switch somehow the magnetization okay okay but the physical meaning of these there I like to say something else so let's imagine that you have now a paramagnet and you make a a sweep on large negative to large positive and then you go back which is the net energy needed for this zero because it's a reversible process for a third magnet this is not zero what is the impact it is the physical impact of this okay it's not reversible but this is just theory in practice what happens There is some energy dissipated, so you are heating up your ferromagnet. When you perform a loop of a ferromagnet, the area of that loop represents the energy dissipated per loop. And what does it mean? It means that in some way you have to dissipate this energy. And the typical way you dissipate this energy is by a thermal channel. What does it mean? But if you take a ferromagnet, you put inside your electromagnet, you sweep it, the temperature will raise up. This is the way. So there is a technique to kill cancer, which is called hyperturmia, which is based on the usage of some particles, magnetic particles, stick on cancer cells. then you excite this particle with some, say, AC field, and the fact that there is some energy dissipated per loop is the reason why you increase the temperature of the particle, and when you reach a temperature which is largely about 40°C, you cause now the death of the cancer cell. This is the hypertonia, which is used in some clinical treatment of cancer. But the point is you have an area like this one, and that area represents the energy dissipated per cycle, per loop. And this must be taken into account. And now I have a question for you. Let's move to diamagnetic material. So the magnetic work is negative. Are you creating energy? Think about that and you will give me the solution tomorrow, okay? But do you understand my question? You have a system, you apply a field, you do something, and you have a negative walk. What does it mean, negative walk? You stimulate the system and the system gives you back some energy. Oh, wonderful. Tell it to Trump, it will be very, very... Sorry, very, very glad of that. And Elon Musk too, Elon Musk knows the solution probably. Not sure, not sure. Elon cannot know everything of course, even though it was represented as the person who will save America, not sure he will do. Anyway, the question is, are you creating energy? Asano says no, why? So the hypothesis is not possible. Ah, okay, good. This is a sort of paradox, okay, which should be understood. But okay, to solve this puzzle, you have to go back to the wall treatment, all the stories that we did so far. I cannot give you right now the answer, otherwise there is no suspense. So see you tomorrow for this, and I will be really eager to know the solutions of this small puzzle, okay? but the question is this one good let's move on yeah in principle this is the work done on the system if it is positive is done on the system if it is negative this means that the system is doing this work on the outside on the environment Yeah, no, no, on the system. By definition, it's on the system. Yeah, I don't want the answer today, Nicola. Think about it. Think about it and tomorrow we will have a discussion. So the task of teachers that are stimulating the discussion, so if we close the discussion right now, there's no stimulation.
\fig{10}{Lecture_2 Thermodynamics of magnetic bodies.pdf}
No stimulus, no party. Thermodynamics, okay? This slide is just a summary of thermodynamics. You know everything. Just to remind you the general framework in which we are moving so that we are on the same page. As you know, when you treat now a thermodynamic system, the way the system exchanges some energy, it's always based on contributions which are written as a product of the intensive variable multiplied by a variation of an extensive one. Okay? This is the general story. So essentially you say that, for instance, you have an extensive action on the system, a pressure for instance, and a state variable, it could be the volume. And how do you write the work done on the hydrostatic system? Minus P by dV. Okay? It's the product of an intensive variable multiplied by a variation of an extensive variable like the volume of the system. And so, essentially, when you write the first law of the thermodynamics, you write that du is equal to the work done on the system written as hx, the intensive variable, by dx plus delta q, which is the variation, so the heat entering in the system, the heat transfer. for this system is minus P by dV plus dQ. Don't forget that the second law of thermodynamics is always valid, which means that in terms of the entropy, you have the delta Q is lower than or equal to T by dS, where T is the temperature and S is the entropy of the system. And the minus, sorry, the equality sign stands for reversible transition and the minor dance stands for irreversible transition those have those happening spontaneously okay and now don't forget that we have different thermodynamic potential we have the internal energy you which is a function of the intensive I or X and s the enthalpy which is the HX and S and then through two free energies which are very very relevant especially for chemistry but also for magnetism as we will see. The L-mode free energy F which is U minus TS and they give free energy G which is a function of HX and T. Just to remind you that F is U minus TS and now what is the beauty of these two? The beauty of these two free energies that they are now function of what of the intensive variable X and another intensive variable which is T the temperature and this is very interesting because many of our experiments they are carried out in a condition in which we control the temperature and all chemical reaction typically takes place in an ambient and environment which tends to keep constant the temperature so many things around us happens at fixed temperature. So having something which is a thermodynamic potential which contains the temperature as a variable is really powerful. And what's the difference between X and T? X is a state variable which is an extensive variable. I'm not sure that I correctly said this before. X is the extensive variable, it could be the volume. While in case of the Gibbs free energy, you have a potential which depends on two intensive variables, HX and T. In case of an anhydrostatic system, it would be what? A function of the pressure and of the temperature. And that's very nice, because very often you control these two variables, the pressure by the force that you apply on your cylinder, and temperature because you have a thermostat or a knitting system, whatever you want. You have a chiller and so on. Now, the interesting story is that just taking the F and the G and calculating now the differential of F, the F differential of G, DG, you find out these two very beautiful expressions. we will review something like that for the magnet magnetic problem the df is lower than or equal to hx dx minus s dt and what does it mean? It means that if you fix t and if you fix x df is lower than or equal to 0 which means that if you fix x and if you fix T the system will spontaneously evolve towards states of minima for F because DF will be equal to 0 which is irreversible transition at equilibrium but in case of irreversible transition the unique value for DF is a negative value which means that spontaneously the system goes from that situation of F to this one which is a lower F. And the same for G. If you keep fixed temperature and this Hx, the G is lower than or equal to zero, which means that for irreversible transformation, spontaneous transformation, the G would be negative, which means you start from the I value of G and you end up with the lower value of G. So the system tends to evolve towards a minimum of G, which means that the equilibrium configuration for your system is that corresponding to the minimum of the Gibbs free energy. You have seen this framework, this theory, exactly in chemistry, for instance. In chemistry, we are using the Gibbs free energy as the thermodynamic potential to be used understanding which is now the evolution of a chemical reaction and the end of the story is that you achieve the equilibrium for the minimum of G and G is the good thermodynamic potential to be used when you control what? The two intensive variable for a chemical reaction the pressure and the temperature. where's the case of chemistry or the case of hydrostatic system gas contain the cylinder and so on now what you have to do is to apply the same theory to magnetism okay and now we are ready to do that because now we know how to write this question for the magnetic world which was missing okay because essentially now we will use for the Delta L the Delta L M okay the magnetic wall what was missing so far was the expression for the magnetic wall when I put my firm an 80 body inside the poles of my little magnet and I change age the The change of H is associated to which kind of magnetic work? Now I know the solution. The solution is this one. The integral of mu zero H A dot delta F. And now let's do it.
\fig{11}{Lecture_2 Thermodynamics of magnetic bodies.pdf}
Yeah, so this is the expression for the magnetic work. In the assumption of no dissipation that we have seen before, no eddy current and so on. And now just to simplify the mathematics, instead of keeping this integral over the volume, let's assume that you have a small, small magnetic body inside this body, M is essentially H of say is uniform and M is uniform. So you carry out this integral, H is uniform inside this small magnetic body. And this integral of M in detail is exactly what? The magnetic moment, the finite magnetic moment, not the magnetization. Magnetization is a magnetic moment per unit volume. Capital M is the magnetic moment per unit volume. Now, I integrate M, capital M, over the volume, and I get what? The total equivalent magnetic dipole of my body, which is this small m. so that the magnetic world can be written like mu0ha.delta m, small m, magnetic moment, which is written in the units of the International System of Units, it would be amps by meters squared. It's really a dipole, magnetic dipole. So delta lm is mu0ha.delta m. But this is exactly the form that you are used to see. It's the product of an intensive variable, which is the applied field. It's an intensive variable. And a variation of the extensive variable, which is the magnetic dipole. It's an extensive variable because if I increase the volume, I fix magnetization. I increase also the magnetic moment. So it's an extensive variable, this one. So this is exactly what we expect. And now, what is dU? dU is the variation of the internal energy, which is lower than or equal to dLm, the magnetic work, plus Tds. And now let's move to the F. Yeah, probably now I can write something. So essentially we found that du is here. is lower than or equal to, and then I take mu zero, H, A, dot delta M, plus T by dS. Okay? Let's move now to the Helmholtz free energy, the F. F is by definition U minus dS. Okay? When I calculate now dF, the F will be equal to DU minus TDS minus SDT so when you now use this expression for the DU you can write down that the DF is lower than or equal to the first term is unchanged 0HA dot delta M and in the second term here, this one will cancel out with that one and you have minus S dT. Okay? Now, let's assume that you are fixing your variable. If you say T is constant and also the applied field, not the applied field, sorry, the M is constant, you have that DF is lower than or equal to zero, which means that if you fix the magnetization and you fix the temperature, the system will evolve towards the minima of DF because for a spontaneous transformation DF is lower than 0 so you always go from IF this is F initial and F final will be this for a spontaneous transformation but in condition of fixed M and fixed temperature Now, is this the typical experimental configuration or not? When I say typical experimental configuration, I mean that now you take Rian. Rian is moving in our lab. Did you make the training for the VSM? No. Okay, so Rian cannot do that. Okay, some other student. The student go down with the sample in the VSM. and you put the sample in the magnetometer, you strip now the field, is that student working at fixed temperature? Yes. You have a cryostat. You can fix the temperature. Is the student working at fixed M? Yes or no? If you strip the field, you put inside a piece of iron, are you working at fixed M? Yes or no? Nicole no definitely you're you're doing in Easter is loop M is not fixed you can revert the magnetization you see my point I forgot your name Alessia Alessia okay Alessia you're switching the magnetization service means M is not constant at all so this situation is not that corresponding to the typical experiment in which you set H okay and you do not control M is M is not the variable of control it's not the control variable of your system is something representing the status of your system is a variable of state of your system it's not the control variable you control H or you control T but you don't control M so the L mode free energy is not very useful for magnetism and so you have to jump to the Gibbs free energy because now if you move to the Gibbs free energy which is the G and the G is the U minus TS which is the F minus HX by X and in our case means minus mu0 HA dot M okay without the Delta minus mu0 HA dot M okay now if you this is easier if you calculate the DG okay the G lower than or equal to what okay sdt minus s S dT will be unchanged. And now, as you have, as before, you can understand that you will exchange now the delta between these two. So it will be minus mu zero, sorry, m dot delta HA minus S dT. And now what you can say is that if T is constant because you are in a cryostat and if HA is constant because you are sending one amps in the coil which is fixing now the applied field which is the result that the DG will be lower than or equal to zero which means that in this condition if you have the initial state and this is the final state this is the G this is a spontaneous transformation leading to a lower value of G what does it mean? when you apply you fix the temperature you are in a cryostat you fix the applied field the system, the magnetization starts to evolve and if you have enough time resolution you can see the evolution of the magnetization and if you are not interested in the time evolution but just in the final state the thermodynamics tells you that the final states will be in equilibrium the minimum of the give free energy for that applied field and for that temperature and this is the beauty of this so after all this story we know now that we have a thermodynamic potential, which is the Gibbs free energy, which is written like this, which is essentially U minus TS minus mu0HA dot M, whose minimization gives you the equilibrium state of your system, the final state of your system. Is that exactly what you want to know? Remember, this was my initial question. I'm trying to predict the configuration of a magnetic body when you apply a field at a given temperature. So that's the answer. You have to write down the suitable expression for the Gibbs free energy and try to minimize it.
\fig{12}{Lecture_2 Thermodynamics of magnetic bodies.pdf}
But there is something to be now, to be discussed, which is in principle the G should be just a function of H, A, and T. But in magnetism, what you put inside is also dependency on M. Does it make sense to you or not? In principle, one could say, okay, if you apply at the given H, at the given P, you should find the equilibrium configuration, which is, it should be unique. But this is not true. Just have a look to this Therese's loop. when you have this Therese's loop, even at zero applied field, you have two states, up and down. System is degenerate in this case. So the first observation is that in magnetism, when you're dealing especially with this theoretic behavior, so the M is not determined by the magnetic field. H, A, and T. So the story of the system is also very relevant to understand which is the real M that you have. And that's the reason why, in principle, the Gibbs free energy in its general expression is also a function of M. So that you jump from the Gibbs free energy to what is called the Landau free energy, the GL, from the name of the physicist, of the Russian physicist Landau. So the GL. But the fact that you have the M here is also very relevant because in terms of your prediction capability, you have to write down an expression including the M. Because if you are able now to write an expression for the GL of your system as a function of M, then you can minimize the GL with respect to the M and find the equilibrium state for your system. So you have to write it as a function of the M. Just for say, okay, let me calculate the GL for all possible configuration of my M. And then I find the minimum with respect to M that correspond to the equilibrium configuration. So in some sense, you can say that the equilibrium configuration will be now the situation in which the first derivative with respect to m, supposing that is just a single variable that you can use, in some cases it is, at fix h a and t equal to zero, and of course the second derivative must be written in zero in order to ensure that it's a stable configuration. It's really minimum, it's not a local maximum. And this is exactly what we are planning to do tomorrow. So tomorrow we will start writing a suitable expression for the G. We have now to understand which are the contributions to be put inside the G in order to find out the suitable GL, suitable land-out free energy, to be minimized in order to find the equilibrium configuration. And this is the basic foundation of micromagnets.\\
Good morning everybody. Due to Lorenzo's request today, we are starting with 20 minutes today, but hopefully this will be successful. And let's see the quality of the registration when we will take a decision. Or probably I could also ask the ASICs to tell me how we can fix it. So just to remind you the final results of last lecture is summarized here. essentially we introduce the energy the function is the GL of the applied field the temperature and the magnetization. The magnetization is valuable. And when you have this functional and you minimize this with respect to M, you find now the minimum of GL which corresponds to the equilibrium coefficient. The reason is that doing any kind of transformation you have to be lower than or equal to zero. For spontaneous transformation, you have to be minus or less than zero, which means that for each transformation you create your system from any state of the community to be moving to the new meaning. Now, it's a function in the sense that this hand, this small particle, is like a detector. But in reality, it's not a field. And if we go back to an extended volume, that's the industry volume, the GHEL must be written as reported there. So here's the F, the N most free energy. Interesting. Yesterday I had a discussion with a colleague of mine for a project in which we are trying to figure out how to model the segregation of epithelial material. I was discussing with this colleague if we have to use either the L3 energy or the G3 energy. But here it's clear. So we have to use the G3 energy. So the F will not depend on H. It will depend on M. So we need function to the function of the control variable of your system, which you control in your experiment. and what you control is the applied field. So it's F minus the integral of the volume of U0 and you have H applied dot M where M is a function of points, ok? This is the general situation. The applied field would be also a function of the point, the point, but you tend to have something that's uniform. Therefore, it's uniform. This is not mandatory. In some of our devices, we don't want to have a uniform field, but okay, that's a complication of the story. Now, it's up to you, but for sure, M will depend on the point. Otherwise, you cannot admit that you're having a main wall. A main wall is a situation in which M is not uniform. So it's varying from one domain to the other one. So this is the idea of the functional. It's not a function of just a individual scalar quantity, but a function of the field and so on. So it's something which is quite common. But the problem is we have to write down a suitable expression for this f. So this is the problem we're working with. So, we have to write down now a suitable expression for the GL, U minus DL.
\fig{13}{Lecture_2 Thermodynamics of magnetic bodies.pdf}
And to do that, let me first clarify something that we will increase in this course. We will essentially neglect the expression for the GL, the entropy constant, by it. This makes sense, of course it makes sense if you are working at very low temperature, so that this contribution is . And this is to prevent that . In real situations, it's not that bad, but the introduction of the entropic contribution is not so easy. So even the simulation software that you are using right now, there are some ad hoc modules that can introduce nanomodulation. But in micro-magnetic theory, what you want to do is to simplify this theory. then you are the school of the as something that's used to use in each problem as something that is shown to the micro-economy. So essentially, the problem is to write down the U, if you think like the 5s. And which have not a contribution to the U, to the energy. The energy, the contribution to the system, as we are dealing with the system ... All adesotropy interaction. Aesotropy energy being serious, ... ... And we have also magnetocrystalline anisotropy which comes from spin-orbit interaction. So when you have a crystal like a big crystal, you have a big crystal. It means that the geometry for the arrangement of the atoms and the end of the day, you will have different contribution which means that the orientation of magnetization with respect to the crystallographic axis makes a difference. And there are other terms that arise when you start considering not just a single magnetic body, but for instance, a multilayer, which is the typical situation of any spintronic device. You don't have just a single layer. Usually, you have different layers, like in a tunneling junction. You have two ferromagnets separated by no magnetic material. You have the interaction with some antiferromagnetic layers and so on. So there are some terms, energetic terms, to be put in the internal energy coming also from the interaction between the different layers. So now the problem is this one. We have to see how to write the expressions, suitable expression, for the different contribution. And then we will have the expression for the G. And then we can apply the machinery, which will be a sort of minimization of GL with respect to the M in order to find the equilibrium configuration. So we want to be able to predict which is the magnetization if M is a function of R in the body in response to an applied field. Better story. And this is also probably the end of this lecture. So we have to move to another one. My computer is working. and now it should be where is it I should have it somewhere wait I have to check that not here, or we also change some one Just to tell you which are lecture number is not here Okay Okay, so this is the lecture in which I tried to analyze the difference.