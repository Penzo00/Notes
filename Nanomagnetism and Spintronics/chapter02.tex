\chapter{Magnetostatic Energy and Shape Anisotropy}
Okay, today I want to see with you something that was written in some of the slides of yesterday's lecture, which is the concept of magnetostatic energy. So yesterday we have seen the magnetostatic framework, which means, say, essentially in case of absence of currents, you can use now a scalar potential, write a Poisson or Laplace equation, solve it for the potential, and then find also the demagnetizing field, which is the field produced by the magnet. And we applied this concept, this machinery, this theory, to three cases. The cylinder uniformly magnetized along its axis, and the same cylinder magnetized in the right direction in the sphere. And we have seen that was the last concept we asked the electric that in case of surfaces of the second order, an ellipsoid, a sphere, a cylinder, all these kind of surfaces containing a thermomagnetic body, you have the peculiar condition in which the demagnetizing field produced by the magnetization is uniform inside the body and is connected to the magnetization by the demagnetizing tensor. Okay, this was the last concept. Okay, Okay, what I would like to do today is to see in more detail the energy cost for building up a magnetic configuration. The cylinder magnetized along its axis, or the cylinder magnetized along its right-hand direction. Do they have the same energy? Do they have the same energy cost for building up these two configurations? The body is the same, but just the fact of magnetizing with the wrong one, acting on the other one, does it create a different cost? And which is the energy cost that we have to pay? That's the basic question. And now to answer this question, we have to see now which kind of energy contribution should come into play. So that's a question for you. Now let's imagine that, and now you're seeing something there. So I should, let's do this. Let's start from scratch. You have a cylinder or a sphere. So let's imagine that we start from our cylinder. And this was our reference ray, XYZ, which is an infinite cylinder, and the magnetization can be pointed like this. Which kind of energy could be involved in the calculation of the total energy of this system? coming from the cause of negative. So which kind of interaction could come into play and give here a contribution to the total energy of the system? Please? Obviously, dipole-dipolar interaction between the magnets. Of course, we can't hear where the first line is. Dipolar interaction between... What's your name? But this shouldn't be the first answer to my question after the course of Jack The first answer should be, if you have a ferromagnet, which is a basic exchange interaction, once you say okay let's calculate the exchange interaction here okay but just to clarify the story when you write down the exchange interaction we will see it I'm not sure today probably tomorrow when you say exchange interaction so So essentially, the Hamiltonian that you have is written like this, minus j s1 dot s2. Well these two are two operators in principle, and the spin space. So everything depends on the angle between the spin. Well let's assume that you are treating now the spin as sort of a quadratic one often. So what this reminds us now, it's interesting, the last one again, is the angle between the speeds. So that you have a difference of field ratio for the same world space in the particle space, okay? So it depends on the angle, power or how many times it makes a difference. And in general, exchange energy comes from an angle between the speeds, which is not equal to zero. So the ground state for the magneto all the things alive. And for any kind of perturbation, you need an angle between negative and positive, you have an exchange energy column. That's the same. And now, with this in mind, we should say that there is no difference in exchange energy between this configuration and the other one. So let's take now the same system, okay, in which same reference frame. Now the magnetization is right there. That's not fantastic. If Ham is pointing like this, there's no difference in exchange, interaction, energy. What's the reason why there's no difference? Because this Hamiltonian is totally isotropic. So what makes a difference, just the angle between the spin? If the spin is at the point of this, or right there, there's no difference at all. See, it's totally antithrombic, the exchange interaction. What makes a difference, just the angle between the spin, not the absolute direction of the spin. So, as soon as you start with an item, a node, another term, a magnet, and you shape it into a cylinder, direction of the speed, followed by the axis of radio, doesn't make any difference in terms of exchange energy. So in terms of exchange, it's not different. And now the real difference that comes into play here is when you consider the bipolar interaction, which is exactly that leading to magnetostatic energy. When you say magnetostatic energy, you are saying dipolar interaction energy.
\fig{3}{Lecture_2 Thermodynamics of magnetic bodies.pdf}
And so the first task of today's lecture is this one. Try to calculate, find a way, the expression for calculating dipolar interaction energy. When you imagine that you have a body like this one, the shape is whatever you want, in which you have now the spin arranged in a way that is arbitrary. They can be pointing in any direction in the space, depending on what? Depending on the way you magnetize the system, depending on the fact that you have a domain wall inside, straight configuration. But in the end of the day, as soon as you have a configuration of the spin of the magnetic moment on each lattice point of your crystal, you have now a configurational energy. It's exactly like electrostatic configurational energy. So it's just the fact that you have an arrangement of spin pointing in an arbitrary direction in the space. That creates now a dipolar energy. What does it mean? It means that if you have this dipolar moment, this magnetic dipole here, it fills the field produced by all the other dipole plates in all the other lattice sites. And the interaction between the field produced by all the other lattice sites and the first one creates an energy that depends on what? On the configuration of the dipoles in your body. So it's something connected to the configuration. so I'm not asking myself which is now the energy cost to create that configuration I'm just saying whatever is the way you create that configuration when you have this configuration there is an energy associated to the interaction between that dipole and the field produced by all the other dipoles in the body this is magnetostatic energy now the question is how can you calculate When you have an extended body with a strange configuration of the dipoles and so on, how can it really tackle this stuff? And this is the way. So first of all, you say, OK. You have a lattice with magnetic moments mu i. Okay, different I is the index with which allows you to distinguish the different dipoles in the lattice. It could be a cubic lattice, could be an hexagonal lattice, depends. Iron is a BCC. When you have nickel, it's an FTC. When you have cobalt, it's an ACP. So different structures, so you have a dipole in the lattice. And then you say that okay, I have the field, you have this dipole, this mu i, and all the other dipoles produce here a field which is the h i, okay? It's produced by all the other dipoles. And now you have the interaction between this mu i and this h i. And this is a sort of Zeeman interaction, okay? It's not an external field, it's the field produced by all the other diaphrases. How do you write down the interaction energy? It's a Zeeman interaction, which is minus mu dot bi, which means minus mu i dot mu zero h i, which is exactly what you write down here. So the interaction energy will be minus mu zero mu i dot h i. Okay? But okay, now you have to pay attention not to count the couple twice, because of course when you do this and you make a summation over the i sides, so you are counting interaction twice, because when you jump from one to two, okay, on one I calculate this. When I jump to two, I calculate the same story, but I'm counting it rationally to one and two twice. So in order to get rid of this, or say, on my smallest mistake, you place in front this factor one out, okay? So this is the interaction energy between that field and this one. But if I want to count now the energy corresponding to all the couples of dipoles in my body, I have to put inside, so outside this factor 1i. And you take the summation over the i, all the sides in your lattice. so this is in principle what you have to do and now the problem is how can you calculate H high okay is the summation now here on all words that's that's interesting okay interesting oh yeah I J and J I remember the meaning of this one. You know, one two two one. You don't have to count the carbon one. Now how can you make it? The basic idea is that represented here. You have a body like this, which is an extended body. Now you say, okay, let me sit on the dipole new pile, this one. Okay, I sit on this, and now I make separation of two regions, a sphere which is concentric to the dipole, nu i, and the region which is outside. The region outside is that dashed here in this board. And I treat the problem in a different way. For the region which is the dashed region, I use the continuum approximation. I don't want to see the crystal structure here. I'm just considering not the peculiar crystal structure, the SCC, DCC, ACP. I'm just considering a magnetic body and I want to use now just a concept of the demagnetizing field coming from the theory of the magnetostatics that we have seen in So I want to use a sort of demagnetizing test of that kind of total magnetic charges and so on. But inside this field, I want to go into the detail of the structure, the crystal structure, and inside the sphere, I really consider that having DCC, FCC of ACP lattice. The question could be, why are you doing that? Because in the end, it's clear that the field, H-I, this one, will reflect also the arrangement of the atoms around the single dipole. But clearly, the fine structure will be very relevant for a distance from the dipole, which is not too large. When the distance become too large all the diapos are more or less in the same position So you understand that the problem of the real crystal lattice is really something which is relevant when you are quite close to the single iron D1 At the distance which is comparable with or to the lattice constant Okay, so when you exceed when you are 10 times the lattice constant far away So the Euclidian structure over there is no more relevant. It's a sort of continuum of . And so what you can say, OK, inside this sphere, in a radius which is larger than the lattice parameter, of course, because the Euclidian is no longer than the lattice parameter, but it's just the signal you want. It must be larger in order to include different lattice sizes, but we will see not too large. And I anticipate something just to say that half half should be what? R is larger than A, the lattice constant, and lower than what is called the exchange length. I will define this parameter in a formal way later on in this course, but this is the typical length, typical distance, over which you see a tilt of the sphere. Let me anticipate something that you already know. Exchange interaction is an interaction that tends to keep all the spin following. The ground states correspond to all the spin following. Any perturbation of this is an excited state, which has an energy quality. So it's just the tendency due to exchange interaction. This is not the unique, the unique, I'll say, energy to be considered for the equilibrium configuration of the magnetization. As a matter of fact, in a body, in an extended body, sorry, I didn't see something. In an extended body, you have domain walls. So the main wall is a surface which separates two parts with a magnetization which is opposite of 90 degrees. So for some reason that we will see, in a real body there is a field between the spaces. There are some regional niches which jump from one domain to another domain, and in that region there is a rotation of the space. Now, even in a domain wall, the rotation cannot take place in just a single light-year structure. The reason is quite simple. The energy cost would be too high. A jump by 180 degrees over just a single light-year structure will imply a cost in terms of the exchange energy that is unaffordable for the system. So the rotation of the spin in the domain takes place over a characteristic distance, which is the domain wall width. So in any case there is a distance over which the spin can rotate. And this distance is defined as the exchange length. We will find a more formal definition of this form later on. But clearly, and that's important at this stage, You chose now a sphere, which is a concentric sphere around the type of mu i, whose radius is lower than Lx. And what is the T-impact of this? Why do you choose that? If R is much lower than the T-impact, you can safely assume that inside the sphere all the T and all the lattice points are parallel. Okay? So what you're saying is that if this is true, then you can say that in your sphere Here you have your dipole which is the new eye But let's imagine that you have a crystal lattice, which is this one which is a cubic one All the spin are parallel And this simplifies a lot the calculation because now the choice is done in a proper way. You choose a surface with the full symmetry sphere and inside the sphere you have all the dipoles are parallel. This is the easiest way for calculating the effect of the peculiar crystal lattice structure that you have in your model. So coming back to our calculation, now our target is calculated h high, which is the field produced by all the other diagrams. And now we have divided the body into two regions, a sphere of radius R, capital R, inside which we are considering the detailed lattice structure. So we are treating the problem exactly like this, with a summation. outside we're using the continuum of approximation. And now how can you proceed? So the way you proceed is this way. You say okay, let me do this. So this is my body, okay. Now I say that the field produced by this can be divided into three contributions. So I take now the field produced by the region which is outside and for doing this I say okay this is equal to this story. I take the same volume here, plus what? A sphere. Okay? Here, let's assume that the magnetization is pointing like this. So it's equal to a body with a magnetization point like this, treated in the continuum, plus what? A sphere whose radius is r with the magnetization point it down. In this way you create an oval, okay? A body with a cavity inside, a straight cavity inside. And then you add, it is MS, if you want to know what I'm using, using just hand. And then I take my sphere of radius r, which is treated in the continuum, not in the continuum, much better. In the case of the real lattice, it could be a cubic lattice. This is the superposition of the three elements, creating a difficult division what sum is exactly in the field that's supposed to have. So essentially I was imagining to have this right here, this is the dipole, this is near high, and now this is the decomposition that I'm using. Okay, now in terms of field, if h high is equal to volt, if I take this body, the demagnetizing field at the point i is what I'm calling here HM. Capital HM is the demagnetized field at the point of the i-dipole produced by the whole body treated in the continuum approximation using the concept of the demagnetizing tensor, magnetic charges and so on. And then I have to, let me do like this, It's more beautiful. And then I have to consider the contribution of this one. What is this? It's a sphere. And what's now the demagnetizing field of the sphere? Minus one-third m. But okay, m, now we should say, in principle this is minus m if you want to be consistent with the same thing. This is m, this is minus m, okay? So you have this minus m, so this means that the demagnetizing field produced by this one is something which is pointing upwards, okay? So I have to write down what ms divided by 3. This is exactly, or ms, sorry. Sorry for the confusion. In the end, it's a saturation magnetization of that body, but I was using just M without this substitute. M divided by 3. Yeah, because, okay, you say it depends. In this case, do you understand this? No, I don't. No, you don't understand. Okay, fine. If you have a sphere, the sphere with the magnetization pointing like this, Okay, the demagnetizing field will be pointing like that. Okay, it's a demagnetizing field. So at the end of the day, it's aligned to M. And the demagnetizing field is one-third, minus one-third M in terms of vectors. But as this one is pointing down, the demagnetizing field is pointing up. So if you want, you put here the M, which is a vector, and you find something which makes sense. Because M is pointing upwards. The first one, in the first element? Yes, exactly. We are considering the demagnetizing of the blood? Yeah, this is the demagnetizing of the whole body. And you should have different signs then, since one is magnetizing one way and the other No, okay. Yeah, you are right. But this is HM, it would be pointing like this. Okay. HM will be pointing down, and the other one will be pointing up. Not exactly down, because it depends on the shape, but it would be opposite to the M. But in case of the sphere, it's really opposite, with the coefficient which is one-third, minus one-third. And HM, which is minus one-third M. In our case, it's the other way around. m minus one-third m, which is pointing up again. Okay? Like m. And this is the original two minuses you have here. If you want this minus one-third m, where m is minus m. This means plus m divided by three. Are you okay? Good. Plus what? plus the contribution at this side of the dipole, which has sit on the different sides here. So plus H I prime, which is the field produced by all the dipoles, considering them as individual dipoles. And see, really, the magnetic field H produced by all these dipoles are arranged according to the lattice that you're considering. Okay, now this is something that you can calculate. This is some, okay. The 9\% is the ring in the left-hand side of the element. Are all the parallels? Yes, definitely. Yes, they are all parallel. We will see this in the calculation in a while. Okay? Good. Let's move on. on. Now essentially this is done, so we in principle depending on the shape we can calculate it using the demagnetizing tensor for instance. This is something which is easy because it's a sphere. Now the problem is how to calculate this contribution here. So the big advantage of this choice is that you have a sphere with all the dipole fibers.
\fig{4}{Lecture_2 Thermodynamics of magnetic bodies.pdf}
So now we have to really go into the detail of the field produced by each dipole. And so you find that, okay, you are inside the sphere, and you have this geometry, the dipole mu i and the dipole mu j. As you see, they are parallel, even in this sketch, okay? They are parallel because of that assumption over there. And now you have to apply the formula for the dipolar field produced by one dipole at the the position of the first one. The formula of this one is a formula that we will use a few times in this course. So it's the dipolar field produced by a dipole at a quite large distance. So it's one over pi minus mu j, mu j. So now the the substitute j is now moving in such a way to account for all the other sides around the mu-tribe. So it's a sum over j for rij, which is lower than capital R, which means inside the sphere, of minus mu j divided by rij to the third. You remember that the dipolar field decays as one over R to the cube. And then you have this much, much complex formula, which is mu j dot r i j by r i j, divided by r i j to the 3. So that in the end, even here, you have a dependency like 1 over r to the cube, which is exactly what you expect for a dipole. So this is exactly the summation that you have to carry out in order to find the value for h i prime. And now probably it's a good time for a small exercise. Let's try to make this calculation, to do this calculation for a simple cubic lattice. Just understand how you can find a real result. Well, we start from the vectorial expression, and now we consider a cubic lattice with the axes which are x, y, and z. And now you say, okay, I start from this formula here, and I take the projection over x, for instance. I consider just one component, which is hx. So it could be hi prime x. So I take this formula here, which means it's a summation. I take 1 over 4 pi in front, everything, and then I get what? Okay, I have a summation of the different terms. Okay, but I don't need this one. The first one is minus mu j, and I take minus mu j x. Divided by r i j to the third. Then I have plus three times. Now I have this dot product, which involves mu j dot r i j. Let's now write this dot product using the Cartesian representation of the vector, which means so I have mu j so it will be mu j x y x i j plus mu j now y yij plus mu jz zij multiplied by now I have to make the projection of rij which is xij divided by rij to the fifth. But now we can simplify the story because we have to remember that inside our sphere, all dipoles are parallel, all dipoles are parallel, which means that I can forget this index j, so it's always the same. So one could say, okay, this is equal to one over four pi summation of this minus mu x as is reported there, divided by Rij to the third plus three times, and then you have mu x xij plus mu y yij plus mu z zij multiplied by xij divided by aij to the cube. And now don't forget that you have to make this summation which is for all the j which are inside the radius. Yes summation over the j. okay and now let's make let's carry out this kind of summation let's consider now the second term here if you now multiply this you will find some terms like this so you have three times sorry and then you have this contribution mu x xij second plus mu y yij xij plus mu z zij xij. Okay? Now, the contribution of this term here and the other term here will be zero when you take the solution. And the reason is very simple. These have all terms both in x and y. So now you have a rep, so in your lattice, you have one on the right, one on the left, okay? And they give you a contribution to the opposite side. So all these terms are odd terms, which means that they carry out the summation inside the sphere. That's the reason for choosing the sphere, is the surface is the body with the maximum symmetry. really capitalize this choice right now. All are parallel and we are using the sphere. So this means that they will give a net zero contribution in the summation and the unique to surviving in this one, because this is the even term, Xij. So you're left with a calculation of the summation of this term here. But now you are in a cubic lattice. as you are in the cubic lattice, x minus z are two limits of change, so they are totally equal. So that I can say that this summation for j inside the sphere of xij the second divided by r i j to the fifth can be written like this one so i have this let me also write three times three times the summation of the j I don't need it, the summation of j, or, yeah, I need it, okay? So let's imagine that you have this one. You can write this one as one-third r ij to the second. Do you see it? r ij, it's not too much space here, but okay. Rij to the second will be xij to the second plus yij to the second plus zij to the second. Okay? If I take the summation of this one, as all these three terms are equivalent, the summation of this one will be exactly one-third the summation of Rij to the second. Okay? So I can find that this one, this summation here, corresponds to this. And so this and this make it out, and finally you are left with the summation over j of 1 over rij to the third. multiplied by what? Multiplied by mu x. And so, with this in mind, as you have this one, so all this term is provided with plus the same term which is here. And the net result is zero. You see? All this term here is giving you what? In the end of the day, you're left with just this term here, with the same term here with the minus. End of story.
\fig{5}{Lecture_2 Thermodynamics of magnetic bodies.pdf}
What is the main result here? The result is that in case of the cubic lattice, the contribution of all the dipoles inside the sphere is zero. The magic of the dot block is zero. You can check the iteration in the final, it's correct, it's zero. interesting what does it mean? It means that in this case, in principle, you can forget about these two points. You just have the first two ones. But this is a general result of... Is it peculiar to the cubic lattice? Is it just peculiar to the cubic lattice? It depends on the theory itself. We have used... We have deeply used this term there, the cubic lattice, where this symmetry is actually the center of the sphere. If you take another crystal lattice, those will not be zero. But again, if you understand now the way you carry out the calculation, those will be in any case connected to the direction of the sphere inside the sphere. And in general, you can easily figure out that this term can be written like this. Minus is a tensor which multiplies the magnetization. Just tell me. Why am I saying that this makes sense? Because inside the summation, you always have the mu. And the mu is what defines the space, is pointing to what, and the single dipole is connected to the magnetization. The magnetization is the magnetic dipole divided by the volume. So in principle, the magnetization is defined like this. is the sum for j, okay, if you want r ij lower than r of what? mu j divided by four pi r to the q. This is the magnetization. The magnetization is proportional to mu, okay, and as you have mu inside all this this summation in the end you will say that depending on the shape of the body sorry depending on the machine depending on the crystal structure you will have that the h high prime will be sorry this is the business living not saying that is equal to this h high prime is equal to that expression here. That's a general expression. And so you are left with this general expression for the magnetostatic energy, the configuration of energy depending on the Euler interaction, the u-ham of these lines, minus mu 0 divided by 2, the summation over i of ui dot h-hi, where h-hi is the sum of h-ham plus one third m minus the stencil lambda capital lambda multiplied by have. Okay? Are you okay with this? The question is for yourself. I have a question. Once again, not now. Why did we lost the weeks we were going from the... What we were going after... From there, I didn't lose it. I was just considering that, okay, If you want, I can write also by mu x here. Okay. So it should be mu x, mu x, and mu x. I was just considering everything without mu x, but I need it because without this one, I will not find the cancellation with the first one. I was just considering everything but this pre-factor. Okay, Mattia? Thank you. Good. Now, in the calculation of the magnetic energy, we are here. Now we know how to carry out the summation, but as you know, as it happens, so the summation is easy if you go to the continuum approximations. Instead of carrying out the summation, you move down to the integral, you say, okay, I associate now to each mui a small cube in a lattice, in a square lattice, in a cubic lattice, and then I move now to the integral in the volume. So instead of mui, now I see the appearance of what? Of M, the magnetization. Because now you can say that if you want, that's another way, the new I can be written by this is the magnetization by A to the cube, okay? And A to the cube is exactly d tau, the unit volume, okay, or d cube half. base association you transform now the summation into integral which is exactly this one. There is nothing particular to that. And now you distribute this scalar product to the different terms appearing here within brackets and you get these different terms alpha, beta and gamma, three distinct integral and the integral is is over the volume of the volume. It's over this volume here, okay? This is the integral. Three different terms, alpha, beta, and gamma. Let's have a look to these different terms and comment on the entity, meaning, and so on. Let's start from beta. Beta is minus mu zero six over six, m squared over in beta. It depends on what? Essentially not in the integration of m, okay? Not integration of m. Depends on the magnitude of m. So it's maximized when m is maximum. When m takes the maximum value, case of full magnetization when exchange interaction is so strong so that all the magnetic moments are parallel and so on. So in this sense is something which is maximized when M is maximized, when the temperature is very low so that you are not activating magnets so that the magnetization is maximum that you can have and so on. And in this sense this is a term which is very similar to the exchange interaction energy. Wow! Can you comment my statement? Now my statement is this is a term which is similar in its form, not in its origin, similar to exchange energy. This seems to be somehow connected to the same thing. Do you remember the Weiss theory of third magnetism? So, if I'm considering M to the third as a means of the analysis of the magnetic energy? Yeah, so in the Weiss theory of third magnetism, which is, of course, was not a theory, was just to say, a phenomenological way for accounting for such a very large value of the cooling temperature. So the idea is that if for iron you have to heat up up to above 1000 Kelvin in order to destroy the magnetic motor, it looks like you have a strong internal field which keeps all the magnetic moments aligned. And in the Weiss theory of permanganate, so you say that there is a sort of internal field, which is responsible for the alignment of the spins, which in the end represent the exchange interaction, but at the time of Weiss theory was the internal field. And so you say that there is an internal field H, how do you call it? I don't know. Vice. The vice field, which was written like a lambda coefficient multiplying M. So that the energy, the exchange energy, the exchange in this theory was written like, Okay, it's the interaction between this field and m, which means minus mu zero, lambda m squared, and d tau. But what about the value of lambda? The value of thousand, okay? And so you immediately see that this term is really similar to this one. But there is a big difference that the lambda factor is missing here. So this means that this contribution is similar to the contribution of exchange energy in the continuum of exhibition, apart from the project is 1000 times low, which means, okay, you can neglect it, forget about it is is there, but it's not very relevant, because as compared to exchange interaction, that will be an energy contribution that we will definitely take into account in the micromagnetism theory. So that contribution is negligible. Good. Let's move to the third contribution, which is gamma. In this contribution, so we have seen, for instance, that in case of a cubic lattice, this contribution will be, the gamma contribution for a cubic lattice will be equal to zero. In other lattices it will be different. And so it is strongly connected to the symmetry of the crystal lattice. So it's an energy contribution that cannot be neglected at all, but it is strongly related to the crystal structure. And it contains the anisotropy of the crystal lattice. In a cubic lattice, if you move along the x direction or along the diagonal of the cube, our situation is different. So if the magnetization points along the direction of the size of the cube or the diagonal, you expect to find the difference, which is coming from the peculiar shape of the capital lambda a tensor that describes how you connect the H-high prime to the magnetization. So this is an anisotropic term that we will find again when we will treat magnetic energy, and especially the so-called two-Ion crystal anisotropy, magnetic crystalline anisotropy. And then we are left with this one. In the end of the day, even though the calculation of the configuration of energy leads to the summation of these three terms, this one can be neglected. This one is conventionally considered into the anisotropy, magnetocrystalline anisotropy energy, and you are left with just the first two, which is the so-called magnetostatic self energy. I will use during this course, not just me, but the international community is using as also the formulation for the magnetostatic energy or the polar energy. Which takes this form here.
\fig{6}{Lecture_2 Thermodynamics of magnetic bodies.pdf}
So far, by solving that problem, we found that expression for the magnetostatic energy, which is minus mu zero divided by two, the integral over the volume of HM dot M md tau. Is the integral over the volume possible? more. Now, here there is a demonstration of something, okay, you can go through that demonstration, it's quite easy, I can just outline it, there's no physics inside, but the result is important in some sense, because this is an integral over the volume, and you can transform now into an integral over the space, the world space, of something else. Let me first comment on the sign of this magnetostatic energy. One could say, ah, my God, there's a minus sign, which means that the energy is negative. No, it's positive, and the reason is quite simple. As we have seen, the demagnetizing field tends to be antiparallel to the magnetization. So this means that this dot product is essentially negative, okay? It's providing you with a negative contribution, and so for this reason, that minus by minus gives you plus. This is a positive contribution. It's a well-defined positive energy, the magnetostatic energy. OK? Now, you can transform this integral in a different way. So I just outlined things. You say, OK, this is an integral over the volume. Yes, but the magnetization is zero, is non-null, just inside the volume. So you can just rewrite the same integral as an integral over the whole space, because outside m is zero, so nothing changes. So this is the transformation integral to the minus sign. And then just you remember that, okay, this is the connection between v, h, and m. v is equal to mu zero h m plus m. So you rewrite instead of m, from this equation, v divided mu zero minus h. Okay? the whole space. And now you say okay, I just distribute this product here, I get two integrals. The first integral is the integral of the whole space of hm dot b in the tau plus mu zero divided by two integral on the whole space of the demagnetizing field of the stray field squared. Strictly, in the field where you find the magnet, the straight field or the free diagonal, demagnetizing field is strictly, they are exactly the same thing. Good. Now what you can see is that this first integral is zero. It's a new integral. And the reason is that you have the dot problem, two fields, and the divergence of one is zero, is the divergence of B is zero, and the curl of H is zero. This is something that we have seen. In magnetostatics, the curl of H is zero, there are no current amounts. And in any case, the divergence of B is zero. Now, with these two ingredients, you just rewind it a bit, because you can say, okay, the divergence of B is zero, you can write B as the curl of A, of this sort of vector potential here. And then you write H dot the curve of A, you use this vector identity, you find here the curve of H is zero because there are no currents, you can't set this contribution, and you're left with the integral of the whole space of the divergence of this. You use the theorem of divergence, which means the integral of the whole volume with divergence of A cross H is the integral of the whole surface, which is at infinity, the whole space of A cross H. But as for finite source, finite magnet, both A and D, they go down at least as 1 over R squared. This integral is equal to 0. And that's the general situation. When we have the integral of the whole space of two vectors, which are characterized by the fact that the divergence of the first one is 0, the curl of the other is zero, and you add that at infinity they go down at least as one over r squared. Okay, this integral is zero. So you get rid of this and you're left with this expression, which means that in the end you can write down the same expression for magnetic static energy in a different way. We have the first possibility, which is this one. The first one. The second one is plus The integral of the volume of hm squared in the tau. Okay? Sorry. Over the whole space. Over the volume minus the zero m dot hm. Over the whole space hm squared. Clearly in the second case, the fact that we have a quantity which is 1, it is well evident. But also in the first case, because HM and M, they are antiparallel. Okay, now we are ready to go back to the result of yesterday. I want to apply this to the three cases that we have seen, okay? And especially in the case of the cylinder. I remind you that we have seen two cases. The first one was a cylinder which was magnetized along the z direction. Okay, the second case was a cylinder which was magnetized along the radial direction. our calculation, so let's say m was pointing like this. The result was that here the demagnetizing field was equal to zero. And the reason is that magnetic charges are pushed, a plus infinity or minus infinity, there is no net contribution. Here the demagnetizing field was equal to minus m divided by 2 by ux, as m was pointing along x, or if you want minus m divided by 2 vector. Okay So probably it's even better to write it like this. Minus m divided by 2. Let's calculate now the magnet's static energy. Let's see, let's take the first expression. Minus mu zero divided by 2, hm dot m, and the tau. But HM is zero HD, sorry, HM is equal to HD, okay, same story. But HM is zero. It's zero. There is no demagnetizing field, so there is no interaction essentially of the magnetization with the field produced by the magnetization itself, because that field is zero. If you move to the other situation, it's not the same. It's not the same at all. Because you find that the magnetic static energy in the second case would be minus U0, the integral over the volume of HM, which is minus M squared divided by 2 in V tau. And so, of course, if it was an infinite cylinder, what we can do is we can calculate the magnetic energy per unit length. Let's assume that we take now a length, which is L. Matteo, any problem? No, no, no. Okay, good. So this means minus mu zero divided by L, minus by minus is plus, m squared, and now we have pi r squared, this was the radius, pi r squared multiplied by L, which is exactly the volume. And this is different from zero. Now, this is the energy cost. Ladies and gentlemen, which is the preferred configuration for the magnetization, 1 or 2? 1. The energy cost is 0. The energy cost in terms of the polar energy. For this same term, it is the same. This was the first statement of my lecture today. but on the other energy, the first configuration is much more convenient, when in the second case you have this energy cost. And what is the outcome to the implication of this? Just because you have a specific shape, there is a clear preference of the system to allow the manifestation in one way or another. This is not written in the exchange system, it is not connected to the drug. This is the reward of dipolar energy. And so, we are seeing what is called shape anisotropy. Chez-Panaisotopy is a consequence of dipolar energy. The term of dipolar energy, of magnetostatic energy, depends now on the direction of the magnetization with respect to a specific axis defining, I don't know, the rotation axis of your system, or I don't know, the cube, the side of the cube of your shape. So, the way your hand handles with respect to the shape of your body makes a difference, a big difference. And this is inherently contained in the form of the demagnetizing tensor. If you assume that you have now the demagnetizing tensor, what you know is that you can always write, if the demagnetizing tensor is well defined, if the surface containing the magnetic body is of second degree, like in case of a cylinder, for instance, like in case of a sphere and so on, like in case of an ellipsoid, in this case you can write that HM is equal to minus N multiplied by N. But if this is true, you can rewrite the magnetic static energy Em like minus mu zero divided by two, the integral over the volume of hm dot m. This is essentially want M with the minus which comes there plus M, M applied to M and beta. What happens when you do this? As I told you, you can always write your demagnetizing tensor in a diagonal form. you can properly choose now your reference frame, which means that the n is always something like n and y and z . All the other components are zero, okay? It assumes this diagonal form. And with this in mind, you can easily prove that We can easily prove that the magnetostatic energy in its general form can be written like this. U0 divided by 2, the integral of the volume of water. Okay, you carry out multiplication, you get Nxx by Mx, then you carry out the dot product you will find Nxx Mx squared plus NyY My squared plus Nzz Mz squared. Okay? Now if this is a general expression of the magnetized austatic energy, you You understand that the anisotropy comes from the peculiar form of the magnetizing tensor. For a sphere, no anisotropy at all, because you have that Nxx is equal to Nyy is equal to Nzz is equal to one third. So in the end you get what? the magnetic static energy will be just mu zero divided by two, and so you get what? Integral of the volume of one-third mx squared plus mz squared, or d tau, which is, for example, mu zero divided by two, m squared divided by three, by four, yeah over the model okay no anisotropy at all the three components are totally equivalent okay but if you take now a cylinder that's not exactly the same infinite cylinder situation is different the tensor is 1 half 1 half and 0 ok this is the tensor this is F this is this expression so when you write in this case the magnetostatic energy it will be minus u0 divided by 2 minus minus is plus and so on and then you get 1 out nx sorry mx square plus 1 out my square plus plus nothing 0 mz square So you immediately realize that. Now if you put your magnetization along Z, you find zero. If you put the magnetization along X, you find something which is different from zero. So using the concept of the demagnetizing tensor, So you immediately find that the form now of the magnetostatic energy contains the shape anisotropy. It's inside. And we can see even a more general situation, which is that of spheroids. Sorry, by the way, I'm jumping back. I'm sorry. If you want to find this, for some reason with the chalk on my fingers it doesn't feel which is not good at all. These things you can find these things more or less here yeah so it is in the first part of this life okay.
\fig{9}{lecture_1 basic magnetostatics.pdf}
So, let's imagine that you have this situation. First case, oblate steroids. They correspond to this shape here. It's like a rugby ball, okay? x, y, and z with the axis of r, b, and c. Okay? The alpha axis, a, b, and c. So this is definitely a second-order surface you can write for defining this. And one now can say, okay, let's write down the magnetostatic energy for this body here. Of course, what you know immediately is that this body has a cylindrical symmetry around the z-axis. So you expect to find that nx will be equal to ny. Sorry, nxx equal to nyy. They will be equal. And then there will be an nzz. And now my question for you is this one. which is the relationship between Nxx and Nzz, which is the bigger. This may be like a mean considering their axes and the inverse proportionality relation. like I imagine that... So for you, an X is larger or smaller? Is larger. Why? Because I have, let's say, less space inside the material to magnetize, so, people are saying that something should be equal, I imagine that the magnetic response points along one axis that is shorter than the other should be written. Interesting. I'm very interested in this. Because, okay, the way you perceive things, the feeling is important in physics, very important. These days, in our lab, there is something which is extremely strange. There is a result that I don't understand. Something that could be if it's true it could be a sort of revolution so i i believe that it's more now don't tell the guy okay my first my experience has been there is something more okay this is true i have to wait some other experiment to see if it's true or not but what makes the difference it's very often the feeling of feeling. I don't really understand what is feeling of love, but with some experience, we start perceiving the feeling that it should be like that. Which means that if someone tells me, how can you say that I would say something like Eduardo is saying, saying something like this. Eduardo is not saying exactly the core of the thing, but I understand the problem in his mind, the idea is a good one. And the feeling is important here in science. Many times the feeling is guiding your choice. After why, after many stories, after some few you understand why. But the feeling is fundamental and you should have a good feeling in science. You know the good feeling is difficult because you can't have the proof. At first sight, you enter a room, you see a spectrum, you say, oh my god, this could be related to that. That's intuition, that's a field, and the first approach is fundamental. Now, let's put what the others are saying in a more quantitative and understandable program for everybody to explain. It's true, it must be bigger, but what's the reason? So, let's try to go back to the general meaning of the component of the demagnetizing tensor. Okay. And we can use also the example of the of the cylinder which is very very representative of the story. Anyway, when you say that HM is equal to minus NXXNYYNZZ MXMYNZ. Okay? This is the way you find things. which means that you can write it down as minus Nxx, Nx, Nyy, Nyz, Nz. So, Nxx gives you the idea in a quantitative way of how the body responds to the application, to the presence of a magnetization along the X. If nx is big, this means that if you have mx, you will have an x component of the magnitude of infinity which is b. If it is zero, this means that the body does not respond with a component of h to mx. So in this respect, now you can understand what happened, for instance, in the case of the cylinder. If the cylinder is magnetizing that direction, okay, along Z, you find that the magnetic charges are so far away, there is no demagnetizing field. So this means that n ZZ is equal to zero, okay? And so you have a general rule, the longer the, that I'll say the size of the body in one direction, the lower the response to a magnetization point in the same direction. In this sense, even though it's not an inverse proportionality, but here roughly say that the body is elongated in one direction, along that direction the components of the diminutive tensor will be small. It's not exactly an inverse proportionality, even though some texts report this, but this is wrong. It's not an inverse proportionality in the end. It's not like one over the side, but it goes like this in terms of inequality, in a qualitative term. Okay? And, in the other case, you have mx, for instance, you have a finite hx, hmx, minus m divided by 2. And this is why. Because magnetic charges are very close each other. So the demagnetizing field, because it is M, the demagnetizing field is not negligible at all. So the smaller the size in that direction, the larger the components of the demagnetizing terms. You see my point? Fabrizio, you okay? Yeah. No, you don't, you're not okay. Now m is constant or over the space? Yeah, of course. m is uniform. It's not, yeah, like also for the Yeah, if m is not uniform, you cannot apply the concept of the demagnetizing tensor. Okay, that's a general assumption. Only in case of uniform m, you use the demagnetizing tensor for calculating the demagnetizing field. That's a basic assumption. If it is not uniform, you must refer to simulation in some way. But, OK, this analytical framework is very powerful because it gives you the way for making some calculations. So in the case of the problem that we have right now, you can run some simulation. And students are doing that extensively. But at some point, I need to write down some equation in which I want to make some modeling of the story to understand the trend, the physical. And for this, you have to make some assumption. In some case, even though M is not perfectly uniform, say, OK, let's assume that I can use demagnetizing tensor just to write a formula that allows you to understand how things are going physically. But the basic assumption is here, M is uniform. Good. So now it should be clear, the story. large size, the body in one direction, small value of the demyelinated ion, so we'll call it along that direction. So coming back here, yeah, I'll cancel this and place it on that space. So in this case, this relation holds true. Nxx is larger than Nzz. So writing down the expression for the magnetostatic energy, we discovered something which is interesting, also called future. So the magnetostatic energy will be near zero divided by two. And so I will have NXXMX squared plus, okay, I have to write NXX, but okay, let me say they are equal. So I will just replace again, MY squared. plus nzz mz squared. Okay, this is the magnetostatic energy in our case. But now I want to say, okay, it's an integral, but if we assume that everything is uniform, it's not really an integral, is a multiplication by the volume of the body. Do you see this? It's really the question by Fabrizio saying about, okay, m is uniform, so in principle all this integral is just an integral of the constant, okay? Nothing changes here. So you can write down the magnetic static energy per unit volume, that will be near zero divided by two, and then you're left with this expression here. And now, just with a very simple mathematical tree, we can find an expression which is more understandable than this one. But the trick is that I will add here nxx by mz squared. Because if I make this addition and I subtract it, here I will find nxx by m squared. So by doing this, I will find out nx as mx squared plus my squared plus mz squared, in the end is m squared, plus, now as a hundredth term I have to subtract the same, nzz minus and x, x, m, z squared. Okay? And now, the interesting part of the story is that, hey, this is just a constant, okay? What does it mean as a constant? So, as it contains m squared, independently on the orientation of m, and it is a very different value of mx and imz, this is a constant. And a constant for an energy term that may cancel. So, energy is defined without peculiar considerations, it's not really interesting. a constant plus the zero divided by two and that's that minus and xx and z squared okay so it seems that the unique dependency that you have is a the dependency on the Z component of the magnetization. So if M for instance is pointing like this, and theta is the polar angle, it depends on what? M squared, why? Because N squared on the angle is left. Okay? And now, what about the sign of this term here within brackets? Is it positive or negative? It's negative. Okay, this is the expression for a uniaxial anisotropic term. What do I mean by uniaxial and rhizotropic? What is clear is that here everything depends on what? On the angle that the magnetization forms with the z-axis. And if this coefficient is negative, so now you can make a plot of the magnetostatic energy as a function of the angle theta from 0, pi and 2 pi, and you will find that, okay, this is negative, okay, it's a negative value, so this means that it was not a very good choice, this one, because of the problem of the blackboard. Okay, as this coefficient is negative, for theta equal to zero, cosine of theta is of course one, okay, so you have what? You have a negative contribution, which is, it's a minimum for this function. It's minus the cosine square of the angle theta. starting from this point here. When you have that, sorry, you have phi over 2, you have the cosine of pi over 2, which is 0, so you are here, and for pi, as you have the cosine square of theta, we're going back here. So this is the the profile of this shape I resolved. You have two minima, one is here, or zero, and the other four pi. And you have a maximum which is at 90 degrees. What does it mean? It means that this configuration is a minimum, but also the other configuration is a minimum. And the maximum is for m in the equatorial plane of this prolate ellipsoid. Any questions? No. This is the uniarcal anisotopy that you can have. And the better case of oblate steroid is a rugby ball. If you move now to the other situation, is that of proletal and so on. spheroid or ellipsoid. In a prolative spheroid, you have a sort of disk. So it's lumped something in there. X, Y, and Z. Again, A, B, and C. So A is equal to B, and is larger than C. Please. Sorry? Yeah, it's exactly the same, sorry, sorry. Yeah, sorry. This is oblate. Yeah. And collate. I can tell you how to, in my mind, when I have to remember this distinction, I use a strange story. You know, the blade is an older of monks here in Italy. And usually they are not very very... they love food. So that's the reason why there was something more of a blade than the prurates or what. But that's my way. With all due respect for these monks. Nevertheless, I made a mistake. Anyway, okay. In this case, you have a different relationship between A, B, and C. And, of course, this implies that when you write down the same expression, Nxx would be lower than Nzz. The mathematics is the same. Now, if I have to write down the magnetostatic energy per unit volume, I will find exactly the same expression, because the symmetry is the same. Now, if I rewrite this, this will be mu zero divided by two, nzz minus nxx by the cosine square of the angle theta by m square. but the difference is that now this two pi will be positive and the implication is that in this case zero pi two pi is the same here, so the magnetostatic energy, the unit volume will be a cosine square of the angle theta. So in this case you have this. So this corresponds to this. Yeah, so this is like this, like that. Then what is now the message? You have a disk here, and now there's a preferential orientation magnetization in the equatorial plane. This is called, this is Yoni Aichalagy-Gyatopi. This is called easy plane and this is a big one. Why? Because, okay, now the easy axis here is the z-axis. What does it mean easy axis? It's the axis along which it's easier to find the magnetization. It's the easy axis. The axis along which the magnetization loves to stay. Here you have an easy plane, so the magnetization allows to stay in the plane.
\fig{10}{lecture_1 basic magnetostatics.pdf}
In the slides, you will find that in case of the steroid, there is a calculation, this quite famous paper by Holtzburg, which calculates the XF4 of the components of the demagnetizing tensor, assuming that A, B and C have a well-known value. This is just a detail if you want. Apart from that, I would like to point out that this second example is very relevant for practical and technological applications. The reason is that all the technology of the semiconductor industry is based on planar technology. Everything starts in the wicker and proceeds with a particular deposition of layers, thin film. Now the question is, which is the shape and the It's not a miracle. It's a miracle. But it behaves like a prolate or a blate. It's very similar. No blates here. So what we expect to see is what we can't tell. Something very similar to the case of the oblate spheroid. And this is the last message of this lecture. But it's very relevant because we will deal with Fin-Fins. So Fin-Fin means something like that. This is a cross-section and if you want, it extends like this, okay? Now the question is, where do you expect to find the magnetization? How it could be oriented, just considering magnetostatic energy? Which is the easy direction? Out of plane or in plane? Ah, 50-50. As today is the election day, so who will vote for out of plane? of plane. Raise your hand. Nobody. For in plane, a lot of preferences for in plane. It's in plane because the thin plane is very similar to an oblate. But also physically you can understand why. Because let's imagine now you have two options. In the out of plane option, if M is pointing like this, we have a lot of magnetic charges appearing. We have positive charges appearing on, so this is a cross-section. So positive and negative. And this distribution of charges creates a lot of free field, a lot of field, hm, filling the space so that the magnetostatic angle would be large. Okay? The other situation is this one, in which the magnetization is starting with this, and you have negative charges here and there, but they are very far away. Thin film means that you have an 8-inch wicker, 200 mm diameter, with a thickness of 10 mm. Nothing has converted to infinite distance to the plate. And what happens is that the magnetic charges appear and the edge is not the grateful. But here, so it's an infinite distance, so it brings me off the magnetizing field. So here, essentially, you have HM, which is almost zero, okay? And this means that the magnetic energy tends to be minimized in this way. That's the reason why by shape and authority, we most likely behave in a way that's consistent with the preferential orientation of the magnetization in the thin-thin-thin. That's a fundamental achievement of today's science. And if you have to write down the equivalent demagnetized tensor, some people would say, no way, it's not the second-order surface, this one. Do you agree with me? So, basically, you cannot apply this to the theory of demagnetized tensor. Briefly speaking, yes, but let's assume that we want to write down something which is in the framework of the Mario tangential. And so the n should have brought us which expression? Zero, one. Zero, one. Zero, one.
\fig{11}{lecture_1 basic magnetostatics.pdf}
\fig{12}{lecture_1 basic magnetostatics.pdf}

\fig{7}{Lecture_2 Thermodynamics of magnetic bodies.pdf}

\fig{8}{Lecture_2 Thermodynamics of magnetic bodies.pdf}

\fig{9}{Lecture_2 Thermodynamics of magnetic bodies.pdf}

\fig{10}{Lecture_2 Thermodynamics of magnetic bodies.pdf}

\fig{11}{Lecture_2 Thermodynamics of magnetic bodies.pdf}