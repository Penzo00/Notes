\chapter{Exchange and Anisotropy in the Landau Free Energy}
\fig{2}{Lecture_3 Magnetic free energy terms.pdf}
So different contribution toward the F, well then in the end, you want U minus Cs minus and what I'm saying is that I will neglect wisdom, and we are just living with this view. I would just like to add another comment, and it is made in order. It is the structure of things. One could say, ah, okay, what are these things? the short-term seaman interaction energy minus the zero agent which is b dot m so in some textbooks not too many but they they skip all the terminology and just say okay you have your internal energy and your internal energy will be made by more objective interactions and more interaction and more seaman interaction the short card in the end of the day that is the real story and then again in the field of thermodynamics with the minimization of thermodynamic potential but you can also say this is a sort of Zima-like term this is what Zima-edged even though it comes from the magnetic field, it behaves like a seaman energy contribution. Why does it help? Ok, now we have to see now the different contributions. And we start with the exchange. Obviously, you have seen exchange interaction in the framework of different courses and very recently in the course of Jarco. of course you are treating what as the Isambrich Hamiltonian. The basic way you treat now, you move from the physical real space with the real coordinates X, Y, Z to the spin space. And you write an Hamiltonian for the spin part of the ion function for your electron. And when you do that, you discover that you're left with this form of the Hamiltonian, which is minus J, S i, dot S j. And you are making now a sum, when you have a crystal lattice, of the near-snake. What does it mean? The equation is quite simple. We exchange time as m from the perpendicular to the negative. So we are exchanging the value of m from the perpendicular to the negative. there is no change in fraction. And that's the reason why we can really make a fraction of 3 1 and 3. 1 and 2 is reasonable. And in some cases it's not enough. The third magnet of hydrogen is the ion. by definition, but this cannot be explained. The dialectic sense is a complex concept. How do you explain the interaction of dialects? It's not dialectic sense. D dialectic sense cannot be explained. Only archetypal interaction can explain So to say that the distance between the ion atoms is large, to have now a time of value for the target, is a case of a chemical complexity. Notice the negation by the transition of the X-P-L-I-T into a behavior-wide direction. only this kind of exchange and value is here. Only in the traditional theory, but in general, when we have a summation of minus A S i S j, if you limit your summation to I was just saying that you are making a summation of one and two. summation over i and j not e right I have to say over i and j and nearest neighborhood I'm not sure it's a summation over i and j but something that okay okay okay And in this summation here we are considering now couples Ij and Ji twice. That means that Y is equal to 1L over 1Lj. So here J is defined according to this equation, which means the energy of a single state. This is the energy. You have minus the energy of the carbon state. It was two times. In this way, you have IJ and JI. The summation you have IJ and JI. You consider both, so you have to take J as ES minus ET divided by 2. as ij plus ji is your exactly this. And this is the third one. And this is the third one. And now, in this expression, when you write down this, this I mean, so this is an operator, minus j s1, s2. These are spin operators. This is a rule. How these things are being used. But in the form of the micro-magnets, we are using the milliliter spinning classical of the field. These are quantum components. In the end, we are now achieving everything as arising from some magnetic moment place where the atoms are placed. So we are considering a level of inspection of our material, well we stop at the level of the single crystal lattice where we imagine that we have some magnetic diamonds. So see my point? And in this sense the exchange interaction is an interaction between these which means that in Stiegel's theory, S1 and S2 are operators. The formulation of the Hamiltonian, which becomes an exchange energy, in which you move from this to the exchange, which is a summation where i and j of order minus j, s1 dot s2, where these two become some vectors, some sort of pseudo-speed, you can imagine they are what? Some vectors, most So, we have the force of polar power in the exhibition value of the S-X, S-Y and S-V. We're moving from the operator to this lattice of pseudo spins. Ok. Now the problem is how to evaluate the summation of this. Ok. In micromagnetism, the idea is this one. So you say, okay, we are going to do a classical spin. First observation. Now, for a single couple of spins, for example, that's pi, that's the other thing, that's j. What about the angle, which is pi ij? Pi ij cannot be large. That is true. why can phi ij be larger than d pi? if you are concentrating on i and j because if you are concentrating on i and j, it is just differentiated by i and j which is typically a lattice. So in iron it is 2.7 along the side of Q so why phi ij cannot be larger? Otherwise, you change energy for only 2 hours. Change energy helps you to be alive as soon as you feel the energy. In practical situation, be alive for 40 minutes. every time you have a rotation, when you have to move one domain, another domain, this cannot be done just to work in a single line. This is simple because if we calculate the energy cost, minus m divided by s1 is 2 pi minus 1. So this will be really a new challenge. So we need to put the rotation over the chain of the atmosphere. So then, the situation of phi ij is equal to the least equal to the relation of the vector. So for this, in the calculation we will assume that phi ij is much lower than 1, so it's a small angle, And if you have a small angle, when you move now to the vector, Si and Sj, they will have the same magnitude. Why this ? Because of iron. For iron, you have the magnetic moment for single atoms . So, this is relevant here. It's just the relative rotation of this pseudo spin. So, what is going on? My computer is leaving. I'm lost. No. The system is not collaborating. Apart from this, I don't want to connect myself. I don't need a connection for doing this. I don't want to have a connection. Apart from this, I don't want to connect myself. I don't need a connection for doing this. I don't want to have a connection. So let me try to do something different. Now you're seeing something. But now, why I don't see? I've lost the camera. Do you see? Don't ask me why. And now it moves. Okay. So what I was saying is that here, of course, when you take now the summation, you are moving from minus j as high as j to minus j hat square by the summation of the cosine of phi ij. But now you can expand the cosine of phi ij as 1 minus phi ij squared divided by 2. It's the first term of the expansion, the Taylor expansion, the cosine of phi ij. Okay? And when you do this, you are left with this. exchange becomes minus j s squared, the relation of i and j of the time of i and j, which means minus j s squared, plus minus 5ij squared, okay? And now you will get a constant here, when minus by minus, so this is constant, Now, how can you know which is phi ij? And with this geometric approximation, you have now rij, which is the vector connecting the side i and the side j. And now, let's introduce something which is the reduced moment, which will be, say, aligned to the speed of the electrons. Now we are moving from this Hamiltonian, which is energy, which is also there on the same side, side, water. We'll see. Also the magnetization. The magnetization of the spins are strongly connected. Very certain signs of magnet. Signs of the magnet. The minus sign of the magnet. The minus sign. Okay? So if the spin is far from there, magnetization is getting more So, what you can do is that now you move from the idea of the spin of the cyto-speed to so called reduced magnetism. This is the reduced magnetism. That is the important concept for this course. In principle, in your body you have a film. He was describing the magnetization as a function of the point. So the magnetization is only for your volume. Okay? Now, what happens? If I take now a small volume, which is d tau, very, very small, what happens is that by exchange interaction I know that inside this smooth volume all the spin that I have that will be parallel. Why? Because as we have seen the angle of the stream which means that if you take down the volume smaller and smaller and smaller inside, if you have the spin, they will be almost parallel. So what happens is that, what you can say is that here, if you have some spins, they will be almost parallel. And so what you can find now is that the new spin is very well. inside this volume you take now the summation of all magnetic diamonds corresponding to the speed and you divide by the volume you will always find what? the saturation magnetization. what is the concept of saturation magnetization? let's see this. when you have a fellow magnet, you measure the magnetization applied here. Now what you find is this torresius loop. It means that it calcates. And you typically define now, it's not very beautiful, but you understand. So So here you have some interesting aspects. This one is to loosen field. Here, this value This value here is ms, the saturation magnetization that you reach at a very large field. And this value here is the remnants magnetization, mr. The value of the magnetization that you have in remnants when you remove the field. Now, the saturation magnetization corresponds to the magnetic moment per unit volume when you have a very large field, so that the Zeeman energy interaction brings all spins aligned to the external field. So it's the Zeeman energy interaction which overcomes all the other energy turns, provided that the external field is large enough, so that all the spins are parallel, all the magnetic moments corresponding to the electrons are parallel, and in the end you reach the maximum possible magnetization of the system. Saturation magnetization, MS, is a characteristic fingerprint of each material, something that you find in the table for the characteristics of the magnetic material, MS. And this is one of the first things that we measure each time we grow a film. So what is the saturation magnetization that we reach? So that's something which depends on the material itself. And this typically tells you if you have a good material or not, if the material is something wrong or not. Anyway, the remnants magnetization depends somewhat on something which is essentially an anisotropy that you have in your magnetic box. And we will see this after. But what I want to say is that sometimes, the remanence magnetization is even smaller than the saturation magnetization for a very simple reason. That here, when you're in saturation, you have a film, which is a film deposited on a 10 by 10 substrate of silk, 10 by 10 millimeters. So you have that, okay, this is a uniform configuration, so all the spins, all the magnetic moments corresponding to the electric field, all are the same. When you are in remnants, the equation could be different. In remnants, you can have a situation like this, for instance, in which you start seeing This is the domain wall, this is something that I've seen yesterday during an MFL measurement. So, you have a domain wall and you have here on this corner a domain with opposite magnetization, which means that in the end, taking our average, you find something which is more. when you make a measurement like the measurement done in a vibrating sample magnetometer you would hold your sample inside your instrument and you measure the average magnetization so that the rule will tell you that in a very large field you have a saturation magnetization when you go to zero if you have a structure like this you measure the average and it would be reduced with respect to the saturation So now this is something which happens at the macroscopic level. When you go to a volume element, which is d tau, smaller, smaller, smaller, there is no way to see a domain wall inside. And the reason is that for observing this, you must have that your sample is larger than the domain wall. domain wall. The domain wall has a characteristic width that could be, I don't know, 100 nanometer or 1 micron, depending on the characteristic. But if you go now to the detail level, unit volume, which means that inside each unit volume, you can consider that the magnetization is always the saturation magnetization. Just to clarify this a little. Here you have two two magnetic domains, which means that the magnetization here will not be uniform when you cross this. But now if you take a small volume here and a small volume there, saturation magnetization will be cheap, will be far. So what I mean is that exchange intervals is a little wide. Very, very locally, you always find that the magnetization is equal to the saturation . The flocculate, when the beta is very, very small, but it's not based on . It's not just a single domain. So the magnetization will be equal to the saturation . So that's a very fundamental problem. So then, in a real object, when you want to know the distribution of the air and the particle of fire, in reality, this is a problem in which what makes sense, what gives you the information, is not the magnitude of that vector. It's the orientation in the space. So the magnitude will be always the saturation magnetization, very low. And what will create now the interesting micro magnetic configuration will be the orientation of that vector in the space. You see my point? That's a fundamental concept for this group. So what you can say is that the magnitude of m, the function of r, will be always equal to the saturation magnitude in a small volume. Because that's the definition, OK? So m as a function of r is what? is delta m over delta tau the limit for delta tau and the limit for delta m over delta tau ok better definition but as soon as delta tau becomes smaller smaller, smaller you are reaching a single configuration which is always a single domain so you find always saturation so the main information is the orientation of the vector and this is described by the so-called reduced magnetization if this is clear what you can say is that ok, so the f, the function of r will be equal to ms multiplied by a vector a unit vector m that will depend now on the specific position. What is interesting is just the orientation, so you just need to know how this unit vector, which is parallel to the magnet position locally, is what is arranged in the space. And this is the content and again, what is going on? Now as soon as I connect I know probably there is a timer here with me something here in the background you see so very something in the background nobody knows what people is seeing at home as we are approaching the break so let me go to the next So essentially, now we understand that the basic information is here. M is a function of r, but you can also write as M divided by saturation. Dimensionless field, and we are interested in marking this dimensionless field. So in some sense now, what you can do is... That's interesting to me. I see. That's really stupid. Anyway. So, something that you can do is that you say okay, s and now if you need to evaluate this summation of i by j square, I can do something like this. I think this was the spin SI, okay? And this is IJ and this is the other speed SJ. Now, you can replace this spin with the reduced magnetic magnetization MI and here MJ. This is the angle phi ij. So to evaluate the phi ij, you can say, okay, phi from the phi ij, you can use this approximation as phi ij is very, very, very small. Okay. Say that this is something you can approximate with mj minus mj. why? because this value is delta m and at this unit vector this delta m corresponds to the angle this delta m can be approximated with the angle itself so in reality you have to evaluate the model square of m i minus m j. And something that we will see after the coffee break, if the coffee is available, is this one. But we can now say this is equal to what? Let's say it's r i j dot gradient of m. That's the approximation material. It's the whole model that you can do. Of course. And then you have to take this. So now we take a break. I'll try to restore now the issue. And then I will explain how this can be calculated for a square lattice and now we can find out a reasonable expression for the exchange intervention. I'm going to do a lesson here, a lesson in a decent way, and that's it. I think we should continue to do this, because even if we stay here and do lessons decently, we have to keep looking at this. You are right about this, but it's a problem we have to solve, so that the mind can do lessons decently. One thing seems reasonable to me, as sometimes it's all about... I'm a bit silent, but there are some slides I have to read. I explain, I write, twice a day I wash my hands, I can't do it, but I can do it. Ah, and if not, I have to go to the bathroom to wash my hands. All right, let's take a little break. Okay, so now What we have to do is to explain the meaning of this expression. This expression is very similar to something you are familiar with, which is the evolution of the difference of a variable, which could be, for instance, a scalar variable, a temperature. So when you have a temperature field, for instance, describing the temperature of the pool, so you want to evaluate the difference in temperature in point T, which is separated by a small distance dr, so you have the gradient of T and you just have to evaluate the scalar product between dr, which is the vector connecting the points, and the gradient. And of course this is slightly, so you take now the projection of the gradient on the different direction, now the derivative you multiply by dr and you find the delta of the value okay but in our case it is slightly different so if i have to follow which is here yeah let's see exactly not exactly i know OK. If you assume that what I wrote makes sense, which is formally very similar to the case of the power of theta, then it can be shown that the energy written as the summation of phi h square which becomes r i j dot nebla of phi m. This is how to be in case for distance of the square lattice an expression like this in which we have a constant which is the stiffness a I will define in a part divided by two by the integral of the volume of the gradient of m x square plus the gradient of my square plus the gradient of mz. And a is this exchange between j, s square, z, y, y, and 4. If you require this for an hexagonal closed pack, a lot of this would be different expression, but in any case it's something connected. So now we have to jump from this to this. And now let's make a calculation which corresponds to this line.
\fig{3}{Lecture_3 Magnetic free energy terms.pdf}
So the exercise finds the expression for the infinite energy in a simple cubic lattice. This is not higher. It's a simple cubic lattice, okay? an atom at each corner of a cube. OK. We start from this expression, the phi ij, which is mj minus m i. And the key point is that yij squared can be written like this, delta mx squared plus delta my squared plus delta mz squared. and now we are back to the scalar field. mx is a function of r, my is a function of r, mz. We can call it the reduced magnitude. Now you can apply what we wrote here for the temperature field, for instance. you can say okay what about delta MX it would be Rij dot what gradient of LX and here you have the expression from the story and this is the meaning of the operator that you wrote before. Now you should be able to see it. What does it mean? rij.NAMLA apply to m. It's exactly this. Okay? r h i dot okay now for the energy now what is the e is the energy for the energy you have to place j s squared divided by two by the summation i and over j now the problem is how to to the only solution for the cubic lattice. So, one, let me write this here so that I will not jump from this one, r i j, you know, the gradient of m y, where it was, r i j, you know, the gradient of m z. okay now you have a simple cubic lattice which means as we have to carry out the summation here, the energy will be j, l squared divided by 2, summation over first index, pi, and then a summation over j, pi, and then a summation over j of everything is inside square brackets. What does it mean? It means that I start sitting on the side i and then I vary j in order to touch all the nearest neighbors. So this is i, that side, and now I have my cubic lattice which means this will be the y direction, this will be x direction, this will be z direction. x, y, z. So I have an atom here that I put here, it is on the crystal lattice point and then I see the nearest neighbor. Let's start from the x coordinate. So what happens is that if the lattice constant is a, I will have the nearest neighbor placed here at plus a and another one at minus a. Okay? So, what I mean is that along x, along x, you have Rij, which will respond to what? Plus a by ux and minus a by ux, where ux is the unit vector in the x direction. Okay? And now you have to evaluate this rij dot the gradient of mx. But as rij is a vector along the x direction, this will select just a partial derivative of mx with respect to x. Okay? So this means that when you start doing this so we are here you will have what? plus a let's consider the first one this one you will have a dot the derivative of mx with respect to x. And this describes, and this of course, and this is describing just this r i j. Then you have the second one which is then corresponding to this vector here. j now is moving there. And for that one, you will find out in principle minus a, okay, but as you have now the square here, this will be exactly equivalent to the first term, there is a factor 2 appearing, okay? And this is the way you obtain 5i, this first term here, Now for the same i and j, you have to evaluate the other contribution. So this is just the first contribution. Now you move to the other one, this one. Now what happens is that, well now r and j which is always the same, which is the same. And now you have that, you have what? The gradient of my. The gradient of my will contain all the derivatives of my with respect to x, y, and z. But as here you have that ij is a log x, you are selecting the partial derivative of my with respect to x. Okay? So you will think A by the derivative of M Y with respect to X. Is it clear? Because you have to carry out this dot product. R I J dot the gradient of M Y. So R I J is equal to A by U X. gradient of my is derivative of my and x by ux plus derivative of my y by uy and so on. When you take the dot product between the two and this one is you are selecting just this one. And when you move to the last term which is this one of course is the same, plus h squared, the partial derivative of mz with respect to h squared. Okay, and this is just the calculation for these two nearest neighbors here, along n. Then you move to the nearest neighbor along y. So in this case you will have these two guys here, Again, plus a and minus a. So if you go along y, you will do the same thing and what will change is that along y, you have that rij will be what? Plus auy minus auy. Which means that now you are selecting the partial derivative with respect to y, not with respect to x. So you will get exactly the same expression here, but instead of derivative with respect to x, you will have it with respect to y. and this is exactly oh my god this is something that is unexpected because I change things in order to avoid now I probably know how to do that one large equal don't ask me why but and you get the second row here, you see the partial derivative with respect to x has been replaced by a dead derivative with respect to y. And when you move now to the third couple of the nearest neighbors, these guys here, along z, of course you are selecting now the partial derivative with respect to z. And this is inferred from this solution. Good. Now, look at this expression and don't look at it by rows, but by columns. And tell me how you can write down this solution. So if you look at the summation of these three elements in the first column, what you can say is that the total energy would be equal to J s squared divided by 2, which was our free factor, multiplied by what? a by gradient of mx squared. is a square multiplied by the gradient of mx squared. And when I consider the second rule again, the derivative of my with respect to x, the derivative of my with respect to y, with respect to z, these are the components of the derivative of my. Plus the gradient of my squared. And for the third form, the same. You see the derivative of mz with respect to x, y, and z. So in the end, you get the gradient of mz. Look. Yeah, maybe it's a 2. I used it to. I too. OK.
\fig{4}{Lecture_3 Magnetic free energy terms.pdf}
Now, just a few comments. Now, you understand the meaning of the expression that I wrote before. And you see that in the expression for the exchange find out that the 2 and the 2 they cancel out by Nicola. Nicola was here yesterday. Nicola, you can change the way you write it. So yesterday Nicola was the first one. The left one. So here is the perturbation of index. Okay, anyway, sorry about that. So it's j squared and so on and this summation. Now, you move from the summation to an integral. The way to do that is that you have to see what is the point element of the tau. And of course, what is now the unit volume is 2. Or your volume, sorry, is a to the third. You make an integral over the volume and you divide by the volume of the unit cell which is a to the cube. So that this a squared and a cubed are partially simplified and you find this expression. So the integral over the volume contains the gradient of mx squared, the gradient of my squared, the gradient of mz squared. And in front of it you have now a constant a divided by 2. and of course according to this definition A is 2 times J square divided by 1 to exchange it. Okay, let's understand now what is the meaning, the physical meaning of this expression that we have. This is the exchange energy cost of the micromagnetic configuration in which you are deviating from the ground. For exchange interaction in a ferromagnetic body, the ground state corresponds to all the spins following. Align, follow, okay? This is the ground state. If for any reason there is a tilt of the local gradients magnetization, giving rise to a certain movement, different domain, there are some locations, here we have whatever you want. So if you have such organization, there is an energy cost which is exchanged to the person that can do that political action. And which is the key ingredient for real point here, is that you have an exchange energy cost only if So it's a non-uniform distribution. Because if you have a non-uniform, you will have a non-none value of the gradient of mx, my, and mz. So this is clearly the expression stating that there is an exchange energy cost only if there is a variation of the . Only if m is not equal. Only if the gradients are non-sine. Otherwise, it's not going to be this way. And there is this factor in front, A, which is providing you with an estimation, what you get is an estimation of the strength of the X-axis. That's why it's called exchange stiffness. So it's really the stiffness of the magnetic field. value will be stiff, it will resist against any tendency to create a tip. And as you see it is proportional to the h constant j. So, if it is pretty strong, h to the h in the large distance, any deviation will create a very very huge curve. the stiffness, it's very steep. Depends now on the magnitude of the spin, angular momentum, s, which is . And because it is inversed, inversed a lot, it's proportional to the . Why? When you see an expression, you have always to ask yourself, they are worth eating, why I eat and do it, which is the physical meaning. They are far away. Yeah, I'll see. The problem here is that J principle already . So this dependence is just stressing the time there. If you take now . even though it's like this one small d and it's strongly on the connect that one piece calculation which you can't put on the on the summation of the i and j to begin anyway for a cubic lattice for simple cubing we found the expression which is times j s squared divided by 2 for a cubic lattice we have to do the calculation that we have done So, this is the way it works. And so, the expression is, the information is called by Z, and you do, and then you ask me, and I'll tell you, and you find that, okay, the exchange quarter can be written in a different way than J is created by E multiplied by Z. And you said, assume the value 1 for the sequence, And now I find out the value of A. A is always in the order of 10 to 5 joules per meter. for firm alloy which is nickel iron nickel 80 iron 20 which is very popular alloy that's really but always future calculation. And this is just exchange interaction. So now we have other interaction to see and we start seeing anisotropy.
\fig{5}{Lecture_3 Magnetic free energy terms.pdf}
Because the power energy is something that we have already seen so we already know the way you can calculate it. So the first point, the first comment is this I want to ask you, is exchange interaction isotropic or anisotropic? In the isomer and the sodium, there is no dependency on the absolute So, if you have a lattice and the whole speech are oriented like this or are oriented like that, it is not difficult. What makes the simple ideas the relatively more efficient are the absence of the simple ideas. Only the high school. That's one of the reasons why Paine's research in predicting some behavior of real systems probably only Ludwig Wagner and Bernadette Schiphol don't know if we can see it. Anyway, isomer Hamiltonian is really isotropic, but in practice magnetic material, they do not display a fully isotropic. So the point is that if you have, for instance, a body as we have seen, a cylinder, so it has a different energy for the magnetization along the z-axis or in the right direction. So you have to find a way to describe the anisotropic energy term. And this is just, in this slide you see, a way for representing an isotopic energy independently on the opposite. So how do you represent it? How do you... That's the answer that you find in this slide. Essentially you say that you start from the magnetization and as we already discussed in this lecture, the magnetization factor locally has only the same level of fluctuation. What makes the difference is just what could be the condition that we have. So what is relevant is the polar angle theta and the azimuthal angle phi, the direction of m in the capital M, all the different components that you can write down there. But now the question is, how, what is the expression for the anisotropy energy, which is usually written like F, anisotropy, capital F means real energy, and you have the small f which gives you the n-th derivative by the capital F by the n-th derivative. And so you have to find an expression for this function f describing now the way, and this depends on the orientation of the m-th unit vector. So it's the function of the unit vector. And the usual way for representing the anisotopy is that considering the concept of anisotopy energy surface. These are surfaces that you can nearly build in a three-dimensional space, and the idea is this one. you build up a surface so that a point on the surface, A, at the intersection between the direction of the vector m and the surface is placed at the distance from the origin which is proportional to f for that particular so that in this case for instance you have that f is larger here and on the equatorial plane. So the distance from this point to the origin is one. I repeat, so essentially you build up a surface so that the distance from one point to the surface to the origin is proportional to the value of the horizontal energy for the magnetization vector along the OA direction. Okay? So this means that in this case, for this specific shape, you see you have two lobes along the z direction, so this means that that direction is a hard axis. Along that direction, the energy is not as strong as along this direction or the other direction. While along first of the y direction you see when because here is smaller, so this means that the system has a good magnetization along the xy in the xy. You see the meaning of the equation. F is an exotically energy where the magnetization is not taking place, which cannot be hidden away. As I was telling you, this is not answering the question which is the origin of anisotropy. It's just saying that if you have anisotropy, you can include it into a low-key energy by introducing an anisotropy energy that will depend only on the absolute aeration of the reduced magnetization with respect to some axes which are referred to in the system. the way you represent it graphically you are using the concept of anisotopy energy surfaces is described in some sense visually the kind of result we stand this one is a sort of easy plane and is of the field. So the easy axis will respond to local minima. So here you will have an easy axis in the plane. And the other axis will correspond to the z, because you have a large video on the x-axis. Let's have a look at something which is better represented here.
\fig{7}{Lecture_3 Magnetic free energy terms.pdf}
It's exactly the same situation as my previous sketch. Let's assume that this is the surface of the energy. You see that along the z direction, because at this point here is a large distance with the stratosphere. This means that if the magnetization is along the z direction, we have a maximum of the energy of the plane. If the magnetization is along the xy plane, we are very very close very close to that. And this other piano too will be more. This is a way for representing realism. The graphical way for representing realism. Another case here. Probably you see that this is a sort of donut with a minimum here inside. in this case this is a easy axis along that direction the distance of the eight-point center okay and for today probably I'm done yes now you have to move to other lecture.
\fig{6}{Lecture_3 Magnetic free energy terms.pdf}
Anyway what I was describing is the so-called uniaxial anisotopy why do you say uniaxial anisotopy it's easy to see that this surface energy like I a by side we will see this Monday that was the question Monday the meeting of the we have lecture in the other any reason why there shouldn't be lecture I don't know let me check what about the other like so do we have our other lecture during Monday? Voi non avete? Senti, voi fate la lezione? E io perché no? Faccio lezione anch'io? No, no, faccio lezione anch'io. Tanto è le pomeriggio, le due, io vado lì. Le hanno occupate? Ah, that one, he says. Maybe there was... In Via Colombo, who wants to go there? That's an abandoned place. In my opinion, no. I'll tell you one thing, though, that I'm not clear about. Thank you. For the exchange, he says. or a old old old old old It is possible to change the temperature of the oil. What are you saying? We are ideally looking at the temperature of the oil. It is because of the temperature of the oil and the substances that we use. Yes, but not completely. This is true, if you think, this is for other discussions, but if you think also about the model of Heisenberg, the model of Ising, But as soon as she turns on the thermos, what happens? that can activate in the excited states. And you have seen that in the Enchiridion, the excited states correspond to the bones, to the bones. What does a bone do? A bone eats one, eats a piece of bone, two pieces of bone. We will give a moment to talk about the cut blood. And so what happens? If you look at this object, the temperature is decreasing. Why is it decreasing? Because as the temperature increases, it activates the excitations that correspond to the system of the maniocs. and so when the altibas is at the same level that is, as soon as the temperature increases, then here is the contribution of the tropic and so on and so on which is just a treatment I mean, I didn't understand at the level of... No, but you have to think that it is always like that, it is not a system that has an energy an internal energy U and that says if you do not consider the tropic term the system minimizes the gypsum for parallel What is the temperature of the oil? The temperature of the oil is the one for which KT is the oil of the I don't know. I don't know if you can see it. It's just at the level of finding a minimum of energy. But, be careful, I break the maximum symmetry to get to that point. Why does it naturally proceed? It's an explanation. Apart from that, there is the Miltonian, it is minimized. So, first of all, to obtain... So, let's start right in reality. I start from a target of Mario Rubato or whatever he wants, I will never do anything. And then I deposit my son. When I have deposited my son, he does not have any interest in me. I open a box. Then what happens? He has his own cycle of hope. I have here, I do the first movement of my fingers, I bring it here, and then when I return to the side, I go out. And it gives me a certain amount of remaining. Why do I have that thing there? Because if I go to see what happens, in terms of energy, If I did this, and then I would have to divide it together, I would have to divide it by two, let's do it like this, let's say that this here is mx and this here is hx, right? So, let's say that I have to take a field with this direction, let's say that I have to take this hx, and I do now my m, let's assume that it is 5 degrees, then in reality the mechanism can also be tested with the energy of the earth This is the angle of the mirror. We would have to start with a minimum of zero. A minimum of zero. This corresponds to this point here. And this here corresponds to this point here. These are two absolutely different gestures. but it generates energy. So what do I do with the magnet? I put it here and I can't jump from the other side because it doesn't work. I mean, no... You can always think that then the entropic tendency, that is the magnetic tendency, So much so that no magnetic memory is foreign. Why? Because the reaction is the same in every part of the transition. The other part of the transition will always depend on the input. Here is its activation energy, that is the transition. So the next transition is the proportional to E to the equal to A divided by the equal to A. The point is to understand how the one with the A is made. There are systems of the hyperparamagnetic field, with which she can make this comparison with K-tines. She must be punctual. She never sees it like this. She must be punctual, because the disorder tendency does not do it. But if I have a much larger K-tine, the system appears blocked. But it appears blocked for the observation time of the eyes. I forgot to ask you I wanted I wanted to know if this is only a curve for the applied field and not about the whole work. The work that is negative is actually spent on the the magnetic energy of the system. In this case I would consider it a solution to the first number of the system. But it is. The point here is that this one, this guy here, is not the torque on the wall. Just to watch only the happening end of the world. Magnetic work. But don't forget that you cannot achieve this without producing and distributing the mind. So this means that in the end you have to spend some money. But the energy that you have to spend is limited. then the energy that you are using comes in. So the point is that you do this entire demagnetization and you have a diamagnetic system. You still have to spend some energy. But energy is less than you would spend without it. And we don't quite understand that. So what you mean is that to create this applied field, you have to spend some energy. You always have to spend some energy. And so in the case of a diamagnetic matter, you will spend less energy. That's the point. That's why it's negative. Just less energy is not the negative end. We're not producing it.
\fig{8}{Lecture_3 Magnetic free energy terms.pdf}

\fig{9}{Lecture_3 Magnetic free energy terms.pdf}

\fig{10}{Lecture_3 Magnetic free energy terms.pdf}